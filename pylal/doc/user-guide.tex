%
% =============================================================================
%
%                                   Premable
%
% =============================================================================
%

\documentclass{book}
\usepackage{fullpage}
\usepackage{pylal}


% Add your name to the author list!  (alphabetical order?)

\title{pyLAL Applications User Guide and Reference}
\author{Kipp Cannon, Thomas Cokelaer, Alexander Dietz, Steve Fairhurst}
\date{\today}


%
% =============================================================================
%
%                                   Document
%
% =============================================================================
%

\begin{document}
\maketitle
\tableofcontents
\listoftables
\listoffigures
\chapter{Coding Conventions}
\section{General}
The executables should contain some standard code at the beginning of the document such
as
\begin{verbatim}
__version__ = "$Revision$"
__date__ = "$Date$"[7:-2]
__id__ = "$Id$"
__name__="plotthinca"
\end{verbatim}
so that the version, and Id can be used within the document if needed.

\section{Naming conventions}
The output file should be 
\begin{verbatim}
<executable name>_<usertag>_<figure name>-<GPS start time>-<duration>.png
<executable name>_<usertag>-<GPS start time>-<duration>.html
<executable name>_<usertag>-<GPS start time>-<duration>.cache
\end{verbatim}
Where, executable name is defined by the variable \prog{\_\_name\_\_} (see above)


\section{arguments}
The authors should try to use the following convention for the optional
argument as much as possible
\begin{description}
\item[\progarg{--cache-file}] to be used to provide a cache file
\item[\progarg{--figure-name}] a tag to be used for naming the output figures
(see Naming conventions).
\item[\progarg{--verbose}] Be verbose
\end{description}

\section{TODO}
sort the list of arguments by alphabetical order.



\chapter{Burst Tools}
\begin{manpage}{\prog{ligolw\_binjfind}}

\manname{\prog{ligolw\_binjfind}}{A program for finding burst injections.}

\mansynopsis{\prog{ligolw\_binjfind} \{\progarg{-c},\progarg{--match-algorithm}\} \progparm{algorithm} [options] [file [file ...]]}

\begin{mansection}{Description}
\prog{ligolw\_binjfind} applies a burst injection identification algorithm
to one or more LIGO Light Weight XML files, each containing a list of burst
triggers and a list of software injections.  If no file names are provided
on the command line, then input is read from \file{stdin}, and output
written to \file{stdout}.  Any file whose name ends in ``\file{.gz}'' is
assumed to be gzip-compressed.

The results of the search are recorded using the standard coincidence table
infrastructure.
\end{mansection}

\begin{mansection}{Options}
\begin{description}
\item[\progarg{--comment} \progparm{text}] Set the string to be recorded as
the comment in the job metadata added to the files.  The default is ``''.

\item[\progarg{-f},\progarg{--force}] Normally input files whose internal
metadata indicates they have already been processed will be skipped.  This
behaviour facilitates the use of \prog{ligolw\_binjfind} in Condor DAGs,
where it is convenient to have one job process many files but where a job
may be evicted and re-started, causing it to re-read each file.  Setting
\progarg{--force} forces files to be re-processed.

\item[\progarg{-h},\progarg{--help}] Show a help message and exit.

\item[\progarg{-c},\progarg{--match-algorithm} \progparm{algorithm}] Set
the algorithm to be used in comparing burst triggers to injections.  There
are currently two algorithms defined, \progarg{excesspower} and
\progarg{stringcusp}.  The former is suitable for use in comparing the
triggers generated by \prog{lalapps\_power} to a list of burst injections,
while the latter is for use in comparing the triggers generated by
\prog{lalapps\_StringSearch}.

\item[\progarg{--verbose}] Be verbose.

\item[\progarg{--version}] Show program's version number and exit.

\end{description}
\end{mansection}

\begin{mansection}{See Also}
\prog{ligolw\_add}, \prog{lalapps\_binj}, \prog{ligolw\_burca}
\end{mansection}

\end{manpage}

\chapter{Inspiral Tools}
\begin{manpage}{\prog{plotnumtemplates}}

\manname{\prog{plotnumtemplates}}{A program for plotting template bank size.}
\mansynopsis{\prog{plotnumtemplates}  [options] }

\begin{mansection}{Description}
\prog{plotnumtemplates} scan template and/or triggered template bank, 
search for nevents in the summary table and create a pictures combining 
the template bank sizes for each ifo provided.  Any file whose name ends
in ``\file{.gz}'' is assumed to be gzip-compressed.

The results of the search can be stored in a png figure.
\end{mansection}

\begin{mansection}{Options}
\begin{description}
\item[to be done]

\item[\progarg{--verbose}] Be verbose.

\item[\progarg{--version}] Show program's version number and exit.

\end{description}
\end{mansection}

\begin{mansection}{See Also}
\prog{plotnumtriggers}, \prog{plotinspiral}
\end{mansection}

\end{manpage}


\end{document}
