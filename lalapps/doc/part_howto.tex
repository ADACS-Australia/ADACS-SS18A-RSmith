%
%
\color{black}
%%%%%%%%%%%%%%%%%%%%%%%%%%%%%%%%%%%%%%%%%%%%%%%%%%%%%%%%%%%%%%%%%%%%%%%%%%%
\section{Getting and installing the software}
%%%%%%%%%%%%%%%%%%%%%%%%%%%%%%%%%%%%%%%%%%%%%%%%%%%%%%%%%%%%%%%%%%%%%%%%%%%

We describe the installation of \lal\ and \lalapps\ for use in
gravitational-wave data analysis.   These packages are under active
development at the present time and the latest functionality is
available only from CVS.   These instructions assume a standard
Red Hat 7.3 or 9 installation including development tools.   

\subsection{Installation of required development tools}

A useful installation of \lal\ and \lalapps\ requires \textsc{fftw3},
the \textsc{frame}, \textsc{metaio} and \textsc{gsl} libraries.   Pre-compiled
versions of these libraries are available as binary RPMs for Red Hat 7.3 and
Red Hat 9. If you do not know know what distribution of Red Hat you are
running, this can be obtained by looking at the contents of the file 
\begin{verbatim}
/etc/issue
\end{verbatim}
On a Red Hat 7.3 system this file will contain the line
\begin{verbatim}
Red Hat Linux release 7.3 (Valhalla)
\end{verbatim}
and on a Red Hat 9 system it will contain the line
\begin{verbatim}
Red Hat Linux release 9 (Shrike)
\end{verbatim}
To install the required development tools from binary RPMs, follow section
\ref{ss:rh73} if you have a Red Hat 7.3 machine and section \ref{ss:rh9} if
you have a Red Hat 9 machine. If your machine has a different type of Linux
installed, you may try obtaining the source RPMs and building them manually as
described in section \ref{ss:srpm} .

\subsection{Red Hat 7.3 Binary RPM Installation}
\label{ss:rh73}

The binary RPMs for Red Hat 7.3 are located at
\begin{verbatim}
http://www.lsc-group.phys.uwm.edu/lal/rpms/redhat-7/
\end{verbatim}
You will need to download and install the following files from this directory
\begin{verbatim}
fftw-3.0.1-4.i386.rpm
frame-6.12-1.i386.rpm
gsl-1.1.1-1.i386.rpm
gsl-devel-1.1.1-1.i386.rpm
metaio-5.4-1.i386.rpm
\end{verbatim}
You may also download the following \emph{optional} RPMs 
\begin{verbatim}
frame-matlab-6.12-1.i386.rpm
metaio-matlab-5.4-1.i386.rpm
\end{verbatim}
if you would like to use some additional Matlab only functionality provided,
however these are \emph{not required} to build and install \lal or \lalapps.
Download the RPM's, log in as root on your computer, then type
\begin{verbatim}
rpm -Uvh fftw-3.0.1-4.i386.rpm
rpm -Uvh frame-6.12-1.i386.rpm
rpm -Uvh gsl-1.1.1-1.i386.rpm
rpm -Uvh gsl-devel-1.1.1-1.i386.rpm
rpm -Uvh metaio-5.4-1.i386.rpm
\end{verbatim}
If you downloaded the optional Matlab RPMs, then type
\begin{verbatim}
rpm -Uvh frame-matlab-6.12-1.i386.rpm
rpm -Uvh metaio-matlab-5.4-1.i386.rpm
\end{verbatim}
Watch for error messages.  If everything installs correctly,  skip to
Sec.~\ref{ss:config},  otherwise skip to section~\ref{ss:srpm}.

\subsection{Red Hat 9 Binary RPM Installation}
\label{ss:rh9}

The binary RPMs for Red Hat 9 are located at
\begin{verbatim}
http://www.lsc-group.phys.uwm.edu/lal/rpms/redhat-9/
\end{verbatim}
You will need to download and install the following files from this directory
\begin{verbatim}
fftw-3.0.1-4.i386.rpm
frame-6.12-2.i386.rpm
gsl-1.1.1-5.i386.rpm
gsl-devel-1.1.1-5.i386.rpm
metaio-5.4-2.i386.rpm
\end{verbatim}
You may also download the following \emph{optional} RPMs 
\begin{verbatim}
frame-matlab-6.12-2.i386.rpm
metaio-matlab-5.4-2.i386.rpm
\end{verbatim}
if you would like to use some additional Matlab only functionality provided,
however these are \emph{not required} to build and install \lal or \lalapps.
Download the RPM's, log in as root on your computer, then type
\begin{verbatim}
rpm -Uvh fftw-3.0.1-4.i386.rpm
rpm -Uvh frame-6.12-2.i386.rpm
rpm -Uvh gsl-1.1.1-5.i386.rpm
rpm -Uvh gsl-devel-1.1.1-5.i386.rpm
rpm -Uvh metaio-5.4-2.i386.rpm
\end{verbatim}
If you downloaded the optional Matlab RPMs, then type
\begin{verbatim}
rpm -Uvh frame-matlab-6.12-1.i386.rpm
rpm -Uvh metaio-matlab-5.4-1.i386.rpm
\end{verbatim}
Watch for error messages.   If everything installs correctly,  skip to
Sec.~\ref{ss:config}, otherwise skip to section~\ref{ss:srpm}.

\subsection{Installing from Source RPMs}
\label{ss:srpm}

If the binary RPM's do not work for some reason or your particular O/S
is not supported,   the source RPM's are also available at the LAL web
at
\begin{verbatim}
http://www.lsc-group.phys.uwm.edu/lal/rpms/source/
\end{verbatim}
Download the files
\begin{verbatim}
fftw-3.0.1-3.src.rpm
frame-6.12-2.src.rpm
gsl-1.1.1-5.src.rpm
metaio-5.4-2.src.rpm
\end{verbatim}
to your computer.  As root,   type
\begin{verbatim}
rpm -Uvh fftw-3.0.1-3.src.rpm
rpm -Uvh frame-6.12-2.src.rpm
rpm -Uvh gsl-1.1.1-5.src.rpm
rpm -Uvh metaio-5.4-2.src.rpm
\end{verbatim}
Watch for errors.  If this fails,  there is a major problem:  e-mail
\verb+lal-discuss@gravity.phys.uwm.edu+.    If it succeeds,  as root
\begin{verbatim}
cd /usr/src/redhat/SPECS
rpm -bb fftw.spec
rpm -bb frame.spec
rpm -bb metaio.spec
rpm -bb gsl.spec
cd /usr/src/redhat/RPMS/i386
rpm -Uvh fftw-3.0.1-3.i386.rpm
rpm -Uvh frame-6.12-2.i386.rpm
rpm -Uvh gsl-1.1.1-5.i386.rpm
rpm -Uvh gsl-devel-1.1.1-5.i386.rpm
rpm -Uvh metaio-5.4-2.i386.rpm
\end{verbatim}
The Matlab add-ons for frame and metaio will not be available if you build the
RPMs from source.  If the all the above fails, then try the following:
\begin{enumerate}
   \item Later versions of the RPM software require \verb+rpm -bb+ to be 
      replaced by \verb+rpmbuild -bb+. 
   \item If \verb+rpm -bb+ \textit{package} (or \verb+rpmbuild -bb+
      \textit{package}) fail, you might be able to build by hand:\\
      \verb+cd ../BUILD/+\textit{package-dir}\\
      \verb+./configure+\\
      \verb+make+\\
      \verb+make install+\\
      where \textit{package} is the package for which \verb+rpmbuild -bb+ 
      did not work. The \textit{package-dir} that belongs to 
      \textit{package} should be obvious. You do not need to do \verb+rpm -Uvh+ 
      \textit{ package.i386.rpm} for packages installed in this way (this
      binary rpm should not even exist).\\
      For fftw the you should use use the flag \texttt{--enable-float}
      when you run \texttt{configure}.
   \item If none of this works, e-mail\verb+lal-discuss@gravity.phys.uwm.edu+.   
\end{enumerate}
If it succeeds,  then you can go ahead to the next stage.   

\color{black}
%%%%%%%%%%%%%%%%%%%%%%%%%%%%%%%%%%%%%%%%%%%%%%%%%%%%%%%%%%%%%%%%%%%%%%%%%%%
\subsection{Configuring your environment}\label{ss:config}
%%%%%%%%%%%%%%%%%%%%%%%%%%%%%%%%%%%%%%%%%%%%%%%%%%%%%%%%%%%%%%%%%%%%%%%%%%%
\color{black}

While developing,  we recommend that you install the \lal\ and \lalapps\
somewhere under your home directory.  If you follow the
instructions below,  the \lal\ and \lalapps\ libraries will be in
\verb+$LALPREFIX/lib+;  the documentation in \verb+$LALPREFIX/doc+; the header
files in \verb+$LALPREFIX/include+; the binaries in \verb+$LALPREFIX/bin+.
The environment \verb+LALPREFIX+ should be set to the absolute path where you
want to install these things.  For example,  user \texttt{patrick} might put
the source code in \verb+/home/patrick/src+ and set \verb+LALPREFIX+ to
\verb+/home/patrick+.

To make the appropriate binaries accesible to you,  check that your path is
set correctly. It is also useful to add environment variables for the CVS
servers. If you have a lal CVS login, change
\verb+anonymous@gravity.phys.uwm.edu+ to
\verb+your_user_name@gravity.phys.uwm.edu+ in the environment variables below.
If you are using \texttt{bash}, add
the following lines to your \texttt{.bash\_profile} file
\begin{verbatim}
export LALPREFIX=${HOME}      # <---- Change this as appropriate
export PATH=${LALPREFIX}/bin:${PATH}
export LSCSOFTCVS=":pserver:anonymous@gravity.phys.uwm.edu:2402/usr/local/cvs/lscsoft"
export MATLABPREFIX=/usr/share/matlab:${MATLABPREFIX}
\end{verbatim}
If you are using \texttt{csh} or a derivative,  add the following lines to
your \texttt{.cshrc} file
\begin{verbatim}
setenv LALPREFIX $HOME
setenv PATH $LALPREFIX/bin:$PATH
setenv LSCSOFTCVS ":pserver:anonymous@gravity.phys.uwm.edu:2402/usr/local/cvs/lscsoft"
setenv MATLABPREFIX /usr/share/matlab:${MATLABPREFIX}
\end{verbatim}
These enviromnent variables must be set before running anything,
so it is a good idea to log out and log back in again before
continuing. You may delete the line that sets the variable 
\texttt{MATLABPREFIX} if you have not installed the optional matlab RPMs.

\color{black}
%%%%%%%%%%%%%%%%%%%%%%%%%%%%%%%%%%%%%%%%%%%%%%%%%%%%%%%%%%%%%%%%%%%%%%%%%%%
\subsection{LAL}\label{ss:lal}
%%%%%%%%%%%%%%%%%%%%%%%%%%%%%%%%%%%%%%%%%%%%%%%%%%%%%%%%%%%%%%%%%%%%%%%%%%%
\color{black}

In the following commands, remember \verb+LALPREFIX+ is the
absolute directory path where you want to install the LAL library.  If
\verb+LALPREFIX+ does not exist,  you must create it:
\begin{verbatim}
mkdir -p $LALPREFIX
\end{verbatim}
Then create the directory into which you wish to put the source files for LAL
and LALApps:
\begin{verbatim}
mkdir $LALPREFIX/src
\end{verbatim}

The LAL software is maintained in a CVS repository -- CVS stands for
Concurrent Version-control System which is tool to allow multiple developers
to manipulate the same software,  merging differences and identifying
conflicts between changes if they arise.  Obtain the LAL package from the CVS
repository as follows:  
\begin{verbatim}
cd $LALPREFIX/src
cvs -d $LSCSOFTCVS login
\end{verbatim}
At this point,  you will be asked for a password.  The password for the
\verb+anonymous+ user is \verb+lscsoft+. If you have your own username, use the
password that you have been given.
The version of LAL that is checked-out is identified by the argument
to the \texttt{-r} option in the commands below.   The HEAD tag checks out the
current development version.  If you want to check out a released
version, replace the \verb+HEAD+ tag by \verb+release-X-Y+ where
\verb+X+ and \verb+Y+ are integers which identify the release version
number as \verb+X.Y+. You may also replace \verb+HEAD+ by a specific tag name,
for example if you want to download the \verb+iulgroup_20040219+ tagged
version.
\begin{verbatim}
cvs -d $LSCSOFTCVS checkout -rHEAD lal
\end{verbatim}
You have now obtained the latest development version of LAL.

\subsubsection{Build and Install}
To build and install the LAL software suite, 
\begin{verbatim}
cd $LALPREFIX/src/lal
./00boot
./configure --prefix=$LALPREFIX \
        --enable-frame --enable-metaio \
        --enable-static --disable-shared \
        --with-extra-cflags=-g
make
make check
make dvi
make install
\end{verbatim}
This completes the installation and testing of LAL.  

Note that if you have version 6.1 of the Intel Math Kernel Library (MKL)
installed on your system, you may enable it by adding the arguments:
\begin{verbatim}
--enable-intelfft=yes --with-extra-cppflags=-I/opt/intel/mkl61/include \
--with-extra-ldflags=-L/opt/intel/mkl61/lib/32
\end{verbatim}
to the LAL configure (after \texttt{--with-extra-cflags=-g}), assuming that
you have installed MKL in the default location, which is
\texttt{/opt/intel/mkl61}.

If you wish to run your LALApps executable under the condor standard universe
against a build of LAL that uses the Intel Math Kernel Library you should
specifiy
\begin{verbatim}
--enable-intelfft=condor
\end{verbatim}
instead of \texttt{--enable-intelfft=yes} on the command line.


\color{black}
%%%%%%%%%%%%%%%%%%%%%%%%%%%%%%%%%%%%%%%%%%%%%%%%%%%%%%%%%%%%%%%%%%%%%%%%%%%
\subsection{LALApps}
%%%%%%%%%%%%%%%%%%%%%%%%%%%%%%%%%%%%%%%%%%%%%%%%%%%%%%%%%%%%%%%%%%%%%%%%%%%
\color{black}

In the following commands, remember \verb+LALPREFIX+ is the absolute
directory path where you want to install the LALApps binaries.  We
assume that you have followed the build instructions for \lal\ and
that the directories \verb+LALPREFIX+ and \verb+$LALPREFIX/src+ exist.

Obtain the LALApps package from the CVS repository as follows:  
\begin{verbatim}
cd $LALPREFIX/src
cvs -d $LSCSOFTCVS login
\end{verbatim}
At this point,  you will be asked for a password.  The password for the
\verb+anonymous+ user is \verb+lal+. If you have your own username, use the
password that you have been given.
The version of LALApps that is checked-out is identified by the argument
to the \texttt{-r} option in the commands below.   The HEAD tag checks out the
current development version.  If you want to check out a released
version, replace the \verb+HEAD+ tag by \verb+release-X-Y+ where
\verb+X+ and \verb+Y+ are integers which identify the release version
number as \verb+X.Y+. You may also replace \verb+HEAD+ by a specific tag name,
for example if you want to download the \verb+iulgroup_20040226+ tagged
version.
\begin{verbatim}
cvs -d $LSCSOFTCVS checkout -rHEAD lalapps
\end{verbatim}
You have now obtained the latest development version of LALApps if you used
\verb+HEAD+ or the specified tagged version.

\subsubsection{Build and Install}
To build and install the LALApps suite, 
\begin{verbatim}
cd $LALPREFIX/src/lalapps
./00boot
./configure --prefix=$LALPREFIX \
    --with-extra-cppflags="-I$LALPREFIX/include" \
    --with-extra-ldflags="-L$LALPREFIX/lib" \
    --enable-frame --enable-metaio \
    --enable-static --disable-shared \
    --with-extra-cflags="-g -static"
make
make check
make dvi
make install
\end{verbatim}
This completes the installation and testing of LALApps.  

Note that if you are building lalapps on a machine with Condor installed and
wish to run in the standard universe, append the option
\texttt{--enable-condor} to the configure for lalapps before you type make.
The lalapps executables will then be linked with \texttt{condor\_compile} and
checkpointing will be available.

\color{black}
%%%%%%%%%%%%%%%%%%%%%%%%%%%%%%%%%%%%%%%%%%%%%%%%%%%%%%%%%%%%%%%%%%%%%%%%%%%
\subsection{Updating LAL/LALApps}
%%%%%%%%%%%%%%%%%%%%%%%%%%%%%%%%%%%%%%%%%%%%%%%%%%%%%%%%%%%%%%%%%%%%%%%%%%%
\color{black}

Since \lal\ and \lalapps\ are constantly being improved,  you will
eventually want to update the versions you have installed.   These
instructions give you come guidance on the process.   As you become a
more expereienced developer,  you will develop your own tricks for
making this more efficient.    To do a complete update to the head of
the CVS:
\begin{enumerate}
\item Remove the currently installed versions:
\begin{verbatim}
cd $LALPREFIX/src/lal
make uninstall
cd $LALPREFIX/src/lalapps
make uninstall
\end{verbatim}

\item Update the source code on your computer:
\begin{verbatim}
cd $LALPREFIX/src/lal
make cvs-clean
cvs update -Ad
cd $LALPREFIX/src/lalapps
make cvs-clean
cvs update -Ad
\end{verbatim}

\item Repeat the \textbf{Build and Install} instructions for \lal\ and then
\lalapps.
\end{enumerate}
To avoid conflicts it is important to \texttt{make uninstall} \emph{before}
updating the the source code from CVS.
