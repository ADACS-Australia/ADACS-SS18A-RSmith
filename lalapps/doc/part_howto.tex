%
% $Id$
%
\color{black}
\chapter{How to get, build and use software for the LSC Data Grid} 
%%%%%%%%%%%%%%%%%%%%%%%%%%%%%%%%%%%%%%%%%%%%%%%%%%%%%%%%%%%%%%%%%%%%%%%%%%%
\section{Getting and installing the software}
%%%%%%%%%%%%%%%%%%%%%%%%%%%%%%%%%%%%%%%%%%%%%%%%%%%%%%%%%%%%%%%%%%%%%%%%%%%

We describe the installation of \lal\ and \lalapps\ for use in
gravitational-wave data analysis.   These packages are under active
development at the present time and the latest functionality is
available only from CVS.   These instructions assume a standard
RedHat installation including development tools.   

\subsection{Installation of required development tools}

A useful installation of \lal\ and \lalapps\ requires \textsc{fftw},
the \textsc{frame} and \textsc{metaio} libraries.   Pre-compiled
versions of these libraries are available as RPM's from
\begin{verbatim}
http://www.lsc-group.phys.uwm.edu/lal/rpms/
\end{verbatim}
The binary versions
\begin{verbatim}
fftw-2.1.3-2lal.i386.rpm   28-Aug-2003 17:35   810k  
frame-6.08-1lal.i386.rpm   28-Aug-2003 17:35   464k  
metaio-4.13-1lal.i386.rpm  28-Aug-2003 17:35    24k  
stow-1.3.3-1lal.i386.rpm   29-Aug-2003 12:27    29k 
\end{verbatim}
are compiled on RedHat7.3 and RedHat 9.0 architecture,  but should work on other
versions of RedHat.   Download the RPM's,  log in as root on your
computer,  then type
\begin{verbatim}
rpm -Uvh fftw-2.1.3-2lal.i386.rpm
rpm -Uvh frame-6.08-1lal.i386.rpm
rpm -Uvh metaio-4.13-1lal.i386.rpm
rpm -Uvh stow-1.3.3-1lal.i386.rpm
\end{verbatim}
Watch for error messages.   If everything installs correctly,  skip to
Sec.~\ref{ss:lal},  otherwise continue through the following steps.  

If the binary RPM's do not work for some reason or your particular O/S
is not supported,   the source RPM's are also available at the LAL web
site.   Download
\begin{verbatim}
fftw-2.1.3-2lal.src.rpm   28-Aug-2003 17:35   810k  
frame-6.08-1lal.src.rpm   28-Aug-2003 17:35   464k  
metaio-4.13-1lal.src.rpm  28-Aug-2003 17:35    24k  
stow-1.3.3-1lal.src.rpm   29-Aug-2003 12:27   139k 
\end{verbatim}
to your computer.  As root,   type
\begin{verbatim}
rpm -Uvh fftw-2.1.3-2lal.src.rpm
rpm -Uvh frame-6.08-1lal.src.rpm
rpm -Uvh metaio-4.13-1lal.src.rpm
rpm -Uvh stow-1.3.3-1lal.src.rpm
\end{verbatim}
Watch for errors.  If this fails,  there is a major problem:  e-mail
\verb+lal-discuss@gravity.phys.uwm.edu+.    If it succeeds,  as root
\begin{verbatim}
cd /usr/src/redhat/SPECS
rpm -ba fftw.spec
rpm -ba frame.spec
rpm -ba metaio.spec
cd /usr/src/redhat/RPMS/i386
rpm -Uvh fftw-2.1.3-2lal.i386.rpm
rpm -Uvh frame-6.08-1lal.i386.rpm
rpm -Uvh metaio-4.13-1lal.i386.rpm
\end{verbatim}
Watch for errors at every step.  If you encounter a problem:   e-mail
\verb+lal-discuss@gravity.phys.uwm.edu+.   If it succeeds,  then you
can go ahead to the next stage.   Later versions of the RPM software
require \verb+rpm+ to be replaced by \verb+rpmbuild+. 

\color{black}
%%%%%%%%%%%%%%%%%%%%%%%%%%%%%%%%%%%%%%%%%%%%%%%%%%%%%%%%%%%%%%%%%%%%%%%%%%%
\subsection{Configuring your environment}
%%%%%%%%%%%%%%%%%%%%%%%%%%%%%%%%%%%%%%%%%%%%%%%%%%%%%%%%%%%%%%%%%%%%%%%%%%%
\color{black}

While developing,  we recommend that you install the \lal\ and \lalapps\
somewhere under your home directory.  If you follow the
instructions below,  the \lal\ and \lalapps\ libraries will be in
\verb+$LALPREFIX/lib+;  the documentation in \verb+$LALPREFIX/doc+; the header
files in \verb+$LALPREFIX/include+; the binaries in \verb+$LALPREFIX/bin+.
The environment \verb+LALPREFIX+ should be set to the absolute path where you
want to install these things.  For example,  user \texttt{patrick} might put
the source code in \verb+/home/patrick/src+ and set \verb+LALPREFIX+ to
\verb+/home/patrick+.

To make the appropriate binaries accesible to you,  check that your path is
set correctly. It is also useful to add environment variables for the CVS
servers. If you have a lal CVS login, change
\verb+anonymous@gravity.phys.uwm.edu+ to
\verb+your_user_name@gravity.phys.uwm.edu+ in the environment variables below.
If you are using \texttt{bash}, add
the following lines to your \texttt{.bash\_profile} file
\begin{verbatim}
export LALPREFIX=${HOME}      # <---- Change this as appropriate
export PATH=${LALPREFIX}/bin:${PATH}
export LALCVS=":pserver:anonymous@gravity.phys.uwm.edu:/usr/local/cvs/lal"
\end{verbatim}
If you are using \texttt{csh} or a derivative,  add the following lines to
your \texttt{.cshrc} file
\begin{verbatim}
setenv LALPREFIX $HOME
setenv PATH $LALPREFIX/bin:$PATH
setenv LALCVS ":pserver:anonymous@gravity.phys.uwm.edu:/usr/local/cvs/lal"
\end{verbatim}
These enviromnent variables must be set before running anything,
so it is a good idea to log out and log back in again before
continuing.   \textbf{Note:} With libtool version 1.4.2, or greater, there is
no need to have the \verb+LD_LIBRARY_PATH+ environment variable set. Library
paths are hard coded into the libraries using the \texttt{-rpath}
compiler option. This is the correct way to do this,  using
\verb+LD_LIBRARY_PATH+ is incorrect.  

\color{black}
%%%%%%%%%%%%%%%%%%%%%%%%%%%%%%%%%%%%%%%%%%%%%%%%%%%%%%%%%%%%%%%%%%%%%%%%%%%
\subsection{LAL}\label{ss:lal}
%%%%%%%%%%%%%%%%%%%%%%%%%%%%%%%%%%%%%%%%%%%%%%%%%%%%%%%%%%%%%%%%%%%%%%%%%%%
\color{black}

In the following commands, remember \verb+LALPREFIX+ is the
absolute directory path where you want to install the LAL library.  If
\verb+LALPREFIX+ does not exist,  you must create it:
\begin{verbatim}
mkdir $LALPREFIX
\end{verbatim}
Then create the directory into which you wish to put the source files for LAL
and LALApps:
\begin{verbatim}
mkdir $LALPREFIX/src
\end{verbatim}

The LAL software is maintained in a CVS repository -- CVS stands for
Concurrent Version-control System which is tool to allow multiple developers
to manipulate the same software,  merging differences and identifying
conflicts between changes if they arise.  Obtain the LAL package from the CVS
repository as follows:  
\begin{verbatim}
cd $LALPREFIX/src
cvs -d $LALCVS login
\end{verbatim}
At this point,  you will be asked for a password.  The password for the
\verb+anonymous+ user is \verb+lal+. If you have your own username, use the
password that you have been given.
The version of LAL that is checked-out is identified by the argument
to the \texttt{-r} option in the commands below.   The HEAD tag checks out the
current development version.  If you want to check out a released
version, replace the \verb+HEAD+ tag by \verb+release-X-Y+ where
\verb+X+ and \verb+Y+ are integers which identify the release version
number as \verb+X.Y+.
\begin{verbatim}
cvs -d $LALCVS checkout -rHEAD lal
\end{verbatim}
You have now obtained the latest development version of LAL.

\subsubsection{Build and Install}
To build and install the LAL software suite, 
\begin{verbatim}
cd $LALPREFIX/src/lal
./00boot
./configure --prefix=$LALPREFIX \
        --enable-frame
make
make check
make dvi
make install prefix=$LALPREFIX/stow_pkgs/lal-howto
cd $LALPREFIX/stow_pkgs
stow lal-howto
\end{verbatim}
This completes the installation and testing of LAL.  

\color{black}
%%%%%%%%%%%%%%%%%%%%%%%%%%%%%%%%%%%%%%%%%%%%%%%%%%%%%%%%%%%%%%%%%%%%%%%%%%%
\subsection{LALApps}
%%%%%%%%%%%%%%%%%%%%%%%%%%%%%%%%%%%%%%%%%%%%%%%%%%%%%%%%%%%%%%%%%%%%%%%%%%%
\color{black}

In the following commands, remember \verb+LALPREFIX+ is the absolute
directory path where you want to install the LALApps binaries.  We
assume that you have followed the build instructions for \lal\ and
that the directories \verb+LALPREFIX+ and \verb+$LALPREFIX/src+ exist.

Obtain the LALApps package from the CVS repository as follows:  
\begin{verbatim}
cd $LALPREFIX/src
cvs -d $LALCVS login
\end{verbatim}
At this point,  you will be asked for a password.  The password for the
\verb+anonymous+ user is \verb+lal+. If you have your own username, use the
password that you have been given.
The version of LALApps that is checked-out is identified by the argument
to the \texttt{-r} option in the commands below.   The HEAD tag checks out the
current development version.  If you want to check out a released
version, replace the \verb+HEAD+ tag by \verb+release-X-Y+ where
\verb+X+ and \verb+Y+ are integers which identify the release version
number as \verb+X.Y+.
\begin{verbatim}
cvs -d $LALCVS checkout -rHEAD lalapps
\end{verbatim}
You have now obtained the latest development version of LALApps.

\subsubsection{Build and Install}
To build and install the LALApps suite, 
\begin{verbatim}
cd $LALPREFIX/src/lalapps
./00boot
./configure --prefix=$LALPREFIX \
    --with-extra-cppflags="-I$LALPREFIX/include" \
    --with-extra-ldflags="-L$LALPREFIX/lib" \
    --enable-frame --enable-metaio
make
make check
make dvi
make install prefix=$LALPREFIX/stow_pkgs/lalapps-howto
cd $LALPREFIX/stow_pkgs
stow lalapps-howto
\end{verbatim}
This completes the installation and testing of LALApps.  

\color{black}
%%%%%%%%%%%%%%%%%%%%%%%%%%%%%%%%%%%%%%%%%%%%%%%%%%%%%%%%%%%%%%%%%%%%%%%%%%%
\subsection{Updating LAL/LALApps}
%%%%%%%%%%%%%%%%%%%%%%%%%%%%%%%%%%%%%%%%%%%%%%%%%%%%%%%%%%%%%%%%%%%%%%%%%%%
\color{black}

Since \lal\ and \lalapps\ are constantly being improved,  you will
eventually want to update the versions you have installed.   These
instructions give you come guidance on the process.   As you become a
more expereienced developer,  you will develop your own tricks for
making this more efficient.    To do a complete update:
\begin{enumerate}
\item Update the source code on your computer
\begin{verbatim}
cd $LALPREFIX/src/lal
make cvs-clean
cvs update -Ad
cd $LALPREFIX/src/lalapps
make cvs-clean
cvs update -Ad
\end{verbatim}

\item Remove the old versions
\begin{verbatim}
cd $LALPREFIX/stow_pkgs
stow --delete lal-howto
stow --delete lalapps-howto
\end{verbatim}

\item Repeat the \textbf{Build and Install} instructions for \lal\ and then
\lalapps,  but \textbf{change the prefix argument} to \verb+make install+.   
For example,   if you compiled on 28 August 2003,  you might want to use
\begin{verbatim}
make install prefix=$LALPREFIX/stow_pkgs/lalapps-030828
cd $LALPREFIX/stow_pkgs
stow lalapps-030828
\end{verbatim}

\end{enumerate}
