\section{Program \texttt{lalapps\_inspiral\_hipe}}
\label{program:inspiral-hipe}
\idx[Program]{inspiral\_hipe.in}

\begin{entry}
\item[Name]
\verb$lalapps_inspiral_hipe$ --- python script to generate Condor DAGs to
run the inspiral hierarchical pipeline.

\item[Synopsis]
\begin{verbatim}
  -h, --help              display this message
  -v, --version           print version information and exit
  -u, --user-tag TAG      tag the job with TAG (overrides value in ini file)
  
  -a, --h1-data           analyze h1 data
  -b, --h2-data           analyze h2 data
  -l, --l1-data           analyze l1 data
 
  -S, --one-ifo           analyze single ifo data
  -D, --two-ifo           analyze two interferometer data
  -T, --three-ifo         analyze three interferometer data
   
  -A, --analyze-all       analyze all ifos and all data (over-rides above)
 
  -d, --datafind          run LSCdataFind to create frame cache files
  -t, --template-bank     run lalapps_tmpltbank to generate template banks
  -i, --inspiral          run lalapps_inspiral to generate triggers
  -c, --coincidence       run lalapps_inca to test for coincidence
  -s, --sire              do sires to sweep up triggers
   
  -P, --priority PRIO     run jobs with condor priority PRIO
 
  -f, --config-file FILE  use configuration file FILE
  -l, --log-path PATH     directory to write condor log file
  -o, --output-segs       output the segment lists of analyzed data
\end{verbatim}

\item[Description] \verb$lalapps_inspiral_pipe$ generates a Condor DAG to run
the hierarchical inspiral analysis pipeline.  The current pipeline is only a
prototype, all the features are not yet implemented.  It currently works for
the three LIGO interferometers.  

The code reads in segment lists for the three instruments.  If one of the
segment files is not specified or is empty, it is assumed that there is no data
from that instrument.  From the segment files, the pipeline calculates three
lists of single ifo segments, for H1, H2 and L1; three lists of double ifo
segments, for H1-H2, H1-L1 and H2-L1; and one list of three ifo segments for
H1-H2-L1.  The options \verb$--h1-data$, \verb$--h2-data$ and \verb$--l1-data$
allow you to choose which of the interferometers' data to analyze.  Similarly,
the \verb$--one-ifo$, \verb$--two-ifo$ and \verb$--three-ifo$ flags determine
whether to analyze times during which one, two or three instruments
respectively were operational.  If the \verb$--analyze-all$ flag is set, the
pipeline will analyze all data from all instruments.  If the
\verb$--output-segments$ option is chosen, the pipeline will output segment
lists for the non-empty data types.  The file names are
"h1\_l1\_segs\_analyzed.txt" etc, or if the analyis is restricted to
playground, they are "h1\_l1\_play\_segs\_analyzed.txt".

There are several steps of the pipeline which have been implemented to date.
The full hierarchical pipeline will involve more steps than are currently
available.  At present, the pipeline can perform the following parts of the
search: \verb$--datafind$, \verb$--template-bank$, \verb$--inspiral$,
\verb$--coincidence$ and \verb$--sire$.  Any or all of these options may be
specified.  However, each step of the pipeline relies on results files
produced by the previous step (and in the case of the \verb$inspiral$ step,
both \verb$datafind$ and \verb$template-bank$ must have been run previously).


The configuration file specifies the parameters needed to run the analysis jobs
contained in the pipeline.  It is specified with the \verb$--config-file$
option.  A typical .ini file is the following: 

\begin{verbatim}
; inspiral pipeline configuration script.
; 
;
; this is the configuration file for the inspiral DAG generation program 
; lalapps_inspiral_pipe that creates a condor DAG to run the inspiral
; analysis pipeline. It can be use to perform a simple single interferometer
; or a double coincident analysis.

[condor]
; setup of condor universe and location of executables
universe    = standard
datafind    = LSCdataFind
tmpltbank   = ./lalapps_tmpltbank
inspiral    = ./lalapps_inspiral
inca        = ./lalapps_inca
thinca      = ./lalapps_thinca
trigtotmplt = ./lalapps_trigbank
sire        = ./lalapps_sire
cohbank     = ./lalapps_coherentbank
chia        = ./lalapps_coherent_inspiral



[pipeline]
; tagging information for the configure script
version = $Id$
cvs-tag = $Name$
; user-tag here can be overidden on the command line of lalapps_inspiral_hipe
user-tag = 
; data choice (playground_only|exclude_playground|all_data)
playground-data-mask = playground_only

[input]
; the segments file should be the output from segwizard with DQ flags applied
; if no segment file if specified, assumed no data from that IFO.
h1-segments = S3H1v05_selectedsegs.txt
h2-segments = S3H2v05_selectedsegs.txt
l1-segments = S3L1v05_selectedsegs.txt
g1-segments = 
ligo-channel= LSC-AS_Q
ligo-type = RDS_R_L3
geo-channel = DER_DATA_H 
geo-type = G1_RDS_C01_LX
geo-bank = H1_bank_4_G1.xml

; injection file (if blank then no injections)
injection-file =
; num slides (if blank or zero, then no time slides are performed)
num-slides = 20

[calibration]
; location of the calibration cache and the cache files
path = /ldas_outgoing/calibration/cache_files
L1 = L1-CAL-V03-751719553-757687373.cache
H1 = H1-CAL-V03-751651153-757672093.cache
H2 = H2-CAL-V03-751654453-757699693.cache

[datafind]
; type of data to use
match = localhost

[data]
; data conditioning parameters common to tmpltbank and inspiral
pad-data = 8
segment-length = 1048576
number-of-segments = 15
sample-rate = 4096
resample-filter = ldas
spectrum-type = median

[ligo-data]
enable-high-pass = 70.0
high-pass-order = 8
high-pass-attenuation = 0.1
low-frequency-cutoff = 70.0

[geo-data]
; data conditioning specific to GEO detector
calibrated-data = real_8
disable-high-pass = 
geo-high-pass-freq = 110.0
geo-high-pass-order = 8
geo-high-pass-atten = 0.1
low-frequency-cutoff = 500.0

[tmpltbank]
; template bank generation parameters -- added to all tmpltbank jobs
minimum-mass =  1.0 
maximum-mass = 20.0
minimal-match = 0.97
high-frequency-cutoff = 2048.0
order = twoPN
approximant = TaylorF2
space = Tau0Tau3
debug-level = 33

[inspiral]
; inspiral analysis parameters -- added to all inspiral jobs
approximant = TaylorF2
segment-overlap = 524288
inverse-spec-length = 16
dynamic-range-exponent = 69.0
enable-output = 
cluster-method = window
cluster-window = 16
maximization-interval = 10000000
debug-level = 33
minimal-match = 0.55

[no-veto-inspiral]
; inspiral parameters specific to the first set of inspirals (pre coinc)
chisq-bins = 0
disable-rsq-veto =

[veto-inspiral]
; inspiral parameters for the second set of inspirals, after coincidence
chisq-bins = 16
enable-rsq-veto =
rsq-veto-window = 2.0
rsq-veto-threshold = 10.0

[h1-inspiral]
; h1 specific inspiral paramters
snr-threshold = 6.5
chisq-threshold = 5.0

[h2-inspiral]
; h2 specific inspiral parameters
snr-threshold = 6.5
chisq-threshold = 5.0

[l1-inspiral]
; l1 specific inspiral parameters
snr-threshold = 6.5
chisq-threshold = 5.0

[g1-inspiral]
; g1 specific inspiral parameters
snr-threshold = 8.0
chisq-threshold = 10.0
minimal-match = 0.55
inverse-spec-length = 24
dynamic-range-exponent = 65.0

[inca]
; common coincidence parameters -- added to all inca jobs
debug-level = 33

[thinca]
; common coincidence parameters -- added to all thinca jobs
debug-level = 33
parameter-test = mchirp_and_eta
h1-time-accuracy = 1
h2-time-accuracy = 1
l1-time-accuracy = 1
g1-time-accuracy = 2
h1-mchirp-accuracy = 0.01
h2-mchirp-accuracy = 0.01
l1-mchirp-accuracy = 0.01
g1-mchirp-accuracy = 0.05
h1-eta-accuracy = 0.1
h2-eta-accuracy = 0.1
l1-eta-accuracy = 0.1
g1-eta-accuracy = 0.3

[thinca-slide]
; time slide parameters
g1-slide = 15
h1-slide = 0
h2-slide = 10
l1-slide = 5
t1-slide = 20
v1-slide = 25

[trigtotmplt]
parameter-test = m1_and_m2

[sire-cluster]
; clustering parameters for sire
cluster-time = 4000
cluster-algorithm = snr

[sire-inj]
; injection coincidence
injection-coincidence = 3 

[cohbank]
;params for the coherent bank code
debug-level = 33
 
[coh-trig]
;params for the trigbank code in the coherent stage
debug-level = 33
parameter-test = no_test

[chia]
;params for the coherent code
debug-level = 33
maximize-over-chirp = 
cohsnr-threshold = 10.0
low-frequency-cutoff = 70.0
write-cohsnr = 
write-events =
sample-rate = 4096
glob-frame-data = 
dynamic-range-exponent = 69.0
segment-length = 1048576
\end{verbatim}
  

The .ini file contains several sections.  The \verb$[condor]$ section contains
the names of the executables which will run the various stages of the
pipeline.  The \verb$[pipeline]$ section gives the CVS details of the
pipeline, the usertag (which can be overwritten on the command line with the
\verb$--user-tag$ option) and the \verb$playground-data-mask$ which must be
set to one of \verb$playground_only$, \verb$exclude_playground$ or
\verb$all_data$.  The \verb$input$ section contains the names of the segment
files for the three interferometers, the name of the injection file and the
channel name.  If any of the segment files are left blank, it is assumed that
there are no segments for  that instrument.  Similarly,  a blank
injection-file signifies that no injections are to be performed.

The remaining sections set options for the various jobs to be run in the
pipeline.  The options in the \verb$[datafind]$, \verb$[tmpltbank]$,
\verb$[inspiral]$ and \verb$[inca]$ sections are added to every instance of
the relevant executable.  Note that these options are set the same for all
interferometers.  The options in the \verb$[data]$ section are added to all
\verb$[inspiral]$ and \verb$[tmpltbank]$ jobs, and the \verb$[calibration]$
information is used to determine the calibration data for these jobs.  In
\verb$[inspiral-thresholds]$ section, the snr and chi squared thresholds for
the various intererometers are set.  The \verb$[coincidence-parameters]$ are
used to determine the ifo dependent arguments to be added to the inca
coincidence jobs.  These give the timing and effective distance accuracies.
The pipeline's final step is to run several sires to concatenate the triggers.
It is currently set up to run two sires for each ifo and analysis type -- one
without clustering and one using the clustering parameters specified in the
\verb$[sire-parameters]$ section.  If it is an injection analysis, then sire
will look for triggers coincident with injections.  The coincidence window is
set by the \verb$injection-coincidence$ option in the \verb$[sire-parameters]$
section.

The science segments are read in from the segment files for each instrument.
These science segments are split up into analysis chunks.  The analysis chunk
size is determined from the number of data segments and their length and
overlap specified in config file. Currently, we are using 256 second analysis
segments.  We use 15 segments in a chunk, and overlap them by 128 seconds to
give a chunk length of 2048 seconds.  The chunks are constructed for each of
the interferometers independently.  Any science segment shorter than the
length of a chunk is not analyzed.  Additionally, we cannot produce triggers
for the first and last overlap/2 seconds of a science segment, due to the
finite length of the inspiral templates.  We then construct segment lists of
analyzable data for each of the seven types of data we may have (three single
ifo, three double ifo and one triple ifo).  If the playground only option is
specified, the segments are restricted to playground times.  We decide which
chunks should be analyzed by testing for overlap between the chunk and the
data we need to analyze.  Note that if the pipeline is restricted to
playground data, then only the playground times are analyzed for triggers in
the inspiral code.  This is done by setting the \verb$trig-start-time$ and
\verb$trig-end-time$ for the inspiral jobs appropriately.

Once the DAG file has been created it should be submitted to the Condor pool
with the \verb$condor_submit_dag$ command.

\item[Options]\leavevmode
\begin{entry}
\item[\texttt{--help}] Display a brief usage summary.
\end{entry}
  
\begin{entry}
\item[\texttt{--version}] Display the version information and exit.
\end{entry}
  
\begin{entry}
\item[\texttt{--user-tag} \textsc{usertag}] Set the user-tag to \textsc{usertag}.  This overrides the user-tag which may have been set in the ini file.  The usertag will be added to the names of the output files of the pipeline.
\end{entry}
  
\begin{entry}
\item[\texttt{--h1-data}] Analyze the H1 data, the times of which are
determined from the \verb$h1-segments$ file specified in the ini file.  If not
set, then no data when H1 was operational will be analyzed.
\end{entry}

\begin{entry}
\item[\texttt{--h2-data}] Analyze the H2 data, the times of which are
determined from the \verb$h2-segments$ file specified in the ini file.  If not
set, then no data when H2 was operational will be analyzed.
\end{entry}

\begin{entry}
\item[\texttt{--l1-data}] Analyze the L1 data, the times of which are
determined from the \verb$l1-segments$ file specified in the ini file.  If not
set, then no data when H1 was operational will be analyzed.
\end{entry}

\begin{entry}
\item[\texttt{--one-ifo}] Analyze any times when one and only one instrument
was operational.  Note that this option works together with the IFO options
given above.  For example if \verb$--one-ifo$ and \verb$--h2-data$ were
specified, then only the single IFO H2 times would be analyzed.
\end{entry}

\begin{entry}
\item[\texttt{--two-ifo}] Analyze any times when two instruments were
operational.  Note that this option works together with the IFO options given
above.  For example if \verb$--two-ifo$, \verb$h1-data$ and \verb$--h2-data$
were specified, then the times when only H1 and H2 were operational would be
analyzed.  However, if only \verb$--two-ifo$ and \verb$h1-data$ were
specified, no data would be analyzed.
\end{entry}

\begin{entry}
\item[\texttt{--three-ifo}] Analyze any times when three instruments were
operational.  Note that this option works together with the IFO options given
above.  For example if \verb$--three-ifo$, \verb$h1-data$, \verb$--h2-data$
and \verb$--l1-data$ were specified, then the times when all of H1, H2 and L1
were operational would be analyzed.  Note that since the triple coincidence
code isn't yet in place, the three ifo data is only run through template bank
and inspiral.
\end{entry}

\begin{entry}
\item[\texttt{--analyze-all}] Analyze all ifos and all data.  This is
equivalent to setting all six of the options above.  Then, all the data is
analyzed.
\end{entry}

\begin{entry}
\item[\texttt{--datafind}] Run the datafind step of the pipeline.
\end{entry}

\begin{entry}
\item[\texttt{--template-bank}] Run the template-bank step of the pipeline.
Note that the template-bank jobs require the cache files created by datafind,
so \verb$--datafind$ must either be run in the pipeline or have been run
previously.
\end{entry}

\begin{entry}
\item[\texttt{--inspiral}] Run the inspiral step of the pipeline.  Note that
the inspiral jobs require the cache files created by datafind and template
banks, so both \verb$--datafind$ and \verb$template-bank$ must either be run
in the pipeline or have been run previously.
\end{entry}

\begin{entry}
\item[\texttt{--coincidence}] Run the coincidence step of the pipeline.  Note
that the inca jobs require the inspiral triggers created by inspiral, so that
\verb$--inspiral$ must either be run in the pipeline or have been run
previously.  For each single IFO segment, the coincidence step simply creates
a file containing all the triggers in the time interval.  For two IFO
segments, the coincidence step performs coincidence and outputs the double
coincidence triggers in two files (one for each instrument).  For the three
IFO segments, the coincidence step currently does nothing.  
\end{entry} 

\begin{entry}
\item[\texttt{--sire}] Run the sire step of the pipeline.  This sweeps up the
triggers from each of the seven types of data.  We get a file containing all
the triggers from the each list single IFO data.  For a given combination of
two active instruments (e.g. H1 and L1) we obtain two files, which contain the
coincident triggers for each of the IFOS (H1 and L1 respectively).  Sire
currently does not do anything for the triple coincident times.  Note that the
sire jobs require the output of the inca jobs as input.  Therefore, the
\verb$--coincidence$ step of the pipeline must either be run or have been run
previously.
\end{entry} 

\begin{entry}
\item[\texttt{--priority} \textsc{PRIO}] Set the condor priority \textsc{PRIO}
of the condor jobs to be run in the pipeline.
\end{entry} 

\begin{entry}
\item[\texttt{--config-file} \texttt{config\_file}] Set the name of the
configuration file to be \textsc{config\_file}.  This is the which is used to
determine the parameters for the pipeline.  This is a required argument.
\end{entry} 
 
\begin{entry}
\item[\texttt{--log-path}] The directory in which to write the condor log
file.  This should generally be a local directory of the condor submit
machine.  This is a required argument.
\end{entry} 

\begin{entry}
\item[\texttt{--output-segs}] Output the segment lists of analyzed data.  Up
to seven files will be output, one for each of the types of interferometer
data (H1, H2, L1, H1\_H2, H1\_L1, H2\_L1, H1\_H2\_L1).  Any segment lists
which are non-empty will we written.
\end{entry} 


\item[Author] 
Steve Fairhurst
\end{entry}
