\documentclass[oneside]{book}
\usepackage{color}
\usepackage{fancyhdr}
\usepackage{geometry}
\usepackage{graphicx}
\usepackage{ifthen}
\usepackage{makeidx}
\usepackage{moreverb}
\usepackage{times}
\usepackage{verbatim}

\geometry{letterpaper,margin=1in}

\pagestyle{fancy}
\fancyhf{}
\lhead{\bf\nouppercase\rightmark}
\rhead{\bf Page\space\thepage}

\newenvironment{entry}%
  {\begin{list}{}{\renewcommand{\makelabel}[1]%
    {\parbox[b]{\labelwidth}{\makebox[0pt][l]{\textbf{##1}}\\}}%
    \setlength{\labelwidth}{1em}%
    \setlength{\labelsep}{1em}%
    \setlength{\leftmargin}{2em}%
    \setlength{\topsep}{\medskipamount}%
    \setlength{\itemsep}{\medskipamount}%
    \setlength{\parsep}{\medskipamount}%
    \setlength{\listparindent}{0pt}}}
  {\end{list}}
\newenvironment{indented}%
  {\begin{list}{}{\setlength{\leftmargin}{2em}}%
    \setlength{\itemsep}{\medskipamount}%
    \setlength{\parsep}{\medskipamount}%
    \setlength{\listparindent}{0pt}\item}
  {\end{list}}

\let\keepttfamily\ttfamily
\def\ttfamily{\color{blue}\keepttfamily}
\def\listinglabel#1{\llap
  {\scriptsize\color{black}\the#1}\hskip\listingoffset\relax}

\newcommand\idx[2][xxx]{%
  \ifthenelse{\equal{#1}{xxx}}{\index{#2@\texttt{#2}}}%
  {\index{#2@\texttt{#2}}\index{#1!#2@\texttt{#2}}}}

\makeindex

\begin{document}

\frontmatter

\title{LALApps --- LSC Algorithm Library Applications}
\author{Various Contributors}
\maketitle

\tableofcontents

\mainmatter

\chapter{LALApps utilities}

Several utilities (macros, global variables, and functions) are provided to
assist in writing programs in LALApps, and for maintaining a standard
look-and-feel.  This chapter describes these utilities and concludes with
the listing of an example program.

\newpage
\section{Header \texttt{lalapps.h}}
\label{header:lalapps}

\begin{indented}
Provides utilities for writing programs for LALApps.

Several macros, global variables, and function prototypes are given that will
assist in writing LALApps programs, and will aid in maintaining a standard
look-and-feel.

To use these utilities, include the header \verb$lalapps.h$ and make sure the
program links to the object \verb$lalapps.o$.
\end{indented}

\newpage
\subsection{Function \texttt{set\_debug\_level}}
\label{function:set-debug-level}
\idx[Function]{set\_debug\_level}
\index{Debug Level}
\idx[Variable]{lalDebugLevel}
\idx[Debug Level]{lalDebugLevel}
\idx[Debug Level]{NDEBUG}
\idx[Debug Level]{ERROR}
\idx[Debug Level]{WARNING}
\idx[Debug Level]{INFO}
\idx[Debug Level]{TRACE}
\idx[Debug Level]{MEMINFO}
\idx[Debug Level]{MEMDBG}
\idx[Debug Level]{MSGLVL1}
\idx[Debug Level]{MSGLVL2}
\idx[Debug Level]{MSGLVL3}
\idx[Debug Level]{ALLDBG}
\idx[Environment]{LAL\_DEBUG\_LEVEL}

\begin{entry}

\item[Name]

\verb$set_debug_level$ --- sets the LAL debug level

\item[Synopsis]

\begin{verbatim}
#include <lalapps.h>
extern int lalDebugLevel;
int set_debug_level( const char *s );
\end{verbatim}

\item[Description]

The function \verb$set_debug_level$ sets the LAL debug level to a value
determined by the string \verb$s$, which can be an absolute debug level
(a string representing an integer) or a string of LAL debug level flags.
Allowed flags are:
\begin{indented}
\begin{entry}
\item[NDEBUG]
  No debugging information is printed and memory debugging code is disabled.
\item[ERROR]
  Error messages are printed.
\item[WARNING]
  Warning messages are printed.
\item[INFO]
  Information messages are printed.
\item[TRACE]
  Function call tracing messages are printed.
\item[MEMINFO]
  Memory allocation information messages are printed.
\item[MEMDBG]
  Debugging of memory allocation routines is enabled bug no messages are
  printed.
\end{entry}
\end{indented}
The following pre-defined composite levels are available:
\begin{indented}
\begin{entry}
\item[MSGLVL1]
  Equivalent to \verb$ERROR$.
\item[MSGLVL2]
  Equivalent to \verb$ERROR | WARNING$.
\item[MSGLVL3]
  Equivalent to \verb$ERROR | WARNING | INFO$.
\item[ALLDBG]
  All debugging messages are printed.
\end{entry}
\end{indented}

If the argument to \verb$set_debug_level$ is \verb$NULL$, then the string
stored in the environment variable \verb$LAL_DEBUG_LEVEL$ is used.  If this
environment is not defined, or if no flags or values are specified in the
string, the debug level is set to 0, which is equivalent to \verb$NDEBUG$.
(This is also the default value for \verb$lalDebugLevel$ unless it is set to
some other value.)

For example, the statement
\begin{indented}
\verb$set_debug_level( "ERROR | INFO" );$
\end{indented}
will set the debug level so that error and information messages are printed
(but not warning messages).  Another example is the statement
\begin{indented}
\verb$set_debug_level( "2" );$
\end{indented}
which would set the debug level to 2 (warning messages are printed).

\item[Return Value]

The return value is the (integer) debug level that is assigned to
\verb$lalDebugLevel$.

\item[Environment]
\leavevmode
\begin{entry}
\item[\texttt{LAL\_DEBUG\_LEVEL}]
  Default LAL debug level string to use.
\end{entry}

\end{entry}

\newpage
\subsection{Function \texttt{clear\_status}}
\label{function:clear-status}
\idx[Function]{clear\_status}
\idx[Variable]{blank\_status}

\begin{entry}

\item[Name]
\verb$clear_status$ --- clears the LAL status structure after a failed LAL
function call

\item[Synopsis]
\begin{verbatim}
#include <lalapps.h>
extern const LALStatus blank_status;
int clear_status( LALStatus *status );
\end{verbatim}

\item[Description]
Clears the LAL status structure and iteratively frees attatched (sic) any
linked status structures.  This is to be used after a failed LAL function
call to restore the status structure to a useable form.  The structure
\verb$blank_status$ contains a blank status structure that can be used to
initialize a status structure in the program.

\item[Example]

The following program calls a routine \verb$LALFailUnlessNegative$ twice,
once with a positive argument (which causes the routine to fail) and once
with a negative argument (which causes the routine to pass).  The function
\verb$clear_status$ is used to clean up the status structure after the
failure and the constant structure \verb$blank_status$ is used to initialize
the status structure.

\begin{indented}
\begin{verbatim}
#include <lalapps.h>
#include <lal/LALStdlib.h>

extern const LALStatus blank_status;

void LALFailUnlessNegative( LALStatus *status, INT4 n )
{
  INITSTATUS( status, "LALFail", "$Id$" );
  ATTATCHSTATUSPTR( status );
  ASSERT( n, status, 1, "Non-negative n" );
  if ( n > 0 )
  {
    TRY( LALFailUnlessNegative( status->statusPtr, n - 1 ), status );
  }
  DETATCHSTATUSPTR( status );
  RETURN( status );
}

int main( void )
{
  LALStatus status = blank_status;
  LALFailUnlessNegative( &status, 5 );
  clear_status( &status );
  LALFailUnlessNegative( &status, -2 );
  return status.statusCode;
}
\end{verbatim}
\end{indented}

\end{entry}


\newpage
\subsection{Macro \texttt{RCSID}}
\label{macro:RCSID}
\idx[Macro]{RCSID}
\idx[Variable]{rcsid}

\begin{entry}

\item[Name]

\verb$RCSID$ --- set the RCS Id variable

\item[Synopsis]

\begin{verbatim}
#include <lalapps.h>
#ifndef RCSID
#define RCSID( id ) static volatile const char *rcsid = (id)
#endif
\end{verbatim}

\item[Description]

\verb$RCSID$ sets the static (i.e., internal-linkage) variable \verb$rcsid$
to the RCS Id string, \$\relax Id\$, which is given as the argument \verb$id$.
The string \$\relax Id\$ is expanded by RCS to contain the identification of
the source file along with its revision number.
For example:
\begin{indented}
\verb+RCSID("$+\verb+Id+\verb+$");+
\end{indented}

\end{entry}

\newpage
\subsection{Macro \texttt{PRINT\_VERSION}}
\label{macro:PRINT-VERSION}
\idx[Macro]{PRINT\_VERSION}

\begin{entry}

\item[Name]
\verb$PRINT_VERSION$ --- prints the LALApps version of the program

\item[Synopsis]
\begin{verbatim}
#include <lalapps.h>
static volatile const char *rcsid="$Id$";
#ifndef PRINT_VERSION
#define PRINT_VERSION( program ) \
  fprintf( stderr, PACKAGE " %s version " VERSION "\n%s\n", program, rcsid )
#endif
\end{verbatim}

\item[Description]
\verb$PRINT_VERSION$ prints the version information for \verb$program$ in a
standard format, along with the RCS Id information.  For example, for the
program \verb$lalapps_hello$, the version information
\begin{indented}
\verb+lalapps hello version 0.1+\\
\verb+$+\verb+Id+\verb+$+
\end{indented}
is printed with the command \verb$lalapps_hello -V$.  The source code to
print this is
\begin{indented}
\verb$PRINT_VERSION( "hello" );$
\end{indented}

Note that \verb$PRINT_VERSION$ requires the string variable \verb$rcsid$ to be
set.

\end{entry}

\newpage
\subsection{Macro \texttt{LAL\_CALL}}
\label{macro:LAL-CALL}
\idx[Macro]{LAL\_CALL}
\idx[Variable]{vrblvl}
\idx[Function]{lal\_errhandler}
\idx[Type]{lal\_errhandler\_t}
\index{Error Handler}
\idx[Error Handler]{LAL\_ERR\_DFLT}
\idx[Error Handler]{LAL\_ERR\_ABRT}
\idx[Error Handler]{LAL\_ERR\_EXIT}
\idx[Error Handler]{LAL\_ERR\_RTRN}

\begin{entry}

\item[Name]
\verb$LAL_CALL$ --- call a LAL routine and handle any errors

\item[Synopsis]
\begin{verbatim}
#include <lalapps.h>

extern int vrblvl;
extern int ( *lal_errhandler )( LALStatus *stat, const char *func,
    const char *file, const int line, volatile const char *id );
extern lal_errhandler_t lal_errhandler;

static volatile const char *rcsid="$Id$";

#ifndef LAL_CALL
#define LAL_CALL( function, statusptr ) \
  ((function),lal_errhandler(statusptr,#function,__FILE__,__LINE__,rcsid))
#endif
\end{verbatim}

\item[Description]
\verb$LAL_CALL$ executes the LAL function \verb$function$ and executes the
error handler \verb$lal_errhandler$, which examines the status structure
\verb$statusptr$ to see if an error occurred.  Typically the error handler
will return with value 0 if there was no error; otherwise it will print a trace
of the execution stack and then perform a specific action.  The action
performed depends on the error handler, which can be set to one of the
following:
\begin{indented}
\begin{entry}
\item[\texttt{LAL\_ERR\_DFLT}]
  The default error handler (same as \verb$LAL_ERR_ABRT$).
\item[\texttt{LAL\_ERR\_ABRT}]
  Raises \verb$SIGABRT$ if there is an error.
\item[\texttt{LAL\_ERR\_EXIT}]
  Exits with the returned status code if there is an error.
\item[\texttt{LAL\_ERR\_RTRN}]
  Returns the status code.
\end{entry}
\end{indented}

Note that \verb$LAL_CALL$ requires the string variable \verb$rcsid$ to be set.

\item[Return Value]
If \verb$LAL_CALL$ returns (rather than terminating execution), the return
value is equal to the status code returned by the LAL function.

\item[Example]
The following example program illustrates the use of \verb$LAL_CALL$.
The routine \verb$LALInvert$ is called incorrectly twice.  The first time
the division by zero error is caught.  The second time, the unexpected null
pointer error is not caught and the default error handler aborts the program.
\begin{indented}
\begin{verbatim}
#include <stdlib.h>
#include <lalapps.h>
#include <lal/LALStdlib.h>

RCSID( "$Id$" );

extern int vrblvl;
extern const LALStatus blank_status;

void LALInvert( LALStatus *status, REAL4 *y, REAL4 x )
{
  INITSTATUS( status, "LALInvert", rcsid );
  ASSERT( y, status, 1, "Null pointer" );
  if ( input == 0 )
  {
    ABORT( status, 1, "Division by zero" );
  }
  *y = 1 / x;
  RETURN( status );
}

int main( void )
{
  LALStatus status = blank_status;
  REAL4 x;
  int code;

  vrblvl = 1;

  lal_errhandler = LAL_ERR_RTRN;
  code = LAL_CALL( LALInvert( &status, &x, 0 ), &status );
  if ( code == 2 )
  {
    puts( "division by zero" );
    clear_status( &status );
  }
  else if ( code )
  {
    exit( code );
  }

  lal_errhandler = LAL_ERR_DFLT;
  LAL_CALL( LALInvert( &status, NULL, 1 ), &status );

  return 0;
}
\end{verbatim}
\end{indented}

\end{entry}

\newpage
\section{Source \texttt{hello.c}}
\label{source:hello.c}
\idx[Source]{hello.c}

\begin{indented}
This is the source code for the program \verb$lalapps_hello$:

\listinginput[10]{1}{hello.c}
\end{indented}

\chapter{LALApps programs}

This chapter describes the programs that are part of the LALApps package.


%%
%% INCLUDE YOUR DOCUMENTATION BELOW
%%

\chapter{Package \texttt{hello}}

This is an example package that illustrates how to write and document LAL
code.  This pacakage contains a routine that prints ``hello, LSC!''

\newpage\input{LALHelloH}



\backmatter

\printindex

\end{document}
