\section{Program \texttt{lalapps\_splitbank}}
\label{program:lalapps-splitbank}
\idx[Program]{lalapps\_splitbank}

\begin{entry}
\item[Name]
\verb$lalapps_splitbank$ --- splits a template bank file into several smaller
files

\item[Synopsis]
\prog{lalapps\_splitbank} \newline \hspace*{0.5in}
\option{--bank-file} \parm{file} \newline \hspace*{0.5in}
\option{--comment} \parm{comment}\newline \hspace*{0.5in}
\option{--help}  \newline \hspace*{0.5in}
\option{--minimal-match} \parm{m} \newline \hspace*{0.5in}
\option{--number-of-banks} \parm{n}\newline \hspace*{0.5in}
\option{--user-tag} \parm{comment}\newline \hspace*{0.5in}
\option{--verbose} \newline \hspace*{0.5in}
\option{--version} \newline \hspace*{0.5in}

\item[Description] 
\verb$lalapps_splitbank$ splits a LIGO\_LW XML file containing inspiral
templates in a \texttt{sngl\_inspiral} table into several smaller bank
files. This allows a template bank to be split across several inspiral 
jobs and then recombined with \texttt{lalapps\_inca} or
\texttt{lalapps\_sire}.

The name of the output template bank files is derived from the name of 
the input bank file and the number of files that the bank should be split
into. For example, if the input bank file:\\
\begin{center}
\texttt{H1-TRIGBANK\_L1-729330491-2048.xml}\\
\end{center}
is split into 3 output files, then these will be named:\\
\begin{center}
\texttt{H1-TRIGBANK\_L1\_00-729330491-2048.xml}\\
\texttt{H1-TRIGBANK\_L1\_01-729330491-2048.xml}\\
\texttt{H1-TRIGBANK\_L1\_02-729330491-2048.xml}\\
\end{center}
The naming convention is to insert the bank file number after the usertag part
of the filename and before the GPS start time part of the file name.

In the case that the input file contains no templates, empty output bank files
are generated. This is done since DAGman does not implement decision rules
yet, so the nodes in the DAG must be identical regardless of the data flowing
through them.

\item[Options]\leavevmode
\begin{entry}

\item[\option{--bank-file} \parm{file}] 
Read the templates from the \texttt{sngl\_inspiral} table in the file \parm{file}.

\item[ \option{--comment}~\parm{comment}]
Add the string \parm{comment} to the \verb$process$ table in the output XML file.

\item[\option{--help}]
Display a usage message and exit.

\item[\option{--minimal-match} \parm{m}] 
Set the minimal match of the output template bank file to \parm{m}.
This option is not really needed for running \verb$lalapps_splitbank$, it just put that value of \parm{m} for the minimal match in all splited template banks.

\item[\option{--number-of-banks} \parm{n}] 
 Split the input template banks into \parm{n} seperate output bank files.

\item[\option{--user-tag} \parm{comment}]
Set the user tag to the string \parm{comment}.  This string must not
contain spaces or dashes (``-'').  This string will appear in the name of
the file to which output information is written, and is recorded in the
various XML tables within the file.

\item[\option{--verbose}]
Print debugging information to the
standard output while executing.

\item[\option{--version}]   
Print the CVS id and exit.

\end{entry}

\item[Example]
\begin{verbatim}
lalapps_splitbank --bank-file L1-TMPLTBANK-732488741-2048.xml \
--number-of-banks 3 --minimal-match 0.97
\end{verbatim}

\item[Algorithm]
\texttt{lalapps\_splitbank} counts the number of templates in the input file.
It increments this by one and divides by the number of template banks to
generate using standard integer division. This gives the upper limit on the
number of templates in a single output file.

\item[Author] 
Duncan Brown and Alexander Dietz
\end{entry}

