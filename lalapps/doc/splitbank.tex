\section{Program \texttt{lalapps\_splitbank}}
\label{program:lalapps-splitbank}
\idx[Program]{lalapps\_splitbank}

\begin{entry}
\item[Name]
\verb$lalapps_splitbank$ --- splits a template bank file into several smaller
files

\item[Synopsis]
\verb$lalapps_splitbank$ 
[\verb$--help$]
[\verb$--verbose$]
[\verb$--version$]
[\verb$--debug-level$ \textsc{level}] \newline
[\verb$--user-tag$ \textsc{string}]
[\verb$--comment$ \textsc{string}] \newline
\verb$--bank-file$ \textsc{file}
\verb$--number-of-banks$ \textsc{n}
\verb$--minimal-match$ \textsc{m}

\item[Description] 
\verb$lalapps_splitbank$ splits a LIGO\_LW XML file containing inspiral
templates in a \texttt{sngl\_inspiral} table into several smaller bank
files. This allows a template bank to be split across several inspiral 
jobs and then recombined with \texttt{lalapps\_inca} or
\texttt{lalapps\_sire}.

The name of the output template bank files is derived from the name of 
the input bank file and the number of files that the bank should be split
into. For example, if the input bank file:\\
\begin{center}
\texttt{H1-TRIGBANK\_L1-729330491-2048.xml}\\
\end{center}
is split into 3 output files, then these will be named:\\
\begin{center}
\texttt{H1-TRIGBANK\_L1\_00-729330491-2048.xml}\\
\texttt{H1-TRIGBANK\_L1\_01-729330491-2048.xml}\\
\texttt{H1-TRIGBANK\_L1\_02-729330491-2048.xml}\\
\end{center}
The naming convention is to insert the bank file number after the usertag part
of the filename and before the GPS start time part of the file name.

In the case that the input file contains no templates, empty output bank files
are generated. This is done since DAGman does not implement decision rules
yet, so the nodes in the DAG must be identical regardless of the data flowing
through them.

\item[Options]\leavevmode
\begin{entry}
\item[\texttt{--help}] Optional. Print a usage summary and exit.

\item[\texttt{--verbose}] Optional. Print debugging information to the
standard output while executing.

\item[\texttt{--version}] Optional.  Print the CVS id and exit.

\item[\texttt{--debug-level} \textsc{level}] Optional. Set the LAL debug
level to \textsc{level}. If not specified the default is 1.

\item[\texttt{--user-tag} \textsc{string}] Optional. Set the user tag in
the \texttt{process\_params} table to \textsc{string}. Note that this does
not set the user tag in the output filename, as it does in other inspiral
pipeline tools. The output filename is derived strictly from the number of
output banks and the input bank file name.

\item[\texttt{--comment} \textsc{string}] Optional. Set the comment column in
the output \texttt{process} table to \textsc{string}.

\item[\texttt{--bank-file} \textsc{file}] Required. Read the templates from
the \texttt{sngl\_inspiral} table in \textsc{file}.

\item[\texttt{--number-of-banks} \textsc{n}] Required. Split the input
template banks into \textsc{n} output bank files.

\item[\texttt{--minimal-match} \textsc{m}] Required. Set the minimal match of
the output template bank file to \textsc{m}.
\end{entry}

\item[Example]
\begin{verbatim}
lalapps_splitbank --bank-file L1-TMPLTBANK-732488741-2048.xml \
--number-of-banks 3 --minimal-match 0.97
\end{verbatim}

\item[Algorithm]
\texttt{lalapps\_splitbank} counts the number of templates in the input file.
It increments this by one and divides by the number of template banks to
generate using standard integer division. This gives the upper limit on the
number of templates in a single output file.

\item[Author] 
Duncan Brown
\end{entry}

