\section{Program \texttt{lalapps\_inca}}
\label{program:lalapps-inca}
\idx[Program]{lalapps\_inca}

\begin{entry}
\item[Name]
\verb$lalapps_inca$ --- program does inspiral coincidence analysis.

\item[Synopsis]
\begin{verbatim}
 [--help ] 
 [--verbose ]
 [--version ] 
 [--user-tag USERTAG ]
 [--ifo-tag IFOTAG ]
 [--comment STRING ] 
 
  --gps-start-time GPSSTARTTIME      
  --gps-end-time GPSENDTIME       
 
 [--silde-time SLIDE_SEC ]
 [--slide-time-ns SLIDE_NS ] 
 
  --ifo-a IFOA              
  --ifo-b IFOB             
 
 [--single-ifo ]
 [--triggered-bank TRIGBANKFILE]
 [--minimal-match M ]
 
 [--epsilon EPSILON ]
 [--kappa KAPPA ]
 [--ifo-b-snr-threshold B_SNR]
 [--ifo-b-range-cut ]
 
  --paramenter-test TEST   
 [--dm DM ]  
 [--dpsi0 DPSI0 ]
 [--dpsi3 DPSI3 ]
 [--alphaf-cut ALPHAFCUT]
 [--dmchirp DM_CHIRP ]
 [--deta  DETA ]
 [--dt DT ]
 
 [--no-playground ]
 [--playground-only ]
 [--write-uniq-triggers ]

\end{verbatim}

\textsc{(LIGO Lightweight XML files)}

\item[Description --- General] 

\verb$lalapps_inca$ runs in three distinct modes.  The first is to
perform coincidence between triggers from two distinct interferomters.
The second is to create a triggered templatebank from a list of inspiral
triggers.  The third is to create a list of triggers from a single
interferometer in a specified time interval.  The way to run the code in
these three ways is descibed below.


\item[Description --- Coincidence Testing]

We begin with the two interferometer coincidence testing.  This is the
default behaviour of the code.  It takes in triggers from two
interferometers and returns those which pass both time and mass
coincidence tests.  The two interferometers are specified with the
\texttt{ifo-a} and \texttt{ifo-b} arguments. The triggers are read in
from a list of LIGO Lightweight XML files given after the last command
line argument.  This list must contain at least one from each of the two
interferometers.  The triggers from these files are read in, and only
those triggers which lie in the interval between \textsc{GPSSTARTTIME} and
\textsc{GPSENDTIME} are kept. The default behaviour is to keep only
playground triggers (this can be explicitly requested with the
\texttt{playground-only} option).  By specifying
\texttt{no-playground}, only non-playground triggers are kept.  The
triggers from the two interferometers are then tested for time and mass
coincidence.  Two triggers are considered time coincident if their end
times differ by less than \textsc{dt} milliseconds.  If a time slide has
been specified, then \textsc{SLIDE\_SEC} seconds plus \textsc{SLIDE\_NS}
nanoseconds is added to the recorded time of each trigger in
\textsc{IFOB} before testing for time coincidence.  Triggers are then
tested for mass coincidence using one of three tests
\texttt{(m1\_and\_m2 | mchirp\_and\_eta | psi0\_and\_psi3)}.  If
\texttt{m1\_and\_m2} is specified then both the mass1 and mass2 fields
of the triggers must differ by less than the specified \textsc{DM}.  If
\texttt{mchirp\_and\_eta} is specified then the chirp masses must differ
by less than \textsc{DM\_CHIRP} and the mass ratios $\eta$ must differ
by less then \textsc{DETA}.  Finally, if \textsc{psi0\_and\_psi3} is
specified the \texttt{psi0} and \texttt{psi3} fields of the trigger must
differ by less than \textsc{PSI0} and \textsc{PSI3}.  

If demanding coincidence over m1 and m2, it then tests 
that 
%
\begin{equation} \frac{\left|D_\mathrm{IFOA} -
  D_\mathrm{IFOA}\right|}{D_\mathrm{IFOA}} <
  \frac{\epsilon}{\rho_\mathrm{IFOB}} + \kappa.  \end{equation}
% 
This is equivalent to testing that 
%
\begin{equation}\label{snrtest} \left|\rho_\mathrm{IFOB} -
\hat{\rho}_\mathrm{IFOB}\right| < \epsilon + \kappa\rho_\mathrm{IFOB},
  \end{equation} 
%
where 
%
\begin{equation} \hat{\rho}_\mathrm{IFOB} = \frac{\sigma_\mathrm{IFOB}}
{\sigma_\mathrm{IFOA}} \rho_\mathrm{IFOA} \, .  \end{equation} 
%

If demanding coincidence over $\psi_0$ and $\psi_3$, there is an additional
cut applied in the triggers. The single-ifo triggers that have values
of $\alpha_F$ greater than ALPHAFCUT (as specified in the 
command line) are rejected. For that reason, the option 
\texttt{--alphaf-cut} is required, if \texttt{--parameter-test} is
set to \texttt{psi0\_and\_psi3}.

If all the tests are passed, the events are considered to be coincident
and written to the output file.


The \texttt{--ifo-b-range-cut} option performs a test similar to
(\ref{snrtest}) above to see whether we should expect a trigger in
\textsc{IFOB}.  There are three possibilities, which depend upon the
value of the \textsc{SNRSTAR} threshold for \textsc{IFOB}, denoted
$\rho_\mathrm{IFOB}^{*}$.

\begin{enumerate}

\item In this case, the expected signal to noise ratio in \textsc{IFOB} is
above our threshold:
%
\begin{equation} 
  \rho_\mathrm{IFOB}^{*} < \frac{(\hat{\rho}_\mathrm{IFOB} - \epsilon)}
  {1 + \kappa} , 
\end{equation}
%
so we look for a coincident trigger.  We only keep the 
\textsc{IFOA} trigger if one is found in coincidence in \textsc{IFOB}.

\item In this case, our the allowed range of signal to noise ratio in 
\textsc{IFOB} is partly above and partly below our threshold: 
%
\begin{equation}
  \frac{(\hat{\rho}_\mathrm{IFOB} - \epsilon)} {1 + \kappa} < 
  \rho_\mathrm{IFOB}^{*} <
  \frac{(\hat{\rho}_\mathrm{IFOB} + \epsilon)} {1 - \kappa} .
\end{equation}
%
We search \textsc{IFOB} for triggers and record a coincident trigger if
found.  Otherwise, we just record the \textsc{IFOA} trigger.

\item In this case, the trigger is not visible to \textsc{IFOB}:
%
\begin{equation}
  \rho_\mathrm{IFOB}^{*} <
  \frac{(\hat{\rho}_\mathrm{IFOB} + \epsilon)} {1 - \kappa} .
\end{equation}
%
We do not search \textsc{IFOB}, but do keep the trigger from \textsc{IFOA}.


\end{enumerate}

The triggers which survive coincidence are output to two LIGO
lightweight XML files.  Two XML output files are written.  The output
files contain \texttt{process}, \texttt{process\_params} and
\texttt{search\_summary} tables that describe the search. The primary
ifo output file contains the triggers from \textsc{IFOA} that are found
to be in coincidence with triggers in \textsc{IFOB}. The secondary
output file contains the triggers from \textsc{IFOB} that are found to
be in coincidence with the triggers from \textsc{IFOA}.  Each trigger in
the \textsc{IFOA} file corresponds to the coincident trigger in the
\textsc{IFOB} file, so there may be duplicate \textsc{IFOA} triggers.
To prevent this, specify the \texttt{write-uniq-triggers} option.

The output files are named in the standard way for inspiral pipeline output.
The primary triggers are in a file named\\
\begin{center}
\texttt{IFOA-INCA\_IFOTAG\_USERTAG-GPSSTARTTIME-DURATION.xml}\\
\end{center}
and the secondary triggers are in a file named\\
\begin{center}
\texttt{IFOB-INCA\_IFOTAG\_USERTAG-GPSSTARTTIME-DURATION.xml}\\
\end{center}

If a \texttt{--user-tag} or \texttt{--ifo-tag} is not specified on the
command line, the \textsc{\_USERTAG} or \texttt{\_IFOTAG} part of the
filename will be omitted.


\item[Description --- Triggered Bank] 
  
If the \texttt{triggered-bank} option is specified, then
\texttt{lalapps\_inca} will produce a triggered template bank from the
input xml files.  In this case, the code expects triggers from only a
single interferometer, {IFOA}.  The triggered bank is formed by first
sorting the templates in time, and discarding any which are before the
\textsc{GPSSTARTTIME} or after the time specified \textsc{GPSENDTIME}.
The templates are then sorted according to the given
\texttt{parameter-test}, which must be one of \texttt{m1\_and\_m2} or
\texttt{psi0\_and\_psi3}.  Duplicate templates (those with identical m1
    and m2 or psio and psi3) are discarded and what
remains is output to the \texttt{TRIGBANKFILE} specified by the
\verb$--triggered-bank$ argument.  The output file contain
\texttt{process}, \texttt{process\_params}, \texttt{search\_summary} and
\texttt{sngl\_inspiral} tables.  

\item[Description --- Single IFO mode]

If the \texttt{single-ifo} option is specified, then
\texttt{lalapps\_inca} reads in triggers from a single interferometer
and returns those within the specified time window.  The time window is
specified by \textsc{GPSSTARTTIME} and \textsc{GPSENDTIME}.
By default, the program returns only playground triggers.  This
behaviour can be explicitly requested with the \texttt{playground-only}
flag.  If \texttt{no-playground} is specified then only those triggers
outside the playground are written to the output file.

The output file is named in the standard way for inspiral pipeline output.
The triggers are in a file named\\
\begin{center}
\texttt{IFOA-INCA\_IFOTAG\_USERTAG-GPSSTARTTIME-DURATION.xml}\\
\end{center}

If a \texttt{--user-tag} or \texttt{--ifo-tag} is not specified on the
command line, the \textsc{\_USERTAG} or \texttt{\_IFOTAG} part of the
filename will be omitted.  The triggers are stored in a
\texttt{sngl\_inspiral} table.  The output file also contains
\texttt{process}, \texttt{process\_params} and \texttt{search\_summary}
tables that describe the search.

\item[Options]\leavevmode
\begin{entry}

\item[\texttt{--triggered-bank} \texttt{TRIGBANKFILE}] Optional.  Run
inca in triggered bank mode.  Output the triggered bank to a file named
\texttt{TRIGBANKFILE}.

\item[\texttt{--single-ifo}] Optional.  Run inca in single ifo mode.

\item[\texttt{--playground-only}] Optional.  Record only triggers that
occur in the playground times.  This is the default behavior.

\item[\texttt{--no-playground}] Optional.  Record all triggers that are
not in playground data.  The default behavior returns only those triggers
which lie in the playground data set.  

\item[\texttt{--ifo-a} \textsc{IFOA}] Required. This is the name of the
interferometer to use as the interferometer A in the coincidence algorithm.
It must be a two letter IFO code e.g. \texttt{L1}, \texttt{H1}, etc.

\item[\texttt{--ifo-b} \textsc{IFOB}] Required for coincidence, not for
trigbank or single ifo. This is the name of the interferometer to use as
the interferometer B in the coincidence algorithm.  It must be a two
letter IFO code e.g. \texttt{L1}, \texttt{H1}, etc.

\item[\texttt{--epsilon} \textsc{$\epsilon$}] Optional. Set the value of
$\epsilon$ in the effective distance test. If not given the default of
$\epsilon = 2$ will be used.

\item[\texttt{--kappa} \textsc{$\kappa$}] Optional. Set the value of
$\kappa$ in the effective distance test. If not given the default of
$\kappa= 0.01$ will be used.

\item[\texttt{--ifo-b-range-cut}] Optional.  Use effective distance test
to see whether \textsc{IFOB} has a chance of seeing trigger before
performing the search.

\item[\texttt{--ifo-b-snr-threshold} \textsc{SNRSTAR}] Optional.  Set the 
value of the signal to noise threshold for \textsc{IFOB}.  This is used in
determining which triggers \textsc{IFOB} has a chance to see.  If not
specified, the default value of 6 is used.

\item[\texttt{--parameter-test} TEST] Required. Choose which parameters
to use when testing for coincidence
(m1\_and\_m2|psi0\_and\_psi3|mchirp\_and\_eta).  Depending on which test
is chosen, the allowed windows on the appropriate parameters should be
set as described below.

\item[\texttt{--dm} \textsc{$\delta m$}] Optional. Accept triggers as
coincident if both m1 and m2 agree within $\delta m$.  If not supplied,
	   then $\delta m = 0$.

\item[\texttt{--dpsi0} \textsc{$\delta \psi_0$}] Optional. Accept
triggers as coincident if \textsc{$\psi_0$} parameters agree within
$\delta \psi_0$.  If not supplied,  then $\delta  \psi_0 = 0$.

\item[\texttt{--dpsi3} \textsc{$\delta \psi_3$}] Optional. Accept
triggers as coincident if \textsc{$\psi_3$} parameters agree within
$\delta \psi_3$.  If not supplied,  then $\delta  \psi_3 = 0$.

\item[\texttt{--alphaf-cut} \textsc{ALPHAFCUT}] Required only if
\texttt{--parameter-test} is set to \texttt{psi0\_and\_psi3}. Accept 
only the single-ifo BCV triggers that have $\alpha_F$ less or
equal to \textsc{ALPHAFCUT}. Affects only the coincidence part of
the code and not the triggered-bank generation.

\item[\texttt{--dmchirp} \textsc{$\delta mchirp$}] Optional. Accept
triggers as coincident if mchirp agrees within $\delta mchirp$.  If not
supplied,  then $\delta mchirp = 0$.

\item[\texttt{--deta} \textsc{$\delta \eta$}] Optional. Accept triggers
as coincident if $eta$ agrees within $\delta \eta$.  If not supplied,
then $\delta \eta = 0$.

\item[\texttt{--dt} \textsc{$\delta t$}] Optional. Accept triggers as
coincident if their end times agree within $\delta t$ milliseconds.  If
not supplied,  then $\textsc{$\delta t$} = 0$.

\item[\texttt{--gps-start-time} \textsc{GPS seconds}] Required.  Look
for coincident triggers with end times after \textsc{GPS seconds}.

\item[\texttt{--gps-end-time} \textsc{GPS seconds}] Required.  Look for
coincident triggers with end times before \textsc{GPS seconds}.

\item[\texttt{--slide-time}] \textsc{SLIDE\_SEC} Optional.  Slide the
triggers from \textsc{IFOB} forwards in time by  \textsc{SLIDE\_SEC}
seconds before testing for coincidence.  Only used in the coincidence
testing mode of inca.

\item[\texttt{--slide-time-ns}] \textsc{SLIDE\_NS} Optional.  Slide the
triggers from \textsc{IFOB} forwards in time by  \textsc{SLIDE\_NS} nano
seconds before testing for coincidence.  Only used in the coincidence
testing mode of inca.

\item[\texttt{--write-uniq-triggers}] Optional.  The default behavior is
to only write all triggers from IFO A. However, a trigger from IFO A may
match two or more triggers from IFO B, so it may be duplicated in the
output. Specifying this option causes only unique IFO A triggers to be
written.

\item[\texttt{--minimal-match} \textsc{M}] Optional.  If running in triggered
bank mode, set the minimal match in the output file to \textsc{M}.

\item[\texttt{--comment} \textsc{string}] Optional. Add \textsc{string}
to the comment field in the process table. If not specified, no comment
is added. 

\item[\texttt{--user-tag} \textsc{USERTAG}] Optional. Set the user tag
for this job to be \textsc{USERTAG}. May also be specified on the command
line as \texttt{-userTag} for LIGO database compatibility.  This will
affect the naming of the output file.

\item[\texttt{--ifo-tag} \textsc{IFOTAG}] Optional. Set the user tag for
this job to be \textsc{IFOTAG}.  This will affect the naming of the
output file.

\item[\texttt{--verbose}] Enable the output of informational messages.


\item[\texttt{--help}] Optional.  Print a help message and exit.

\item[\texttt{--version}] Optional.  Print out the author, CVS version and
tag information and exit.

\end{entry}

\item[Arguments]\leavevmode
\begin{entry}
\item[\texttt{[LIGO Lightweight XML files]}] The arguments to the program
should be a list of LIGO Lightweight XML files containing the triggers from
the two interferometers. The input files can be in any order and do not need
to be time ordered as \texttt{inca} will sort all the triggers once they are
read in. If the program encounters a LIGO Lightweight XML containing triggers
from an unknown interferometer (i.e. not IFO A or IFO B) it will exit with an
error.
\end{entry}

\item[Example]
\begin{verbatim}
lalapps_inca \
--playground-only  --dm 0.03 --kappa 1000.0 --ifo-b H1 --ifo-a L1 \
--user-tag SNR6_INJ --gps-start-time 734323079
--gps-end-time 734324999 --epsilon 2.0 --dt 11.0 \
L1-INSPIRAL_INJ-734323015-2048.xml H1-INSPIRAL_INJ-734323015-2048.xml
\end{verbatim}

\item[Algorithm]
The code maintains two pointers to triggers from each ifo,
\texttt{currentTrigger[0]} and \texttt{currentTrigger[1]}, corresponding to
the current trigger from IFO A and B respectively.

\begin{enumerate}
\item An empty linked list of triggers from each interferometer is created.
Each input file is read in and the code determines which IFO the triggers in
the file correspond to. The triggers are appended to the linked list for the
corresponding interferometer.

\item If there are no triggers read in from either of the interferometers,
the code exits cleanly.

\item The triggers for each interferometer is sorted by the \texttt{end\_time}
of the trigger.

\item \texttt{currentTrigger[0]} is set to point to the first trigger from IFO
A that is after the specified GPS start time for coincidence. If no trigger is
found after the start time, the code exits cleanly.

\item Loop over each trigger from IFO A that occurs before the specified GPS
end time for coincidence:
\begin{enumerate}
\item \texttt{currentTrigger[1]} is set to point to the first trigger from IFO
B that is within the time coincidence window, $\delta t$, of
\texttt{currentTrigger[0]}. If no IFO B trigger exists within this window,
\texttt{currentTrigger[0]} is incremented to the next trigger from IFO A and
the loop over IFO A triggers restarts.

\item If the trigger \texttt{currentTrigger[0]} \emph{is, is not} in the
playground data, start looping over triggers from IFO B.
\begin{enumerate}
\item For each trigger from IFO B that is within $\delta t$ of
\texttt{currentTrigger[0]}
\item Call \texttt{LALCompareSnglInspiral()} to check if the triggers match as
determined by the options on the command line. If the trigger match, record
them for later output as coincident triggers.
\end{enumerate}

\item Increment \texttt{currentTrigger[0]} and continue loop over triggers
from IFO A.
\end{enumerate}
\end{enumerate}

\item[Author] 
Patrick Brady, Duncan Brown and Steve Fairhurst
\end{entry}


