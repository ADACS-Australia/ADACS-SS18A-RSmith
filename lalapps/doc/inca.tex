\section{Program \texttt{lalapps\_inca}}
\label{program:lalapps-inca}
\idx[Program]{lalapps\_inca}

\begin{entry}
\item[Name]
\verb$lalapps_inca$ --- program does inspiral coincidence analysis.

\item[Synopsis]
\begin{verbatim}
  --help                    display this message
  --verbose                 print progress information
  --version                 print version information and exit
  --debug-level LEVEL       set the LAL debug level to LEVEL
  --user-tag STRING         set the process_params usertag to STRING
  --ifo-tag STRING          set the ifo-tag to STRING - for file naming
  --comment STRING          set the process table comment to STRING
 
  --gps-start-time SEC      GPS second of data start time
  --gps-end-time SEC        GPS second of data end time
 
  --silde-time SEC          slide all triggers of IFOB by SEC
  --slide-time-ns NS        slide all triggers of IFOB by NS
 
  --ifo-a IFOA              name of first ifo (e.g. L1, H1 or H2)
  --ifo-b IFOB              name of second ifo (e.g. L1, H1 or H2)
 
  --triggered-bank FILE     write a triggered bank insted of doing inca
  --minimal-match M         set minimal match of triggered bank to M
 
  --epsilon ERROR           set effective distance test epsilon (default 2)
  --kappa ERROR             set effective distance test kappa (default 0.01)
  --ifo-b-snr-threshold SNR set minimum snr in IFO B (default 6)
  --ifo-b-range-cut         test range of IFO B to see if sensitive to trigger
  --paramenter-test TEST    set the desired parameters to test coincidence
                            for inca: (m1_and_m2|psi0_and_psi3|mchirp_and_eta)
                            for triggered bank (m1_and_m2|psi0_and_psi3)
  --dm Dm                   mass coincidence window (default 0)
  --dpsi0 Dpsi0             psi0 coincidence window
  --dpsi3 Dpsi3             psi3 coincidence window
  --dmchirp Dmchirp         mchirp coincidence window
  --deta  Deta              eta coincidence window
  --dt Dt                   time coincidence window (milliseconds)
 
  --no-playground           do not select triggers from playground
  --playground-only         only use triggers that are in playground
  --write-uniq-triggers     make sure triggers from IFO A are unique

\end{verbatim}

\textsc{(LIGO Lightweight XML files)}

\item[Description --- Coincidence Testing] 

\verb$lalapps_inca$ performs coincidence on triggers from the inspiral search
code.  At present it works for only two interferometers.  The names of the two
interferometers must be given. Output is written to LIGO lightweight XML files.
Two XML output files are written.  The output files contain \texttt{process},
\texttt{process\_params} and \texttt{search\_summary} tables that describe the
search. The primary ifo output file contains the triggers from \textsc{IFOA}
that are found to be in coincidence with triggers in \textsc{IFOB}. The
secondary output file contains the triggers from \textsc{IFOB} that are found
to be in coincidence with the triggers from \textsc{IFOA}.  Each trigger in the
\textsc{IFOA} file corresponds to the coincident trigger in the \textsc{IFOB}
file, so there may be duplicate \textsc{IFOA} triggers.  To prevent this,
specify the \verb$--write-uniq-triggers$ option.

The output files are named in the standard way for inspiral pipeline output.
The primary triggers are in a file named\\
\begin{center}
\texttt{IFOA-INCA\_IFOTAG\_USERTAG-GPSSTARTTIME-DURATION.xml}\\
\end{center}
and the secondary triggers are in a file named\\
\begin{center}
\texttt{IFOB-INCA\_IFOTAG\_USERTAG-GPSSTARTTIME-DURATION.xml}\\
\end{center}

If a \texttt{--user-tag} or \texttt{--ifo-tag} is not specified on the command
line, the \texttt{\_USERTAG} or \texttt{\_IFOTAG} part of the filename will be
omitted.

The default behaviour outputs triggers during playground times only. To obtain
those triggers that are not in the playground, use the \verb$--no-playground$
flag.     

\texttt{lalapps\_inca} tests for coincidence in time as well as another set of
trigger parameters.  The allowed parameter choices are specified by
\verb$--parameter-test$ \texttt{TEST}, where \texttt{TEST} is one of
m1\_and\_m2, psi0\_and\_psi3, mchirp\_and\_eta.  \texttt{lalapps\_inca} calls
the LAL function \texttt{LALCompareSnglInspiral()} to test if two triggers are
coincident. This first tests that the time of the triggers is coincidence to
within $\delta t$.  It then tests that the other parameters of the trigger are
coincident within the input ranges provided.  Additionally, if demanding
coincidence over m1 and m2, it then tests to that 
%
\begin{equation} 
  \frac{\left|D_\mathrm{IFOA} - D_\mathrm{IFOA}\right|}{D_\mathrm{IFOA}} < 
  \frac{\epsilon}{\rho_\mathrm{IFOB}} + \kappa.  
\end{equation}
% 
This is equivalent to testing that 
%
\begin{equation}\label{snrtest} 
  \left|\rho_\mathrm{IFOB} - \hat{\rho}_\mathrm{IFOB}\right| < 
  \epsilon + \kappa\rho_\mathrm{IFOB},
\end{equation} 
%
where 
%
\begin{equation} 
  \hat{\rho}_\mathrm{IFOB} = \frac{\sigma_\mathrm{IFOB}}
  {\sigma_\mathrm{IFOA}} \rho_\mathrm{IFOA} \, .  
\end{equation} 
%
If all the tests are passed, the events are considered to be coincident and
written to the output file.

The \texttt{--ifo-b-range-cut} option performs a test similar to
(\ref{snrtest}) above to see whether we should expect a trigger in
\textsc{IFOB}.  There are three possibilities, which depend upon the value of
the \textsc{SNRSTAR} threshold for \textsc{IFOB}, denoted
$\rho_\mathrm{IFOB}^{*}$.

\begin{enumerate}

\item In this case, the expected signal to noise ratio in \textsc{IFOB} is
above our threshold:
%
\begin{equation} 
  \rho_\mathrm{IFOB}^{*} < \frac{(\hat{\rho}_\mathrm{IFOB} - \epsilon)}
  {1 + \kappa} , 
\end{equation}
%
so we look for a coincident trigger.  We only keep the 
\textsc{IFOA} trigger if one is found in coincidence in \textsc{IFOB}.

\item In this case, our the allowed range of signal to noise ratio in 
\textsc{IFOB} is partly above and partly below our threshold: 
%
\begin{equation}
  \frac{(\hat{\rho}_\mathrm{IFOB} - \epsilon)} {1 + \kappa} < 
  \rho_\mathrm{IFOB}^{*} <
  \frac{(\hat{\rho}_\mathrm{IFOB} + \epsilon)} {1 - \kappa} .
\end{equation}
%
We search \textsc{IFOB} for triggers and record a coincident trigger if found.
Otherwise, we just record the \textsc{IFOA} trigger.

\item In this case, the trigger is not visible to \textsc{IFOB}:
%
\begin{equation}
  \rho_\mathrm{IFOB}^{*} <
  \frac{(\hat{\rho}_\mathrm{IFOB} + \epsilon)} {1 - \kappa} .
\end{equation}
%
We do not search \textsc{IFOB}, but do keep the trigger from \textsc{IFOA}.

\end{enumerate}

\item[Description --- Triggered Bank] 
  
If the option \verb$--triggered-bank$ is specified, then \texttt{lalapps\_inca}
will produce a triggered template bank from the input xml files.  In this case,
the code expects triggers from only a single interferometer, {IFOA}.  The
triggered bank is formed by first sorting the templates in time, and discarding
any which are before the time specified with \verb$--gps-start-time$ or after
the time specified with \verb$--gps-end-time$.  The templates are then sorted
according to the given \verb$--parameter-test$, which must be one of m1\_and\_m2
or psi0\_and\_psi3.  Duplicate templates are discarded and what remains is
output to the \texttt{FILE} specified by the \verb$--triggered-bank$ argument.
The output file contain \texttt{process}, \texttt{process\_params},
\texttt{search\_summary} and \texttt{sngl\_inspiral} tables.  

\item[Options]\leavevmode
\begin{entry}
\item[\texttt{--no-playground}] Optional.  Record all triggers that are
not in playground data.  The default behavior returns only those triggers
which lie in the playground data set.  

\item[\texttt{--playground-only}] Optional.  Record only triggers that
occur in the playground times.  This is the default behavior.

\item[\texttt{--ifo-a} \textsc{IFOA}] Required. This is the name of the
interferometer to use as the interferometer A in the coincidence algorithm.
It must be a two letter IFO code e.g. \texttt{L1}, \texttt{H1}, etc.

\item[\texttt{--ifo-b} \textsc{IFOB}] Required. This is the name of the
interferometer to use as the interferometer B in the coincidence algorithm.
It must be a two letter IFO code e.g. \texttt{L1}, \texttt{H1}, etc.

\item[\texttt{--epsilon} \textsc{$\epsilon$}] Optional. Set the value of
$\epsilon$ in the effective distance test. If not given the default of
$\epsilon = 2$ will be used.

\item[\texttt{--kappa} \textsc{$\kappa$}] Optional. Set the value of
$\kappa$ in the effective distance test. If not given the default of
$\kappa= 0.01$ will be used.

\item[\texttt{--ifo-b-snr-threshold} \textsc{SNRSTAR}] Optional.  Set the 
value of the signal to noise threshold for \textsc{IFOB}.  This is used in
determining which triggers \textsc{IFOB} has a chance to see.  The default
value 

\item[\texttt{--ifo-b-range-cut}] Optional.  Use effective distance test to
see whether \textsc{IFOB} has a chance of seeing trigger before performing 
the search.

\item[\texttt{--parameter-test} TEST] Required. Choose which parameters to use 
when testing for coincidence (m1\_and\_m2|psi0\_and\_psi3|mchirp\_and\_eta).  
Depending on which test is chosen, the allowed windows on the appropriate 
parameters should be set as described below.

\item[\texttt{--dm} \textsc{$\delta m$}] Optional. Accept triggers as
coincident if both m1 and m2 agree within $\delta m$.  If not
supplied,  then $\delta m = 0$.

\item[\texttt{--dpsi0} \textsc{$\delta \psi_0$}] Optional. Accept triggers as
coincident if \textsc{$\psi_0$} parameters agree within $\delta \psi_0$.  If not
supplied,  then $\delta  \psi_0 = 0$.

\item[\texttt{--dpsi3} \textsc{$\delta \psi_3$}] Optional. Accept triggers as
coincident if \textsc{$\psi_3$} parameters agree within $\delta \psi_3$. 
 If not supplied,  then $\delta  \psi_3 = 0$.

\item[\texttt{--dmchirp} \textsc{$\delta mchirp$}] Optional. Accept triggers as
coincident if mchirp agrees within $\delta mchirp$.  If not
supplied,  then $\delta mchirp = 0$.

\item[\texttt{--deta} \textsc{$\delta \eta$}] Optional. Accept triggers as
coincident if $eta$ agrees within $\delta \eta$.  If not
supplied,  then $\delta \eta = 0$.

\item[\texttt{--dt} \textsc{$\delta t$}] Optional. Accept triggers as
coincident if their end times agree within $\delta t$ milliseconds.  If not
supplied,  then $\textsc{$\delta t$} = 0$.

\item[\texttt{--gps-start-time} \textsc{GPS seconds}] Required.  Look for
coincident triggers with end times after \textsc{GPS seconds}.

\item[\texttt{--gps-end-time} \textsc{GPS seconds}] Required.  Look for
coincident triggers with end times before \textsc{GPS seconds}.

\item[\texttt{--write-uniq-triggers}] Optional.  The default behavior is to
only write all triggers from IFO A. However, a trigger from IFO A
may match two or more triggers from IFO B, so it may be duplicated in the 
output. Specifying this option causes only unique IFO A triggers to be
written.

\item[\texttt{--comment} \textsc{string}] Optional. Add \textsc{string} to the
comment field in the process table. If not specified, no comment is added. 

\item[\texttt{--user-tag} \textsc{string}] Optional. Set the user tag for this
job to be \textsc{string}. May also be specified on the command line as 
\texttt{-userTag} for LIGO database compatibility.

\item[\texttt{--help}] Optional.  Print a help message.

\item[\texttt{--debug-level} \textsc{level}] Optional. Set the LAL debug
level to \textsc{level}. If not specified the default is 1.

\end{entry}

\item[Arguments]\leavevmode
\begin{entry}
\item[\texttt{[LIGO Lightweight XML files]}] The arguments to the program
should be a list of LIGO Lightweight XML files containing the triggers from
the two interferometers. The input files can be in any order and do not need
to be time ordered as \texttt{inca} will sort all the triggers once they are
read in. If the program encounters a LIGO Lightweight XML containing triggers
from an unknown interferometer (i.e. not IFO A or IFO B) it will exit with an
error.
\end{entry}

\item[Example]
\begin{verbatim}
lalapps_inca \
--playground-only  --dm 0.03 --kappa 1000.0 --ifo-b H1 --ifo-a L1 \
--user-tag SNR6_INJ --debug-level 33 --gps-start-time 734323079
--gps-end-time 734324999 --epsilon 2.0 --dt 11.0 \
L1-INSPIRAL_INJ-734323015-2048.xml H1-INSPIRAL_INJ-734323015-2048.xml
\end{verbatim}

\item[Algorithm]
The code maintains two pointers to triggers from each ifo,
\texttt{currentTrigger[0]} and \texttt{currentTrigger[1]}, corresponding to
the current trigger from IFO A and B respectively.

\begin{enumerate}
\item An empty linked list of triggers from each interferometer is created.
Each input file is read in and the code determines which IFO the triggers in
the file correspond to. The triggers are appended to the linked list for the
corresponding interferometer.

\item If there are no triggers read in from either of the interferometers,
the code exits cleanly.

\item The triggers for each interferometer is sorted by the \texttt{end\_time}
of the trigger.

\item \texttt{currentTrigger[0]} is set to point to the first trigger from IFO
A that is after the specified GPS start time for coincidence. If no trigger is
found after the start time, the code exits cleanly.

\item Loop over each trigger from IFO A that occurs before the specified GPS
end time for coincidence:
\begin{enumerate}
\item \texttt{currentTrigger[1]} is set to point to the first trigger from IFO
B that is within the time coincidence window, $\delta t$, of
\texttt{currentTrigger[0]}. If no IFO B trigger exists within this window,
\texttt{currentTrigger[0]} is incremented to the next trigger from IFO A and
the loop over IFO A triggers restarts.

\item If the trigger \texttt{currentTrigger[0]} \emph{is, is not} in the
playground data, start looping over triggers from IFO B.
\begin{enumerate}
\item For each trigger from IFO B that is within $\delta t$ of
\texttt{currentTrigger[0]}
\item Call \texttt{LALCompareSnglInspiral()} to check if the triggers match as
determined by the options on the command line. If the trigger match, record
them for later output as coincident triggers.
\end{enumerate}

\item Increment \texttt{currentTrigger[0]} and continue loop over triggers
from IFO A.
\end{enumerate}
\end{enumerate}

\item[Author] 
Patrick Brady, Duncan Brown and Steve Fairhurst
\end{entry}


