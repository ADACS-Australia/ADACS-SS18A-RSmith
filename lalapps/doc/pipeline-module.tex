%
% API Documentation
% Module pipeline
%
% Generated by epydoc 2.0
% [Wed Sep 24 18:39:38 2003]
%

%%%%%%%%%%%%%%%%%%%%%%%%%%%%%%%%%%%%%%%%%%%%%%%%%%%%%%%%%%%%%%%%%%%%%%%%%%%
%%                          Module Description                           %%
%%%%%%%%%%%%%%%%%%%%%%%%%%%%%%%%%%%%%%%%%%%%%%%%%%%%%%%%%%%%%%%%%%%%%%%%%%%

    \index{pipeline \textit{(module)}|(}
\section{Python Module \texttt{pipeline.py}}

    \label{pipeline}
pipeline.py - condor functions used by pipeline scripts

\$Id$


%%%%%%%%%%%%%%%%%%%%%%%%%%%%%%%%%%%%%%%%%%%%%%%%%%%%%%%%%%%%%%%%%%%%%%%%%%%
%%                               Functions                               %%
%%%%%%%%%%%%%%%%%%%%%%%%%%%%%%%%%%%%%%%%%%%%%%%%%%%%%%%%%%%%%%%%%%%%%%%%%%%

  \subsection{Functions}

    \label{pipeline:s2play}
    \index{pipeline \textit{(module)}!s2play \textit{(function)}}
    \vspace{0.5ex}

    \noindent\begin{boxedminipage}{\textwidth}

    \raggedright \textbf{s2play}(\textit{t})

    \vspace{-1.5ex}

    \rule{\textwidth}{0.5\fboxrule}
    Return 1 if t is in the S2 playground, 0 otherwise t = GPS time to 
    test if playground

    \vspace{1ex}

    \end{boxedminipage}


%%%%%%%%%%%%%%%%%%%%%%%%%%%%%%%%%%%%%%%%%%%%%%%%%%%%%%%%%%%%%%%%%%%%%%%%%%%
%%                               Variables                               %%
%%%%%%%%%%%%%%%%%%%%%%%%%%%%%%%%%%%%%%%%%%%%%%%%%%%%%%%%%%%%%%%%%%%%%%%%%%%

  \subsection{Variables}

\begin{longtable}{|p{.30\textwidth}|p{.62\textwidth}|l}
\cline{1-2}
\cline{1-2} \centering \textbf{Name} & \centering \textbf{Description}& \\
\cline{1-2}
\endhead\cline{1-2}\multicolumn{3}{r}{\small\textit{continued on next page}}\\\endfoot\cline{1-2}
\endlastfoot\raggedright \_\-\_\-a\-u\-t\-h\-o\-r\-\_\-\_\- & \raggedright \textbf{Value:} 
{\tt '\-D\-u\-n\-c\-a\-n\-~\-B\-r\-o\-w\-n\-~\-{\textless}\-d\-u\-n\-c\-a\-n\-@\-g\-r\-a\-v\-i\-t\-y\-.\-p\-h\-y\-s\-.\-u\-w\-m\-.\-e\-d\-u\-{\textgreater}\-'\-}&\\
\cline{1-2}
\raggedright \_\-\_\-d\-a\-t\-e\-\_\-\_\- & \raggedright \textbf{Value:} 
{\tt '\-\$\-D\-a\-t\-e\-:\-~\-2\-0\-0\-3\-/\-0\-9\-/\-2\-3\-~\-1\-6\-:\-3\-3\-:\-4\-6\-~\-\$\-'\-}&\\
\cline{1-2}
\raggedright \_\-\_\-v\-e\-r\-s\-i\-o\-n\-\_\-\_\- & \raggedright \textbf{Value:} 
{\tt '\-1\-.\-2\-'\-}&\\
\cline{1-2}
\end{longtable}

    \index{pipeline \textit{(module)}!AnalysisChunk \textit{(class)}|(}

%%%%%%%%%%%%%%%%%%%%%%%%%%%%%%%%%%%%%%%%%%%%%%%%%%%%%%%%%%%%%%%%%%%%%%%%%%%
%%                           Class Description                           %%
%%%%%%%%%%%%%%%%%%%%%%%%%%%%%%%%%%%%%%%%%%%%%%%%%%%%%%%%%%%%%%%%%%%%%%%%%%%

\subsection{Class AnalysisChunk}

    \label{pipeline:AnalysisChunk}
An AnalysisCunk is the unit of data that a node works with, usually some 
subset of a ScienceSegment.


%%%%%%%%%%%%%%%%%%%%%%%%%%%%%%%%%%%%%%%%%%%%%%%%%%%%%%%%%%%%%%%%%%%%%%%%%%%
%%                                Methods                                %%
%%%%%%%%%%%%%%%%%%%%%%%%%%%%%%%%%%%%%%%%%%%%%%%%%%%%%%%%%%%%%%%%%%%%%%%%%%%

  \subsubsection{Methods}

    \label{pipeline:AnalysisChunk:__init__}
    \index{pipeline \textit{(module)}!AnalysisChunk \textit{(class)}!\_\_init\_\_ \textit{(method)}}
    \vspace{0.5ex}

    \noindent\begin{boxedminipage}{\textwidth}

    \raggedright \textbf{\_\_init\_\_}(\textit{self}, \textit{start}, \textit{end})

    \vspace{-1.5ex}

    \rule{\textwidth}{0.5\fboxrule}
    start = GPS start time of the chunk. end = GPS end time of the chunk.

    \vspace{1ex}

    \end{boxedminipage}

    \label{pipeline:AnalysisChunk:__repr__}
    \index{pipeline \textit{(module)}!AnalysisChunk \textit{(class)}!\_\_repr\_\_ \textit{(method)}}
    \vspace{0.5ex}

    \noindent\begin{boxedminipage}{\textwidth}

    \raggedright \textbf{\_\_repr\_\_}(\textit{self})

    \end{boxedminipage}

    \label{pipeline:AnalysisChunk:dur}
    \index{pipeline \textit{(module)}!AnalysisChunk \textit{(class)}!dur \textit{(method)}}
    \vspace{0.5ex}

    \noindent\begin{boxedminipage}{\textwidth}

    \raggedright \textbf{dur}(\textit{self})

    \vspace{-1.5ex}

    \rule{\textwidth}{0.5\fboxrule}
    Returns the length (duration) of the chunk in seconds.

    \vspace{1ex}

    \end{boxedminipage}

    \label{pipeline:AnalysisChunk:end}
    \index{pipeline \textit{(module)}!AnalysisChunk \textit{(class)}!end \textit{(method)}}
    \vspace{0.5ex}

    \noindent\begin{boxedminipage}{\textwidth}

    \raggedright \textbf{end}(\textit{self})

    \vspace{-1.5ex}

    \rule{\textwidth}{0.5\fboxrule}
    Returns the GPS end time of the chunk.

    \vspace{1ex}

    \end{boxedminipage}

    \label{pipeline:AnalysisChunk:start}
    \index{pipeline \textit{(module)}!AnalysisChunk \textit{(class)}!start \textit{(method)}}
    \vspace{0.5ex}

    \noindent\begin{boxedminipage}{\textwidth}

    \raggedright \textbf{start}(\textit{self})

    \vspace{-1.5ex}

    \rule{\textwidth}{0.5\fboxrule}
    Returns the GPS start time of the chunk.

    \vspace{1ex}

    \end{boxedminipage}

    \index{pipeline \textit{(module)}!AnalysisChunk \textit{(class)}|)}
    \index{pipeline \textit{(module)}!AnalysisJob \textit{(class)}|(}

%%%%%%%%%%%%%%%%%%%%%%%%%%%%%%%%%%%%%%%%%%%%%%%%%%%%%%%%%%%%%%%%%%%%%%%%%%%
%%                           Class Description                           %%
%%%%%%%%%%%%%%%%%%%%%%%%%%%%%%%%%%%%%%%%%%%%%%%%%%%%%%%%%%%%%%%%%%%%%%%%%%%

\subsection{Class AnalysisJob}

    \label{pipeline:AnalysisJob}
Describes a generic analysis job that filters LIGO data as configured by 
an ini file.


%%%%%%%%%%%%%%%%%%%%%%%%%%%%%%%%%%%%%%%%%%%%%%%%%%%%%%%%%%%%%%%%%%%%%%%%%%%
%%                                Methods                                %%
%%%%%%%%%%%%%%%%%%%%%%%%%%%%%%%%%%%%%%%%%%%%%%%%%%%%%%%%%%%%%%%%%%%%%%%%%%%

  \subsubsection{Methods}

    \label{pipeline:AnalysisJob:__init__}
    \index{pipeline \textit{(module)}!AnalysisJob \textit{(class)}!\_\_init\_\_ \textit{(method)}}
    \vspace{0.5ex}

    \noindent\begin{boxedminipage}{\textwidth}

    \raggedright \textbf{\_\_init\_\_}(\textit{self}, \textit{cp})

    \vspace{-1.5ex}

    \rule{\textwidth}{0.5\fboxrule}
    cp = ConfigParser object that contains the configuration for this 
    job.

    \vspace{1ex}

    \end{boxedminipage}

    \label{pipeline:AnalysisJob:calibration}
    \index{pipeline \textit{(module)}!AnalysisJob \textit{(class)}!calibration \textit{(method)}}
    \vspace{0.5ex}

    \noindent\begin{boxedminipage}{\textwidth}

    \raggedright \textbf{calibration}(\textit{self}, \textit{ifo})

    \vspace{-1.5ex}

    \rule{\textwidth}{0.5\fboxrule}
    Returns the name of the calibration file to use for the given IFO. 
    ifo = name of interferomener (e.g. L1, H1 or H2).

    \vspace{1ex}

    \end{boxedminipage}

    \label{pipeline:AnalysisJob:channel}
    \index{pipeline \textit{(module)}!AnalysisJob \textit{(class)}!channel \textit{(method)}}
    \vspace{0.5ex}

    \noindent\begin{boxedminipage}{\textwidth}

    \raggedright \textbf{channel}(\textit{self})

    \vspace{-1.5ex}

    \rule{\textwidth}{0.5\fboxrule}
    Returns the name of the channel that this job is filtering. Note that 
    channel is defined to be IFO independent, so this may be LSC-AS\_Q or 
    IOO-MC\_F. The IFO is set on a per node basis, not a per job basis.

    \vspace{1ex}

    \end{boxedminipage}

    \label{pipeline:AnalysisJob:get_config}
    \index{pipeline \textit{(module)}!AnalysisJob \textit{(class)}!get\_config \textit{(method)}}
    \vspace{0.5ex}

    \noindent\begin{boxedminipage}{\textwidth}

    \raggedright \textbf{get\_config}(\textit{self}, \textit{sec}, \textit{opt})

    \vspace{-1.5ex}

    \rule{\textwidth}{0.5\fboxrule}
    Get the configration variable in a particular section of this jobs 
    ini file. sec = ini file section. opt = option from section sec.

    \vspace{1ex}

    \end{boxedminipage}

    \index{pipeline \textit{(module)}!AnalysisJob \textit{(class)}|)}
    \index{pipeline \textit{(module)}!AnalysisNode \textit{(class)}|(}

%%%%%%%%%%%%%%%%%%%%%%%%%%%%%%%%%%%%%%%%%%%%%%%%%%%%%%%%%%%%%%%%%%%%%%%%%%%
%%                           Class Description                           %%
%%%%%%%%%%%%%%%%%%%%%%%%%%%%%%%%%%%%%%%%%%%%%%%%%%%%%%%%%%%%%%%%%%%%%%%%%%%

\subsection{Class AnalysisNode}

    \label{pipeline:AnalysisNode}
\begin{tabular}{cccccc}
% Line for pipeline.CondorDAGNode, linespec=[0]
\multicolumn{2}{r}{\settowidth{\BCL}{pipeline.CondorDAGNode}\multirow{2}{\BCL}{pipeline.CondorDAGNode}}
&&
  \\\cline{3-3}
  &&\multicolumn{1}{c|}{}
&&
  \\
&&\multicolumn{2}{l}{\textbf{AnalysisNode}}
\end{tabular}

Contains the methods that allow an object to be built to analyse LIGO 
data in a Condor DAG.


%%%%%%%%%%%%%%%%%%%%%%%%%%%%%%%%%%%%%%%%%%%%%%%%%%%%%%%%%%%%%%%%%%%%%%%%%%%
%%                                Methods                                %%
%%%%%%%%%%%%%%%%%%%%%%%%%%%%%%%%%%%%%%%%%%%%%%%%%%%%%%%%%%%%%%%%%%%%%%%%%%%

  \subsubsection{Methods}

    \label{pipeline:AnalysisNode:__init__}
    \index{pipeline \textit{(module)}!AnalysisNode \textit{(class)}!\_\_init\_\_ \textit{(method)}}
    \vspace{0.5ex}

    \noindent\begin{boxedminipage}{\textwidth}

    \raggedright \textbf{\_\_init\_\_}(\textit{self})

      Overrides: pipeline.CondorDAGNode.\_\_init\_\_

    \end{boxedminipage}

    \label{pipeline:AnalysisNode:get_end}
    \index{pipeline \textit{(module)}!AnalysisNode \textit{(class)}!get\_end \textit{(method)}}
    \vspace{0.5ex}

    \noindent\begin{boxedminipage}{\textwidth}

    \raggedright \textbf{get\_end}(\textit{self})

    \vspace{-1.5ex}

    \rule{\textwidth}{0.5\fboxrule}
    Get the GPS end time of the node.

    \vspace{1ex}

    \end{boxedminipage}

    \label{pipeline:AnalysisNode:get_ifo}
    \index{pipeline \textit{(module)}!AnalysisNode \textit{(class)}!get\_ifo \textit{(method)}}
    \vspace{0.5ex}

    \noindent\begin{boxedminipage}{\textwidth}

    \raggedright \textbf{get\_ifo}(\textit{self})

    \vspace{-1.5ex}

    \rule{\textwidth}{0.5\fboxrule}
    Returns the two letter IFO code for this node.

    \vspace{1ex}

    \end{boxedminipage}

    \label{pipeline:AnalysisNode:get_input}
    \index{pipeline \textit{(module)}!AnalysisNode \textit{(class)}!get\_input \textit{(method)}}
    \vspace{0.5ex}

    \noindent\begin{boxedminipage}{\textwidth}

    \raggedright \textbf{get\_input}(\textit{self})

    \vspace{-1.5ex}

    \rule{\textwidth}{0.5\fboxrule}
    Get the file that will be passed as input.

    \vspace{1ex}

    \end{boxedminipage}

    \label{pipeline:AnalysisNode:get_output}
    \index{pipeline \textit{(module)}!AnalysisNode \textit{(class)}!get\_output \textit{(method)}}
    \vspace{0.5ex}

    \noindent\begin{boxedminipage}{\textwidth}

    \raggedright \textbf{get\_output}(\textit{self})

    \vspace{-1.5ex}

    \rule{\textwidth}{0.5\fboxrule}
    Get the file that will be passed as output.

    \vspace{1ex}

    \end{boxedminipage}

    \label{pipeline:AnalysisNode:get_start}
    \index{pipeline \textit{(module)}!AnalysisNode \textit{(class)}!get\_start \textit{(method)}}
    \vspace{0.5ex}

    \noindent\begin{boxedminipage}{\textwidth}

    \raggedright \textbf{get\_start}(\textit{self})

    \vspace{-1.5ex}

    \rule{\textwidth}{0.5\fboxrule}
    Get the GPS start time of the node.

    \vspace{1ex}

    \end{boxedminipage}

    \label{pipeline:AnalysisNode:set_cache}
    \index{pipeline \textit{(module)}!AnalysisNode \textit{(class)}!set\_cache \textit{(method)}}
    \vspace{0.5ex}

    \noindent\begin{boxedminipage}{\textwidth}

    \raggedright \textbf{set\_cache}(\textit{self}, \textit{file})

    \vspace{-1.5ex}

    \rule{\textwidth}{0.5\fboxrule}
    Set the LAL frame cache to to use. The frame cache is passed to the 
    job with the --frame-cache argument. file = calibration file to use.

    \vspace{1ex}

    \end{boxedminipage}

    \label{pipeline:AnalysisNode:set_end}
    \index{pipeline \textit{(module)}!AnalysisNode \textit{(class)}!set\_end \textit{(method)}}
    \vspace{0.5ex}

    \noindent\begin{boxedminipage}{\textwidth}

    \raggedright \textbf{set\_end}(\textit{self}, \textit{time})

    \vspace{-1.5ex}

    \rule{\textwidth}{0.5\fboxrule}
    Set the GPS end time of the analysis node by setting a --gps-end-time 
    option to the node when it is executed. time = GPS end time of job.

    \vspace{1ex}

    \end{boxedminipage}

    \label{pipeline:AnalysisNode:set_ifo}
    \index{pipeline \textit{(module)}!AnalysisNode \textit{(class)}!set\_ifo \textit{(method)}}
    \vspace{0.5ex}

    \noindent\begin{boxedminipage}{\textwidth}

    \raggedright \textbf{set\_ifo}(\textit{self}, \textit{ifo})

    \vspace{-1.5ex}

    \rule{\textwidth}{0.5\fboxrule}
    Set the channel name to analyze and add a calibration file for that 
    channel. The name of the ifo is prepended to the channel name 
    obtained from the job configuration file and passed with a 
    --channel-name option. A calibration file is obtained from the ini 
    file and passed with a --calibration-cache option. ifo = two letter 
    ifo code (e.g. L1, H1 or H2).

    \vspace{1ex}

    \end{boxedminipage}

    \label{pipeline:AnalysisNode:set_input}
    \index{pipeline \textit{(module)}!AnalysisNode \textit{(class)}!set\_input \textit{(method)}}
    \vspace{0.5ex}

    \noindent\begin{boxedminipage}{\textwidth}

    \raggedright \textbf{set\_input}(\textit{self}, \textit{file})

    \vspace{-1.5ex}

    \rule{\textwidth}{0.5\fboxrule}
    Add an input to the node by adding a --input option. file = option 
    argument to pass as input.

    \vspace{1ex}

    \end{boxedminipage}

    \label{pipeline:AnalysisNode:set_output}
    \index{pipeline \textit{(module)}!AnalysisNode \textit{(class)}!set\_output \textit{(method)}}
    \vspace{0.5ex}

    \noindent\begin{boxedminipage}{\textwidth}

    \raggedright \textbf{set\_output}(\textit{self}, \textit{file})

    \vspace{-1.5ex}

    \rule{\textwidth}{0.5\fboxrule}
    Add an output to the node by adding a --output option. file = option 
    argument to pass as output.

    \vspace{1ex}

    \end{boxedminipage}

    \label{pipeline:AnalysisNode:set_start}
    \index{pipeline \textit{(module)}!AnalysisNode \textit{(class)}!set\_start \textit{(method)}}
    \vspace{0.5ex}

    \noindent\begin{boxedminipage}{\textwidth}

    \raggedright \textbf{set\_start}(\textit{self}, \textit{time})

    \vspace{-1.5ex}

    \rule{\textwidth}{0.5\fboxrule}
    Set the GPS start time of the analysis node by setting a 
    --gps-start-time option to the node when it is executed. time = GPS 
    start time of job.

    \vspace{1ex}

    \end{boxedminipage}

  \textbf{Inherited from CondorDAGNode:}
    \_\_repr\_\_,
    add\_parent,
    add\_var,
    job,
    set\_log\_file,
    set\_name,
    set\_retry,
    write
    \index{pipeline \textit{(module)}!AnalysisNode \textit{(class)}|)}
    \index{pipeline \textit{(module)}!CondorDAG \textit{(class)}|(}

%%%%%%%%%%%%%%%%%%%%%%%%%%%%%%%%%%%%%%%%%%%%%%%%%%%%%%%%%%%%%%%%%%%%%%%%%%%
%%                           Class Description                           %%
%%%%%%%%%%%%%%%%%%%%%%%%%%%%%%%%%%%%%%%%%%%%%%%%%%%%%%%%%%%%%%%%%%%%%%%%%%%

\subsection{Class CondorDAG}

    \label{pipeline:CondorDAG}
A CondorDAG is a Condor Directed Acyclic Graph that describes a 
collection of Condor jobs and the order in which to run them. All Condor 
jobs in the DAG must write their Codor logs to the same file. NOTE: The 
log file must not be on an NFS mounted system as the Condor jobs must be 
able to get an exclusive file lock on the log file.


%%%%%%%%%%%%%%%%%%%%%%%%%%%%%%%%%%%%%%%%%%%%%%%%%%%%%%%%%%%%%%%%%%%%%%%%%%%
%%                                Methods                                %%
%%%%%%%%%%%%%%%%%%%%%%%%%%%%%%%%%%%%%%%%%%%%%%%%%%%%%%%%%%%%%%%%%%%%%%%%%%%

  \subsubsection{Methods}

    \label{pipeline:CondorDAG:__init__}
    \index{pipeline \textit{(module)}!CondorDAG \textit{(class)}!\_\_init\_\_ \textit{(method)}}
    \vspace{0.5ex}

    \noindent\begin{boxedminipage}{\textwidth}

    \raggedright \textbf{\_\_init\_\_}(\textit{self}, \textit{log})

    \vspace{-1.5ex}

    \rule{\textwidth}{0.5\fboxrule}
    log = path to log file which must not be on an NFS mounted file 
    system.

    \vspace{1ex}

    \end{boxedminipage}

    \label{pipeline:CondorDAG:add_node}
    \index{pipeline \textit{(module)}!CondorDAG \textit{(class)}!add\_node \textit{(method)}}
    \vspace{0.5ex}

    \noindent\begin{boxedminipage}{\textwidth}

    \raggedright \textbf{add\_node}(\textit{self}, \textit{node})

    \vspace{-1.5ex}

    \rule{\textwidth}{0.5\fboxrule}
    Add a CondorDAGNode to this DAG. The CondorJob that the node uses is 
    also added to the list of Condor jobs in the DAG so that a list of 
    the submit files needed by the DAG can be maintained. Each unique 
    CondorJob will be added once to prevent duplicate submit files being 
    written. node = CondorDAGNode to add to the CondorDAG.

    \vspace{1ex}

    \end{boxedminipage}

    \label{pipeline:CondorDAG:set_dag_file}
    \index{pipeline \textit{(module)}!CondorDAG \textit{(class)}!set\_dag\_file \textit{(method)}}
    \vspace{0.5ex}

    \noindent\begin{boxedminipage}{\textwidth}

    \raggedright \textbf{set\_dag\_file}(\textit{self}, \textit{path})

    \vspace{-1.5ex}

    \rule{\textwidth}{0.5\fboxrule}
    Set the name of the file into which the DAG is written. path = path 
    to DAG file.

    \vspace{1ex}

    \end{boxedminipage}

    \label{pipeline:CondorDAG:write_dag}
    \index{pipeline \textit{(module)}!CondorDAG \textit{(class)}!write\_dag \textit{(method)}}
    \vspace{0.5ex}

    \noindent\begin{boxedminipage}{\textwidth}

    \raggedright \textbf{write\_dag}(\textit{self})

    \vspace{-1.5ex}

    \rule{\textwidth}{0.5\fboxrule}
    Write all the nodes in the DAG to the DAG file.

    \vspace{1ex}

    \end{boxedminipage}

    \label{pipeline:CondorDAG:write_sub_files}
    \index{pipeline \textit{(module)}!CondorDAG \textit{(class)}!write\_sub\_files \textit{(method)}}
    \vspace{0.5ex}

    \noindent\begin{boxedminipage}{\textwidth}

    \raggedright \textbf{write\_sub\_files}(\textit{self})

    \vspace{-1.5ex}

    \rule{\textwidth}{0.5\fboxrule}
    Write all the submit files used by the dag to disk. Each submit file 
    is written to the file name set in the CondorJob.

    \vspace{1ex}

    \end{boxedminipage}

    \index{pipeline \textit{(module)}!CondorDAG \textit{(class)}|)}
    \index{pipeline \textit{(module)}!CondorDAGError \textit{(class)}|(}

%%%%%%%%%%%%%%%%%%%%%%%%%%%%%%%%%%%%%%%%%%%%%%%%%%%%%%%%%%%%%%%%%%%%%%%%%%%
%%                           Class Description                           %%
%%%%%%%%%%%%%%%%%%%%%%%%%%%%%%%%%%%%%%%%%%%%%%%%%%%%%%%%%%%%%%%%%%%%%%%%%%%

\subsection{Class CondorDAGError}

    \label{pipeline:CondorDAGError}
\begin{tabular}{cccccccc}
% Line for exceptions.Exception, linespec=[0, 0]
\multicolumn{2}{r}{\settowidth{\BCL}{exceptions.Exception}\multirow{2}{\BCL}{exceptions.Exception}}
&&
&&
  \\\cline{3-3}
  &&\multicolumn{1}{c|}{}
&&
&&
  \\
% Line for pipeline.CondorError, linespec=[0]
\multicolumn{4}{r}{\settowidth{\BCL}{pipeline.CondorError}\multirow{2}{\BCL}{pipeline.CondorError}}
&&
  \\\cline{5-5}
  &&&&\multicolumn{1}{c|}{}
&&
  \\
&&&&\multicolumn{2}{l}{\textbf{CondorDAGError}}
\end{tabular}


%%%%%%%%%%%%%%%%%%%%%%%%%%%%%%%%%%%%%%%%%%%%%%%%%%%%%%%%%%%%%%%%%%%%%%%%%%%
%%                                Methods                                %%
%%%%%%%%%%%%%%%%%%%%%%%%%%%%%%%%%%%%%%%%%%%%%%%%%%%%%%%%%%%%%%%%%%%%%%%%%%%

  \subsubsection{Methods}

  \textbf{Inherited from Exception:}
    \_\_getitem\_\_,
    \_\_str\_\_
    \\
  \textbf{Inherited from CondorError:}
    \_\_init\_\_
    \index{pipeline \textit{(module)}!CondorDAGError \textit{(class)}|)}
    \index{pipeline \textit{(module)}!CondorDAGJob \textit{(class)}|(}

%%%%%%%%%%%%%%%%%%%%%%%%%%%%%%%%%%%%%%%%%%%%%%%%%%%%%%%%%%%%%%%%%%%%%%%%%%%
%%                           Class Description                           %%
%%%%%%%%%%%%%%%%%%%%%%%%%%%%%%%%%%%%%%%%%%%%%%%%%%%%%%%%%%%%%%%%%%%%%%%%%%%

\subsection{Class CondorDAGJob}

    \label{pipeline:CondorDAGJob}
\begin{tabular}{cccccc}
% Line for pipeline.CondorJob, linespec=[0]
\multicolumn{2}{r}{\settowidth{\BCL}{pipeline.CondorJob}\multirow{2}{\BCL}{pipeline.CondorJob}}
&&
  \\\cline{3-3}
  &&\multicolumn{1}{c|}{}
&&
  \\
&&\multicolumn{2}{l}{\textbf{CondorDAGJob}}
\end{tabular}

A Condor DAG job never notifies the user on completion and can have 
variable arguments that are set for a particular node in the DAG. 
Inherits methods from a CondorJob.


%%%%%%%%%%%%%%%%%%%%%%%%%%%%%%%%%%%%%%%%%%%%%%%%%%%%%%%%%%%%%%%%%%%%%%%%%%%
%%                                Methods                                %%
%%%%%%%%%%%%%%%%%%%%%%%%%%%%%%%%%%%%%%%%%%%%%%%%%%%%%%%%%%%%%%%%%%%%%%%%%%%

  \subsubsection{Methods}

    \label{pipeline:CondorDAGJob:__init__}
    \index{pipeline \textit{(module)}!CondorDAGJob \textit{(class)}!\_\_init\_\_ \textit{(method)}}
    \vspace{0.5ex}

    \noindent\begin{boxedminipage}{\textwidth}

    \raggedright \textbf{\_\_init\_\_}(\textit{self}, \textit{universe}, \textit{executable})

    \vspace{-1.5ex}

    \rule{\textwidth}{0.5\fboxrule}
    universe = the condor universe to run the job in. executable = the 
    executable to run in the DAG.

    \vspace{1ex}

      Overrides: pipeline.CondorJob.\_\_init\_\_

    \end{boxedminipage}

    \label{pipeline:CondorDAGJob:add_var_arg}
    \index{pipeline \textit{(module)}!CondorDAGJob \textit{(class)}!add\_var\_arg \textit{(method)}}
    \vspace{0.5ex}

    \noindent\begin{boxedminipage}{\textwidth}

    \raggedright \textbf{add\_var\_arg}(\textit{self}, \textit{arg})

    \vspace{-1.5ex}

    \rule{\textwidth}{0.5\fboxrule}
    Add a variable (or macro) option to the condor job. The option is 
    added to the submit file and a different argument to the option can 
    be set fot each node in the DAG. arg = name of option to add.

    \vspace{1ex}

    \end{boxedminipage}

  \textbf{Inherited from CondorJob:}
    add\_arg,
    add\_ini\_args,
    get\_stderr\_file,
    get\_stdout\_file,
    get\_sub\_file,
    set\_log\_file,
    set\_notifcation,
    set\_stderr\_file,
    set\_stdout\_file,
    set\_sub\_file,
    write\_sub\_file
    \index{pipeline \textit{(module)}!CondorDAGJob \textit{(class)}|)}
    \index{pipeline \textit{(module)}!CondorDAGNode \textit{(class)}|(}

%%%%%%%%%%%%%%%%%%%%%%%%%%%%%%%%%%%%%%%%%%%%%%%%%%%%%%%%%%%%%%%%%%%%%%%%%%%
%%                           Class Description                           %%
%%%%%%%%%%%%%%%%%%%%%%%%%%%%%%%%%%%%%%%%%%%%%%%%%%%%%%%%%%%%%%%%%%%%%%%%%%%

\subsection{Class CondorDAGNode}

    \label{pipeline:CondorDAGNode}
\textbf{Known Subclasses:} AnalysisNode

A CondorDAGNode represents a node in the DAG. It corresponds to a 
particular condor job (and so a particular submit file). If the job has 
variable (macro) arguments, they can be set here so each nodes executes 
with the correct arguments.


%%%%%%%%%%%%%%%%%%%%%%%%%%%%%%%%%%%%%%%%%%%%%%%%%%%%%%%%%%%%%%%%%%%%%%%%%%%
%%                                Methods                                %%
%%%%%%%%%%%%%%%%%%%%%%%%%%%%%%%%%%%%%%%%%%%%%%%%%%%%%%%%%%%%%%%%%%%%%%%%%%%

  \subsubsection{Methods}

    \label{pipeline:CondorDAGNode:__init__}
    \index{pipeline \textit{(module)}!CondorDAGNode \textit{(class)}!\_\_init\_\_ \textit{(method)}}
    \vspace{0.5ex}

    \noindent\begin{boxedminipage}{\textwidth}

    \raggedright \textbf{\_\_init\_\_}(\textit{self}, \textit{job})

    \vspace{-1.5ex}

    \rule{\textwidth}{0.5\fboxrule}
    job = the CondorJob that this node corresponds to.

    \vspace{1ex}

    \end{boxedminipage}

    \label{pipeline:CondorDAGNode:__repr__}
    \index{pipeline \textit{(module)}!CondorDAGNode \textit{(class)}!\_\_repr\_\_ \textit{(method)}}
    \vspace{0.5ex}

    \noindent\begin{boxedminipage}{\textwidth}

    \raggedright \textbf{\_\_repr\_\_}(\textit{self})

    \end{boxedminipage}

    \label{pipeline:CondorDAGNode:add_parent}
    \index{pipeline \textit{(module)}!CondorDAGNode \textit{(class)}!add\_parent \textit{(method)}}
    \vspace{0.5ex}

    \noindent\begin{boxedminipage}{\textwidth}

    \raggedright \textbf{add\_parent}(\textit{self}, \textit{node})

    \vspace{-1.5ex}

    \rule{\textwidth}{0.5\fboxrule}
    Add a parent to this node. This node will not be executed until the 
    parent node has run sucessfully. node = CondorDAGNode to add as a 
    parent.

    \vspace{1ex}

    \end{boxedminipage}

    \label{pipeline:CondorDAGNode:add_var}
    \index{pipeline \textit{(module)}!CondorDAGNode \textit{(class)}!add\_var \textit{(method)}}
    \vspace{0.5ex}

    \noindent\begin{boxedminipage}{\textwidth}

    \raggedright \textbf{add\_var}(\textit{self}, \textit{var}, \textit{value})

    \vspace{-1.5ex}

    \rule{\textwidth}{0.5\fboxrule}
    Add the a variable (macro) arguments for this node. If the option 
    specified does not exist in the CondorJob, it is added so the submit 
    file will be correct when written. var = option name. value = value 
    of the option for this node in the DAG.

    \vspace{1ex}

    \end{boxedminipage}

    \label{pipeline:CondorDAGNode:job}
    \index{pipeline \textit{(module)}!CondorDAGNode \textit{(class)}!job \textit{(method)}}
    \vspace{0.5ex}

    \noindent\begin{boxedminipage}{\textwidth}

    \raggedright \textbf{job}(\textit{self})

    \vspace{-1.5ex}

    \rule{\textwidth}{0.5\fboxrule}
    Return the CondorJob that this node is associated with.

    \vspace{1ex}

    \end{boxedminipage}

    \label{pipeline:CondorDAGNode:set_log_file}
    \index{pipeline \textit{(module)}!CondorDAGNode \textit{(class)}!set\_log\_file \textit{(method)}}
    \vspace{0.5ex}

    \noindent\begin{boxedminipage}{\textwidth}

    \raggedright \textbf{set\_log\_file}(\textit{self}, \textit{log})

    \vspace{-1.5ex}

    \rule{\textwidth}{0.5\fboxrule}
    Set the Condor log file to be used by this CondorJob. log = path of 
    Condor log file.

    \vspace{1ex}

    \end{boxedminipage}

    \label{pipeline:CondorDAGNode:set_name}
    \index{pipeline \textit{(module)}!CondorDAGNode \textit{(class)}!set\_name \textit{(method)}}
    \vspace{0.5ex}

    \noindent\begin{boxedminipage}{\textwidth}

    \raggedright \textbf{set\_name}(\textit{self})

    \vspace{-1.5ex}

    \rule{\textwidth}{0.5\fboxrule}
    Generate a unique name for this node in the DAG.

    \vspace{1ex}

    \end{boxedminipage}

    \label{pipeline:CondorDAGNode:set_retry}
    \index{pipeline \textit{(module)}!CondorDAGNode \textit{(class)}!set\_retry \textit{(method)}}
    \vspace{0.5ex}

    \noindent\begin{boxedminipage}{\textwidth}

    \raggedright \textbf{set\_retry}(\textit{self}, \textit{retry})

    \vspace{-1.5ex}

    \rule{\textwidth}{0.5\fboxrule}
    Set the number of times that this node in the DAG should retry. retry 
    = number of times to retry node.

    \vspace{1ex}

    \end{boxedminipage}

    \label{pipeline:CondorDAGNode:write}
    \index{pipeline \textit{(module)}!CondorDAGNode \textit{(class)}!write \textit{(method)}}
    \vspace{0.5ex}

    \noindent\begin{boxedminipage}{\textwidth}

    \raggedright \textbf{write}(\textit{self}, \textit{fh})

    \vspace{-1.5ex}

    \rule{\textwidth}{0.5\fboxrule}
    Write the DAG entry for this node to the DAG file descriptor. fh = 
    descriptor of open DAG file.

    \vspace{1ex}

    \end{boxedminipage}

    \index{pipeline \textit{(module)}!CondorDAGNode \textit{(class)}|)}
    \index{pipeline \textit{(module)}!CondorDAGNodeError \textit{(class)}|(}

%%%%%%%%%%%%%%%%%%%%%%%%%%%%%%%%%%%%%%%%%%%%%%%%%%%%%%%%%%%%%%%%%%%%%%%%%%%
%%                           Class Description                           %%
%%%%%%%%%%%%%%%%%%%%%%%%%%%%%%%%%%%%%%%%%%%%%%%%%%%%%%%%%%%%%%%%%%%%%%%%%%%

\subsection{Class CondorDAGNodeError}

    \label{pipeline:CondorDAGNodeError}
\begin{tabular}{cccccccc}
% Line for exceptions.Exception, linespec=[0, 0]
\multicolumn{2}{r}{\settowidth{\BCL}{exceptions.Exception}\multirow{2}{\BCL}{exceptions.Exception}}
&&
&&
  \\\cline{3-3}
  &&\multicolumn{1}{c|}{}
&&
&&
  \\
% Line for pipeline.CondorError, linespec=[0]
\multicolumn{4}{r}{\settowidth{\BCL}{pipeline.CondorError}\multirow{2}{\BCL}{pipeline.CondorError}}
&&
  \\\cline{5-5}
  &&&&\multicolumn{1}{c|}{}
&&
  \\
&&&&\multicolumn{2}{l}{\textbf{CondorDAGNodeError}}
\end{tabular}


%%%%%%%%%%%%%%%%%%%%%%%%%%%%%%%%%%%%%%%%%%%%%%%%%%%%%%%%%%%%%%%%%%%%%%%%%%%
%%                                Methods                                %%
%%%%%%%%%%%%%%%%%%%%%%%%%%%%%%%%%%%%%%%%%%%%%%%%%%%%%%%%%%%%%%%%%%%%%%%%%%%

  \subsubsection{Methods}

  \textbf{Inherited from Exception:}
    \_\_getitem\_\_,
    \_\_str\_\_
    \\
  \textbf{Inherited from CondorError:}
    \_\_init\_\_
    \index{pipeline \textit{(module)}!CondorDAGNodeError \textit{(class)}|)}
    \index{pipeline \textit{(module)}!CondorError \textit{(class)}|(}

%%%%%%%%%%%%%%%%%%%%%%%%%%%%%%%%%%%%%%%%%%%%%%%%%%%%%%%%%%%%%%%%%%%%%%%%%%%
%%                           Class Description                           %%
%%%%%%%%%%%%%%%%%%%%%%%%%%%%%%%%%%%%%%%%%%%%%%%%%%%%%%%%%%%%%%%%%%%%%%%%%%%

\subsection{Class CondorError}

    \label{pipeline:CondorError}
\begin{tabular}{cccccc}
% Line for exceptions.Exception, linespec=[0]
\multicolumn{2}{r}{\settowidth{\BCL}{exceptions.Exception}\multirow{2}{\BCL}{exceptions.Exception}}
&&
  \\\cline{3-3}
  &&\multicolumn{1}{c|}{}
&&
  \\
&&\multicolumn{2}{l}{\textbf{CondorError}}
\end{tabular}

\textbf{Known Subclasses:}
CondorDAGError,
    CondorDAGNodeError,
    CondorJobError,
    CondorSubmitError

Error thrown by Condor Jobs


%%%%%%%%%%%%%%%%%%%%%%%%%%%%%%%%%%%%%%%%%%%%%%%%%%%%%%%%%%%%%%%%%%%%%%%%%%%
%%                                Methods                                %%
%%%%%%%%%%%%%%%%%%%%%%%%%%%%%%%%%%%%%%%%%%%%%%%%%%%%%%%%%%%%%%%%%%%%%%%%%%%

  \subsubsection{Methods}

    \label{pipeline:CondorError:__init__}
    \index{pipeline \textit{(module)}!CondorError \textit{(class)}!\_\_init\_\_ \textit{(method)}}
    \vspace{0.5ex}

    \noindent\begin{boxedminipage}{\textwidth}

    \raggedright \textbf{\_\_init\_\_}(\textit{self}, \textit{args}=\texttt{N\-o\-n\-e\-})

      Overrides: exceptions.Exception.\_\_init\_\_

    \end{boxedminipage}

  \textbf{Inherited from Exception:}
    \_\_getitem\_\_,
    \_\_str\_\_
    \index{pipeline \textit{(module)}!CondorError \textit{(class)}|)}
    \index{pipeline \textit{(module)}!CondorJob \textit{(class)}|(}

%%%%%%%%%%%%%%%%%%%%%%%%%%%%%%%%%%%%%%%%%%%%%%%%%%%%%%%%%%%%%%%%%%%%%%%%%%%
%%                           Class Description                           %%
%%%%%%%%%%%%%%%%%%%%%%%%%%%%%%%%%%%%%%%%%%%%%%%%%%%%%%%%%%%%%%%%%%%%%%%%%%%

\subsection{Class CondorJob}

    \label{pipeline:CondorJob}
\textbf{Known Subclasses:} CondorDAGJob

Generic condor job class. Provides methods to set the options in the 
condor submit file for a particular executable


%%%%%%%%%%%%%%%%%%%%%%%%%%%%%%%%%%%%%%%%%%%%%%%%%%%%%%%%%%%%%%%%%%%%%%%%%%%
%%                                Methods                                %%
%%%%%%%%%%%%%%%%%%%%%%%%%%%%%%%%%%%%%%%%%%%%%%%%%%%%%%%%%%%%%%%%%%%%%%%%%%%

  \subsubsection{Methods}

    \label{pipeline:CondorJob:__init__}
    \index{pipeline \textit{(module)}!CondorJob \textit{(class)}!\_\_init\_\_ \textit{(method)}}
    \vspace{0.5ex}

    \noindent\begin{boxedminipage}{\textwidth}

    \raggedright \textbf{\_\_init\_\_}(\textit{self}, \textit{universe}, \textit{executable}, \textit{queue})

    \vspace{-1.5ex}

    \rule{\textwidth}{0.5\fboxrule}
    universe = the condor universe to run the job in. executable = the 
    executable to run. queue = number of jobs to queue.

    \vspace{1ex}

    \end{boxedminipage}

    \label{pipeline:CondorJob:add_arg}
    \index{pipeline \textit{(module)}!CondorJob \textit{(class)}!add\_arg \textit{(method)}}
    \vspace{0.5ex}

    \noindent\begin{boxedminipage}{\textwidth}

    \raggedright \textbf{add\_arg}(\textit{self}, \textit{arg}, \textit{value})

    \vspace{-1.5ex}

    \rule{\textwidth}{0.5\fboxrule}
    Add a command line argument to the executable. arg = command line 
    argument to add. value = value to pass to the argument (None for no 
    argument).

    \vspace{1ex}

    \end{boxedminipage}

    \label{pipeline:CondorJob:add_ini_args}
    \index{pipeline \textit{(module)}!CondorJob \textit{(class)}!add\_ini\_args \textit{(method)}}
    \vspace{0.5ex}

    \noindent\begin{boxedminipage}{\textwidth}

    \raggedright \textbf{add\_ini\_args}(\textit{self}, \textit{cp}, \textit{section})

    \vspace{-1.5ex}

    \rule{\textwidth}{0.5\fboxrule}
    Parse command line arguments from a given section in an ini file and 
    pass to the executable. cp = ConfigParser object pointing to the ini 
    file. section = section of the ini file to add to the arguments.

    \vspace{1ex}

    \end{boxedminipage}

    \label{pipeline:CondorJob:get_stderr_file}
    \index{pipeline \textit{(module)}!CondorJob \textit{(class)}!get\_stderr\_file \textit{(method)}}
    \vspace{0.5ex}

    \noindent\begin{boxedminipage}{\textwidth}

    \raggedright \textbf{get\_stderr\_file}(\textit{self})

    \vspace{-1.5ex}

    \rule{\textwidth}{0.5\fboxrule}
    Get the file to which Condor directs the stderr of the job.

    \vspace{1ex}

    \end{boxedminipage}

    \label{pipeline:CondorJob:get_stdout_file}
    \index{pipeline \textit{(module)}!CondorJob \textit{(class)}!get\_stdout\_file \textit{(method)}}
    \vspace{0.5ex}

    \noindent\begin{boxedminipage}{\textwidth}

    \raggedright \textbf{get\_stdout\_file}(\textit{self})

    \vspace{-1.5ex}

    \rule{\textwidth}{0.5\fboxrule}
    Get the file to which Condor directs the stdout of the job.

    \vspace{1ex}

    \end{boxedminipage}

    \label{pipeline:CondorJob:get_sub_file}
    \index{pipeline \textit{(module)}!CondorJob \textit{(class)}!get\_sub\_file \textit{(method)}}
    \vspace{0.5ex}

    \noindent\begin{boxedminipage}{\textwidth}

    \raggedright \textbf{get\_sub\_file}(\textit{self})

    \vspace{-1.5ex}

    \rule{\textwidth}{0.5\fboxrule}
    Get the name of the file which the Condor submit file will be written 
    to when write\_sub\_file() is called. path = path to submit file.

    \vspace{1ex}

    \end{boxedminipage}

    \label{pipeline:CondorJob:set_log_file}
    \index{pipeline \textit{(module)}!CondorJob \textit{(class)}!set\_log\_file \textit{(method)}}
    \vspace{0.5ex}

    \noindent\begin{boxedminipage}{\textwidth}

    \raggedright \textbf{set\_log\_file}(\textit{self}, \textit{path})

    \vspace{-1.5ex}

    \rule{\textwidth}{0.5\fboxrule}
    Set the Condor log file. path = path to log file.

    \vspace{1ex}

    \end{boxedminipage}

    \label{pipeline:CondorJob:set_notifcation}
    \index{pipeline \textit{(module)}!CondorJob \textit{(class)}!set\_notifcation \textit{(method)}}
    \vspace{0.5ex}

    \noindent\begin{boxedminipage}{\textwidth}

    \raggedright \textbf{set\_notifcation}(\textit{self}, \textit{value})

    \vspace{-1.5ex}

    \rule{\textwidth}{0.5\fboxrule}
    Set the email address to send notification to. value = email address 
    or never for no notification.

    \vspace{1ex}

    \end{boxedminipage}

    \label{pipeline:CondorJob:set_stderr_file}
    \index{pipeline \textit{(module)}!CondorJob \textit{(class)}!set\_stderr\_file \textit{(method)}}
    \vspace{0.5ex}

    \noindent\begin{boxedminipage}{\textwidth}

    \raggedright \textbf{set\_stderr\_file}(\textit{self}, \textit{path})

    \vspace{-1.5ex}

    \rule{\textwidth}{0.5\fboxrule}
    Set the file to which Condor directs the stderr of the job. path = 
    path to stderr file.

    \vspace{1ex}

    \end{boxedminipage}

    \label{pipeline:CondorJob:set_stdout_file}
    \index{pipeline \textit{(module)}!CondorJob \textit{(class)}!set\_stdout\_file \textit{(method)}}
    \vspace{0.5ex}

    \noindent\begin{boxedminipage}{\textwidth}

    \raggedright \textbf{set\_stdout\_file}(\textit{self}, \textit{path})

    \vspace{-1.5ex}

    \rule{\textwidth}{0.5\fboxrule}
    Set the file to which Condor directs the stdout of the job. path = 
    path to stdout file.

    \vspace{1ex}

    \end{boxedminipage}

    \label{pipeline:CondorJob:set_sub_file}
    \index{pipeline \textit{(module)}!CondorJob \textit{(class)}!set\_sub\_file \textit{(method)}}
    \vspace{0.5ex}

    \noindent\begin{boxedminipage}{\textwidth}

    \raggedright \textbf{set\_sub\_file}(\textit{self}, \textit{path})

    \vspace{-1.5ex}

    \rule{\textwidth}{0.5\fboxrule}
    Set the name of the file to write the Condor submit file to when 
    write\_sub\_file() is called. path = path to submit file.

    \vspace{1ex}

    \end{boxedminipage}

    \label{pipeline:CondorJob:write_sub_file}
    \index{pipeline \textit{(module)}!CondorJob \textit{(class)}!write\_sub\_file \textit{(method)}}
    \vspace{0.5ex}

    \noindent\begin{boxedminipage}{\textwidth}

    \raggedright \textbf{write\_sub\_file}(\textit{self})

    \vspace{-1.5ex}

    \rule{\textwidth}{0.5\fboxrule}
    Write a submit file for this Condor job.

    \vspace{1ex}

    \end{boxedminipage}

    \index{pipeline \textit{(module)}!CondorJob \textit{(class)}|)}
    \index{pipeline \textit{(module)}!CondorJobError \textit{(class)}|(}

%%%%%%%%%%%%%%%%%%%%%%%%%%%%%%%%%%%%%%%%%%%%%%%%%%%%%%%%%%%%%%%%%%%%%%%%%%%
%%                           Class Description                           %%
%%%%%%%%%%%%%%%%%%%%%%%%%%%%%%%%%%%%%%%%%%%%%%%%%%%%%%%%%%%%%%%%%%%%%%%%%%%

\subsection{Class CondorJobError}

    \label{pipeline:CondorJobError}
\begin{tabular}{cccccccc}
% Line for exceptions.Exception, linespec=[0, 0]
\multicolumn{2}{r}{\settowidth{\BCL}{exceptions.Exception}\multirow{2}{\BCL}{exceptions.Exception}}
&&
&&
  \\\cline{3-3}
  &&\multicolumn{1}{c|}{}
&&
&&
  \\
% Line for pipeline.CondorError, linespec=[0]
\multicolumn{4}{r}{\settowidth{\BCL}{pipeline.CondorError}\multirow{2}{\BCL}{pipeline.CondorError}}
&&
  \\\cline{5-5}
  &&&&\multicolumn{1}{c|}{}
&&
  \\
&&&&\multicolumn{2}{l}{\textbf{CondorJobError}}
\end{tabular}


%%%%%%%%%%%%%%%%%%%%%%%%%%%%%%%%%%%%%%%%%%%%%%%%%%%%%%%%%%%%%%%%%%%%%%%%%%%
%%                                Methods                                %%
%%%%%%%%%%%%%%%%%%%%%%%%%%%%%%%%%%%%%%%%%%%%%%%%%%%%%%%%%%%%%%%%%%%%%%%%%%%

  \subsubsection{Methods}

  \textbf{Inherited from Exception:}
    \_\_getitem\_\_,
    \_\_str\_\_
    \\
  \textbf{Inherited from CondorError:}
    \_\_init\_\_
    \index{pipeline \textit{(module)}!CondorJobError \textit{(class)}|)}
    \index{pipeline \textit{(module)}!CondorSubmitError \textit{(class)}|(}

%%%%%%%%%%%%%%%%%%%%%%%%%%%%%%%%%%%%%%%%%%%%%%%%%%%%%%%%%%%%%%%%%%%%%%%%%%%
%%                           Class Description                           %%
%%%%%%%%%%%%%%%%%%%%%%%%%%%%%%%%%%%%%%%%%%%%%%%%%%%%%%%%%%%%%%%%%%%%%%%%%%%

\subsection{Class CondorSubmitError}

    \label{pipeline:CondorSubmitError}
\begin{tabular}{cccccccc}
% Line for exceptions.Exception, linespec=[0, 0]
\multicolumn{2}{r}{\settowidth{\BCL}{exceptions.Exception}\multirow{2}{\BCL}{exceptions.Exception}}
&&
&&
  \\\cline{3-3}
  &&\multicolumn{1}{c|}{}
&&
&&
  \\
% Line for pipeline.CondorError, linespec=[0]
\multicolumn{4}{r}{\settowidth{\BCL}{pipeline.CondorError}\multirow{2}{\BCL}{pipeline.CondorError}}
&&
  \\\cline{5-5}
  &&&&\multicolumn{1}{c|}{}
&&
  \\
&&&&\multicolumn{2}{l}{\textbf{CondorSubmitError}}
\end{tabular}


%%%%%%%%%%%%%%%%%%%%%%%%%%%%%%%%%%%%%%%%%%%%%%%%%%%%%%%%%%%%%%%%%%%%%%%%%%%
%%                                Methods                                %%
%%%%%%%%%%%%%%%%%%%%%%%%%%%%%%%%%%%%%%%%%%%%%%%%%%%%%%%%%%%%%%%%%%%%%%%%%%%

  \subsubsection{Methods}

  \textbf{Inherited from Exception:}
    \_\_getitem\_\_,
    \_\_str\_\_
    \\
  \textbf{Inherited from CondorError:}
    \_\_init\_\_
    \index{pipeline \textit{(module)}!CondorSubmitError \textit{(class)}|)}
    \index{pipeline \textit{(module)}!ScienceData \textit{(class)}|(}

%%%%%%%%%%%%%%%%%%%%%%%%%%%%%%%%%%%%%%%%%%%%%%%%%%%%%%%%%%%%%%%%%%%%%%%%%%%
%%                           Class Description                           %%
%%%%%%%%%%%%%%%%%%%%%%%%%%%%%%%%%%%%%%%%%%%%%%%%%%%%%%%%%%%%%%%%%%%%%%%%%%%

\subsection{Class ScienceData}

    \label{pipeline:ScienceData}
An object that can contain all the science data used in an analysis. Can 
contain multiple ScienceSegments and has a method to generate these from 
a text file produces by the LIGOtools segwizard program.


%%%%%%%%%%%%%%%%%%%%%%%%%%%%%%%%%%%%%%%%%%%%%%%%%%%%%%%%%%%%%%%%%%%%%%%%%%%
%%                                Methods                                %%
%%%%%%%%%%%%%%%%%%%%%%%%%%%%%%%%%%%%%%%%%%%%%%%%%%%%%%%%%%%%%%%%%%%%%%%%%%%

  \subsubsection{Methods}

    \label{pipeline:ScienceData:__init__}
    \index{pipeline \textit{(module)}!ScienceData \textit{(class)}!\_\_init\_\_ \textit{(method)}}
    \vspace{0.5ex}

    \noindent\begin{boxedminipage}{\textwidth}

    \raggedright \textbf{\_\_init\_\_}(\textit{self})

    \end{boxedminipage}

    \label{pipeline:ScienceData:__getitem__}
    \index{pipeline \textit{(module)}!ScienceData \textit{(class)}!\_\_getitem\_\_ \textit{(method)}}
    \vspace{0.5ex}

    \noindent\begin{boxedminipage}{\textwidth}

    \raggedright \textbf{\_\_getitem\_\_}(\textit{self}, \textit{i})

    \vspace{-1.5ex}

    \rule{\textwidth}{0.5\fboxrule}
    Allows direct access to or iteration over the ScienceSegments 
    associated with the ScienceData.

    \vspace{1ex}

    \end{boxedminipage}

    \label{pipeline:ScienceData:__len__}
    \index{pipeline \textit{(module)}!ScienceData \textit{(class)}!\_\_len\_\_ \textit{(method)}}
    \vspace{0.5ex}

    \noindent\begin{boxedminipage}{\textwidth}

    \raggedright \textbf{\_\_len\_\_}(\textit{self})

    \vspace{-1.5ex}

    \rule{\textwidth}{0.5\fboxrule}
    Returns the number of ScienceSegments associated with the 
    ScienceData.

    \vspace{1ex}

    \end{boxedminipage}

    \label{pipeline:ScienceData:__repr__}
    \index{pipeline \textit{(module)}!ScienceData \textit{(class)}!\_\_repr\_\_ \textit{(method)}}
    \vspace{0.5ex}

    \noindent\begin{boxedminipage}{\textwidth}

    \raggedright \textbf{\_\_repr\_\_}(\textit{self})

    \end{boxedminipage}

    \label{pipeline:ScienceData:make_chunks}
    \index{pipeline \textit{(module)}!ScienceData \textit{(class)}!make\_chunks \textit{(method)}}
    \vspace{0.5ex}

    \noindent\begin{boxedminipage}{\textwidth}

    \raggedright \textbf{make\_chunks}(\textit{self}, \textit{length}, \textit{overlap}, \textit{play})

    \vspace{-1.5ex}

    \rule{\textwidth}{0.5\fboxrule}
    Divide each ScienceSegment contained in this object into 
    AnalysisChunks. length = length of chunk in seconds. overlap = 
    overlap between segments. play = if true, only generate chunks that 
    overlap with S2 playground data.

    \vspace{1ex}

    \end{boxedminipage}

    \label{pipeline:ScienceData:read}
    \index{pipeline \textit{(module)}!ScienceData \textit{(class)}!read \textit{(method)}}
    \vspace{0.5ex}

    \noindent\begin{boxedminipage}{\textwidth}

    \raggedright \textbf{read}(\textit{self}, \textit{file})

    \vspace{-1.5ex}

    \rule{\textwidth}{0.5\fboxrule}
    Parse the science segments from the segwizard output contained in 
    file. file = input text file containing a list of science segments 
    generated by segwizard.

    \vspace{1ex}

    \end{boxedminipage}

    \index{pipeline \textit{(module)}!ScienceData \textit{(class)}|)}
    \index{pipeline \textit{(module)}!ScienceSegment \textit{(class)}|(}

%%%%%%%%%%%%%%%%%%%%%%%%%%%%%%%%%%%%%%%%%%%%%%%%%%%%%%%%%%%%%%%%%%%%%%%%%%%
%%                           Class Description                           %%
%%%%%%%%%%%%%%%%%%%%%%%%%%%%%%%%%%%%%%%%%%%%%%%%%%%%%%%%%%%%%%%%%%%%%%%%%%%

\subsection{Class ScienceSegment}

    \label{pipeline:ScienceSegment}
A ScienceSegment is a period of time where the experimenters determine 
that the inteferometer is in a state where the data is suitable for 
scientific analysis. A science segment can have a list of AnalysisChunks 
asscociated with it that break the segment up into (possibly overlapping) 
smaller time intervals for analysis.


%%%%%%%%%%%%%%%%%%%%%%%%%%%%%%%%%%%%%%%%%%%%%%%%%%%%%%%%%%%%%%%%%%%%%%%%%%%
%%                                Methods                                %%
%%%%%%%%%%%%%%%%%%%%%%%%%%%%%%%%%%%%%%%%%%%%%%%%%%%%%%%%%%%%%%%%%%%%%%%%%%%

  \subsubsection{Methods}

    \label{pipeline:ScienceSegment:__init__}
    \index{pipeline \textit{(module)}!ScienceSegment \textit{(class)}!\_\_init\_\_ \textit{(method)}}
    \vspace{0.5ex}

    \noindent\begin{boxedminipage}{\textwidth}

    \raggedright \textbf{\_\_init\_\_}(\textit{self}, \textit{segment})

    \vspace{-1.5ex}

    \rule{\textwidth}{0.5\fboxrule}
    segemnt = a tuple containing the (segment id, gps start time, gps end 
    time, duration) of the segment.

    \vspace{1ex}

    \end{boxedminipage}

    \label{pipeline:ScienceSegment:__getitem__}
    \index{pipeline \textit{(module)}!ScienceSegment \textit{(class)}!\_\_getitem\_\_ \textit{(method)}}
    \vspace{0.5ex}

    \noindent\begin{boxedminipage}{\textwidth}

    \raggedright \textbf{\_\_getitem\_\_}(\textit{self}, \textit{i})

    \vspace{-1.5ex}

    \rule{\textwidth}{0.5\fboxrule}
    Allows iteration over and direct access to the AnalysisChunks 
    contained in this ScienceSegment.

    \vspace{1ex}

    \end{boxedminipage}

    \label{pipeline:ScienceSegment:__len__}
    \index{pipeline \textit{(module)}!ScienceSegment \textit{(class)}!\_\_len\_\_ \textit{(method)}}
    \vspace{0.5ex}

    \noindent\begin{boxedminipage}{\textwidth}

    \raggedright \textbf{\_\_len\_\_}(\textit{self})

    \vspace{-1.5ex}

    \rule{\textwidth}{0.5\fboxrule}
    Returns the number of AnalysisChunks contained in this 
    ScienceSegment.

    \vspace{1ex}

    \end{boxedminipage}

    \label{pipeline:ScienceSegment:__repr__}
    \index{pipeline \textit{(module)}!ScienceSegment \textit{(class)}!\_\_repr\_\_ \textit{(method)}}
    \vspace{0.5ex}

    \noindent\begin{boxedminipage}{\textwidth}

    \raggedright \textbf{\_\_repr\_\_}(\textit{self})

    \end{boxedminipage}

    \label{pipeline:ScienceSegment:add_chunk}
    \index{pipeline \textit{(module)}!ScienceSegment \textit{(class)}!add\_chunk \textit{(method)}}
    \vspace{0.5ex}

    \noindent\begin{boxedminipage}{\textwidth}

    \raggedright \textbf{add\_chunk}(\textit{self}, \textit{start}, \textit{end})

    \vspace{-1.5ex}

    \rule{\textwidth}{0.5\fboxrule}
    Add an AnalysisChunk to the list associated with this ScienceSegment. 
    start = GPS start time of chunk. end = GPS end time of chunk.

    \vspace{1ex}

    \end{boxedminipage}

    \label{pipeline:ScienceSegment:dur}
    \index{pipeline \textit{(module)}!ScienceSegment \textit{(class)}!dur \textit{(method)}}
    \vspace{0.5ex}

    \noindent\begin{boxedminipage}{\textwidth}

    \raggedright \textbf{dur}(\textit{self})

    \vspace{-1.5ex}

    \rule{\textwidth}{0.5\fboxrule}
    Returns the length (duration) in seconds of this ScienceSegment.

    \vspace{1ex}

    \end{boxedminipage}

    \label{pipeline:ScienceSegment:end}
    \index{pipeline \textit{(module)}!ScienceSegment \textit{(class)}!end \textit{(method)}}
    \vspace{0.5ex}

    \noindent\begin{boxedminipage}{\textwidth}

    \raggedright \textbf{end}(\textit{self})

    \vspace{-1.5ex}

    \rule{\textwidth}{0.5\fboxrule}
    Returns the GPS end time of this ScienceSegment.

    \vspace{1ex}

    \end{boxedminipage}

    \label{pipeline:ScienceSegment:id}
    \index{pipeline \textit{(module)}!ScienceSegment \textit{(class)}!id \textit{(method)}}
    \vspace{0.5ex}

    \noindent\begin{boxedminipage}{\textwidth}

    \raggedright \textbf{id}(\textit{self})

    \vspace{-1.5ex}

    \rule{\textwidth}{0.5\fboxrule}
    Returns the ID of this ScienceSegment.

    \vspace{1ex}

    \end{boxedminipage}

    \label{pipeline:ScienceSegment:make_chunks}
    \index{pipeline \textit{(module)}!ScienceSegment \textit{(class)}!make\_chunks \textit{(method)}}
    \vspace{0.5ex}

    \noindent\begin{boxedminipage}{\textwidth}

    \raggedright \textbf{make\_chunks}(\textit{self}, \textit{length}=\texttt{0\-}, \textit{overlap}=\texttt{0\-}, \textit{play}=\texttt{0\-})

    \vspace{-1.5ex}

    \rule{\textwidth}{0.5\fboxrule}
    Divides the science segment into chunks of length seconds overlapped 
    by overlap seconds. If the play option is set, only chunks that 
    contain S2 playground data are generated. If the user has a more 
    complicated way of generating chunks, this method should be overriden 
    in a sub-class. Any data at the end of the ScienceSegment that is too 
    short to contain a chunk is ignored. The length of this unused data 
    is stored and can be retrieved with the unused() method. length = 
    length of chunk in seconds. overlap = overlap between chunks in 
    seconds. play = only generate chunks that overlap with S2 playground 
    data.

    \vspace{1ex}

    \end{boxedminipage}

    \label{pipeline:ScienceSegment:start}
    \index{pipeline \textit{(module)}!ScienceSegment \textit{(class)}!start \textit{(method)}}
    \vspace{0.5ex}

    \noindent\begin{boxedminipage}{\textwidth}

    \raggedright \textbf{start}(\textit{self})

    \vspace{-1.5ex}

    \rule{\textwidth}{0.5\fboxrule}
    Returns the GPS start time of this ScienceSegment.

    \vspace{1ex}

    \end{boxedminipage}

    \label{pipeline:ScienceSegment:unused}
    \index{pipeline \textit{(module)}!ScienceSegment \textit{(class)}!unused \textit{(method)}}
    \vspace{0.5ex}

    \noindent\begin{boxedminipage}{\textwidth}

    \raggedright \textbf{unused}(\textit{self})

    \vspace{-1.5ex}

    \rule{\textwidth}{0.5\fboxrule}
    Returns the length of data in the science segment not used to make 
    chunks.

    \vspace{1ex}

    \end{boxedminipage}

    \index{pipeline \textit{(module)}!ScienceSegment \textit{(class)}|)}
    \index{pipeline \textit{(module)}!SegmentError \textit{(class)}|(}

%%%%%%%%%%%%%%%%%%%%%%%%%%%%%%%%%%%%%%%%%%%%%%%%%%%%%%%%%%%%%%%%%%%%%%%%%%%
%%                           Class Description                           %%
%%%%%%%%%%%%%%%%%%%%%%%%%%%%%%%%%%%%%%%%%%%%%%%%%%%%%%%%%%%%%%%%%%%%%%%%%%%

\subsection{Class SegmentError}

    \label{pipeline:SegmentError}
\begin{tabular}{cccccc}
% Line for exceptions.Exception, linespec=[0]
\multicolumn{2}{r}{\settowidth{\BCL}{exceptions.Exception}\multirow{2}{\BCL}{exceptions.Exception}}
&&
  \\\cline{3-3}
  &&\multicolumn{1}{c|}{}
&&
  \\
&&\multicolumn{2}{l}{\textbf{SegmentError}}
\end{tabular}


%%%%%%%%%%%%%%%%%%%%%%%%%%%%%%%%%%%%%%%%%%%%%%%%%%%%%%%%%%%%%%%%%%%%%%%%%%%
%%                                Methods                                %%
%%%%%%%%%%%%%%%%%%%%%%%%%%%%%%%%%%%%%%%%%%%%%%%%%%%%%%%%%%%%%%%%%%%%%%%%%%%

  \subsubsection{Methods}

    \label{pipeline:SegmentError:__init__}
    \index{pipeline \textit{(module)}!SegmentError \textit{(class)}!\_\_init\_\_ \textit{(method)}}
    \vspace{0.5ex}

    \noindent\begin{boxedminipage}{\textwidth}

    \raggedright \textbf{\_\_init\_\_}(\textit{self}, \textit{args}=\texttt{N\-o\-n\-e\-})

      Overrides: exceptions.Exception.\_\_init\_\_

    \end{boxedminipage}

  \textbf{Inherited from Exception:}
    \_\_getitem\_\_,
    \_\_str\_\_
    \index{pipeline \textit{(module)}!SegmentError \textit{(class)}|)}
    \index{pipeline \textit{(module)}|)}
