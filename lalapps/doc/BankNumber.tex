\subsection{Program \texttt{lalapps\_BankNumberOfTemplates}}
\label{program:lalapps-BankNumberOfTemplates}
\idx[Program]{lalapps\_BankNumberOfTemplates}
{\LARGE{{\bf Documentation In progress}}}

\begin{entry}

\item[Name]
\verb$lalapps_BankNumberOfTemplates$ --- Compute the number of templates given 
by a bank.

\item[Synopsis]
\verb$lalapps_BankNumberOfTemplates$ [\verb$-h$] 
[\verb$--alpha-range$   \textit{$\alpha_{Min}$}\texttt{\&}\textit{$\alpha_{Max}$}]
[\verb$--d-range$   \textit{step in $\alpha$}]
[\verb$--fend-bcv$     \textit{$f_{cut}Min$}\texttt{\&}\textit{$f_{cut}Max$}]
[\verb$--fl$	\textit{lower cutoff frequency}]
[\verb$--mm$ \textit{MinimalMatch}]
[\verb$--number-fcut$ \textit{number of layers } \textit{ in $f_{cut}$ dimension}]
[\verb$--psi0-range$ \textit{$\psi0_{Min}$}\texttt{\&}\textit{$\psi0_{Max}$}]
[\verb$--psi3-range$ \textit{$\psi3_{Min}$}\texttt{\&}\textit{$\psi3_{Max}$}]







\item[Description]
\verb$lalapps_BankNumberOfTemplates$ This application compute the number of 
templates in  a bank of templates based on the SPA bank or BCV bank.

To compute the number of templates one need the minimal match, the lower cutoff
frequency and a range of parameters such as masses or $\tau_i$ in the SPA case or 
$\psi_0$ and $\psi_3$ in the  BCV case. 

Concerning BCV, we also need the alpha parameter. 

In the BCV case, a range of alpha value is given [0-1] by default as well as a step
 (0.01 by default). Thus it provides an easy way to see the evolution of the number of
templates with respect to the $\alpha$ parameter. 

Later, we might add options to do the same but with a range in the lower frequency for
instance.


\item[Options related to the simulations]\leavevmode
\begin{entry}
\item[\texttt{--alpha-range}]
alpha parameter to compute the metric.
\item[\texttt{--dalpha}]
step in alpha parameter.
\item[\texttt{--fend-bcv}]
Range of cutoff frequency for the BCV template creation. User must provide two parameters: 
the lowest and upper frequency cutoff. Those frequency are not expressed in Hz but in term
of distance betwwen the two objets. for instance at the last stable orbit, the distance 
between the two objects is 6GM. In the case of an EOB description, the distance between 
the two objects is closer to 3GM. A good compromise is to use fend-bcv between 8GM and 3GM
and using the options --number-fcut = 5. 
\item[\texttt{-h}]
Print a help message.
\item[\texttt{--mm}]
Fixe the Minimal Match of the bank.
\item[\texttt{--n}]
Number of trials, i.e number of injections.
\item[\texttt{--number-fcut}]
Number of layers to create in the BCV bank in the fend-bcv dimension. 
Given the lower and upper frequency (see option fend-bcv), the bank creates
n layers each of them having a frequency between lowest fend-bcv and highest fend-bcv.
\item[\texttt{--psi0-range}]
Range of paramter psi0. User must give two positive numbers.
\item[\texttt{--psi3-range}]
Range of paramter psi3. User must give two negative numbers.
\item[\texttt{--seed}]
Random seed.
\item[\texttt{--space}]
When --bank is given we should also provide the space of parameter. In the case
of a BCV bank is it Psi0Psi3 and there is only one space. But in the SPA case on 
might provide either Tau0Tau3 or Tau0Tau2 option. 

\end{entry}


\item[EXAMPLE]
\begin{verbatim}
lalapps_BankNumberOfTemplates --dalpha 0.001 --alpha-range 0 1 
 --bank BCV

lalapps_BankNumberOfTemplates  --bank SPA --space Tau0Tau3
\end{verbatim}
\item[Author]
Thomas Cokelaer

\end{entry}
