\subsection{Program \texttt{lalapps\_InspiralInjection}}
\label{program:lalapps-InspiralInjection}
\idx[Program]{lalapps\_InspiralInjection}
{\LARGE{{\bf Documentation In progress}}}

\begin{entry}

\item[Name]
\verb$lalapps_InspiralInjection$ --- Test the inspiral injection waveforms.

\item[Synopsis]
\verb$lalapps_InspiralInjection$ [\verb$-h$] 

\item[Description]
\verb$lalapps_BankEfficiency$ This application first read an XML file which has been created
by \texttt{inspinj.c}. Basically this file contains the input parameters of one or more
 inspiral waveforms. Then, it creates the corresponding waveforms in the time domain using 
either inject package or inspiral package. And, finally the waveforms are printed in an output
 file namely "injection.dat". 

This application does not required any input parameters from the user. 




\item[Author]
Thomas Cokelaer

\end{entry}
