\section{Program \texttt{lalapps\_inspmultiawg}}
\label{program:lalapps-inspmultiawg}
\idx[Program]{lalapps\_inspmultiawg}

\begin{entry}
\item[Name]
\verb$lalapps_inspmultiawg$ --- injects specified inspiral chirps into zero
data.  Intended for producing the hardware injection data.

\item[Synopsis]
\begin{verbatim}
--help               display this message
--source SFILE       source file containing details of injection
--actuation ACTFILE  file containing the actuation function
--darm2inj DCFACTOR  calibration between darm_ctrl and injection point
--summary SUMFILE    write details of injections to file
--ifo IFO            name of interferomter (optional)
--flow FSTART        start frequency of injection (default 40 Hz)
--fhigh FSTOP        end frequency of injection (default: end at ISCO)
--smooth QFAC        ringdown the end of the injection with Q factor Qfac
--length LENGTH      length of the data (default 64 seconds)
--samplerate FREQ    rate at which data is sampled (default 16384Hz)
--debug-level DEBUG  the lal debug level
--user-tag TAG       user-tag added to output file names
\end{verbatim}

\item[Description] 
\verb$lalapps_inspmultiawg$ injects inspiral chirps into zero data.  The 
details of several chirps can be specified using the command \verb$--source$,  
otherwise, a single inspiral of two 1.4 solar mass neutron stars will be 
injected.  Each chirp is injected into a new file containing zero data of
$\texttt{LENGTH}$ seconds, sampled at $\texttt{FREQ}$ Hz, and each injection begins at the beginning of the data.  The actuation function can be provided in
$\texttt{ACTFILE}$.  Additionally, there is a parameter $\texttt{DCFACTOR}$ 
which is used to convert between the actuation function and the "response 
function" between ETM\_EXC and strain.  The chirp is output to a file
named
\begin{verbatim}
TAG_inspiral_NUMBER_IFO.out
\end{verbatim}
where $\texttt{TAG}$ is the user-tag, $\texttt{IFO}$ is the name of the 
interferometer, and $\texttt{NUMBER}$ is the injection number.  A summary of
the injections performed can be saved in $\texttt{SUMFILE}$.

\item[Options]\leavevmode
\begin{entry}
\item[\texttt{--sourcefile} \textsc{sfile}] Optional.  Reads source information 
from the file \textsc{sfile}.  If absent, it injects a single 
1.4$M_\odot$--1.4$M_\odot$ inspiral, optimally oriented, at a distance
of $1.0 \textrm{Mpc}$.

\item[\texttt{--actuation} \textsc{actfile}] Optional. Reads a detector
 actuation function from the file \textsc{actfile}.  If absent, it generates 
raw dimensionless strain.  The actuation function should contain 3 columns; frequency, followed by amplitude and phase. 


\item[\texttt{--darm2inj} \textsc{dcfactor}] Optional.  Multiplicative
factor used in calculating the response function from the actuation
function.  More specifically, the actuation function is given by 
%
\begin{equation}
   x_{C} = A(f) \times \mathrm{DARM}\_\mathrm{CRTL} \, .
\end{equation}
%
while we are taking the response function to be:
%
\begin{equation}
   h(f) = R(f) \times \mathrm{ETM}\_\mathrm{EXC} \, .
\end{equation}
%
Then, $\textsc{dcfactor}$ is given by
%
\begin{equation}
   A(f) = \mathrm{DCFACTOR} \times R(f) \, .
\end{equation}

\item[\texttt{--smooth} \textsc{qfac}]  Optional.  This will smooth out
the end of the injected inspiral.  After the inspiral, the frequency of
the waveform will be held constant while the amplitude is exponentially
decreased as $a = a_{\mathrm{final}} \exp(- \pi f_{\mathrm{final}} t /
\textsc{qfac})$.  Here, $t$ is the amount of time after the end of the
inspiral.

\item[\texttt{--summary} \textsc{sumfile}] Optional. The \textsc{sumfile} format
is LIGO lightweight XML with \texttt{process}, \texttt{process\_params} and
\texttt{sim\_inspiral} tables.  The \texttt{sim\_inspiral} table contains
details of all the injections performed.  The details of the injection are
obtained from the source file.

\item[\texttt{--length} \textsc{sec}] Optional.  Specify the length of data into
which the signal will be injected.  The default is 64 seconds.

\item[\texttt{--samplerate} \textsc{freq}] Optional.  Specify the rate at which
the data is sampled.  The default is 16384 Hz.

\item[\texttt{--ifo} \textsc{ifo}] Optional.  Give the name of the
interferometer for which the injections are intended.  This is only used in
naming the output files.

\item[\texttt{--user-tag} \textsc{tag}] Optional.  A user-tag which is
used in naming the output files.

\item[\texttt{--flow} \textsc{fstart}] Optional.  Give the start frequency
\textsc{fstart} for the inspiral.  The default is 40 Hz

\item[\texttt{--fhigh} \textsc{fstop}] Optional.  Give the end frequency
\textsc{fstop} for the inspiral.  The default behaviour is that the inspiral
will continue to ISCO.  If set to a negative number, the generator will use its
absolute value as the terminating frequency, but will ignore post-Newtonian
breakdown. 

\item[\texttt{--debug-level} \textsc{debug}]  Optional.  Set the LAL debug
level.  The default is 33.

\item[\texttt{--help}] Optional.  Print a help message.
\end{entry}

\paragraph{Format for \texttt{sourcefile}:} The source file consists
of any number of lines of data, each specifying a chirp waveform.
Each line must begin with a character code (\verb@CHAR@ equal to one
of \verb@'i'@, \verb@'f'@, or \verb@'c'@), followed by 6
whitespace-delimited numerical fields: the epoch of the chirp
(\verb@INT8@ seconds), the two binary masses (\verb@REAL4@
$M_\odot$), the distance to the source (\verb@REAL4@ Mpc), and the
source's inclination and phase at coalescence (\verb@REAL4@ degrees).
The character codes have the following meanings:
\begin{itemize}
\item[\texttt{'i'}] The epoch represents the GPS time of the start of
the chirp waveform.
\item[\texttt{'f'}] The epoch represents the GPS time of the end of
the chirp waveform.
\item[\texttt{'c'}] The epoch represents the GPS time when the
binaries would coalesce in the point-mass approximation.
\end{itemize}
Since the injection is started at time $t=0$, it is recommended that the
\texttt{'i'} option is used.

Thus a typical input line for two $1.4M_\odot$ objects at 1.1 Mpc
inclined $30^\circ$ with an initial phase of $45^\circ$, beginning at
70 seconds (after the start of the injections), will have the following line in the input
file:
\begin{verbatim}
i 70 1.4 1.4 1.1 30.0 45.0
\end{verbatim}
The time parameter (in this case 70 sec) does not affect the output data in any
way.  It is simply stored in the \texttt{sim\_inspiral} table of the
\textsc{sumfile}, in order to make analysis of the injections easier.

\paragraph{Format for \texttt{actfile}:} The actuation function $A(f)$
gives the transformation from DARM\_CTRL to $X_{C}$ via $X_{C} = A(f) 
\times \mathrm{DARM}\_\mathrm{CRTL}$.   It is combined with
\textsc{dcfactor} as desribed above to give the response function from
the exitation channel to strain as
$\tilde{h}(f)=R(f) \times \mathrm{ETM}_\mathrm{EXC}(f)$.  This is inverted internally 
to give the \emph{transfer function} $T(f)=1/R(f)$. 

The format for \verb@actfile@ is three columns of \verb@REAL4@ data giving the 
frequency followed by the amplitude and phase of the actuation function
at that frequency.  This is the format which is produced by the
calibration team.

\paragraph{Format for the data output:} The data output in the files
\verb$TAG_inspiral_NUMBER_IFO.txt$ is a single column of \verb$REAL4$ ADC data.

\item[Example]
\begin{verbatim}
lalapps_inspmultiawg --source s3.sources \
--actuation E10-H1-ACTUATION.txt --darm2inj 8000\ 
--summary summ.xml --ifo H1 --smooth 5
\end{verbatim}

\item[Author] 
Steve Fairhurst
\end{entry}

