\section{Program \texttt{lalapps\_coinext}}
\label{program:lalapps-coinext}
\idx[Program]{lalapps\_coinext}

\begin{entry}
\item[Name]
\verb$lalapps_coinext$ --- analyzing coincident external triggers online
\item[Synopsis]
\prog{coinExt} \newline 
[\option{--help}] \newline
[\option{--test}] \newline
[\option{--dirInspiral} \parm{dir\_insp}] \newline 
[\option{--dirData}     \parm{dir\_data}] \newline 
[\option{--refresh}     \parm{t\_refresh}] \newline
[\option{--timeWindow}  \parm{t\_window}]  \newline
[\option{--snrCut}      \parm{snr}]       \newline
[\option{--ifo}         \parm{name\_ifo}]  \newline
[\option{--trigger}     \parm{trigger\_file}] \newline
[\option{--restart}] \newline
[\option{--recalc}] \newline

\item[Description]
This online looks for coincidences on external and inspiral triggers, creating and transfering a XML file containing both data.\\

After a time \parm{t\_refresh} has elapsed the actual list of external triggers, placed on {\tt http://www.uoregon.edu/~ileonor/ligo/a4/grb/online/currentgrbs\_html\_A4.txt}, is downloaded. This file is updated when ever an external trigger from the GCN (The GRB Coordinates Network, see {\tt http://gcn.gsfc.nasa.gov/}) has been recieved. Since it takes some time from the observation of a Gamma-Ray Burst to the notification, about 10-20 minutes elapses until the trigger is placed into the downloaded file.\\


{\tt coinext} looks either for inspiral triggers specified in the directory \parm{dir\_insp} or it creates an own DAG for creating the inspiral triggers by itself (when setting the option \option{--recalc}). The DAG is run in the directory  {\tt OnlineAnalysis/Jobs} and the final inspiral triggers are put into {\tt Trigger}. \\

All inspiral triggers within a time window of \parm{t\_window} and the information of the external trigger are stored into a XML file in the directory {\tt Data} or \parm{dir\_data}. The XML file also is transfered to the machine {\tt adietz.phys.lsu.edu} for further analysis.\\

If the option \option{--restart} is specified, all intermediate daa files will be deleted, such as the local list of external triggers with the state of each trigger ({\it later more}).

\item[Options]\leavevmode
\begin{entry}

 
\item[\option{--dirData} \parm{dir\_data}]
Optional argument, specifying the directory where the resulting XML files are written to. The default directory is \option{CoinExt/Data}.

\item[\option{--dirInspiral} \parm{dir\_insp}]
Optional argument, specifying the directory searched for inspiral trigger XML-files. The default directory is \option{Triggers/Inspiral/???}

\item[\option{--help}]
Prints a help message and exits.

\item[\option{--refresh} \parm{t\_refresh}]
Setting the refresh rate to \parm{t\_refresh} minutes. This is the time the program waits until it checks for new external triggers. The default value is 30 minutes.

\item[\option{--test}]
With this option set the test mode is activated. The external triggers are read from a file {\tt CoinExt/ProgCoin/Test/triggers.xml???} and the refresh rate is set to 60 seconds.

\item[\option{--timeWindow} \parm{t\_window}]
Sets the coincident time window that is searched for coincidences.

\item[\option{--snrCut} \parm{cut}]
Sets a thresold for the selection of inspiral triggers. Only triggers with a SNR value greater than \parm{cut} are considered.

\item[\option{--ifo} \parm{ifo\_name}]
Specifies at which site the code it running. Possible parameters are {\tt LLO} and {\tt LHO}.

\item[\option{--trigger} \parm{trigger\_file}]
When specifying this option coinExt used always the trigger file {\tt trigger\_file} instead downloading a new one.

\item[\option{--restart}]
With this option the whole analysis on the triggers is restarted.

\item[\option{--recalc}]
With this option set for each GRb trigger an inspiral DAG will be run on the cluster to calculate the inspiral triggers. 

\end{entry}



\item[Example 1] 
{\it will follow soon}

\item[Notes]
This program requires {\tt tconvert} to be installed.


\item[Author] 
Alexander Dietz
\end{entry}


