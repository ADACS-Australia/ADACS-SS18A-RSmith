\section{Program \texttt{lalapps\_sire}}
\label{program:lalapps-sire}
\idx[Program]{lalapps\_sire}

\begin{entry}
\item[Name]
\verb$lalapps_sire$ --- single inspiral trigger reader and inspiral injections
analysis

\item[Synopsis]
\verb$lalapps_sire$ \quad
[\texttt{--help}]
[\texttt{--verbose}]
[\texttt{--debug-level} \textsc{level}] \newline
%
[\texttt{--user-tag} \textsc{usertag}] 
[\texttt{--comment} \textsc{comment}] \newline
%
(\texttt{--glob} \textsc{globfiles} $|$ \texttt{--input} \textsc{inputfile})
\texttt{--output} \textsc{outfile}
[\texttt{--summary-file} \textsc{summfile}] \newline
%
(\texttt{--playground-only} $|$ 
\texttt{--exclude-playground} $|$
\texttt{--all-data})\newline
%
[\texttt{--snr-threshold} \textsc{$\rho_\ast$}]\quad
[\texttt{--cluster-algorithm} \textsc{choice}
\texttt{--cluster-time} \textsc{$t_\mathrm{coinc}$}]\newline
%
[\texttt{--injection-file} \textsc{injfile} 
\texttt{--injection-coincidence} \textsc{$t_\mathrm{inj}$}\newline
\quad[\texttt{--missed-injections} \textsc{missedfile}]
[\texttt{--hardware-injections} \textsc{$t_\mathrm{hardware}$}]\newline
%
[\texttt{--tama-output} \textsc{tamafile}] \quad
[\texttt{--disable-trig-start-time}]

\item[Description] 
\verb$lalapps_sire$ processes the LIGO lightweight XML files produced by the
standalone inspiral analysis code \verb$lalapps_inspiral$ or the inspiral
coincidence analysis code \verb$lalapps_inca$. It can be used to concatenate
individual \verb$sngl_inspiral$ tables from multiple XML files which contain a
\verb$search_summary$ table into a single XML file. This may be performed with
or without clustering and signal-to-noise ratio cuts. It can also write a
summary file containing the number of triggers and the total time analyzed,
computed from the \verb$search_summary$ table.

The list of input files may be specified by either of POSIX system glob,
\textsc{globfiles}, or by giving the path to a text file, \textsc{inputfile},
that contains relative or absolute paths to the required input files.

If the \texttt{--injection-file} option is specified, \verb$lalapps_sire$ also
reads in a list of \verb$sim_inspiral$ rows from the file \textsc{injfile}.
It determines how may of the injections have a trigger coincident to within
$t_\mathrm{inj}$ milliseconds. The output file, \textsc{outfile}, will contain
\verb$sim_inspiral$ rows for all coincident (found) events. The summary file
will contain numbers of missed and found events and the efficiency of
detection of the injections.  Only injections that are within the input data
times are processed, so the injection file can span a time larger than the
input data and efficiencies will be correct. Missed injections can be written
to the file \textsc{missedfile}, if desired.

Note that the default LAL debug level for \verb$lalapps_sire$ is $33$, rather
than the usual $1$. This turns of memory allocation and leak checking to
increase the speed of run time. The \texttt{--debug-level} option can be used
to increase the level of debug checking used by the LAL functions.

\item[Options]\leavevmode
\begin{entry}
\item[\texttt{--help}] Optional. Print a help message to the standard output
and then exit with status code $0$.

\item[\texttt{--verbose}] Print progress information to the standard output 
while executing.

\item[\texttt{--debug-level} \textsc{level}] Optional. Set the LAL debug level
to \textsc{level}. Note that the default debug level is $33$ which includes
LAL error checking but excludes memory checking.  To enable memory checking
set the debug level to 1.

\item[\texttt{--user-tag} \textsc{usertag}] Optional. Add the user tag string 
\textsc{usertag} to the \verb$process_params$ table in the output XML file.

\item[\texttt{--comment} \textsc{comment}] Optional. Add the string
\textsc{comment} to the \verb$process$ table in the output XML file.

\item[\texttt{--glob} \textsc{globfiles}] Must be given if the
\texttt{--input} option is not used. Read the input triggers from the
LIGO lightweight XML files that match the regular expression
\textsc{globfiles}. The POSIX system call \verb$glob()$ is used to determine
which files are read in. Mutually exclusive with the \texttt{--input} option.

\item[\texttt{--input} \textsc{inputfile}] Must be given if the
\texttt{--glob} option is not used. Read the input triggers from the list of
LIGO lightweight XML files in \textsc{inputfile} which must be a plain text
file containing relative or absolute paths to the files.  Mutually exclusive
with the \texttt{--glob} option.

\item[\texttt{--output} \textsc{outfile}] Required. Write the concatenated
\verb$sngl_inspiral$ tables to the LIGO lightweight XML file \textsc{outfile}.
If injection analysis is performed the \verb$sim_inspiral$ rows from the input
injection file that are coincident with a trigger are also written to this
file (i.e. the found injections).

\item[\texttt{--playground-only}] Analyze \emph{only} triggers (and
injections) that \emph{are} in playground times specified by the post-S1
playground algorithm.

\item[\texttt{--exclude-playground}] Analyze \emph{only} triggers (and
injections) that \emph{are not} in playground times specified by the post-S1
playground algorithm.

\item[\texttt{--all-data}] Analyze \emph{all} triggers (and injections) from
the input files.

\item[\texttt{--snr-threshold} \textsc{$\rho_\ast$}] Optional. Discard all
input triggers that have a signal-to-noise ratio $\rho < \rho_\ast$.

\item[\texttt{--cluster-algorithm} \textsc{choice}] Optional. Use the
clustering algorithm \textsc{choice} to cluster the \verb$sngl_inspiral$ rows
in the output file before writing them to disk. The options for
\textsc{choice} are \verb$snr_and_chisq$, \verb$snrsq_over_chisq$ or
\verb$snr$. The clustering is performed by the LAL function
\verb$LALClusterSnglInspiralTable()$ and documentation for the clustering can
be found in the \texttt{tools} package of the LAL Software Documentation.

\item[\texttt{--cluster-time} \textsc{$t_\mathrm{coinc}$}] Required if the 
\texttt{--cluster-algorithm} option is specified. Use the time window
$t_\mathrm{coinc}$ for the clustering algorithm.

\item[\texttt{--injection-file} \textsc{injfile}] Optional. If this option is
given, \verb$lalapps_sire$ reads in \verb$sim_inspiral$ rows from the file
\textsc{injfile} and performs an injection alaysis of the triggers.

\item[\texttt{--injection-coincidence} \textsc{$t_\mathrm{inj}$}] Required if
the \texttt{--injection-file} option is specified. Set the injection
coincidence window to $\pm t_\mathrm{inj}$ milliseconds.

\item[\texttt{--missed-injections} \textsc{missedfile}] Optional, can only be
specified if \texttt{--injection-file} has been specified. If any injections
are \emph{not} found, write the \verb$sim_inspiral$ rows for these missed
injections to the LIGO lightweight file \textsc{missedfile}.

\item[\texttt{--hardware-injections} \textsc{$t_\mathrm{hardware}$}] Optional,
can only be specified if \texttt{--injection-file} has been specified.
Increment the end times of the injections read from \textsc{injfile} by
$t_\mathrm{hardware}$ seconds. Used for injection analysis of hardware
injections where the input \verb$sim_inspiral$ rows contain the time offset of
the injection from  $t_\mathrm{hardware}$.

\item[\texttt{--tama-output} \textsc{tamafile}] Optional, if specified produces
an output text file \textsc{tamafile} for use in collaboration with TAMA, in
addition to the usual LIGO lightweight XML file.  The following quantities are
recorded for each trigger in the text file:

\begin{itemize}
\item trigger time (as a double precision real)
\item total mass, $M_{\mathrm{TOT}}$
\item the mass ratio, $\eta$
\item the signal to noise ratio, $\rho$
\item the value of $\chi^2$
\item the effective distance to the trigger, $d_{eff}$.
\end{itemize}


\item[\texttt{--disable-trig-start-time}] Optional, should only be used by
maintainers. Disable checking of the \texttt{--trig-start-time} option in the
input files. \emph{Using this option may caused total analyzed times to be
reported incorrectly.} See note in algorithm section below.
\end{entry}

\item[Example 1] Read in all playground triggers files from the current
directory that match the expression 
\begin{verbatim}
L1-INSPIRAL_INJ-7*
\end{verbatim}
Discard all triggers below signal-to-noise ratio $10$ and report the number of
injections from file 
\begin{verbatim}
HL-INJECTIONS_45-729273613-5094000.xml
\end{verbatim}
that are coincident to within $20$ milliseconds with the remaining triggers.
Write an XML file containing the coincident triggers and injections, an XML
file containing the injections not coincident with a trigger and a text
summary file of the analysis, which will contain the total time analyzed and
the efficiency. Report the progress to the standard output and perform LAL
memory checking:
\begin{verbatim}
lalapps_sire \
--glob "L1-INSPIRAL_INJ-7*" --output out_10.xml --summary-file summ_10.txt \
--playground-only --verbose --debug-level 1 --snr-threshold 10.0 \
--injection-file HL-INJECTIONS_45-729273613-5094000.xml \
--injection-coincidence 20 --missed-injections missed_10.xml
\end{verbatim}

\item[Example 2] Read in all the XML files from the list in the plain text
file \verb$H1-INSPIRAL.txt$ and discard all the triggers that \emph{are} in
the playground. Write the remaining triggers to the XML file
\verb$H1-INSPIRAL.xml$ and write a text summary file containing the time
analyzed to \verb$H1-INSPIRAL_summary.txt$:
\begin{verbatim}
lalapps_sire \
--input H1-INSPIRAL.txt --exclude-playground \
--output H1-INSPIRAL.xml --summary-file H1-INSPIRAL_summary.txt
\end{verbatim}

\item[Notes]
\begin{enumerate}
\item The post-S1 playground algorithm is defined to be
\begin{equation}
t \ \textrm{is playground} \iff t - 729273613 < 600 (\textrm{mod}\ 6370).
\end{equation}

\item If a given trigger \verb$end_time,end_time_ns$, $t_\mathrm{trig}$ is
coincident to within $\pm t_\mathrm{inj}$ seconds of an injection \emph{site}
end time, given by \verb$h_end_time,h_end_time_ns$ or
\verb$l_end_time,l_end_time_ns$ then the injection is considered to be found
and the trigger coincident with an injection.

\item Early versions of the inspiral code contained a bug that causes the
\verb$out_start_time$ column of the \verb$search_summary$ table to be set
incorrectly if a non-zero \texttt{--trig-start-time} option is specified.
\verb$lalapps_sire$ corrects for this by checking for the value of
\texttt{--trig-start-time} in the \verb$process_params$ table and using it to
override the value of \verb$out_start_time$ in the \verb$search_summary$
table. To disable this behaviour, use the \texttt{--disable-trig-start-time}
option. Note that specifying this option may cause some analyzed data times to
be double counted and so the amount of analyzed data will be incorrectly
reported.
\end{enumerate}

\item[Author] 
Patrick Brady, Duncan Brown and Steve Fairhurst
\end{entry}


