\section{Program \texttt{lalapps\_trump}}
\label{program:lalapps-trump}
\idx[Program]{lalapps\_trump}

\begin{entry}

\item[Name]
\verb$lalapps_trump$ --- postprocess data generated by the inspiral
search code to determine an upper limit or perform detection. 

\item[Synopsis]
\verb$lalapps_trump$ \verb$--veto$ \textit{filename}
\verb$--trigger$ \textit{filename} \verb$--times$ \textit{filename}
\textit{ovlap} [\verb$--detection$] [\verb$--help$] 

\item[Description]
\verb$lalapps_trump$ manipulates the results of the inspiral search code
and veto data stored in XML format to produce lists of candidate
events for individual interferometers.   It can also be used to
perform a coincidence analysis using the output written for each
interferometer.  

\item[Options]\leavevmode
\begin{entry}
\item[\texttt{--help}]
Print a help message.
\item[\texttt{--veto} \textit{filename}]
This is required. 
Name of the file containing metadata about the vetos to be applied to
inspiral triggers.   This file can contain multiple lines.  Each line
is a semi-colon separated list of the form:
\begin{quote}
\textit{name};\textit{filename};\textit{column};\textit{threshold};\textit{minusdtime};\textit{plusdtime};
\end{quote}
Here \textit{name} is a character string;  \textit{filename} is a
character string naming the xml file with the veto triggers;
\textit{column} is the database column name to use for veto
construction;  \textit{threshold} is a number (float),  vetos have the
value in \textit{column} bigger than this;  \textit{minusdtime} and 
\textit{plusdtime} give the interval of time (in seconds) which should
be ignored before and after the veto trigger.
\item[\texttt{--trigger} \textit{filename}]
This is required. 
Name of the file containing metadata about the inspiral triggers and
simulated injections to determine the pipeline efficiency.    This file 
contains a single, semi-colon separated line:  
\begin{quote}
\textit{name};\textit{trigfile};\textit{injfile};\textit{snr*};\textit{chisq*};\textit{dtime};\textit{injlog};
\end{quote}
Here \textit{name} is a character string;  \textit{trigfile} is a
character string naming the xml file with the inspiral triggers;
\textit{injfile} is a character string naming the xml file with the
inspiral triggers when software injections are made; \textit{snr*} is
a number (float),  triggers have \textit{SNR} greater than this;
\textit{chisq*} is a number (float),  triggers have \textit{CHISQ}
less than this; \textit{dtime} is a number (float) giving the interval
of time (in seconds) between threshold crossings for clustering;
\textit{injlog} is a character string naming the ascii file with the
information about software injections.  
\item[\texttt{--detection}] Use this option if you want to run in
detection mode.  The code will terminate after generating a list of
triggers.  
\item[\texttt{--times} \textit{filename} \textit{ovlap}]
Name of the ascii file with the list of data chunks analyzed.   The
expected format is three columns:  (i) flag to indicate if it was
analyzed or not,  (ii) GPS start time, and (iii) GPS stop time.  The
floating point number \textit{ovlap} indicates how much data in
seconds is ignored in each
chunk.   The code assumes that \textit{ovlap}/2 is ignored at the
begining and end of each chunk.
\end{entry}


\item[Example]
To run the program,  type:
\begin{verbatim}
lalapps_trump --veto l1vetoes --trigger l1cand \ 
   --times segment_S1_play_03.out 32.0 --detection
\end{verbatim}
This command will look for information about vetoes in the file
\verb$l1vetoes$,  information about triggers in the file
\verb$l1cand$,  and information about the times analyzed in
\verb$segment_S1_play_03.out$ with an overlap of 32 seconds.  Since
detection is turned on,  a list of candidates is generated in a file
called \verb$triggers.dat$ in the direction from which you execute the
command. 

\item[Uses] This code uses LAL and the dataflow library.  The code in
\verb$event_utils.c$ includes some code written by Peter Shawhan to
access the xml files.   
\item[Author]
Patrick Brady 

\end{entry}
