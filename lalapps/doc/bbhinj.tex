\section{Program \texttt{lalapps\_bbhinj}}
\label{program:lalapps-bbhinj}
\idx[Program]{lalapps\_bbhinj}

\begin{entry}
\item[Name]
\verb$lalapps_bbhinj$ --- produces inspiral injection data files.

\item[Synopsis]
\verb$lalapps_bbhinj$ \newline
%
[\texttt{--help}]\newline
[\texttt{--verbose}]\newline
%
[\texttt{--gps-start-time} \textsc{tstart}]\newline 
[\texttt{--gps-end-time} \textsc{tend}]\newline
%
[\texttt{--time-step} \textsc{tstep}] \newline
[\texttt{--time-interval} \textsc{tinterval}]\newline
%
[\texttt{--seed} \textsc{seed}]\newline
[\texttt{--usertag} \textsc{tag}]\newline
%
[\texttt{--min-mass} \textsc{mmin}] \newline
[\texttt{--max-mass} \textsc{mmax}] \newline
[\texttt{--max-total-mass} \textsc{totalmass}] \newline
%
[\texttt{--min-distance} \textsc{dmin}] \newline
[\texttt{--max-distance} \textsc{dmax}] \newline
%
[\texttt{--m-distr} \textsc{mdistr}] \newline
[\texttt{--d-distr} \textsc{ddistr}] \newline
%
[\texttt{--waveform} \textsc{wvf}] \newline
%


\item[Description] 
\verb$lalapps_bbhinj$
generates a number of inspiral  parameters suitable  for using in a Monte
Carlo injection to test the efficiency of an inspiral search.  The  various
parameters (detailed  below)  are randomly chosen and are appropriate for
a population of binaries that extends over all space between
the minimum and maximum distances specified.
Despite its name, it can be used for BNS and for BBH parameter generation.

The output of this program  is  a  list  of  the  injected events,  starting
at  the specified start time and ending at the specified end time.  One 
injection with random inspiral parameters will be made every specified time
step, and will be randomly placed within the specified time interval.  
The output is written to a file name in the standard inspiral pipeline format:
\begin{center}
\begin{verbatim}
HL-INJECTIONS_USERTAG_SEED-GPSSTART-DURATION.xml
\end{verbatim}
\end{center}
where \verb$USERTAG$ is \textsc{tag} as specfied on the command line, 
\verb$SEED$ is the  value  of  the random number seed chosen and 
\verb$GPSSTART$ and \verb$DURATION$ describes the GPS time interval that
the file covers. The file is in the standard LIGO lightweight XML format
containing a \texttt{sim\_inspiral} table that describes the injections.
In addition, an ASCII log file called \verb$injlog.txt$ is also written.
If a \texttt{--user-tag} is not specified on the command line, the
\texttt{\_USERTAG} part of the filename will be omitted.

The GEO, TAMA and VIRGO end times and effective distances are currently
not being populated.

\item[Options]\leavevmode
\begin{entry}
\item[\texttt{--help}] Print a help message.

\item[\texttt{--gps-start-time} \textsc{tstart}]
Optional.  Start time of the injection data to be created. Defaults to the
start of S2, Feb 14 2003 16:00:00 UTC (GPS time 729273613)

\item[\texttt{--gps-end-time} \textsc{tend}]
Optional. End time of the injection data to be created. Defaults to the end of
S2, Apr 14 2003 15:00:00 UTC (GPS time 734367613).

\item[\texttt{--time-step} \textsc{tstep}]
Optional. Sets the time step interval between injections. The injections will
occur with an average spacing of \textsc{tstep} seconds. Defaults to 
$2630/\pi$.

\item[\texttt{--time-interval} \textsc{tinterval}]
Optional. Sets the time interval during which an injection can occur. 
Injections are uniformly distributed over the interval.  Setting \textsc{tstep}
to $6370$ and \textsc{tinterval} to 600 guarantees there will be one injection
into each playground segment and they will be randomly distributed within the
playground times.

\item[\texttt{--seed} \textsc{seed}]
Optional. Seed the random number generator with the integer \textsc{seed}.
Defaults to $1$.

\item[\texttt{--user-tag} \textsc{string}] Optional. Set the user tag for this
job to be \textsc{string}. May also be specified on the command line as 
\texttt{-userTag} for LIGO database compatibility.

\item[\texttt{--min-mass} \textsc{mmin}] Optional. The minimum component 
mass of the binary. If not specified, it is automatically set to 3 
$M_\odot$.

\item[\texttt{--max-mass} \textsc{mmax}] Optional. The maximum component 
mass of the binary. If not specified, it is automatically set to 20 
$M_\odot$.

\item[\texttt{-max-total-mass} \textsc{totalmass}] Optional. The maximum of the total masses of the two components. If not specified, no restriction is set to the total mass of the two components. 

\item[\texttt{--min-distance} \textsc{dmin}] Optional. The minimum distance
(in kpc) from the earth of the signals. If not specified, it is automatically 
set to 1 kpc.

\item[\texttt{--max-distance} \textsc{dmax}] Optional. The maximum distance
(in kpc) from the earth of the signals. If not specified, it is automatically 
set to 20 Mpc (20000 kpc).

\item[\texttt{--m-distr} \textsc{mdistr}] Optional. It lets the user specify
the mass distribution of the population. The choices are:
 \begin{itemize} 
 \item \textsc{mdistr} = 0 the distribution is uniform over total mass
 (DEFAULT VALUE)
 \item \textsc{mdistr} = 1 the distribution is uniform over mass1 and
    over mass2
 \end{itemize}

\item[\texttt{--d-distr} \textsc{ddistr}] Optional. It lets the used
specify the distance distribution of the population. The choices are:
 \begin{itemize} 
 \item \textsc{ddistr} = 0 uniform-distance distribution
 (DEFAULT VALUE)
 \item \textsc{ddistr} = 1 uniform $\log$(distance) distribution
 \item \textsc{ddistr} = 2 uniform volume distribution
 \end{itemize}

\item[\texttt{--waveform} \textsc{wvf}] Optional. Sets the waveforms to 
be injected to \textsc{wvf}. The \textsc{wvf} consists of two parts,
which are specified as one word. The first part is one of:
  \begin{itemize}
  \item EOB
  \item PadeT1
  \item TaylorT1
  \item TaylorT3
  \item GeneratePPN
  \end{itemize}
The second part is one of:
  \begin{itemize}
  \item newtonian
  \item onePN
  \item onePointFivePN
  \item twoPN (ONLY CHOICE if GeneratePPN is used before!)
  \item twoPointFivePN
  \item threePN
  \end{itemize}
If nothing is specified, the default value is EOBtwoPN. It should be noted 
that if GeneratePPNtwoPN is used as the waveform, the code used to
perform the injections is different than otherwise. For GeneratePPNtwoPN, the
older injection code (that does only standard post-Newtonian injections) is
used. That is the recommended approximant for the case of binary neutron
star injections.

\end{entry}

\item[Example:] 
The following command will produce an injection file for a population 
of binary black holes of total mass between 6 and 20 $M_\odot$ (component
mass between 3 and 10 $M_\odot$), with uniform-distance distribution
between 100 kpc and 3 Mpc. Since no mass-distribution is specified, the 
mass-distribution will be uniform over total mass (default value).
\begin{verbatim}
lalapps_bbhinj --min-mass 3.0 --max-mass 10.0 --min-distance 100 
--max-distance 3000 --d-distr 0 --waveform PadeT1twoPN
\end{verbatim}

\item[Author]
Jolien Creighton, Patrick Brady, Duncan Brown and Eirini Messaritaki
\end{entry}
