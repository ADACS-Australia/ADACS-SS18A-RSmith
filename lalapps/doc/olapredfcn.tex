
\section{Program \prog{lalapps\_olapredfcn}}
\label{program:lalapps-olapredfcn}
\idx[Program]{lalapps\_olapredfcn}

\begin{entry}

\item[Name]
\prog{lalapps\_olapredfcn} --- computes overlap reduction function given
a pair of known detectors.

\item[Synopsis]
\prog{lalapps\_olapredfcn} \newline \hspace*{0.5in}
[\option{-h}] \newline \hspace*{0.5in}
[\option{-q}] \newline \hspace*{0.5in}
[\option{-v}] \newline \hspace*{0.5in}
[\option{-d}~\parm{debugLevel}] \newline \hspace*{0.5in}
\option{-s}~\parm{siteID1} \newline \hspace*{0.5in}
[\option{-a}~\parm{azimuth1}] \newline \hspace*{0.5in}
\option{-t}~\parm{siteID2} \newline \hspace*{0.5in}
[\option{-b}~\parm{azimuth2}] \newline \hspace*{0.5in}
[\option{-f}~\parm{fLow}] \newline \hspace*{0.5in}
\option{-e}~\parm{deltaF} \newline \hspace*{0.5in}
\option{-n}~\parm{numPoints} \newline \hspace*{0.5in}
\option{-o}~\parm{outfile}

\item[Description]
\prog{lalapps\_olapredfcn} computes the overlap reduction function
$\gamma(f)$ for a pair of known gravitational wave detectors. It uses
the LAL function \texttt{LALOverlapReductionFunction()}, which is
documented in the LAL Software Documentation under the
\texttt{stochastic} package.

\item[Options]\leavevmode
\begin{entry}
\item[\option{-h}]
Print a help message.

\item[\option{-q}]
Run silently (redirect standard input and error to \texttt{/dev/null}).

\item[\option{-v}]
Run in verbose mode.

\item[\option{-d}~\parm{debugLevel}]
Set the LAL debug level to \parm{debugLevel}.

\item[\option{-s}~\parm{siteID1}, \option{-t}~\parm{siteID2}]
Use detector sites identified by \parm{siteID1} and \parm{siteID2}; ID
numbers between \texttt{LALNumCachedDetectors} (defined in the
\texttt{tools} package of LAL) refer to detectors cached in the constant
array \texttt{lalCachedDetectors[]}.  (At this point, these are all
interferometers.)  Additionally, the five resonant bar detectors of the
IGEC collaboration can be specified. The bar geometry data (summarized
in table~\ref{table:cachedBars}) is used by the function
\texttt{LALCreateDetector()} from the \texttt{tools} package of LAL to
generate the Cartesian position vector and response tensor which are
used to calculate the overlap reduction function.  The ID numbers for
the bars depend on the value of \texttt{LALNumCachedDetectors}; the
correct ID numbers can be obtained by with the command
\begin{verbatim}
> lalapps_olapredfcn -h
\end{verbatim}

\item[\option{-a}~\parm{azimuth1}, \option{-b}~\parm{azimuth2}]
If \parm{siteID1} (\parm{siteID2}) is a bar detector, assume it has an
azimuth of \parm{azimuth1} (\parm{azimuth2}) degrees East of North
rather than the default IGEC orientation given in
table~\ref{table:cachedBars}.  Note that this convention, measuring
azimuth in degrees clockwise from North is not the same as that used in
LAL (which comes from the frame spec).  Note also that any specified
azimuth angle is ignored if the corresponding detector is an
interferometer.

\item[\option{-f}~\parm{fLow}]
Begin the frequency series at a frequency of \parm{fLow}\,Hz; if this
is omitted, the default value of 0\,Hz is used.

\item[\option{-e}~\parm{deltaF}]
Construct the frequency series with a frequency spacing of
\parm{deltaF}\,Hz

\item[\option{-n}~\parm{numPoints}]
Construct a frequency series with \parm{numPoints} points.

\item[\option{-o}~\parm{outfile}]
Write the output to file \textit{outfile}. The format of this file is
that output by the routine \texttt{LALPrintFrequencySeries()} in the
\texttt{support} package of LAL, which consists of a header describing
metadata followed by two-column rows, each containing the doublet
$\{f,\gamma(f)\}$.
\end{entry}

\begin{table}[tbp]
\begin{center}
\begin{tabular}{|c|c|c|c|} \hline
Name & Longitude & Latitude & Azimuth \\\hline
AURIGA & $11^\circ56'54''$E & $45^\circ21'12''$N & N$44^\circ$E \\\hline
NAUTILUS & $12^\circ40'21''$E & $41^\circ49'26''$N & N$44^\circ$E \\\hline
EXPLORER & $6^\circ12'$E & $46^\circ27'$N & N$39^\circ$E \\\hline
ALLEGRO & $91^\circ10'43.\!\!''766$W & $30^\circ24'45.\!\!''110$N & N$40^\circ$W \\\hline
NIOBE & $115^\circ49'$E & $31^\circ56'$S & N$0^\circ$E \\\hline
\end{tabular}
\caption{Location and orientation data for the five IGEC resonant bar
detectors, stored in the \texttt{lalCachedBars[]} array. The data are
taken from
\texttt{http://igec.lnl.infn.it/cgi-bin/browser.pl?Level=0,3,1}
except for the latitude and longitude of ALLEGRO, which were taken from
Finn \& Lazzarini, gr-qc/0104040.  Note that the elevation above the
WGS-84 reference ellipsoid and altitude angle for each bar is not given,
and therefore set to zero.}
\label{table:cachedBars}
\end{center}
\end{table}

\item[Example]
To compute the overlap reduction function for LIGO Hanford and LIGO
Livingston, with a resolution of 1\,Hz from 0\,Hz to 1024\,Hz:

\begin{verbatim}
> lalapps_olapredfcn -s 0 -t 1 -e 1 -n 1025 -o LHOLLO.dat
\end{verbatim}
  
To compute the overlap reduction function for ALLEGRO in its optimal
orientation of $72.\!\!^\circ08$ West of South (see Finn \& Lazzarini,
gr-qc/0104040) and LIGO Livingston, with a resolution of 0.5\,Hz
from 782.5\,Hz to 1032\,Hz (assuming \texttt{lalNumCachedBars} is 6):

\begin{verbatim}
> lalapps_olapredfcn -s 9 -a 252.08 -t 1 -f 782.5 -e 0.5 -n 500 -o ALLEGROLHO.dat
\end{verbatim}

\item[Author]
John T.~Whelan

\end{entry}
