\subsection{Program \texttt{lalapps\_olapredfcn}}
\label{program:lalapps-olapredfcn}
\idx[Program]{lalapps\_olapredfcn}

\begin{entry}

\item[Name]
%
  \verb$lal_olapredfcn$ --- computes overlap reduction function given
  a pair of known detectors.

\item[Synopsis]
%
  \verb$lal_olapredfcn $[\verb$-h$]\verb$ $[\verb$-q$]\verb$ $[\verb$-v$]
  \verb$ $[\verb$-d debugLevel $]\verb+ \+\newline
  \verb$   $
  \verb$-s siteID1 $[\verb$-a azimuth1$]
  \verb$-t siteID2 $[\verb$-b azimuth2$]\verb+ \+\newline
  \verb$   $
  [\verb$-f fLow$]\verb$ -e deltaF$\verb$ -n numPoints$\verb$ -o outfile$
                         
\item[Description]
%
  \verb$lal_olapredfcn$ computes the overlap reduction function
  $\gamma(f)$ for a pair of known gravitational wave detectors.  It
  uses the LAL function \verb$LALOverlapReductionFunction()$, which is
  documented in the LAL Software Documentation under the
  \texttt{stochastic} package.

\item[Options]\leavevmode
\begin{entry}
\item[\texttt{-h}]
  Print a help message.
\item[\texttt{-q}]
  Run silently (redirect standard input and error to \texttt{/dev/null}).
\item[\texttt{-v}]
  Run in verbose mode.
\item[\texttt{-d} \textit{debugLevel}]
  Set the LAL debug level to \textit{debugLevel}.
\item[\texttt{-s} \textit{siteID1} \texttt{-t} \textit{siteID2}]
  Use detector sites identified by \textit{siteID1} and
  \textit{siteID2}; ID numbers between \texttt{LALNumCachedDetectors}
  (defined in the \texttt{tools} package of LAL) refer to detectors
  cached in the constant array \verb$lalCachedDetectors[]$.  (At this
  point, these are all interferometers.)  Additionally, the five
  resonant bar detectors of the IGEC collaboration can be specified.
  The bar geometry data (summarized in table~\ref{table:cachedBars})
  is used by the fucntion \verb$LALCreateDetector()$ from the
  \texttt{tools} package of LAL to generate the Cartesian position
  vector and response tensor which are used to calculate the overlap
  reduction function.  The ID numbers for the bars depend on the value
  of \texttt{LALNumCachedDetectors}; the correct ID numbers can be
  obtained by with the command
\begin{verbatim}
./lalapps_olapredfcn -h
\end{verbatim}
\item[\texttt{-a} \textit{azimuth1} \texttt{-b} \textit{azimuth2}]
%
  If \textit{siteID1} (\textit{siteID2}) is a bar detector, assume it
  has an azimuth of \textit{azimuth1} (\textit{azimuth2}) degrees East
  of North rather than the default IGEC orientation given in
  table~\ref{table:cachedBars}.  Note that this convention, measuring
  azimuth in degrees clockwise from North is not the same as that used
  in LAL (which comes from the frame spec).  Note also that any
  specified azimuth angle is ignored if the corresponding detector is
  an interferometer.
\item[\texttt{-f} \textit{fLow}]
  Begin the frequency series at a frequency of \textit{fLow}\,Hz; if this
  is omitted, the default value of 0\,Hz is used.
\item[\texttt{-e} \textit{deltaF}]
  Construct the frequency series with a frequency spacing of
  \textit{deltaF}\,Hz
\item[\texttt{-n} \textit{numPoints}]
  Construct a frequency series with \textit{numPoints} points.
\item[\texttt{-o} \textit{outfile}]
  Write the output to file \textit{outfile}.  The format of this file
  is that output by the routine \verb$LALPrintFrequencySeries()$ in
  the \texttt{support} package of LAL, which consists of a header
  describing metadata followed by two-column rows, each containing the
  doublet $\{f,\gamma(f)\}$.
\end{entry}

\begin{table}[tbp]
  \begin{center}
    \begin{tabular}{|c|c|c|c|}
\hline
      Name & Longitude & Latitude & Azimuth
\\ \hline
\verb$AURIGA$ & $11^\circ56'54''$E & $45^\circ21'12''$N & N$44^\circ$E 
\\ \hline
\verb$NAUTILUS$ & $12^\circ40'21''$E & $41^\circ49'26''$N & N$44^\circ$E 
\\ \hline
\verb$EXPLORER$ & $6^\circ12'$E & $46^\circ27'$N & N$39^\circ$E 
\\ \hline
\verb$ALLEGRO$ & $91^\circ10'43.\!\!''766$W & $30^\circ24'45.\!\!''110$N 
& N$40^\circ$W
\\ \hline
\verb$NIOBE$ & $115^\circ49'$E & $31^\circ56'$S & N$0^\circ$E 
\\ \hline
    \end{tabular}
    \caption{Location and orientation data for the five IGEC resonant
      bar detectors, stored in the \texttt{lalCachedBars[]}
      array.  The data are taken from
      \texttt{http://igec.lnl.infn.it/cgi-bin/browser.pl?Level=0,3,1}
      except for the latitude and longitude of ALLEGRO, which were
      taken from Finn \& Lazzarini, gr-qc/0104040.  Note that the
      elevation above the WGS-84 reference ellipsoid and altitude
      angle for each bar is not given, and therefore set to zero.}
    \label{table:cachedBars}
  \end{center}
\end{table}


\item[Example usage]
  To compute the overlap reduction function for LIGO Hanford and
  LIGO Livingston, with a resolution of 1\,Hz from 0\,Hz to 1024\,Hz:
\begin{verbatim}
lalapps_olapredfcn -s 0 -t 1 -e 1 -n 1025 -o LHOLLO.dat
\end{verbatim}
  
  To compute the overlap reduction function for ALLEGRO in its optimal
  orientation of $72.\!\!^\circ08$ West of South (see Finn \& Lazzarini,
  gr-qc/0104040) and LIGO Livingston, with a resolution of 0.5\,Hz from
  782.5\,Hz to 1032\,Hz (assuming \texttt{lalNumCachedBars} is 6):
\begin{verbatim}
lalapps_olapredfcn -s 9 -a 252.08 -t 1 -f 782.5 -e 0.5 -n 500 -o ALLEGROLHO.dat
\end{verbatim}

\item[Author]
John T.~Whelan

\end{entry}
