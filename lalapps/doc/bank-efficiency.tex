\subsection{Program \texttt{lalapps\_BankEfficiency}}
\label{program:lalapps-BankEfficiency}
\idx[Program]{lalapps\_BankEfficiency}
{\LARGE{{\bf Documentation In progress}}}

\begin{entry}

\item[Name]
\verb$lalapps_BankEfficiency$ --- produces Monte Carlo simulation to test bank efficiency
in the framework of inspiral binaries. It is dedicated to BCV bank validation.

\item[Synopsis]
\verb$lalapps_BankEfficiency$ [\verb$-h$] 
[\verb$-d$ \textit{dbglvl}]
[\verb$--psi0-range$ \textit{psi0 min}\texttt{-}\textit{psi0 max}]
[\verb$--psi3-range$ \textit{psi3 min}\texttt{-}\textit{psi3 max}]
[\verb$-mm$ \textit{Minimal Match}]

[\verb$-r$ \textit{range}]
[\verb$-s$ \textit{seed}]
[\verb$-t$ \textit{tstart}\texttt{-}\textit{tend}]
[\verb$-u$ \textit{tstep}]

\item[Description]
\verb$lalapps_BankEfficiency$ This application create a bank of templates based on the
SPA bank or BCV bank. Then, it performs a Monte Carlo Simulation by injecting known signals
such as TaylorT1, PadeT1 or EOB and computing the overlap between the signal and the 
bank of templates.
 
It is dedicated to BCV overlap tough  classical overlap in quadrature and SPA bank might
be used as well. 

In the case of BCV bank, the user have to provide the following parameters: the minimal 
match $MM$, the alpha parameter (0.01 by default), a range of psi0-psi3 and possibly
some number of layers for the bank generation. 

For the time being the Bank code is provided by the bank package through the function
\texttt{LALCreateCoarseBank}. If the option "template" is BCV then the bank will be a BCV bank
using the fucntion \texttt{LALInspiralBCVFcutBank} (see bnak package for details).



\item[Options related to the simulations]\leavevmode
\begin{entry}
\item[\texttt{--alpha-bank}]
alpha parameter to compute the metric.
\item[\texttt{--alpha-signal}]
alpha parameter for an injected BCV waveform.
\item[\texttt{--fend-bcv}]
Range of cutoff frequency for the BCV template creation. User must provide two parameters: 
the lowest and upper frequency cutoff. Those frequency are not expressed in Hz but in term
of distance betwwen the two objets. for instance at the last stable orbit, the distance 
between the two objects is 6GM. In the case of an EOB description, the distance between 
the two objects is closer to 3GM. A good compromise is to use fend-bcv between 8GM and 3GM
and using the options --number-fcut = 5. 
\item[\texttt{-h}]
Print a help message.
\item[\texttt{--mm}]
Fixe the Minimal Match of the bank.
\item[\texttt{--n}]
Number of trials, i.e number of injections.
\item[\texttt{--number-fcut}]
Number of layers to create in the BCV bank in the fend-bcv dimension. 
Given the lower and upper frequency (see option fend-bcv), the bank creates
n layers each of them having a frequency between lowest fend-bcv and highest fend-bcv.
\item[\texttt{--psi0-range}]
Range of paramter psi0. User must give two positive numbers.
\item[\texttt{--psi3-range}]
Range of paramter psi3. User must give two negative numbers.
\item[\texttt{-seed}]
Random seed.
\end{entry}

\item[Options related to output]\leavevmode
\begin{entry}
\item[\texttt{--print-bank}]
Print the bank of templates on stdout. If a BCV-template bank is used then the ouput is composed of
5 columns: psi0, psi3, number of layer, total mass and final frequency. If it is a SPA bank
then 4 columns are print: tau0, tau3, and corresponding mass1 and mass2. The number of templates is 
also printed.

\end{entry}


For example, the command
\begin{indented}
\verb$lalapps_BankEfficiency -d "ERROR | INFO"$
\end{indented}
will set the debug level so that error and information messages are printed.

\item[Environment]\leavevmode

\begin{entry}
\item[\texttt{LAL\_DEBUG\_LEVEL}]
Default LAL debug level to use.
\end{entry}

\item[Author]
Thomas Cokelaer

\end{entry}
