\section{Power Tools}
\label{section:powertools}

This section of \textsc{LALApps} contains programs that can be used to
perform burst searches using the excess power algorithm.   

Put a summary of the excess power algorithm here .........

\clearpage


\subsection{Program \texttt{lalapps\_power}}
\label{program:lalapps-power}
\idx[Program]{lalapps\_power}

\begin{entry}

\item[Name]
\verb$lalapps_power$ --- runs excess power code on chunks of (simulated) data.

\item[Synopsis]
\verb$lalapps_power$ \verb$--npts$ \textsc{npts} \verb$--nseg$ \textsc{nseg}
\verb$--olap$ \textsc{olap} \verb$--olapfctr$ \textsc{olapfctr}
$\backslash$ \newline \hspace*{0.25in}
\verb$--minfbin$ \textsc{minfbin} \verb$--mintbin$ \textsc{mintbin} 
\verb$--flow$ \textsc{flow} \verb$--delf$ \textsc{delf} 
\verb$--lngth$ \textsc{lngth} 
$\backslash$ \newline \hspace*{0.25in}
\verb$--nsigma$ \textsc{nsigma}
\verb$--alphdef$ \textsc{alphdef}  \verb$--segdcle$ \textsc{segdcle} 
\verb$--threshold$ \textsc{threshold} 
$\backslash$ \newline \hspace*{0.25in}
\verb$--etomstr$ \textsc{etomstr}
\verb$--channel$ \textsc{channel} \verb$--simtype$ \textsc{simtype} 
\verb$--spectype$ \textsc{spectype} 
$\backslash$ \newline \hspace*{0.25in}
\verb$--window$ \textsc{window} \verb$--epoch$ \textsc{sec} \textsc{nsec} 
\verb$--numpts$ \textsc{numpts} [\verb$--printSpectrum$] 
$\backslash$ \newline \hspace*{0.25in}
[\verb$--noise$ \textsc{var} \textsc{seed}] [\verb$--outfile$ \textsc{outfile}]

\item[Description] 
\verb$lal_power$ runs the excess power code from LAL on a chunk of
real or simulated data.    Consider searching  for signals with the
following properties:
\begin{itemize}
\item Maximum signal time duration $T=2^a$ seconds where $a$ is a
positive or negative integer;  the sampling rate of the data stream is
taken assummed $\mbox{\texttt{srate}} = 2^b$ Hz.

\item The frequency band of the signal is between $f_{\mathrm{low}}$
Hz and ${f_{\mathrm{high}}}$ Hz.  Current versions of the code
expect ${f_{\mathrm{high}}}-{f_{\mathrm{low}}}=2^d$ Hz where
$d$ is an integer. 

\item Minimum time duration,

\item Minimum frequency bandwidth.
\end{itemize}

\item[Options]\leavevmode
\begin{entry}
\item[\texttt{--npts}  \textsc{npts}] Number of data points in a
segment is determined by
\[
\mbox{\textsc{npts}} = (T \times \mbox{\textsc{srate}}) \; .
\]  

\item[\texttt{--nseg} \textsc{nseg}] Number of overlapping segments into
which data should be divided for filtering;  must be an integer. 

\item[\texttt{--olap} \textsc{olap}] Number of points overlap between
segments.  This is an argument for completeness,  but in general it
should be $\textsc{npts}/2$.

\item[\texttt{--olapfctr} \textsc{olapfctr}] Amount of overlap between
neighboring TF tiles; must be an integer.  A reasonable value for this
parameter is $3$.  See LAL \texttt{burstsearch} package for details.  

\item[\texttt{--minfbin} \textsc{minfbin}] Smallest extent in frequency of
TF tiles to search;  must be an integer.  A reasonable value for this
parameter is $2$.   The product $\mbox{\textsc{minfbin}} \times
\mbox{\textsc{mintbin}}$ is the minimum time-frequency volume to
be searched.  See LAL \texttt{burstsearch} package for details.

\item[\texttt{--mintbin} \textsc{mintbin}] Smallest extent in time of TF
tiles to search;  must be an integer. A reasonable value for this
parameter is $2$.   The product $\mbox{\textsc{minfbin}} \times
\mbox{\textsc{mintbin}}$ is the minimum time-frequency volume to
be searched.  See LAL \texttt{burstsearch} package for details.

\item[\texttt{--flow} \textsc{flow}] Lowest frequency in Hz to be searched;
a real number.  This is obviously $f_{\mathrm{low}}$ Hz from our
description of the desired signal parameters above.

\item[\texttt{--delf} \textsc{delf}] This input should be set to $1/T$ Hz but
it is ignored by the current version of the code;
a real number.

\item[\texttt{--lngth} \textsc{lngth}] may be determined by the following
formula
\[
\mbox{\textsc{lngth}} = T \times ({f_{\mathrm{high}} - f_{\mathrm{low}}}) .
\]
This is an integer which determines the maximum frequency bandwidth
over which a signal is expected.  

\item[\texttt{--nsigma} \textsc{nsigma}] threshold number of sigma;
a real number.
Currently see LAL \texttt{burstsearch} package for details.

\item[\texttt{--alphdef} --\textsc{alphdef}] default alpha value for tiles
with sigma < numSigmaMin;
a real number. Currently see LAL \texttt{burstsearch}
package for details.

\item[\texttt{--segdcle} \textsc{segdcle}] Number of segments analyzed
at a time;  must be an integer.  The current code uses
\textsc{segdcle} overlapping segments to compute the average (or
median) power spectral estimate for use inside the code.

\item[\texttt{--threshold} \textsc{threshold}] Identify events with alpha
less than this;
a real number.  Currently see LAL \texttt{burstsearch} package for
details.

\item[\texttt{--etomstr} \textsc{etomstr}] Number of events to be
accepted from a search over \textsc{npts} of data;  must be an
integer.

\item[\texttt{--channel} \textsc{channel}] The name used to identify the
data to be analyzed;  a character string matching
the channel name in the frame files,  e.g. \texttt{H2:LSC-AS\_Q}.

\item[\texttt{--simtype} \textsc{simtype}] Type of simulation. Set it to 0,
although it is ignored in the current version of the code.

\item[\texttt{--spectype} \textsc{spectype}] Spectrum estimator for whitening
data;  a character string [\texttt{useMean}, \texttt{useMedian}].

\item[\texttt{--window} \textsc{window}] Type of window to use on the data;
must be an integer.  [Possible values are:  0=Rectangular, 1=Hann,
2=Welch, 3=Bartlett, 4=Parzen, 5=Papoulis, 6=Hamming.]
\end{entry}


\item[Example]
To run the program,  type:
\begin{verbatim}
lalapps_power --npts 2048 --nseg 719 --olap 1024 --olapfctr 3 \
    --minfbin 2 --mintbin 2 --flow 100.0 --delf 1.0 --lngth 512 --nsigma 2.0 \
    --alphdef 0.5 --segdcle 32 --threshold 1.0e-13 --etomstr 10 \
    --channel S1:LSC-AS_Q --simtype 0 --spectype useMedian --window 2 \
    --epoch  693768279 0 --numpts 1048576 --printSpectrum --noise 1 11 \
    --outfile sim-inj.xml
\end{verbatim}

\item[Author]
Patrick Brady
\end{entry}


%%%%%%%%%%%%%%%%%%%%%%%%%%%%%%%%%%%%%%%%%%%%%%%%%%%%%%%%%%%%%%%%%%%%%%%%%%%%%%
% manipulate the events from power code
%%%%%%%%%%%%%%%%%%%%%%%%%%%%%%%%%%%%%%%%%%%%%%%%%%%%%%%%%%%%%%%%%%%%%%%%%%%%%%
\subsection{Program \texttt{lalapps\_snglBurstHistogram}}
\label{program:lalapps-snglBurstHistogram}
\idx[Program]{lalapps\_snglBurstHistogram}

\begin{entry}

\item[Name]
\verb$lalapps_snglBurstHistogram$ --- runs over LIGO lightweight files and histograms
the triggers by frequency.   

\item[Synopsis]
\verb$lalapps_snglBurstHistogram$ \verb$--input$ \textsc{infile} 
[\verb$--threshold$ \textsc{threshold}]    
$\backslash$ \newline \hspace*{0.25in}
\verb$--freq$ \textsc{fstart} \textsc{fstop} \textsc{df} 
[\verb$--help$]

\item[Description] 
\verb$lal_snglBurstHistogram$ reads in LIGO lightweight files and
constructs a histogram of the triggers over frequency.   The code
should be extensible to do a number of similar things like this.

\item[Options]\leavevmode
\begin{entry}
\item[\texttt{--input}  \textsc{infile}] The name of a file containing
a list of XML files to parse;  one XML file per line of
\textsc{infile}.

\item[\texttt{--threshold} \textsc{threshold}] Identify events with alpha
less than this; a real number.  The meaning of the threshold is the same as
\verb$lalapps_power$ itself.  (Optional)

\item[\texttt{--freq} \textsc{fstart} \textsc{fstop} \textsc{df}]
Parameters which define a frequency histogram.  \textsc{fstart} is the
lowest frequency;  \textsc{fstop} is the highest frequency; \textsc{df} 
is the width of abin.

\item[\texttt{--help}] Print usage instructions.
\end{entry}


\item[Example]
To run the program,  type:
\begin{verbatim}
lalapps_power --npts 2048 --nseg 719 --olap 1024 --olapfctr 3 \
    --minfbin 2 --mintbin 2 --flow 100.0 --delf 1.0 --lngth 512 --nsigma 2.0 \
    --alphdef 0.5 --segdcle 32 --threshold 1.0e-13 --etomstr 10 \
    --channel S1:LSC-AS_Q --simtype 0 --spectype useMedian --window 2 \
    --epoch  693768279 0 --numpts 1048576 --printSpectrum --noise 1 11 \
    --outfile sim-inj.xml
\end{verbatim}

\item[Author]
Patrick Brady

\end{entry}
