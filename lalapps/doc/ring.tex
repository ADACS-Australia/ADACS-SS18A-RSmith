\chapter{Ringdown Search Code}
\label{chapter:ringdown}

\section{Program \texttt{lalapps\_ring}}
\label{program:lalapps-ring}
\idx[Program]{lalapps\_ring}

\begin{entry}

\item[Name]
\verb$lalapps_ring$ --- filters data through a bank of ringdown filters.

\item[Synopsis]
\verb$lalapps_ring$ [\verb$-h$] [\verb$-V$] [\verb$-v$]
  [\verb$-d$ \textit{dbglvl}] [\verb$-i$ \textit{infile}]
  [\verb$-o$ \textit{outfile}] [\verb$-k$] [\verb$-s$]
  [\verb$-f$ \textit{framefile}] [\verb$-r$ \textit{respfile}]
  [\verb$-c$ \textit{channel}] [\verb$-n$ \textit{numpoints}]
  [\verb$-t$ \textit{starttime}] [\verb$-b$ \textit{b0}[,\textit{b1}]]
  [\verb$--$ \textit{filterparams}]

\item[Description]
\verb$lalapps_ring$ uses matched filtering to search for ringdown waveforms in
gravitational wave data.

\item[Options]\leavevmode
\begin{entry}
\item[\texttt{-h}]
Print a help message.
\item[\texttt{-V}]
Print the version information.
\item[\texttt{-v}]
Verbose output.
\item[\texttt{-d} \textit{dbglvl}]
Set LAL debug level to \textit{dbglvl}.
\item[\texttt{-i} \textit{infile}]
Read  filter  parameters  from  input  file \textit{infile} [stdin].
\item[\texttt{-o} \textit{outfile}]
Write the output to file \textit{outfile} [stdout].
\item[\texttt{-k}]
Keep filtering results (for use with option \texttt{-s}).
\item[\texttt{-s}]
Save  intermediate  filtering  results  as .dat and snr- files.
\item[\texttt{-f} \textit{framefile}]
Read channel data from \textit{framefile} [$\ast$.gwf].
\item[\texttt{-r} \textit{respfile}]
Read   response   function   data   from  \textit{respfile} [response.asc].
\item[\texttt{-c} \textit{channel}]
Use channel \textit{channel} data [H1:LSC-AS\_Q].
\item[\texttt{-n} \textit{numpoints}]
Use \textit{numpoints} points of data [65536].
\item[\texttt{-t} \textit{starttime}]
Use  data  starting at GPS time \textit{starttime} [start of frame data].
\item[\texttt{-b} \textit{b0},\textit{b1}]
Filter only template numbers \textit{b0} to \textit{b1} in bank [0,end of bank].
\item[\texttt{--} \textit{filterparams}]
Specify filter parameters as command line arguments \textit{filterparams}
(see below for filter parameters).
\end{entry}

\item[Filter parameters]\leavevmode
The filter parameters can be specified either on the  command  line  as
arguments  following the \texttt{--} option or in a resource file that is input
using the \texttt{-i} option  (or  from stdin).  As a resource file, each
option-value pair should have their own line.
\begin{entry}
\item[\texttt{-segsz} \textit{npts}]
Set the size of segments analyzed to \textit{npts} points.
\item[\texttt{-speclen} \textit{len}]
Set the size of inverse spectrum truncation to \textit{len} points [0].
\item[\texttt{-flow} \textit{flow}]
Set the low frequency cutoff to \textit{flow} Hz.
\item[\texttt{-fmin} \textit{fmin}]
Set the minimum frequency for the bank to \textit{fmin} Hz.
\item[\texttt{-fmax} \textit{fmax}]
Set the maximum frequency for the bank to \textit{fmax} Hz.
\item[\texttt{-qmin} \textit{qmin}]
Set the minimum quality for the bank to \textit{qmin}.
\item[\texttt{-qmax} \textit{qmax}]
Set the maximum quality for the bank to \textit{qmax}.
\item[\texttt{-maxmm} \textit{maxmm}]
Set  the  maximum  allowed mismatch for the bank to \textit{maxmm}.
\item[\texttt{-thresh} \textit{thresh}]
Set the ringdown event signal-to-noise ratio threshold to \textit{thresh}.
\item[\texttt{-scale} \textit{scale}]
Scale the response function by a dynamic range factor of \textit{scale} [1].
\end{entry}

\item[Debug levels]
The LAL debug level can be specified as an integer or as a string of flags:
\begin{entry}
\item[\texttt{NDEBUG}]
No debugging information is printed and memory debugging code is disabled.
\item[\texttt{ERROR}]
Error messages are printed.
\item[\texttt{WARNING}]
Warning messages are printed.
\item[\texttt{INFO}]
Information messages are printed.
\item[\texttt{TRACE}]
Function call tracing messages are printed.
\item[\texttt{MEMINFO}]
Memory  allocation  information messages are printed.
\item[\texttt{MEMDBG}]
Debugging of memory allocation routines is enabled but no messages are printed.
\end{entry}
The following composite levels are available:
\begin{entry}
\item[\texttt{MSGLVL1}]
Equivalent to \verb$ERROR$
\item[\texttt{MSGLVL2}]
Equivalent to \verb$ERROR | WARNING$
\item[\texttt{MSGLVL3}]
Equivalent to \verb$ERROR | WARNING | INFO$
\item[\texttt{ALLDBG}]
All debugging messages are printed.
\end{entry}

For example, the command
\begin{indented}
\verb$lalapps_ring -d "ERROR | INFO"$ ...
\end{indented}
will set the debug level so that error and information messages are printed.

\item[Environment]\leavevmode

\begin{entry}
\item[\texttt{LAL\_DEBUG\_LEVEL}]
Default LAL debug level to use.
\end{entry}

\item[Author]
Jolien Creighton

\end{entry}
