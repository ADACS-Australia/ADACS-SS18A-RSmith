%
% API Documentation
% Module inspiral
%
% Generated by epydoc 2.0
% [Thu Sep 25 14:02:41 2003]
%

%%%%%%%%%%%%%%%%%%%%%%%%%%%%%%%%%%%%%%%%%%%%%%%%%%%%%%%%%%%%%%%%%%%%%%%%%%%
%%                          Module Description                           %%
%%%%%%%%%%%%%%%%%%%%%%%%%%%%%%%%%%%%%%%%%%%%%%%%%%%%%%%%%%%%%%%%%%%%%%%%%%%

    \index{inspiral \textit{(module)}|(}
\section{Python Module \texttt{inspiral}}

    \label{inspiral}
Classes needed for the inspiral analysis pipeline. This script produced 
the necessary condor submit and dag files to run the standalone inspiral 
code on LIGO data


%%%%%%%%%%%%%%%%%%%%%%%%%%%%%%%%%%%%%%%%%%%%%%%%%%%%%%%%%%%%%%%%%%%%%%%%%%%
%%                               Variables                               %%
%%%%%%%%%%%%%%%%%%%%%%%%%%%%%%%%%%%%%%%%%%%%%%%%%%%%%%%%%%%%%%%%%%%%%%%%%%%

  \subsection{Variables}

\begin{longtable}{|p{.30\textwidth}|p{.62\textwidth}|l}
\cline{1-2}
\cline{1-2} \centering \textbf{Name} & \centering \textbf{Description}& \\
\cline{1-2}
\endhead\cline{1-2}\multicolumn{3}{r}{\small\textit{continued on next page}}\\\endfoot\cline{1-2}
\endlastfoot\raggedright \_\-\_\-a\-u\-t\-h\-o\-r\-\_\-\_\- & \raggedright \textbf{Value:} 
{\tt '\-D\-u\-n\-c\-a\-n\-~\-B\-r\-o\-w\-n\-~\-{\textless}\-d\-u\-n\-c\-a\-n\-@\-g\-r\-a\-v\-i\-t\-y\-.\-p\-h\-y\-s\-.\-u\-w\-m\-.\-e\-d\-u\-{\textgreater}\-'\-}&\\
\cline{1-2}
\raggedright \_\-\_\-d\-a\-t\-e\-\_\-\_\- & \raggedright \textbf{Value:} 
{\tt '\-\$\-D\-a\-t\-e\-:\-~\-2\-0\-0\-3\-/\-1\-0\-/\-0\-1\-~\-0\-9\-:\-0\-5\-:\-0\-8\-~\-\$\-'\-}&\\
\cline{1-2}
\raggedright \_\-\_\-v\-e\-r\-s\-i\-o\-n\-\_\-\_\- & \raggedright \textbf{Value:} 
{\tt '\-1\-.\-1\-4\-'\-}&\\
\cline{1-2}
\end{longtable}

    \index{inspiral \textit{(module)}!DataFindJob \textit{(class)}|(}

%%%%%%%%%%%%%%%%%%%%%%%%%%%%%%%%%%%%%%%%%%%%%%%%%%%%%%%%%%%%%%%%%%%%%%%%%%%
%%                           Class Description                           %%
%%%%%%%%%%%%%%%%%%%%%%%%%%%%%%%%%%%%%%%%%%%%%%%%%%%%%%%%%%%%%%%%%%%%%%%%%%%

\subsection{Class DataFindJob}

    \label{inspiral:DataFindJob}
\begin{tabular}{cccccccc}
% Line for pipeline.AnalysisJob, linespec=[0]
\multicolumn{4}{r}{\settowidth{\BCL}{pipeline.AnalysisJob}\multirow{2}{\BCL}{pipeline.AnalysisJob}}
&&
  \\\cline{5-5}
  &&&&\multicolumn{1}{c|}{}
&&
  \\
% Line for pipeline.CondorJob, linespec=[0, 1]
\multicolumn{2}{r}{\settowidth{\BCL}{pipeline.CondorJob}\multirow{2}{\BCL}{pipeline.CondorJob}}
&&
&&\multicolumn{1}{|c}{}
  \\\cline{3-3}
  &&\multicolumn{1}{c|}{}
&&
&\multicolumn{1}{|c}{}&
  \\
% Line for pipeline.CondorDAGJob, linespec=[1]
\multicolumn{4}{r}{\settowidth{\BCL}{pipeline.CondorDAGJob}\multirow{2}{\BCL}{pipeline.CondorDAGJob}}
&&\multicolumn{1}{|c}{}
  \\\cline{5-5}
  &&&&\multicolumn{1}{c|}{}
&\multicolumn{1}{|c}{}&
  \\
&&&&\multicolumn{2}{l}{\textbf{DataFindJob}}
\end{tabular}

A LALdataFind job used by the inspiral pipeline. The static options are 
read from the section [datafind] in the ini file. The stdout from 
LALdataFind contains the paths to the frame files and is directed to a 
file in the cache directory named by site and GPS start and end times. 
The stderr is directed to the logs directory. The job always runs in the 
scheduler universe. The path to the executable is determined from the ini 
file.


%%%%%%%%%%%%%%%%%%%%%%%%%%%%%%%%%%%%%%%%%%%%%%%%%%%%%%%%%%%%%%%%%%%%%%%%%%%
%%                                Methods                                %%
%%%%%%%%%%%%%%%%%%%%%%%%%%%%%%%%%%%%%%%%%%%%%%%%%%%%%%%%%%%%%%%%%%%%%%%%%%%

  \subsubsection{Methods}

    \label{inspiral:DataFindJob:__init__}
    \index{inspiral \textit{(module)}!DataFindJob \textit{(class)}!\_\_init\_\_ \textit{(method)}}
    \vspace{0.5ex}

    \noindent\begin{boxedminipage}{\textwidth}

    \raggedright \textbf{\_\_init\_\_}(\textit{self}, \textit{cp})

    \vspace{-1.5ex}

    \rule{\textwidth}{0.5\fboxrule}
    cp = ConfigParser object from which options are read.

    \vspace{1ex}

      Overrides: pipeline.CondorDAGJob.\_\_init\_\_

    \end{boxedminipage}

  \noindent\textbf{Inherited from AnalysisJob:}
    calibration,
    channel,
    get\_config
    \\
  \noindent\textbf{Inherited from CondorDAGJob:}
    add\_var\_arg,
    add\_var\_opt
    \\
  \noindent\textbf{Inherited from CondorJob:}
    add\_arg,
    add\_condor\_cmd,
    add\_ini\_opts,
    add\_opt,
    get\_stderr\_file,
    get\_stdout\_file,
    get\_sub\_file,
    set\_log\_file,
    set\_notification,
    set\_stderr\_file,
    set\_stdout\_file,
    set\_sub\_file,
    write\_sub\_file
    \index{inspiral \textit{(module)}!DataFindJob \textit{(class)}|)}
    \index{inspiral \textit{(module)}!DataFindNode \textit{(class)}|(}

%%%%%%%%%%%%%%%%%%%%%%%%%%%%%%%%%%%%%%%%%%%%%%%%%%%%%%%%%%%%%%%%%%%%%%%%%%%
%%                           Class Description                           %%
%%%%%%%%%%%%%%%%%%%%%%%%%%%%%%%%%%%%%%%%%%%%%%%%%%%%%%%%%%%%%%%%%%%%%%%%%%%

\subsection{Class DataFindNode}

    \label{inspiral:DataFindNode}
\begin{tabular}{cccccccc}
% Line for pipeline.CondorDAGNode, linespec=[0, 0]
\multicolumn{2}{r}{\settowidth{\BCL}{pipeline.CondorDAGNode}\multirow{2}{\BCL}{pipeline.CondorDAGNode}}
&&
&&
  \\\cline{3-3}
  &&\multicolumn{1}{c|}{}
&&
&&
  \\
% Line for pipeline.AnalysisNode, linespec=[0]
\multicolumn{4}{r}{\settowidth{\BCL}{pipeline.AnalysisNode}\multirow{2}{\BCL}{pipeline.AnalysisNode}}
&&
  \\\cline{5-5}
  &&&&\multicolumn{1}{c|}{}
&&
  \\
% Line for pipeline.CondorDAGNode, linespec=[1]
\multicolumn{4}{r}{\settowidth{\BCL}{pipeline.CondorDAGNode}\multirow{2}{\BCL}{pipeline.CondorDAGNode}}
&&\multicolumn{1}{|c}{}
  \\\cline{5-5}
  &&&&\multicolumn{1}{c|}{}
&\multicolumn{1}{|c}{}&
  \\
&&&&\multicolumn{2}{l}{\textbf{DataFindNode}}
\end{tabular}

A DataFindNode runs an instance of datafind in a Condor DAG.


%%%%%%%%%%%%%%%%%%%%%%%%%%%%%%%%%%%%%%%%%%%%%%%%%%%%%%%%%%%%%%%%%%%%%%%%%%%
%%                                Methods                                %%
%%%%%%%%%%%%%%%%%%%%%%%%%%%%%%%%%%%%%%%%%%%%%%%%%%%%%%%%%%%%%%%%%%%%%%%%%%%

  \subsubsection{Methods}

    \label{inspiral:DataFindNode:__init__}
    \index{inspiral \textit{(module)}!DataFindNode \textit{(class)}!\_\_init\_\_ \textit{(method)}}
    \vspace{0.5ex}

    \noindent\begin{boxedminipage}{\textwidth}

    \raggedright \textbf{\_\_init\_\_}(\textit{self}, \textit{job})

    \vspace{-1.5ex}

    \rule{\textwidth}{0.5\fboxrule}
    job = A CondorDAGJob that can run an instance of LALdataFind.

    \vspace{1ex}

      Overrides: pipeline.AnalysisNode.\_\_init\_\_

    \end{boxedminipage}

    \label{inspiral:DataFindNode:get_output}
    \index{inspiral \textit{(module)}!DataFindNode \textit{(class)}!get\_output \textit{(method)}}
    \vspace{0.5ex}

    \noindent\begin{boxedminipage}{\textwidth}

    \raggedright \textbf{get\_output}(\textit{self})

    \vspace{-1.5ex}

    \rule{\textwidth}{0.5\fboxrule}
    Return the output file, i.e. the file containing the frame cache 
    data.

    \vspace{1ex}

      Overrides: pipeline.AnalysisNode.get\_output

    \end{boxedminipage}

    \label{inspiral:DataFindNode:set_end}
    \index{inspiral \textit{(module)}!DataFindNode \textit{(class)}!set\_end \textit{(method)}}
    \vspace{0.5ex}

    \noindent\begin{boxedminipage}{\textwidth}

    \raggedright \textbf{set\_end}(\textit{self}, \textit{time})

    \vspace{-1.5ex}

    \rule{\textwidth}{0.5\fboxrule}
    Set the end time of the datafind query. time = GPS end time of query.

    \vspace{1ex}

      Overrides: pipeline.AnalysisNode.set\_end

    \end{boxedminipage}

    \label{inspiral:DataFindNode:set_ifo}
    \index{inspiral \textit{(module)}!DataFindNode \textit{(class)}!set\_ifo \textit{(method)}}
    \vspace{0.5ex}

    \noindent\begin{boxedminipage}{\textwidth}

    \raggedright \textbf{set\_ifo}(\textit{self}, \textit{ifo})

    \vspace{-1.5ex}

    \rule{\textwidth}{0.5\fboxrule}
    Set the IFO to retrieve data for. Since the data from both Hanford 
    interferometers is stored in the same frame file, this takes the 
    first letter of the IFO (e.g. L or H) and passes it to the 
    --instrument option of LALdataFind. ifo = IFO to obtain data for.

    \vspace{1ex}

      Overrides: pipeline.AnalysisNode.set\_ifo

    \end{boxedminipage}

    \label{inspiral:DataFindNode:set_start}
    \index{inspiral \textit{(module)}!DataFindNode \textit{(class)}!set\_start \textit{(method)}}
    \vspace{0.5ex}

    \noindent\begin{boxedminipage}{\textwidth}

    \raggedright \textbf{set\_start}(\textit{self}, \textit{time})

    \vspace{-1.5ex}

    \rule{\textwidth}{0.5\fboxrule}
    Set the start time of the datafind query. time = GPS start time of 
    query.

    \vspace{1ex}

      Overrides: pipeline.AnalysisNode.set\_start

    \end{boxedminipage}

  \noindent\textbf{Inherited from AnalysisNode:}
    get\_end,
    get\_ifo,
    get\_input,
    get\_start,
    set\_cache,
    set\_input,
    set\_output
    \\
  \noindent\textbf{Inherited from CondorDAGNode:}
    \_\_repr\_\_,
    add\_parent,
    add\_var\_arg,
    add\_var\_opt,
    job,
    set\_log\_file,
    set\_name,
    set\_retry,
    write\_job,
    write\_parents,
    write\_vars
    \index{inspiral \textit{(module)}!DataFindNode \textit{(class)}|)}
    \index{inspiral \textit{(module)}!IncaJob \textit{(class)}|(}

%%%%%%%%%%%%%%%%%%%%%%%%%%%%%%%%%%%%%%%%%%%%%%%%%%%%%%%%%%%%%%%%%%%%%%%%%%%
%%                           Class Description                           %%
%%%%%%%%%%%%%%%%%%%%%%%%%%%%%%%%%%%%%%%%%%%%%%%%%%%%%%%%%%%%%%%%%%%%%%%%%%%

\subsection{Class IncaJob}

    \label{inspiral:IncaJob}
\begin{tabular}{cccccccc}
% Line for pipeline.AnalysisJob, linespec=[0]
\multicolumn{4}{r}{\settowidth{\BCL}{pipeline.AnalysisJob}\multirow{2}{\BCL}{pipeline.AnalysisJob}}
&&
  \\\cline{5-5}
  &&&&\multicolumn{1}{c|}{}
&&
  \\
% Line for pipeline.CondorJob, linespec=[0, 1]
\multicolumn{2}{r}{\settowidth{\BCL}{pipeline.CondorJob}\multirow{2}{\BCL}{pipeline.CondorJob}}
&&
&&\multicolumn{1}{|c}{}
  \\\cline{3-3}
  &&\multicolumn{1}{c|}{}
&&
&\multicolumn{1}{|c}{}&
  \\
% Line for pipeline.CondorDAGJob, linespec=[1]
\multicolumn{4}{r}{\settowidth{\BCL}{pipeline.CondorDAGJob}\multirow{2}{\BCL}{pipeline.CondorDAGJob}}
&&\multicolumn{1}{|c}{}
  \\\cline{5-5}
  &&&&\multicolumn{1}{c|}{}
&\multicolumn{1}{|c}{}&
  \\
&&&&\multicolumn{2}{l}{\textbf{IncaJob}}
\end{tabular}

A lalapps\_inca job used by the inspiral pipeline. The static options are 
read from the section [inca] in the ini file. The stdout and stderr from 
the job are directed to the logs directory. The job always runs in the 
scheduler universe. The path to the executable is determined from the ini 
file.


%%%%%%%%%%%%%%%%%%%%%%%%%%%%%%%%%%%%%%%%%%%%%%%%%%%%%%%%%%%%%%%%%%%%%%%%%%%
%%                                Methods                                %%
%%%%%%%%%%%%%%%%%%%%%%%%%%%%%%%%%%%%%%%%%%%%%%%%%%%%%%%%%%%%%%%%%%%%%%%%%%%

  \subsubsection{Methods}

    \label{inspiral:IncaJob:__init__}
    \index{inspiral \textit{(module)}!IncaJob \textit{(class)}!\_\_init\_\_ \textit{(method)}}
    \vspace{0.5ex}

    \noindent\begin{boxedminipage}{\textwidth}

    \raggedright \textbf{\_\_init\_\_}(\textit{self}, \textit{cp})

    \vspace{-1.5ex}

    \rule{\textwidth}{0.5\fboxrule}
    cp = ConfigParser object from which options are read.

    \vspace{1ex}

      Overrides: pipeline.CondorDAGJob.\_\_init\_\_

    \end{boxedminipage}

  \noindent\textbf{Inherited from AnalysisJob:}
    calibration,
    channel,
    get\_config
    \\
  \noindent\textbf{Inherited from CondorDAGJob:}
    add\_var\_arg,
    add\_var\_opt
    \\
  \noindent\textbf{Inherited from CondorJob:}
    add\_arg,
    add\_condor\_cmd,
    add\_ini\_opts,
    add\_opt,
    get\_stderr\_file,
    get\_stdout\_file,
    get\_sub\_file,
    set\_log\_file,
    set\_notification,
    set\_stderr\_file,
    set\_stdout\_file,
    set\_sub\_file,
    write\_sub\_file
    \index{inspiral \textit{(module)}!IncaJob \textit{(class)}|)}
    \index{inspiral \textit{(module)}!IncaNode \textit{(class)}|(}

%%%%%%%%%%%%%%%%%%%%%%%%%%%%%%%%%%%%%%%%%%%%%%%%%%%%%%%%%%%%%%%%%%%%%%%%%%%
%%                           Class Description                           %%
%%%%%%%%%%%%%%%%%%%%%%%%%%%%%%%%%%%%%%%%%%%%%%%%%%%%%%%%%%%%%%%%%%%%%%%%%%%

\subsection{Class IncaNode}

    \label{inspiral:IncaNode}
\begin{tabular}{cccccccc}
% Line for pipeline.CondorDAGNode, linespec=[0, 0]
\multicolumn{2}{r}{\settowidth{\BCL}{pipeline.CondorDAGNode}\multirow{2}{\BCL}{pipeline.CondorDAGNode}}
&&
&&
  \\\cline{3-3}
  &&\multicolumn{1}{c|}{}
&&
&&
  \\
% Line for pipeline.AnalysisNode, linespec=[0]
\multicolumn{4}{r}{\settowidth{\BCL}{pipeline.AnalysisNode}\multirow{2}{\BCL}{pipeline.AnalysisNode}}
&&
  \\\cline{5-5}
  &&&&\multicolumn{1}{c|}{}
&&
  \\
% Line for pipeline.CondorDAGNode, linespec=[1]
\multicolumn{4}{r}{\settowidth{\BCL}{pipeline.CondorDAGNode}\multirow{2}{\BCL}{pipeline.CondorDAGNode}}
&&\multicolumn{1}{|c}{}
  \\\cline{5-5}
  &&&&\multicolumn{1}{c|}{}
&\multicolumn{1}{|c}{}&
  \\
&&&&\multicolumn{2}{l}{\textbf{IncaNode}}
\end{tabular}

An IncaNode runs an instance of the inspiral coincidence code in a Condor 
DAG.


%%%%%%%%%%%%%%%%%%%%%%%%%%%%%%%%%%%%%%%%%%%%%%%%%%%%%%%%%%%%%%%%%%%%%%%%%%%
%%                                Methods                                %%
%%%%%%%%%%%%%%%%%%%%%%%%%%%%%%%%%%%%%%%%%%%%%%%%%%%%%%%%%%%%%%%%%%%%%%%%%%%

  \subsubsection{Methods}

    \label{inspiral:IncaNode:__init__}
    \index{inspiral \textit{(module)}!IncaNode \textit{(class)}!\_\_init\_\_ \textit{(method)}}
    \vspace{0.5ex}

    \noindent\begin{boxedminipage}{\textwidth}

    \raggedright \textbf{\_\_init\_\_}(\textit{self}, \textit{job})

    \vspace{-1.5ex}

    \rule{\textwidth}{0.5\fboxrule}
    job = A CondorDAGJob that can run an instance of lalapps\_inca.

    \vspace{1ex}

      Overrides: pipeline.AnalysisNode.\_\_init\_\_

    \end{boxedminipage}

    \label{inspiral:IncaNode:get_ifo_a}
    \index{inspiral \textit{(module)}!IncaNode \textit{(class)}!get\_ifo\_a \textit{(method)}}
    \vspace{0.5ex}

    \noindent\begin{boxedminipage}{\textwidth}

    \raggedright \textbf{get\_ifo\_a}(\textit{self})

    \vspace{-1.5ex}

    \rule{\textwidth}{0.5\fboxrule}
    Returns the IFO code of the primary interferometer.

    \vspace{1ex}

    \end{boxedminipage}

    \label{inspiral:IncaNode:get_ifo_b}
    \index{inspiral \textit{(module)}!IncaNode \textit{(class)}!get\_ifo\_b \textit{(method)}}
    \vspace{0.5ex}

    \noindent\begin{boxedminipage}{\textwidth}

    \raggedright \textbf{get\_ifo\_b}(\textit{self})

    \vspace{-1.5ex}

    \rule{\textwidth}{0.5\fboxrule}
    Returns the IFO code of the primary interferometer.

    \vspace{1ex}

    \end{boxedminipage}

    \label{inspiral:IncaNode:get_output}
    \index{inspiral \textit{(module)}!IncaNode \textit{(class)}!get\_output \textit{(method)}}
    \vspace{0.5ex}

    \noindent\begin{boxedminipage}{\textwidth}

    \raggedright \textbf{get\_output}(\textit{self})

    \vspace{-1.5ex}

    \rule{\textwidth}{0.5\fboxrule}
    Returns the file name of output from the inca code. This must be kept 
    synchronized with the name of the output file in inca.c.

    \vspace{1ex}

      Overrides: pipeline.AnalysisNode.get\_output

    \end{boxedminipage}

    \label{inspiral:IncaNode:set_ifo_a}
    \index{inspiral \textit{(module)}!IncaNode \textit{(class)}!set\_ifo\_a \textit{(method)}}
    \vspace{0.5ex}

    \noindent\begin{boxedminipage}{\textwidth}

    \raggedright \textbf{set\_ifo\_a}(\textit{self}, \textit{ifo})

    \vspace{-1.5ex}

    \rule{\textwidth}{0.5\fboxrule}
    Set the interferometer code to use as IFO A. ifo = IFO code (e.g. L1, 
    H1 or H2).

    \vspace{1ex}

    \end{boxedminipage}

    \label{inspiral:IncaNode:set_ifo_b}
    \index{inspiral \textit{(module)}!IncaNode \textit{(class)}!set\_ifo\_b \textit{(method)}}
    \vspace{0.5ex}

    \noindent\begin{boxedminipage}{\textwidth}

    \raggedright \textbf{set\_ifo\_b}(\textit{self}, \textit{ifo})

    \vspace{-1.5ex}

    \rule{\textwidth}{0.5\fboxrule}
    Set the interferometer code to use as IFO B. ifo = IFO code (e.g. L1, 
    H1 or H2).

    \vspace{1ex}

    \end{boxedminipage}

  \noindent\textbf{Inherited from AnalysisNode:}
    get\_end,
    get\_ifo,
    get\_input,
    get\_start,
    set\_cache,
    set\_end,
    set\_ifo,
    set\_input,
    get\_output,
    set\_start
    \\
  \noindent\textbf{Inherited from CondorDAGNode:}
    \_\_repr\_\_,
    add\_parent,
    add\_var\_arg,
    add\_var\_opt,
    job,
    set\_log\_file,
    set\_name,
    set\_retry,
    write\_job,
    write\_parents,
    write\_vars
    \index{inspiral \textit{(module)}!IncaNode \textit{(class)}|)}
    \index{inspiral \textit{(module)}!InspiralError \textit{(class)}|(}

%%%%%%%%%%%%%%%%%%%%%%%%%%%%%%%%%%%%%%%%%%%%%%%%%%%%%%%%%%%%%%%%%%%%%%%%%%%
%%                           Class Description                           %%
%%%%%%%%%%%%%%%%%%%%%%%%%%%%%%%%%%%%%%%%%%%%%%%%%%%%%%%%%%%%%%%%%%%%%%%%%%%

\subsection{Class InspiralError}

    \label{inspiral:InspiralError}
\begin{tabular}{cccccc}
% Line for exceptions.Exception, linespec=[0]
\multicolumn{2}{r}{\settowidth{\BCL}{exceptions.Exception}\multirow{2}{\BCL}{exceptions.Exception}}
&&
  \\\cline{3-3}
  &&\multicolumn{1}{c|}{}
&&
  \\
&&\multicolumn{2}{l}{\textbf{InspiralError}}
\end{tabular}


%%%%%%%%%%%%%%%%%%%%%%%%%%%%%%%%%%%%%%%%%%%%%%%%%%%%%%%%%%%%%%%%%%%%%%%%%%%
%%                                Methods                                %%
%%%%%%%%%%%%%%%%%%%%%%%%%%%%%%%%%%%%%%%%%%%%%%%%%%%%%%%%%%%%%%%%%%%%%%%%%%%

  \subsubsection{Methods}

    \label{inspiral:InspiralError:__init__}
    \index{inspiral \textit{(module)}!InspiralError \textit{(class)}!\_\_init\_\_ \textit{(method)}}
    \vspace{0.5ex}

    \noindent\begin{boxedminipage}{\textwidth}

    \raggedright \textbf{\_\_init\_\_}(\textit{self}, \textit{args}=\texttt{N\-o\-n\-e\-})

      Overrides: exceptions.Exception.\_\_init\_\_

    \end{boxedminipage}

  \noindent\textbf{Inherited from Exception:}
    \_\_getitem\_\_,
    \_\_str\_\_
    \index{inspiral \textit{(module)}!InspiralError \textit{(class)}|)}
    \index{inspiral \textit{(module)}!InspiralJob \textit{(class)}|(}

%%%%%%%%%%%%%%%%%%%%%%%%%%%%%%%%%%%%%%%%%%%%%%%%%%%%%%%%%%%%%%%%%%%%%%%%%%%
%%                           Class Description                           %%
%%%%%%%%%%%%%%%%%%%%%%%%%%%%%%%%%%%%%%%%%%%%%%%%%%%%%%%%%%%%%%%%%%%%%%%%%%%

\subsection{Class InspiralJob}

    \label{inspiral:InspiralJob}
\begin{tabular}{cccccccc}
% Line for pipeline.AnalysisJob, linespec=[0]
\multicolumn{4}{r}{\settowidth{\BCL}{pipeline.AnalysisJob}\multirow{2}{\BCL}{pipeline.AnalysisJob}}
&&
  \\\cline{5-5}
  &&&&\multicolumn{1}{c|}{}
&&
  \\
% Line for pipeline.CondorJob, linespec=[0, 1]
\multicolumn{2}{r}{\settowidth{\BCL}{pipeline.CondorJob}\multirow{2}{\BCL}{pipeline.CondorJob}}
&&
&&\multicolumn{1}{|c}{}
  \\\cline{3-3}
  &&\multicolumn{1}{c|}{}
&&
&\multicolumn{1}{|c}{}&
  \\
% Line for pipeline.CondorDAGJob, linespec=[1]
\multicolumn{4}{r}{\settowidth{\BCL}{pipeline.CondorDAGJob}\multirow{2}{\BCL}{pipeline.CondorDAGJob}}
&&\multicolumn{1}{|c}{}
  \\\cline{5-5}
  &&&&\multicolumn{1}{c|}{}
&\multicolumn{1}{|c}{}&
  \\
&&&&\multicolumn{2}{l}{\textbf{InspiralJob}}
\end{tabular}

A lalapps\_inspiral job used by the inspiral pipeline. The static options 
are read from the sections [data] and [inspiral] in the ini file. The 
stdout and stderr from the job are directed to the logs directory. The 
job runs in the universe specfied in the ini file. The path to the 
executable is determined from the ini file.


%%%%%%%%%%%%%%%%%%%%%%%%%%%%%%%%%%%%%%%%%%%%%%%%%%%%%%%%%%%%%%%%%%%%%%%%%%%
%%                                Methods                                %%
%%%%%%%%%%%%%%%%%%%%%%%%%%%%%%%%%%%%%%%%%%%%%%%%%%%%%%%%%%%%%%%%%%%%%%%%%%%

  \subsubsection{Methods}

    \label{inspiral:InspiralJob:__init__}
    \index{inspiral \textit{(module)}!InspiralJob \textit{(class)}!\_\_init\_\_ \textit{(method)}}
    \vspace{0.5ex}

    \noindent\begin{boxedminipage}{\textwidth}

    \raggedright \textbf{\_\_init\_\_}(\textit{self}, \textit{cp})

    \vspace{-1.5ex}

    \rule{\textwidth}{0.5\fboxrule}
    cp = ConfigParser object from which options are read.

    \vspace{1ex}

      Overrides: pipeline.CondorDAGJob.\_\_init\_\_

    \end{boxedminipage}

  \noindent\textbf{Inherited from AnalysisJob:}
    calibration,
    channel,
    get\_config
    \\
  \noindent\textbf{Inherited from CondorDAGJob:}
    add\_var\_arg,
    add\_var\_opt
    \\
  \noindent\textbf{Inherited from CondorJob:}
    add\_arg,
    add\_condor\_cmd,
    add\_ini\_opts,
    add\_opt,
    get\_stderr\_file,
    get\_stdout\_file,
    get\_sub\_file,
    set\_log\_file,
    set\_notification,
    set\_stderr\_file,
    set\_stdout\_file,
    set\_sub\_file,
    write\_sub\_file
    \index{inspiral \textit{(module)}!InspiralJob \textit{(class)}|)}
    \index{inspiral \textit{(module)}!InspiralNode \textit{(class)}|(}

%%%%%%%%%%%%%%%%%%%%%%%%%%%%%%%%%%%%%%%%%%%%%%%%%%%%%%%%%%%%%%%%%%%%%%%%%%%
%%                           Class Description                           %%
%%%%%%%%%%%%%%%%%%%%%%%%%%%%%%%%%%%%%%%%%%%%%%%%%%%%%%%%%%%%%%%%%%%%%%%%%%%

\subsection{Class InspiralNode}

    \label{inspiral:InspiralNode}
\begin{tabular}{cccccccc}
% Line for pipeline.CondorDAGNode, linespec=[0, 0]
\multicolumn{2}{r}{\settowidth{\BCL}{pipeline.CondorDAGNode}\multirow{2}{\BCL}{pipeline.CondorDAGNode}}
&&
&&
  \\\cline{3-3}
  &&\multicolumn{1}{c|}{}
&&
&&
  \\
% Line for pipeline.AnalysisNode, linespec=[0]
\multicolumn{4}{r}{\settowidth{\BCL}{pipeline.AnalysisNode}\multirow{2}{\BCL}{pipeline.AnalysisNode}}
&&
  \\\cline{5-5}
  &&&&\multicolumn{1}{c|}{}
&&
  \\
% Line for pipeline.CondorDAGNode, linespec=[1]
\multicolumn{4}{r}{\settowidth{\BCL}{pipeline.CondorDAGNode}\multirow{2}{\BCL}{pipeline.CondorDAGNode}}
&&\multicolumn{1}{|c}{}
  \\\cline{5-5}
  &&&&\multicolumn{1}{c|}{}
&\multicolumn{1}{|c}{}&
  \\
&&&&\multicolumn{2}{l}{\textbf{InspiralNode}}
\end{tabular}

An InspiralNode runs an instance of the inspiral code in a Condor DAG.


%%%%%%%%%%%%%%%%%%%%%%%%%%%%%%%%%%%%%%%%%%%%%%%%%%%%%%%%%%%%%%%%%%%%%%%%%%%
%%                                Methods                                %%
%%%%%%%%%%%%%%%%%%%%%%%%%%%%%%%%%%%%%%%%%%%%%%%%%%%%%%%%%%%%%%%%%%%%%%%%%%%

  \subsubsection{Methods}

    \label{inspiral:InspiralNode:__init__}
    \index{inspiral \textit{(module)}!InspiralNode \textit{(class)}!\_\_init\_\_ \textit{(method)}}
    \vspace{0.5ex}

    \noindent\begin{boxedminipage}{\textwidth}

    \raggedright \textbf{\_\_init\_\_}(\textit{self}, \textit{job})

    \vspace{-1.5ex}

    \rule{\textwidth}{0.5\fboxrule}
    job = A CondorDAGJob that can run an instance of lalapps\_inspiral.

    \vspace{1ex}

      Overrides: pipeline.AnalysisNode.\_\_init\_\_

    \end{boxedminipage}

    \label{inspiral:InspiralNode:get_output}
    \index{inspiral \textit{(module)}!InspiralNode \textit{(class)}!get\_output \textit{(method)}}
    \vspace{0.5ex}

    \noindent\begin{boxedminipage}{\textwidth}

    \raggedright \textbf{get\_output}(\textit{self})

    \vspace{-1.5ex}

    \rule{\textwidth}{0.5\fboxrule}
    Returns the file name of output from the inspiral code. This must be 
    kept synchronized with the name of the output file in inspiral.c.

    \vspace{1ex}

      Overrides: pipeline.AnalysisNode.get\_output

    \end{boxedminipage}

    \label{inspiral:InspiralNode:set_bank}
    \index{inspiral \textit{(module)}!InspiralNode \textit{(class)}!set\_bank \textit{(method)}}
    \vspace{0.5ex}

    \noindent\begin{boxedminipage}{\textwidth}

    \raggedright \textbf{set\_bank}(\textit{self}, \textit{bank})

    \end{boxedminipage}

  \noindent\textbf{Inherited from AnalysisNode:}
    get\_end,
    get\_ifo,
    get\_input,
    get\_start,
    set\_cache,
    set\_end,
    set\_ifo,
    set\_input,
    set\_output,
    set\_start
    \\
  \noindent\textbf{Inherited from CondorDAGNode:}
    \_\_repr\_\_,
    add\_parent,
    add\_var\_arg,
    add\_var\_opt,
    job,
    set\_log\_file,
    set\_name,
    set\_retry,
    write\_job,
    write\_parents,
    write\_vars
    \index{inspiral \textit{(module)}!InspiralNode \textit{(class)}|)}
    \index{inspiral \textit{(module)}!TmpltBankJob \textit{(class)}|(}

%%%%%%%%%%%%%%%%%%%%%%%%%%%%%%%%%%%%%%%%%%%%%%%%%%%%%%%%%%%%%%%%%%%%%%%%%%%
%%                           Class Description                           %%
%%%%%%%%%%%%%%%%%%%%%%%%%%%%%%%%%%%%%%%%%%%%%%%%%%%%%%%%%%%%%%%%%%%%%%%%%%%

\subsection{Class TmpltBankJob}

    \label{inspiral:TmpltBankJob}
\begin{tabular}{cccccccc}
% Line for pipeline.AnalysisJob, linespec=[0]
\multicolumn{4}{r}{\settowidth{\BCL}{pipeline.AnalysisJob}\multirow{2}{\BCL}{pipeline.AnalysisJob}}
&&
  \\\cline{5-5}
  &&&&\multicolumn{1}{c|}{}
&&
  \\
% Line for pipeline.CondorJob, linespec=[0, 1]
\multicolumn{2}{r}{\settowidth{\BCL}{pipeline.CondorJob}\multirow{2}{\BCL}{pipeline.CondorJob}}
&&
&&\multicolumn{1}{|c}{}
  \\\cline{3-3}
  &&\multicolumn{1}{c|}{}
&&
&\multicolumn{1}{|c}{}&
  \\
% Line for pipeline.CondorDAGJob, linespec=[1]
\multicolumn{4}{r}{\settowidth{\BCL}{pipeline.CondorDAGJob}\multirow{2}{\BCL}{pipeline.CondorDAGJob}}
&&\multicolumn{1}{|c}{}
  \\\cline{5-5}
  &&&&\multicolumn{1}{c|}{}
&\multicolumn{1}{|c}{}&
  \\
&&&&\multicolumn{2}{l}{\textbf{TmpltBankJob}}
\end{tabular}

A lalapps\_tmpltbank job used by the inspiral pipeline. The static 
options are read from the sections [data] and [tmpltbank] in the ini 
file. The stdout and stderr from the job are directed to the logs 
directory. The job runs in the universe specfied in the ini file. The 
path to the executable is determined from the ini file.


%%%%%%%%%%%%%%%%%%%%%%%%%%%%%%%%%%%%%%%%%%%%%%%%%%%%%%%%%%%%%%%%%%%%%%%%%%%
%%                                Methods                                %%
%%%%%%%%%%%%%%%%%%%%%%%%%%%%%%%%%%%%%%%%%%%%%%%%%%%%%%%%%%%%%%%%%%%%%%%%%%%

  \subsubsection{Methods}

    \label{inspiral:TmpltBankJob:__init__}
    \index{inspiral \textit{(module)}!TmpltBankJob \textit{(class)}!\_\_init\_\_ \textit{(method)}}
    \vspace{0.5ex}

    \noindent\begin{boxedminipage}{\textwidth}

    \raggedright \textbf{\_\_init\_\_}(\textit{self}, \textit{cp})

    \vspace{-1.5ex}

    \rule{\textwidth}{0.5\fboxrule}
    cp = ConfigParser object from which options are read.

    \vspace{1ex}

      Overrides: pipeline.CondorDAGJob.\_\_init\_\_

    \end{boxedminipage}

  \noindent\textbf{Inherited from AnalysisJob:}
    calibration,
    channel,
    get\_config
    \\
  \noindent\textbf{Inherited from CondorDAGJob:}
    add\_var\_arg,
    add\_var\_opt
    \\
  \noindent\textbf{Inherited from CondorJob:}
    add\_arg,
    add\_condor\_cmd,
    add\_ini\_opts,
    add\_opt,
    get\_stderr\_file,
    get\_stdout\_file,
    get\_sub\_file,
    set\_log\_file,
    set\_notification,
    set\_stderr\_file,
    set\_stdout\_file,
    set\_sub\_file,
    write\_sub\_file
    \index{inspiral \textit{(module)}!TmpltBankJob \textit{(class)}|)}
    \index{inspiral \textit{(module)}!TmpltBankNode \textit{(class)}|(}

%%%%%%%%%%%%%%%%%%%%%%%%%%%%%%%%%%%%%%%%%%%%%%%%%%%%%%%%%%%%%%%%%%%%%%%%%%%
%%                           Class Description                           %%
%%%%%%%%%%%%%%%%%%%%%%%%%%%%%%%%%%%%%%%%%%%%%%%%%%%%%%%%%%%%%%%%%%%%%%%%%%%

\subsection{Class TmpltBankNode}

    \label{inspiral:TmpltBankNode}
\begin{tabular}{cccccccc}
% Line for pipeline.CondorDAGNode, linespec=[0, 0]
\multicolumn{2}{r}{\settowidth{\BCL}{pipeline.CondorDAGNode}\multirow{2}{\BCL}{pipeline.CondorDAGNode}}
&&
&&
  \\\cline{3-3}
  &&\multicolumn{1}{c|}{}
&&
&&
  \\
% Line for pipeline.AnalysisNode, linespec=[0]
\multicolumn{4}{r}{\settowidth{\BCL}{pipeline.AnalysisNode}\multirow{2}{\BCL}{pipeline.AnalysisNode}}
&&
  \\\cline{5-5}
  &&&&\multicolumn{1}{c|}{}
&&
  \\
% Line for pipeline.CondorDAGNode, linespec=[1]
\multicolumn{4}{r}{\settowidth{\BCL}{pipeline.CondorDAGNode}\multirow{2}{\BCL}{pipeline.CondorDAGNode}}
&&\multicolumn{1}{|c}{}
  \\\cline{5-5}
  &&&&\multicolumn{1}{c|}{}
&\multicolumn{1}{|c}{}&
  \\
&&&&\multicolumn{2}{l}{\textbf{TmpltBankNode}}
\end{tabular}

A TmpltBankNode runs an instance of the template bank generation job in a 
Condor DAG.


%%%%%%%%%%%%%%%%%%%%%%%%%%%%%%%%%%%%%%%%%%%%%%%%%%%%%%%%%%%%%%%%%%%%%%%%%%%
%%                                Methods                                %%
%%%%%%%%%%%%%%%%%%%%%%%%%%%%%%%%%%%%%%%%%%%%%%%%%%%%%%%%%%%%%%%%%%%%%%%%%%%

  \subsubsection{Methods}

    \label{inspiral:TmpltBankNode:__init__}
    \index{inspiral \textit{(module)}!TmpltBankNode \textit{(class)}!\_\_init\_\_ \textit{(method)}}
    \vspace{0.5ex}

    \noindent\begin{boxedminipage}{\textwidth}

    \raggedright \textbf{\_\_init\_\_}(\textit{self}, \textit{job})

    \vspace{-1.5ex}

    \rule{\textwidth}{0.5\fboxrule}
    job = A CondorDAGJob that can run an instance of lalapps\_tmpltbank.

    \vspace{1ex}

      Overrides: pipeline.AnalysisNode.\_\_init\_\_

    \end{boxedminipage}

    \label{inspiral:TmpltBankNode:get_output}
    \index{inspiral \textit{(module)}!TmpltBankNode \textit{(class)}!get\_output \textit{(method)}}
    \vspace{0.5ex}

    \noindent\begin{boxedminipage}{\textwidth}

    \raggedright \textbf{get\_output}(\textit{self})

    \vspace{-1.5ex}

    \rule{\textwidth}{0.5\fboxrule}
    Returns the file name of output from the template bank code. This 
    must be kept synchronized with the name of the output file in 
    tmpltbank.c.

    \vspace{1ex}

      Overrides: pipeline.AnalysisNode.get\_output

    \end{boxedminipage}

  \noindent\textbf{Inherited from AnalysisNode:}
    get\_end,
    get\_ifo,
    get\_input,
    get\_start,
    set\_cache,
    set\_end,
    set\_ifo,
    set\_input,
    set\_output,
    set\_start
    \\
  \noindent\textbf{Inherited from CondorDAGNode:}
    \_\_repr\_\_,
    add\_parent,
    add\_var\_arg,
    add\_var\_opt,
    job,
    set\_log\_file,
    set\_name,
    set\_retry,
    write\_job,
    write\_parents,
    write\_vars
    \index{inspiral \textit{(module)}!TmpltBankNode \textit{(class)}|)}
    \index{inspiral \textit{(module)}!TrigToTmpltJob \textit{(class)}|(}

%%%%%%%%%%%%%%%%%%%%%%%%%%%%%%%%%%%%%%%%%%%%%%%%%%%%%%%%%%%%%%%%%%%%%%%%%%%
%%                           Class Description                           %%
%%%%%%%%%%%%%%%%%%%%%%%%%%%%%%%%%%%%%%%%%%%%%%%%%%%%%%%%%%%%%%%%%%%%%%%%%%%

\subsection{Class TrigToTmpltJob}

    \label{inspiral:TrigToTmpltJob}
\begin{tabular}{cccccccc}
% Line for pipeline.AnalysisJob, linespec=[0]
\multicolumn{4}{r}{\settowidth{\BCL}{pipeline.AnalysisJob}\multirow{2}{\BCL}{pipeline.AnalysisJob}}
&&
  \\\cline{5-5}
  &&&&\multicolumn{1}{c|}{}
&&
  \\
% Line for pipeline.CondorJob, linespec=[0, 1]
\multicolumn{2}{r}{\settowidth{\BCL}{pipeline.CondorJob}\multirow{2}{\BCL}{pipeline.CondorJob}}
&&
&&\multicolumn{1}{|c}{}
  \\\cline{3-3}
  &&\multicolumn{1}{c|}{}
&&
&\multicolumn{1}{|c}{}&
  \\
% Line for pipeline.CondorDAGJob, linespec=[1]
\multicolumn{4}{r}{\settowidth{\BCL}{pipeline.CondorDAGJob}\multirow{2}{\BCL}{pipeline.CondorDAGJob}}
&&\multicolumn{1}{|c}{}
  \\\cline{5-5}
  &&&&\multicolumn{1}{c|}{}
&\multicolumn{1}{|c}{}&
  \\
&&&&\multicolumn{2}{l}{\textbf{TrigToTmpltJob}}
\end{tabular}

A lalapps\_trigtotmplt job used by the inspiral pipeline. The static 
options are read from the section [trigtotmplt] in the ini file. The 
stdout and stderr from the job are directed to the logs directory. The 
job always runs in the scheduler universe. The path to the executable is 
determined from the ini file.


%%%%%%%%%%%%%%%%%%%%%%%%%%%%%%%%%%%%%%%%%%%%%%%%%%%%%%%%%%%%%%%%%%%%%%%%%%%
%%                                Methods                                %%
%%%%%%%%%%%%%%%%%%%%%%%%%%%%%%%%%%%%%%%%%%%%%%%%%%%%%%%%%%%%%%%%%%%%%%%%%%%

  \subsubsection{Methods}

    \label{inspiral:TrigToTmpltJob:__init__}
    \index{inspiral \textit{(module)}!TrigToTmpltJob \textit{(class)}!\_\_init\_\_ \textit{(method)}}
    \vspace{0.5ex}

    \noindent\begin{boxedminipage}{\textwidth}

    \raggedright \textbf{\_\_init\_\_}(\textit{self}, \textit{cp})

    \vspace{-1.5ex}

    \rule{\textwidth}{0.5\fboxrule}
    cp = ConfigParser object from which options are read.

    \vspace{1ex}

      Overrides: pipeline.CondorDAGJob.\_\_init\_\_

    \end{boxedminipage}

  \noindent\textbf{Inherited from AnalysisJob:}
    calibration,
    channel,
    get\_config
    \\
  \noindent\textbf{Inherited from CondorDAGJob:}
    add\_var\_arg,
    add\_var\_opt
    \\
  \noindent\textbf{Inherited from CondorJob:}
    add\_arg,
    add\_condor\_cmd,
    add\_ini\_opts,
    add\_opt,
    get\_stderr\_file,
    get\_stdout\_file,
    get\_sub\_file,
    set\_log\_file,
    set\_notification,
    set\_stderr\_file,
    set\_stdout\_file,
    set\_sub\_file,
    write\_sub\_file
    \index{inspiral \textit{(module)}!TrigToTmpltJob \textit{(class)}|)}
    \index{inspiral \textit{(module)}!TrigToTmpltNode \textit{(class)}|(}

%%%%%%%%%%%%%%%%%%%%%%%%%%%%%%%%%%%%%%%%%%%%%%%%%%%%%%%%%%%%%%%%%%%%%%%%%%%
%%                           Class Description                           %%
%%%%%%%%%%%%%%%%%%%%%%%%%%%%%%%%%%%%%%%%%%%%%%%%%%%%%%%%%%%%%%%%%%%%%%%%%%%

\subsection{Class TrigToTmpltNode}

    \label{inspiral:TrigToTmpltNode}
\begin{tabular}{cccccccc}
% Line for pipeline.CondorDAGNode, linespec=[0, 0]
\multicolumn{2}{r}{\settowidth{\BCL}{pipeline.CondorDAGNode}\multirow{2}{\BCL}{pipeline.CondorDAGNode}}
&&
&&
  \\\cline{3-3}
  &&\multicolumn{1}{c|}{}
&&
&&
  \\
% Line for pipeline.AnalysisNode, linespec=[0]
\multicolumn{4}{r}{\settowidth{\BCL}{pipeline.AnalysisNode}\multirow{2}{\BCL}{pipeline.AnalysisNode}}
&&
  \\\cline{5-5}
  &&&&\multicolumn{1}{c|}{}
&&
  \\
% Line for pipeline.CondorDAGNode, linespec=[1]
\multicolumn{4}{r}{\settowidth{\BCL}{pipeline.CondorDAGNode}\multirow{2}{\BCL}{pipeline.CondorDAGNode}}
&&\multicolumn{1}{|c}{}
  \\\cline{5-5}
  &&&&\multicolumn{1}{c|}{}
&\multicolumn{1}{|c}{}&
  \\
&&&&\multicolumn{2}{l}{\textbf{TrigToTmpltNode}}
\end{tabular}

A TrigToTmpltNode runs an instance of the triggered bank generator in a 
Condor DAG.


%%%%%%%%%%%%%%%%%%%%%%%%%%%%%%%%%%%%%%%%%%%%%%%%%%%%%%%%%%%%%%%%%%%%%%%%%%%
%%                                Methods                                %%
%%%%%%%%%%%%%%%%%%%%%%%%%%%%%%%%%%%%%%%%%%%%%%%%%%%%%%%%%%%%%%%%%%%%%%%%%%%

  \subsubsection{Methods}

    \label{inspiral:TrigToTmpltNode:__init__}
    \index{inspiral \textit{(module)}!TrigToTmpltNode \textit{(class)}!\_\_init\_\_ \textit{(method)}}
    \vspace{0.5ex}

    \noindent\begin{boxedminipage}{\textwidth}

    \raggedright \textbf{\_\_init\_\_}(\textit{self}, \textit{job})

    \vspace{-1.5ex}

    \rule{\textwidth}{0.5\fboxrule}
    job = A CondorDAGJob that can run an instance of 
    lalapps\_trigtotmplt.

    \vspace{1ex}

      Overrides: pipeline.AnalysisNode.\_\_init\_\_

    \end{boxedminipage}

  \noindent\textbf{Inherited from AnalysisNode:}
    get\_end,
    get\_ifo,
    get\_input,
    get\_output,
    get\_start,
    set\_cache,
    set\_end,
    set\_ifo,
    set\_input,
    set\_output,
    set\_start
    \\
  \noindent\textbf{Inherited from CondorDAGNode:}
    \_\_repr\_\_,
    add\_parent,
    add\_var\_arg,
    add\_var\_opt,
    job,
    set\_log\_file,
    set\_name,
    set\_retry,
    write\_job,
    write\_parents,
    write\_vars
    \index{inspiral \textit{(module)}!TrigToTmpltNode \textit{(class)}|)}
    \index{inspiral \textit{(module)}|)}

\clearpage

\section{Inspiral Search Programs}
\label{section:inspiral}

This section of \textsc{LALApps} contains programs that can be used to search
interferometer data for inspiral signals using templated matched filtering and
associated veto strategies.

\clearpage
\subsection{Program \texttt{lalapps\_inspiral\_pipe}}
\label{program:inspiral-pipeline}
\idx[Program]{inspiral\_pipeline.py}

\begin{entry}
\item[Name]
\verb$lalapps_inspiral_pipe$ --- python script to generate Condor DAGs to
run the inspiral pipeline.

\item[Synopsis]
\begin{verbatim}
  -h, --help               display this message
  -v, --version            print version information and exit
  -u, --user-tag TAG       tag the job with TAG (overrides value in ini file)

  -d, --datafind           run LALdataFind to create frame cache files
  -t, --template-bank      run lalapps_tmpltbank to generate a template bank
  -i, --inspiral           run lalapps_inspiral on the first IFO
  -T, --triggered-bank     run lalapps_trigtotmplt to generate a triggered bank
  -I, --triggered-inspiral run lalapps_inspiral on the second IFO
  -C, --coincidence        run lalapps_inca on the triggers from both IFOs

  -j, --injections         add simulated inspirals from injection file

  -p, --playground-only    only create chunks that overlap with playground
  -P, --priority PRIO      run jobs with condor priority PRIO

  -f, --config-file FILE   use configuration file FILE
  -l, --log-path PATH      directory to write condor log file
\end{verbatim}

\item[Description] \verb$lalapps_inspiral_pipe$ generates a Condor DAG to run
the inspiral analysis pipeline. The configuration file should specify the
parameters needed to run the jobs and must be specified with the
\verb$--config-file$ option.  A file containing science segments to be
analyzed should be specified in the \verb$[input]$ section of the
configuration file with a line such as
\begin{verbatim}
segments = S2H1L1v03_selectedsegs.txt
\end{verbatim}
This should contain four whitespace separated columns:
\begin{verbatim}
  segment_id    gps_start_time  gps_end_time    duration
\end{verbatim}
that define the science segments to be used. Lines starting with an octothorpe
are ignored.

The analysis chunk size is determined from the number of data segments and
their length and overlap specified in config file. A chunk length is typically
this is 1024 seconds for S2.  The chunks start and stop times are computed
from the science segment times and used to build the DAG.

The once the DAG file has been created it should be submitted to the Condor
pool with the \verb$condor_submit_dag$ command.

\item[Options]\leavevmode
\begin{entry}
\item[\texttt{--help}] Display a brief usage summary.
\end{entry}

\item[Example]
Generate a DAG to run an inspiral search on the first IFO. The generated
DAG is then submitted with \texttt{condor\_submit\_dag}
\begin{verbatim}
lalapps_inspiral_pipe --log-path /people/duncan/dag_logs \
--datafind --template-bank --inspiral --playground-only \
--config-file l1_s2.ini

condor_submit_dag l1_s2.dag
\end{verbatim}

\item[Author] 
Duncan Brown
\end{entry}

\clearpage
\subsection{Program \texttt{lalapps\_tmpltbank}}
\label{program:lalapps-tmpltbank}
\idx[Program]{lalapps\_tmpltbank}

\begin{entry}
\item[Name]
\verb$lalapps_tmpltbank$ --- program to generate inspiral template banks.

\item[Synopsis]
\begin{verbatim}
  --help                       display this message
  --verbose                    print progress information
  --debug-level LEVEL          set the LAL debug level to LEVEL
  --user-tag STRING            set the process_params usertag to STRING
  --comment STRING             set the process table comment to STRING

  --gps-start-time SEC         GPS second of data start time
  --gps-end-time SEC           GPS second of data end time
  --pad-data T                 pad the data start and end time by T seconds

  --frame-cache                obtain frame data from LAL frame cache FILE
  --calibration-cache FILE     obtain calibration from LAL frame cache FILE
  --channel-name CHAN          read data from interferometer channel CHAN

  --sample-rate F              filter data at F Hz, downsampling if necessary
  --resample-filter TYPE       set resample filter to TYPE [ldas|butterworth]

  --disable-high-pass          turn off the IIR highpass filter
  --enable-high-pass F         high pass data above F Hz using an IIR filter
  --spectrum-type TYPE         use PSD estimator TYPE [mean|median]

  --segment-length N           set data segment length to N points
  --number-of-segments N       set number of data segments to N

  --low-frequency-cutoff F     do not filter below F Hz
  --high-frequency-cutoff F    upper frequency cutoff in Hz

  --minimum-mass MASS          set minimum component mass of bank to MASS
  --maximum-mass MASS          set maximum component mass of bank to MASS
  --minimal-match M            generate bank with minimal match M

  --order ORDER                set post-Newtonian order of the waveform to ORDER
                                 (newtonian|oneHalfPN|onePN|onePointFivePN|
                                 twoPN|twoPointFive|threePN|threePointFivePN)
  --approximant APPROX         set approximant of the waveform to APPROX
                                 (TaylorT1|TaylorT2|TaylorT3|TaylorF1|TaylorF2|
                                 PadeT1|PadeT2|EOB|BCV|SpinTaylorT3)
  --space SPACE                grid up template bank with mass parameters SPACE
                                 (Tau0Tau2|Tau0Tau3)

  --write-raw-data             write raw data to a frame file
  --write-response             write the computed response function to a frame
  --write-spectrum             write the uncalibrated psd to a frame
  --write-strain-spectrum      write the calibrated strain psd to a text file
\end{verbatim}

\item[Description] 
\verb$lalapps_tmpltbank$ is a stand alone code for generating inspiral
template banks for LIGO data with the LAL bank package.  The code generates a
calibrated power spectrum at the specified time for the requested channel and
uses this to compute the template bank.  See the LAL bank package
documentation for detailed information on the algorithms used to generate the
banks.

\item[Options]\leavevmode
\begin{entry}
\item[\texttt{--help}] Display a brief usage summary.
\end{entry}

\item[Example]
\begin{verbatim}
lalapps_tmpltbank \
--gps-start-time 734357353 --gps-end-time 734358377 \
--frame-cache cache/L-734357345-734361107.cache \
--segment-length 1048576 --number-of-segments 7 \
--pad-data 7 --sample-rate 4096 --resample-filter ldas \
--enable-high-pass 5.000000e+01 --spectrum-type median
--low-frequency-cutoff 7.000000e+01 --high-frequency-cutoff 2.048000e+03 \
--minimum-mass 1.000000e+00  --maximum-mass 3.000000e+00 \
--minimal-match 9.700000e-01 --calibration-cache  \
/ldas_outgoing/calibration/cache_files/L1-CAL-V03-729273600-734367600.cache \
--space Tau0Tau3 --approximant TaylorT1 --order twoPN \
--channel-name L1:LSC-AS_Q --debug-level 33
\end{verbatim}

\item[Author] 
Duncan Brown
\end{entry}
\clearpage

\subsection{Program \texttt{lalapps\_inspiral}}
\label{program:lalapps-inspiral}
\idx[Program]{lalapps\_inspiral}

\begin{entry}
\item[Name]
\verb$lalapps_inspiral$ --- stand alone inspiral search code

\item[Synopsis]
\begin{verbatim}
  --help                       display this message
  --verbose                    print progress information
  --debug-level LEVEL          set the LAL debug level to LEVEL
  --user-tag STRING            set the process_params usertag to STRING
  --comment STRING             set the process table comment to STRING

  --gps-start-time SEC         GPS second of data start time
  --gps-start-time-ns NS       GPS nanosecond of data start time
  --gps-end-time SEC           GPS second of data end time
  --gps-end-time-ns NS         GPS nanosecond of data end time
  --pad-data T                 pad the data start and end time by T seconds

  --frame-cache                obtain frame data from LAL frame cache FILE
  --calibration-cache FILE     obtain calibration from LAL frame cache FILE
  --channel-name CHAN          read data from interferometer channel CHAN

  --injection-file FILE        inject simulated inspiral signals from FILE
  --inject-overhead            inject signals directly overhead detector

  --bank-file FILE             read template bank parameters from FILE
  --minimal-match M            override bank minimal match with M (sets delta)
  --start-template N           start filtering at template number N in bank
  --stop-templateN             stop filtering at template number N in bank

  --sample-rate F              filter data at F Hz, downsampling if necessary
  --resample-filter TYPE       set resample filter to TYPE (ldas|butterworth)

  --disable-high-pass          turn off the IIR highpass filter
  --enable-high-pass F         high pass data above F Hz using an IIR filter
  --spectrum-type TYPE         use PSD estimator TYPE (mean|median)

  --segment-length N           set data segment length to N points
  --number-of-segments N       set number of data segments to N
  --segment-overlap N          overlap data segments by N points

  --low-frequency-cutoff F     do not filter below F Hz
  --inverse-spec-length T      set length of inverse spectrum to T seconds
  --dynamic-range-exponent X   set dynamic range scaling to 2^X

  --chisq-bins P               set number of chisq veto bins to P
  --snr-threshold RHO          set signal-to-noise threshold to RHO
  --chisq-threshold X          threshold on chi^2 < X * ( p + rho^2 * delta^2 )
  --enable-event-cluster       turn on maximization over chirp length
  --disable-event-cluster      turn off maximization over chirp length

  --approximant APPROX         approximant to be used. {TaylorF2|BCV}

  --enable-output              write the results to a LIGO LW XML file
  --disable-output             do not write LIGO LW XML output file

  --write-raw-data             write raw data to a frame file
  --write-filter-data          write data that is passed to filter to a frame
  --write-response             write the computed response function to a frame
  --write-spectrum             write the uncalibrated psd to a frame
  --write-snrsq                write the snr time series for each data segment
  --write-chisq                write the r^2 time series for each data segment
\end{verbatim}

\item[Description] 
\verb$lalapps_inspiral$ is a stand alone code for performing matched filtering
of LIGO data for gravitational wave signals and Monte Carlo analysis.

\item[Options]\leavevmode
\begin{entry}
\item[\texttt{--help}] Display a brief usage summary.
\end{entry}

\item[Example]
\begin{verbatim}
lalapps_inspiral \
--enable-output --inverse-spec-length 16  --segment-length 1048576 \
--low-frequency-cutoff 7.000000e+01 --pad-data 8 \
--bank-file L1-TMPLTBANK-734357353-1024.xml \
--sample-rate 4096 --chisq-threshold 20.0 --resample-filter ldas \
--channel-name L1:LSC-AS_Q --calibration-cache \
/ldas_outgoing/calibration/cache_files/L1-CAL-V03-729273600-734367600.cache \
--segment-overlap 524288  --snr-threshold 8.0 \
--frame-cache cache/L-734357345-734361107.cache \
--number-of-segments 7 --dynamic-range-exponent 6.900000e+01 \
--enable-high-pass 5.000000e+01 --debug-level 33 \
--gps-start-time 734357353 --gps-end-time 734358377 \
--chisq-bins 8 --spectrum-type median --enable-event-cluster \
--minimal-match 9.700000e-01 --approximant TaylorF2
\end{verbatim}

\item[Author] 
Duncan Brown
\end{entry}
\clearpage

\subsection{Program \texttt{lalapps\_inca}}
\label{program:lalapps-inca}
\idx[Program]{lalapps\_inca}

\begin{entry}
\item[Name]
\verb$lalapps_inca$ --- program does inspiral coincidence analysis.

\item[Synopsis]
\begin{verbatim}
  --help                    display this message
  --verbose                 print progress information
  --version                 print version information and exit
  --debug-level LEVEL       set the LAL debug level to LEVEL
  --user-tag STRING         set the process_params usertag to STRING
  --ifo-tag STRING          set the ifo-tag to STRING - for file naming
  --comment STRING          set the process table comment to STRING
 
  --gps-start-time SEC      GPS second of data start time
  --gps-end-time SEC        GPS second of data end time
 
  --silde-time SEC          slide all triggers of IFOB by SEC
  --slide-time-ns NS        slide all triggers of IFOB by NS
 
  --ifo-a IFOA              name of first ifo (e.g. L1, H1 or H2)
  --ifo-b IFOB              name of second ifo (e.g. L1, H1 or H2)
 
  --triggered-bank FILE     write a triggered bank insted of doing inca
  --minimal-match M         set minimal match of triggered bank to M
 
  --epsilon ERROR           set effective distance test epsilon (default 2)
  --kappa ERROR             set effective distance test kappa (default 0.01)
  --ifo-b-snr-threshold SNR set minimum snr in IFO B (default 6)
  --ifo-b-range-cut         test range of IFO B to see if sensitive to trigger
  --paramenter-test TEST    set the desired parameters to test coincidence
                            for inca: (m1_and_m2|psi0_and_psi3|mchirp_and_eta)
                            for triggered bank (m1_and_m2|psi0_and_psi3)
  --dm Dm                   mass coincidence window (default 0)
  --dpsi0 Dpsi0             psi0 coincidence window
  --dpsi3 Dpsi3             psi3 coincidence window
  --dmchirp Dmchirp         mchirp coincidence window
  --deta  Deta              eta coincidence window
  --dt Dt                   time coincidence window (milliseconds)
 
  --no-playground           do not select triggers from playground
  --playground-only         only use triggers that are in playground
  --write-uniq-triggers     make sure triggers from IFO A are unique

\end{verbatim}

\textsc{(LIGO Lightweight XML files)}

\item[Description --- Coincidence Testing] 

\verb$lalapps_inca$ performs coincidence on triggers from the inspiral search
code.  At present it works for only two interferometers.  The names of the two
interferometers must be given. Output is written to LIGO lightweight XML files.
Two XML output files are written.  The output files contain \texttt{process},
\texttt{process\_params} and \texttt{search\_summary} tables that describe the
search. The primary ifo output file contains the triggers from \textsc{IFOA}
that are found to be in coincidence with triggers in \textsc{IFOB}. The
secondary output file contains the triggers from \textsc{IFOB} that are found
to be in coincidence with the triggers from \textsc{IFOA}.  Each trigger in the
\textsc{IFOA} file corresponds to the coincident trigger in the \textsc{IFOB}
file, so there may be duplicate \textsc{IFOA} triggers.  To prevent this,
specify the \verb$--write-uniq-triggers$ option.

The output files are named in the standard way for inspiral pipeline output.
The primary triggers are in a file named\\
\begin{center}
\texttt{IFOA-INCA\_IFOTAG\_USERTAG-GPSSTARTTIME-DURATION.xml}\\
\end{center}
and the secondary triggers are in a file named\\
\begin{center}
\texttt{IFOB-INCA\_IFOTAG\_USERTAG-GPSSTARTTIME-DURATION.xml}\\
\end{center}

If a \texttt{--user-tag} or \texttt{--ifo-tag} is not specified on the command
line, the \texttt{\_USERTAG} or \texttt{\_IFOTAG} part of the filename will be
omitted.

The default behaviour outputs triggers during playground times only. To obtain
those triggers that are not in the playground, use the \verb$--no-playground$
flag.     

\texttt{lalapps\_inca} tests for coincidence in time as well as another set of
trigger parameters.  The allowed parameter choices are specified by
\verb$--parameter-test$ \texttt{TEST}, where \texttt{TEST} is one of
m1\_and\_m2, psi0\_and\_psi3, mchirp\_and\_eta.  \texttt{lalapps\_inca} calls
the LAL function \texttt{LALCompareSnglInspiral()} to test if two triggers are
coincident. This first tests that the time of the triggers is coincidence to
within $\delta t$.  It then tests that the other parameters of the trigger are
coincident within the input ranges provided.  Additionally, if demanding
coincidence over m1 and m2, it then tests to that 
%
\begin{equation} 
  \frac{\left|D_\mathrm{IFOA} - D_\mathrm{IFOA}\right|}{D_\mathrm{IFOA}} < 
  \frac{\epsilon}{\rho_\mathrm{IFOB}} + \kappa.  
\end{equation}
% 
This is equivalent to testing that 
%
\begin{equation}\label{snrtest} 
  \left|\rho_\mathrm{IFOB} - \hat{\rho}_\mathrm{IFOB}\right| < 
  \epsilon + \kappa\rho_\mathrm{IFOB},
\end{equation} 
%
where 
%
\begin{equation} 
  \hat{\rho}_\mathrm{IFOB} = \frac{\sigma_\mathrm{IFOB}}
  {\sigma_\mathrm{IFOA}} \rho_\mathrm{IFOA} \, .  
\end{equation} 
%
If all the tests are passed, the events are considered to be coincident and
written to the output file.

The \texttt{--ifo-b-range-cut} option performs a test similar to
(\ref{snrtest}) above to see whether we should expect a trigger in
\textsc{IFOB}.  There are three possibilities, which depend upon the value of
the \textsc{SNRSTAR} threshold for \textsc{IFOB}, denoted
$\rho_\mathrm{IFOB}^{*}$.

\begin{enumerate}

\item In this case, the expected signal to noise ratio in \textsc{IFOB} is
above our threshold:
%
\begin{equation} 
  \rho_\mathrm{IFOB}^{*} < \frac{(\hat{\rho}_\mathrm{IFOB} - \epsilon)}
  {1 + \kappa} , 
\end{equation}
%
so we look for a coincident trigger.  We only keep the 
\textsc{IFOA} trigger if one is found in coincidence in \textsc{IFOB}.

\item In this case, our the allowed range of signal to noise ratio in 
\textsc{IFOB} is partly above and partly below our threshold: 
%
\begin{equation}
  \frac{(\hat{\rho}_\mathrm{IFOB} - \epsilon)} {1 + \kappa} < 
  \rho_\mathrm{IFOB}^{*} <
  \frac{(\hat{\rho}_\mathrm{IFOB} + \epsilon)} {1 - \kappa} .
\end{equation}
%
We search \textsc{IFOB} for triggers and record a coincident trigger if found.
Otherwise, we just record the \textsc{IFOA} trigger.

\item In this case, the trigger is not visible to \textsc{IFOB}:
%
\begin{equation}
  \rho_\mathrm{IFOB}^{*} <
  \frac{(\hat{\rho}_\mathrm{IFOB} + \epsilon)} {1 - \kappa} .
\end{equation}
%
We do not search \textsc{IFOB}, but do keep the trigger from \textsc{IFOA}.

\end{enumerate}

\item[Description --- Triggered Bank] 
  
If the option \verb$--triggered-bank$ is specified, then \texttt{lalapps\_inca}
will produce a triggered template bank from the input xml files.  In this case,
the code expects triggers from only a single interferometer, {IFOA}.  The
triggered bank is formed by first sorting the templates in time, and discarding
any which are before the time specified with \verb$--gps-start-time$ or after
the time specified with \verb$--gps-end-time$.  The templates are then sorted
according to the given \verb$--parameter-test$, which must be one of m1\_and\_m2
or psi0\_and\_psi3.  Duplicate templates are discarded and what remains is
output to the \texttt{FILE} specified by the \verb$--triggered-bank$ argument.
The output file contain \texttt{process}, \texttt{process\_params},
\texttt{search\_summary} and \texttt{sngl\_inspiral} tables.  

\item[Options]\leavevmode
\begin{entry}
\item[\texttt{--no-playground}] Optional.  Record all triggers that are
not in playground data.  The default behavior returns only those triggers
which lie in the playground data set.  

\item[\texttt{--playground-only}] Optional.  Record only triggers that
occur in the playground times.  This is the default behavior.

\item[\texttt{--ifo-a} \textsc{IFOA}] Required. This is the name of the
interferometer to use as the interferometer A in the coincidence algorithm.
It must be a two letter IFO code e.g. \texttt{L1}, \texttt{H1}, etc.

\item[\texttt{--ifo-b} \textsc{IFOB}] Required. This is the name of the
interferometer to use as the interferometer B in the coincidence algorithm.
It must be a two letter IFO code e.g. \texttt{L1}, \texttt{H1}, etc.

\item[\texttt{--epsilon} \textsc{$\epsilon$}] Optional. Set the value of
$\epsilon$ in the effective distance test. If not given the default of
$\epsilon = 2$ will be used.

\item[\texttt{--kappa} \textsc{$\kappa$}] Optional. Set the value of
$\kappa$ in the effective distance test. If not given the default of
$\kappa= 0.01$ will be used.

\item[\texttt{--ifo-b-snr-threshold} \textsc{SNRSTAR}] Optional.  Set the 
value of the signal to noise threshold for \textsc{IFOB}.  This is used in
determining which triggers \textsc{IFOB} has a chance to see.  The default
value 

\item[\texttt{--ifo-b-range-cut}] Optional.  Use effective distance test to
see whether \textsc{IFOB} has a chance of seeing trigger before performing 
the search.

\item[\texttt{--parameter-test} TEST] Required. Choose which parameters to use 
when testing for coincidence (m1\_and\_m2|psi0\_and\_psi3|mchirp\_and\_eta).  
Depending on which test is chosen, the allowed windows on the appropriate 
parameters should be set as described below.

\item[\texttt{--dm} \textsc{$\delta m$}] Optional. Accept triggers as
coincident if both m1 and m2 agree within $\delta m$.  If not
supplied,  then $\delta m = 0$.

\item[\texttt{--dpsi0} \textsc{$\delta \psi_0$}] Optional. Accept triggers as
coincident if \textsc{$\psi_0$} parameters agree within $\delta \psi_0$.  If not
supplied,  then $\delta  \psi_0 = 0$.

\item[\texttt{--dpsi3} \textsc{$\delta \psi_3$}] Optional. Accept triggers as
coincident if \textsc{$\psi_3$} parameters agree within $\delta \psi_3$. 
 If not supplied,  then $\delta  \psi_3 = 0$.

\item[\texttt{--dmchirp} \textsc{$\delta mchirp$}] Optional. Accept triggers as
coincident if mchirp agrees within $\delta mchirp$.  If not
supplied,  then $\delta mchirp = 0$.

\item[\texttt{--deta} \textsc{$\delta \eta$}] Optional. Accept triggers as
coincident if $eta$ agrees within $\delta \eta$.  If not
supplied,  then $\delta \eta = 0$.

\item[\texttt{--dt} \textsc{$\delta t$}] Optional. Accept triggers as
coincident if their end times agree within $\delta t$ milliseconds.  If not
supplied,  then $\textsc{$\delta t$} = 0$.

\item[\texttt{--gps-start-time} \textsc{GPS seconds}] Required.  Look for
coincident triggers with end times after \textsc{GPS seconds}.

\item[\texttt{--gps-end-time} \textsc{GPS seconds}] Required.  Look for
coincident triggers with end times before \textsc{GPS seconds}.

\item[\texttt{--write-uniq-triggers}] Optional.  The default behavior is to
only write all triggers from IFO A. However, a trigger from IFO A
may match two or more triggers from IFO B, so it may be duplicated in the 
output. Specifying this option causes only unique IFO A triggers to be
written.

\item[\texttt{--comment} \textsc{string}] Optional. Add \textsc{string} to the
comment field in the process table. If not specified, no comment is added. 

\item[\texttt{--user-tag} \textsc{string}] Optional. Set the user tag for this
job to be \textsc{string}. May also be specified on the command line as 
\texttt{-userTag} for LIGO database compatibility.

\item[\texttt{--help}] Optional.  Print a help message.

\item[\texttt{--debug-level} \textsc{level}] Optional. Set the LAL debug
level to \textsc{level}. If not specified the default is 1.

\end{entry}

\item[Arguments]\leavevmode
\begin{entry}
\item[\texttt{[LIGO Lightweight XML files]}] The arguments to the program
should be a list of LIGO Lightweight XML files containing the triggers from
the two interferometers. The input files can be in any order and do not need
to be time ordered as \texttt{inca} will sort all the triggers once they are
read in. If the program encounters a LIGO Lightweight XML containing triggers
from an unknown interferometer (i.e. not IFO A or IFO B) it will exit with an
error.
\end{entry}

\item[Example]
\begin{verbatim}
lalapps_inca \
--playground-only  --dm 0.03 --kappa 1000.0 --ifo-b H1 --ifo-a L1 \
--user-tag SNR6_INJ --debug-level 33 --gps-start-time 734323079
--gps-end-time 734324999 --epsilon 2.0 --dt 11.0 \
L1-INSPIRAL_INJ-734323015-2048.xml H1-INSPIRAL_INJ-734323015-2048.xml
\end{verbatim}

\item[Algorithm]
The code maintains two pointers to triggers from each ifo,
\texttt{currentTrigger[0]} and \texttt{currentTrigger[1]}, corresponding to
the current trigger from IFO A and B respectively.

\begin{enumerate}
\item An empty linked list of triggers from each interferometer is created.
Each input file is read in and the code determines which IFO the triggers in
the file correspond to. The triggers are appended to the linked list for the
corresponding interferometer.

\item If there are no triggers read in from either of the interferometers,
the code exits cleanly.

\item The triggers for each interferometer is sorted by the \texttt{end\_time}
of the trigger.

\item \texttt{currentTrigger[0]} is set to point to the first trigger from IFO
A that is after the specified GPS start time for coincidence. If no trigger is
found after the start time, the code exits cleanly.

\item Loop over each trigger from IFO A that occurs before the specified GPS
end time for coincidence:
\begin{enumerate}
\item \texttt{currentTrigger[1]} is set to point to the first trigger from IFO
B that is within the time coincidence window, $\delta t$, of
\texttt{currentTrigger[0]}. If no IFO B trigger exists within this window,
\texttt{currentTrigger[0]} is incremented to the next trigger from IFO A and
the loop over IFO A triggers restarts.

\item If the trigger \texttt{currentTrigger[0]} \emph{is, is not} in the
playground data, start looping over triggers from IFO B.
\begin{enumerate}
\item For each trigger from IFO B that is within $\delta t$ of
\texttt{currentTrigger[0]}
\item Call \texttt{LALCompareSnglInspiral()} to check if the triggers match as
determined by the options on the command line. If the trigger match, record
them for later output as coincident triggers.
\end{enumerate}

\item Increment \texttt{currentTrigger[0]} and continue loop over triggers
from IFO A.
\end{enumerate}
\end{enumerate}

\item[Author] 
Patrick Brady, Duncan Brown and Steve Fairhurst
\end{entry}


\clearpage
\subsection{Program \texttt{lalapps\_sire}}
\label{program:lalapps-sire}
\idx[Program]{lalapps\_sire}

\begin{entry}
\item[Name]
\verb$lalapps_sire$ --- single inspiral trigger reader and inspiral injections
analysis

\item[Synopsis]
\verb$lalapps_sire$ \quad
[\texttt{--help}]
[\texttt{--verbose}]
[\texttt{--debug-level} \textsc{level}] \newline
%
[\texttt{--user-tag} \textsc{usertag}] 
[\texttt{--comment} \textsc{comment}] \newline
%
(\texttt{--glob} \textsc{globfiles} $|$ \texttt{--input} \textsc{inputfile})
\texttt{--output} \textsc{outfile}
[\texttt{--summary-file} \textsc{summfile}] \newline
%
(\texttt{--playground-only} $|$ 
\texttt{--exclude-playground} $|$
\texttt{--all-data})\newline
%
[\texttt{--snr-threshold} \textsc{$\rho_\ast$}]\quad
[\texttt{--cluster-algorithm} \textsc{choice}
\texttt{--cluster-time} \textsc{$t_\mathrm{coinc}$}]\newline
%
[\texttt{--injection-file} \textsc{injfile} 
\texttt{--injection-coincidence} \textsc{$t_\mathrm{inj}$}\newline
\quad[\texttt{--missed-injections} \textsc{missedfile}]
[\texttt{--hardware-injections} \textsc{$t_\mathrm{hardware}$}]\newline
%
[\texttt{--tama-output} \textsc{tamafile}] \quad
[\texttt{--disable-trig-start-time}]

\item[Description] 
\verb$lalapps_sire$ processes the LIGO lightweight XML files produced by the
standalone inspiral analysis code \verb$lalapps_inspiral$ or the inspiral
coincidence analysis code \verb$lalapps_inca$. It can be used to concatenate
individual \verb$sngl_inspiral$ tables from multiple XML files which contain a
\verb$search_summary$ table into a single XML file. This may be performed with
or without clustering and signal-to-noise ratio cuts. It can also write a
summary file containing the number of triggers and the total time analyzed,
computed from the \verb$search_summary$ table.

The list of input files may be specified by either of POSIX system glob,
\textsc{globfiles}, or by giving the path to a text file, \textsc{inputfile},
that contains relative or absolute paths to the required input files.

If the \texttt{--injection-file} option is specified, \verb$lalapps_sire$ also
reads in a list of \verb$sim_inspiral$ rows from the file \textsc{injfile}.
It determines how may of the injections have a trigger coincident to within
$t_\mathrm{inj}$ milliseconds. The output file, \textsc{outfile}, will contain
\verb$sim_inspiral$ rows for all coincident (found) events. The summary file
will contain numbers of missed and found events and the efficiency of
detection of the injections.  Only injections that are within the input data
times are processed, so the injection file can span a time larger than the
input data and efficiencies will be correct. Missed injections can be written
to the file \textsc{missedfile}, if desired.

Note that the default LAL debug level for \verb$lalapps_sire$ is $33$, rather
than the usual $1$. This turns of memory allocation and leak checking to
increase the speed of run time. The \texttt{--debug-level} option can be used
to increase the level of debug checking used by the LAL functions.

\item[Options]\leavevmode
\begin{entry}
\item[\texttt{--help}] Optional. Print a help message to the standard output
and then exit with status code $0$.

\item[\texttt{--verbose}] Print progress information to the standard output 
while executing.

\item[\texttt{--debug-level} \textsc{level}] Optional. Set the LAL debug level
to \textsc{level}. Note that the default debug level is $33$ which includes
LAL error checking but excludes memory checking.  To enable memory checking
set the debug level to 1.

\item[\texttt{--user-tag} \textsc{usertag}] Optional. Add the user tag string 
\textsc{usertag} to the \verb$process_params$ table in the output XML file.

\item[\texttt{--comment} \textsc{comment}] Optional. Add the string
\textsc{comment} to the \verb$process$ table in the output XML file.

\item[\texttt{--glob} \textsc{globfiles}] Must be given if the
\texttt{--input} option is not used. Read the input triggers from the
LIGO lightweight XML files that match the regular expression
\textsc{globfiles}. The POSIX system call \verb$glob()$ is used to determine
which files are read in. Mutually exclusive with the \texttt{--input} option.

\item[\texttt{--input} \textsc{inputfile}] Must be given if the
\texttt{--glob} option is not used. Read the input triggers from the list of
LIGO lightweight XML files in \textsc{inputfile} which must be a plain text
file containing relative or absolute paths to the files.  Mutually exclusive
with the \texttt{--glob} option.

\item[\texttt{--output} \textsc{outfile}] Required. Write the concatenated
\verb$sngl_inspiral$ tables to the LIGO lightweight XML file \textsc{outfile}.
If injection analysis is performed the \verb$sim_inspiral$ rows from the input
injection file that are coincident with a trigger are also written to this
file (i.e. the found injections).

\item[\texttt{--playground-only}] Analyze \emph{only} triggers (and
injections) that \emph{are} in playground times specified by the post-S1
playground algorithm.

\item[\texttt{--exclude-playground}] Analyze \emph{only} triggers (and
injections) that \emph{are not} in playground times specified by the post-S1
playground algorithm.

\item[\texttt{--all-data}] Analyze \emph{all} triggers (and injections) from
the input files.

\item[\texttt{--snr-threshold} \textsc{$\rho_\ast$}] Optional. Discard all
input triggers that have a signal-to-noise ratio $\rho < \rho_\ast$.

\item[\texttt{--cluster-algorithm} \textsc{choice}] Optional. Use the
clustering algorithm \textsc{choice} to cluster the \verb$sngl_inspiral$ rows
in the output file before writing them to disk. The options for
\textsc{choice} are \verb$snr_and_chisq$, \verb$snrsq_over_chisq$ or
\verb$snr$. The clustering is performed by the LAL function
\verb$LALClusterSnglInspiralTable()$ and documentation for the clustering can
be found in the \texttt{tools} package of the LAL Software Documentation.

\item[\texttt{--cluster-time} \textsc{$t_\mathrm{coinc}$}] Required if the 
\texttt{--cluster-algorithm} option is specified. Use the time window
$t_\mathrm{coinc}$ for the clustering algorithm.

\item[\texttt{--injection-file} \textsc{injfile}] Optional. If this option is
given, \verb$lalapps_sire$ reads in \verb$sim_inspiral$ rows from the file
\textsc{injfile} and performs an injection alaysis of the triggers.

\item[\texttt{--injection-coincidence} \textsc{$t_\mathrm{inj}$}] Required if
the \texttt{--injection-file} option is specified. Set the injection
coincidence window to $\pm t_\mathrm{inj}$ milliseconds.

\item[\texttt{--missed-injections} \textsc{missedfile}] Optional, can only be
specified if \texttt{--injection-file} has been specified. If any injections
are \emph{not} found, write the \verb$sim_inspiral$ rows for these missed
injections to the LIGO lightweight file \textsc{missedfile}.

\item[\texttt{--hardware-injections} \textsc{$t_\mathrm{hardware}$}] Optional,
can only be specified if \texttt{--injection-file} has been specified.
Increment the end times of the injections read from \textsc{injfile} by
$t_\mathrm{hardware}$ seconds. Used for injection analysis of hardware
injections where the input \verb$sim_inspiral$ rows contain the time offset of
the injection from  $t_\mathrm{hardware}$.

\item[\texttt{--tama-output} \textsc{tamafile}] Optional, if specified produces
an output text file \textsc{tamafile} for use in collaboration with TAMA, in
addition to the usual LIGO lightweight XML file.  The following quantities are
recorded for each trigger in the text file:

\begin{itemize}
\item trigger time (as a double precision real)
\item total mass, $M_{\mathrm{TOT}}$
\item the mass ratio, $\eta$
\item the signal to noise ratio, $\rho$
\item the value of $\chi^2$
\item the effective distance to the trigger, $d_{eff}$.
\end{itemize}


\item[\texttt{--disable-trig-start-time}] Optional, should only be used by
maintainers. Disable checking of the \texttt{--trig-start-time} option in the
input files. \emph{Using this option may caused total analyzed times to be
reported incorrectly.} See note in algorithm section below.
\end{entry}

\item[Example 1] Read in all playground triggers files from the current
directory that match the expression 
\begin{verbatim}
L1-INSPIRAL_INJ-7*
\end{verbatim}
Discard all triggers below signal-to-noise ratio $10$ and report the number of
injections from file 
\begin{verbatim}
HL-INJECTIONS_45-729273613-5094000.xml
\end{verbatim}
that are coincident to within $20$ milliseconds with the remaining triggers.
Write an XML file containing the coincident triggers and injections, an XML
file containing the injections not coincident with a trigger and a text
summary file of the analysis, which will contain the total time analyzed and
the efficiency. Report the progress to the standard output and perform LAL
memory checking:
\begin{verbatim}
lalapps_sire \
--glob "L1-INSPIRAL_INJ-7*" --output out_10.xml --summary-file summ_10.txt \
--playground-only --verbose --debug-level 1 --snr-threshold 10.0 \
--injection-file HL-INJECTIONS_45-729273613-5094000.xml \
--injection-coincidence 20 --missed-injections missed_10.xml
\end{verbatim}

\item[Example 2] Read in all the XML files from the list in the plain text
file \verb$H1-INSPIRAL.txt$ and discard all the triggers that \emph{are} in
the playground. Write the remaining triggers to the XML file
\verb$H1-INSPIRAL.xml$ and write a text summary file containing the time
analyzed to \verb$H1-INSPIRAL_summary.txt$:
\begin{verbatim}
lalapps_sire \
--input H1-INSPIRAL.txt --exclude-playground \
--output H1-INSPIRAL.xml --summary-file H1-INSPIRAL_summary.txt
\end{verbatim}

\item[Notes]
\begin{enumerate}
\item The post-S1 playground algorithm is defined to be
\begin{equation}
t \ \textrm{is playground} \iff t - 729273613 < 600 (\textrm{mod}\ 6370).
\end{equation}

\item If a given trigger \verb$end_time,end_time_ns$, $t_\mathrm{trig}$ is
coincident to within $\pm t_\mathrm{inj}$ seconds of an injection \emph{site}
end time, given by \verb$h_end_time,h_end_time_ns$ or
\verb$l_end_time,l_end_time_ns$ then the injection is considered to be found
and the trigger coincident with an injection.

\item Early versions of the inspiral code contained a bug that causes the
\verb$out_start_time$ column of the \verb$search_summary$ table to be set
incorrectly if a non-zero \texttt{--trig-start-time} option is specified.
\verb$lalapps_sire$ corrects for this by checking for the value of
\texttt{--trig-start-time} in the \verb$process_params$ table and using it to
override the value of \verb$out_start_time$ in the \verb$search_summary$
table. To disable this behaviour, use the \texttt{--disable-trig-start-time}
option. Note that specifying this option may cause some analyzed data times to
be double counted and so the amount of analyzed data will be incorrectly
reported.
\end{enumerate}

\item[Author] 
Patrick Brady, Duncan Brown and Steve Fairhurst
\end{entry}


\clearpage
\subsection{Program \texttt{lalapps\_inspinj}}
\label{program:lalapps-inspinj}
\idx[Program]{lalapps\_inspinj}

\begin{entry}
\item[Name]
\verb$lalapps_inspinj$ --- produces inspiral injection data files.

\item[Synopsis]
\verb$lalapps_inspinj$ \quad
%
[\texttt{--help}]\quad
\texttt{--source-file} \textsc{sfile} \quad
\texttt{--mass-file} \textsc{mfile}\newline
%
[\texttt{--gps-start-time} \textsc{tstart}]\quad 
[\texttt{--gps-end-time} \textsc{tend}]\newline
%
[\texttt{--time-step} \textsc{tstep}]
[\texttt{--time-interval} \textsc{tinterval}]\newline
%
[\texttt{--seed} \textsc{seed}]\quad
[\texttt{--waveform} \textsc{wave}]\quad
[\texttt{--usertag} \textsc{tag}]\newline
%
[\texttt{--tama-output}]\quad
[\texttt{--ilwd}]

\item[Description] 
\verb$lalapps_inspinj$
generates a number of inspiral  parameters suitable  for using in a Monte
Carlo injection to test the efficiency of a inspiral search.  The  various
parameters (detailed  below)  are randomly chosen and are appropriate for a
particular population of binary neutron stars  whose spatial  distribution
includes the Milky Way and a number of extragalactic objects that are  input
in  a  datafile.  The  possible  mass pairs for the binary neutron star com-
panions are also specified in a (different) datafile.

The output of this program  is  a  list  of  the  injected events,  starting
at  the specified start time and ending at the specified end time.  One 
injection with random inspiral parameters will be made every specified time
step, and will be randomly placed within the specified time interval.  
The output is written to a file name in the standard inspiral pipeline format:
\begin{center}
\begin{verbatim}
HL-INJECTIONS_USERTAG_SEED-GPSSTART-DURATION.xml
\end{verbatim}
\end{center}
where \verb$USERTAG$ is \textsc{tag} as specfied on the command line, 
\verb$SEED$ is the  value  of  the random number seed chosen and 
\verb$GPSSTART$ and \verb$DURATION$ describes the GPS time interval that
the file covers. The file is in the standard LIGO lightweight XML format
containing a \texttt{sim\_inspiral} table that describes the injections.
In addition, an ASCII log file called \verb$injlog.txt$ is also written.
If a \texttt{--user-tag} is not specified on the command line, the
\texttt{\_USERTAG} part of the filename will be omitted.

\item[Options]\leavevmode
\begin{entry}
\item[\texttt{--help}] Print a help message.

\item[\texttt{--source-file} \textsc{sfile}]
Optional. Data file containing spatial distribution of  extragalactic  objects.
Default  is  the file \verb+inspsrcs.dat+ provided by LALApps.

\item[\texttt{--mass-file} \textsc{mfile}]
Optional. Data file containing mass pairs  for  the binary  neutron  star
companions.   Default is the file \verb+BNSMasses.dat+ provided by LALApps.

\item[\texttt{--gps-start-time} \textsc{tstart}]
Optional.  Start time of the injection data to be created. Defaults to the
start of S2, Feb 14 2003 16:00:00 UTC (GPS time 729273613)

\item[\texttt{--gps-end-time} \textsc{tend}]
Optional. End time of the injection data to be created. Defaults to the end of
S2, Apr 14 2003 15:00:00 UTC (GPS time 734367613).

\item[\texttt{--time-step} \textsc{tstep}]
Optional. Sets the time step interval between injections. The injections will
occur with an average spacing of \textsc{tstep} seconds. Defaults to 
$2630/\pi$.

\item[\texttt{--time-interval} \textsc{tinterval}]
Optional. Sets the time interval during which an injection can occur. 
Injections are uniformly distributed over the interval.  Setting \textsc{tstep}
to $6370$ and \textsc{tinterval} to 600 guarantees there will be one injection
into each playground segment and they will be randomly distributed within the
playground times.

\item[\texttt{--seed} \textsc{seed}]
Optional. Seed the random number generator with the integer \textsc{seed}.
Defaults to $1$.

\item[\texttt{--waveform} \textsc{wave}]
Optional. The string \textsc{wave} will be written into the \texttt{waveform}
column of the \texttt{sim\_inspiral} table output. This is used by the
inspiral code to determine which type of waveforms it should inject into the
data. Defaults is \texttt{GeneratePPNtwoPN}.

\item[\texttt{--user-tag} \textsc{string}] Optional. Set the user tag for this
job to be \textsc{string}. May also be specified on the command line as 
\texttt{-userTag} for LIGO database compatibility.

\item[\texttt{--tama-output}]
Optional.  If this option is given, \verb+lalapps_inspinj+ also produces a 
text output file:
\begin{center}
\begin{verbatim}
HLT-INJECTIONS_USERTAG_SEED-GPSSTART-DURATION.txt
\end{verbatim}
\end{center}
which contains the following fields:

\begin{itemize}
\item geocentric end time
\item Hanford end time
\item Livingston end time
\item Tama end time
\item total mass, $M_{\mathrm{TOT}}$
\item mass ratio, $\eta$
\item distance to source (in Mpc)
\item longitude
\item latitude
\item inclination
\item coalescence phase
\item polarization
\end{itemize}

In the above, all times are recorded as double precision real numbers and all
angles are in radians.  

\item[\texttt{--ilwd}] Optional. If this option is given,
\verb+lalapps_inspinj+ also produces two ILWD-format files, injepochs.ilwd and
injparams.ilwd, that contain, respectively, the  GPS  times  suitable for
inspiral injections, and the intrinsic inspiral signal parameters to be used
for  those injections.

The  file  injepochs.ilwd  contains  a sequence of integer pairs representing
the injection GPS time in  seconds  and residual  nano-seconds.   The file
injparams.ilwd contains the intrinsic binary parameters for each injection,
which is  a  sequence  of  eight  real  numbers representing (in order) (1) the
total mass of the binary system  (in  solar masses),  (2)  the  dimensionless
reduced mass --- reduced mass per unit total mass --- in the range from  0
(extreme mass  ratio)  to  0.25 (equal masses), (3) the distance to the system
in meters, (4) the inclination  of  the  binary system  orbit  to the plane of
the sky in radians, (5) the coalescence phase in radians, (6)  the  longitude
to  the direction  of  the  source in radians, (7) the latitude to the
direction of the source in radians, (8) and the polar- ization angle of the
source in radians.
\end{entry}

\item[Example]
\begin{verbatim}
lalapps_inspinj --seed 45\
--source-file inspsrcs.dat --mass-file BNSMasses.dat
\end{verbatim}

\item[Environment]\leavevmode
\begin{entry}
\item[LALAPPS\_DATA\_PATH] Directory to look for the default mass
file \verb+BNSMasses.dat+ and the default source file \verb+inspsrcs.dat+.
\end{entry}


\item[Author] 
Jolien Creighton, Patrick Brady, Duncan Brown
\end{entry}







%%%%%%%%%%%%%%%%%%%%%%%%%%%%%%%%%%%%%%%%%%%%%%%%%%%%%%%%%%%%%%%%%%%%%%%%%%%%%%
%
% PROGRAM:   inspmultiawg
% 
%%%%%%%%%%%%%%%%%%%%%%%%%%%%%%%%%%%%%%%%%%%%%%%%%%%%%%%%%%%%%%%%%%%%%%%%%%%%%%
\clearpage
\subsection{Program \texttt{lalapps\_inspmultiawg}}
\label{program:lalapps-inspmultiawg}
\idx[Program]{lalapps\_inspmultiawg}

\begin{entry}
\item[Name]
\verb$lalapps_inspmultiawg$ --- injects specified inspiral chirps into zero
data.  Intended for producing the hardware injection data.

\item[Synopsis]
\begin{verbatim}
--help               display this message
--source SFILE       source file containing details of injection
--actuation ACTFILE  file containing the actuation function
--darm2inj DCFACTOR  calibration between darm_ctrl and injection point
--summary SUMFILE    write details of injections to file
--ifo IFO            name of interferomter (optional)
--flow FSTART        start frequency of injection (default 40 Hz)
--fhigh FSTOP        end frequency of injection (default: end at ISCO)
--smooth QFAC        ringdown the end of the injection with Q factor Qfac
--length LENGTH      length of the data (default 64 seconds)
--samplerate FREQ    rate at which data is sampled (default 16384Hz)
--debug-level DEBUG  the lal debug level
--user-tag TAG       user-tag added to output file names
\end{verbatim}

\item[Description] 
\verb$lalapps_inspmultiawg$ injects inspiral chirps into zero data.  The 
details of several chirps can be specified using the command \verb$--source$,  
otherwise, a single inspiral of two 1.4 solar mass neutron stars will be 
injected.  Each chirp is injected into a new file containing zero data of
$\texttt{LENGTH}$ seconds, sampled at $\texttt{FREQ}$ Hz, and each injection begins at the beginning of the data.  The actuation function can be provided in
$\texttt{ACTFILE}$.  Additionally, there is a parameter $\texttt{DCFACTOR}$ 
which is used to convert between the actuation function and the "response 
function" between ETM\_EXC and strain.  The chirp is output to a file
named
\begin{verbatim}
TAG_inspiral_NUMBER_IFO.out
\end{verbatim}
where $\texttt{TAG}$ is the user-tag, $\texttt{IFO}$ is the name of the 
interferometer, and $\texttt{NUMBER}$ is the injection number.  A summary of
the injections performed can be saved in $\texttt{SUMFILE}$.

\item[Options]\leavevmode
\begin{entry}
\item[\texttt{--sourcefile} \textsc{sfile}] Optional.  Reads source information 
from the file \textsc{sfile}.  If absent, it injects a single 
1.4$M_\odot$--1.4$M_\odot$ inspiral, optimally oriented, at a distance
of $1.0 \textrm{Mpc}$.

\item[\texttt{--actuation} \textsc{actfile}] Optional. Reads a detector
 actuation function from the file \textsc{actfile}.  If absent, it generates 
raw dimensionless strain.  The actuation function should contain 3 columns; frequency, followed by amplitude and phase. 


\item[\texttt{--darm2inj} \textsc{dcfactor}] Optional.  Multiplicative
factor used in calculating the response function from the actuation
function.  More specifically, the actuation function is given by 
%
\begin{equation}
   x_{C} = A(f) \times \mathrm{DARM}\_\mathrm{CRTL} \, .
\end{equation}
%
while we are taking the response function to be:
%
\begin{equation}
   h(f) = R(f) \times \mathrm{ETM}\_\mathrm{EXC} \, .
\end{equation}
%
Then, $\textsc{dcfactor}$ is given by
%
\begin{equation}
   A(f) = \mathrm{DCFACTOR} \times R(f) \, .
\end{equation}

\item[\texttt{--smooth} \textsc{qfac}]  Optional.  This will smooth out
the end of the injected inspiral.  After the inspiral, the frequency of
the waveform will be held constant while the amplitude is exponentially
decreased as $a = a_{\mathrm{final}} \exp(- \pi f_{\mathrm{final}} t /
\textsc{qfac})$.  Here, $t$ is the amount of time after the end of the
inspiral.

\item[\texttt{--summary} \textsc{sumfile}] Optional. The \textsc{sumfile} format
is LIGO lightweight XML with \texttt{process}, \texttt{process\_params} and
\texttt{sim\_inspiral} tables.  The \texttt{sim\_inspiral} table contains
details of all the injections performed.  The details of the injection are
obtained from the source file.

\item[\texttt{--length} \textsc{sec}] Optional.  Specify the length of data into
which the signal will be injected.  The default is 64 seconds.

\item[\texttt{--samplerate} \textsc{freq}] Optional.  Specify the rate at which
the data is sampled.  The default is 16384 Hz.

\item[\texttt{--ifo} \textsc{ifo}] Optional.  Give the name of the
interferometer for which the injections are intended.  This is only used in
naming the output files.

\item[\texttt{--user-tag} \textsc{tag}] Optional.  A user-tag which is
used in naming the output files.

\item[\texttt{--flow} \textsc{fstart}] Optional.  Give the start frequency
\textsc{fstart} for the inspiral.  The default is 40 Hz

\item[\texttt{--fhigh} \textsc{fstop}] Optional.  Give the end frequency
\textsc{fstop} for the inspiral.  The default behaviour is that the inspiral
will continue to ISCO.  If set to a negative number, the generator will use its
absolute value as the terminating frequency, but will ignore post-Newtonian
breakdown. 

\item[\texttt{--debug-level} \textsc{debug}]  Optional.  Set the LAL debug
level.  The default is 33.

\item[\texttt{--help}] Optional.  Print a help message.
\end{entry}

\paragraph{Format for \texttt{sourcefile}:} The source file consists
of any number of lines of data, each specifying a chirp waveform.
Each line must begin with a character code (\verb@CHAR@ equal to one
of \verb@'i'@, \verb@'f'@, or \verb@'c'@), followed by 6
whitespace-delimited numerical fields: the epoch of the chirp
(\verb@INT8@ seconds), the two binary masses (\verb@REAL4@
$M_\odot$), the distance to the source (\verb@REAL4@ Mpc), and the
source's inclination and phase at coalescence (\verb@REAL4@ degrees).
The character codes have the following meanings:
\begin{itemize}
\item[\texttt{'i'}] The epoch represents the GPS time of the start of
the chirp waveform.
\item[\texttt{'f'}] The epoch represents the GPS time of the end of
the chirp waveform.
\item[\texttt{'c'}] The epoch represents the GPS time when the
binaries would coalesce in the point-mass approximation.
\end{itemize}
Since the injection is started at time $t=0$, it is recommended that the
\texttt{'i'} option is used.

Thus a typical input line for two $1.4M_\odot$ objects at 1.1 Mpc
inclined $30^\circ$ with an initial phase of $45^\circ$, beginning at
70 seconds (after the start of the injections), will have the following line in the input
file:
\begin{verbatim}
i 70 1.4 1.4 1.1 30.0 45.0
\end{verbatim}
The time parameter (in this case 70 sec) does not affect the output data in any
way.  It is simply stored in the \texttt{sim\_inspiral} table of the
\textsc{sumfile}, in order to make analysis of the injections easier.

\paragraph{Format for \texttt{actfile}:} The actuation function $A(f)$
gives the transformation from DARM\_CTRL to $X_{C}$ via $X_{C} = A(f) 
\times \mathrm{DARM}\_\mathrm{CRTL}$.   It is combined with
\textsc{dcfactor} as desribed above to give the response function from
the exitation channel to strain as
$\tilde{h}(f)=R(f) \times \mathrm{ETM}_\mathrm{EXC}(f)$.  This is inverted internally 
to give the \emph{transfer function} $T(f)=1/R(f)$. 

The format for \verb@actfile@ is three columns of \verb@REAL4@ data giving the 
frequency followed by the amplitude and phase of the actuation function
at that frequency.  This is the format which is produced by the
calibration team.

\paragraph{Format for the data output:} The data output in the files
\verb$TAG_inspiral_NUMBER_IFO.txt$ is a single column of \verb$REAL4$ ADC data.

\item[Example]
\begin{verbatim}
lalapps_inspmultiawg --source s3.sources \
--actuation E10-H1-ACTUATION.txt --darm2inj 8000\ 
--summary summ.xml --ifo H1 --smooth 5
\end{verbatim}

\item[Author] 
Steve Fairhurst
\end{entry}
\clearpage


\subsection{Program \texttt{lalapps\_splitbank}}
\label{program:lalapps-splitbank}
\idx[Program]{lalapps\_splitbank}

\begin{entry}
\item[Name]
\verb$lalapps_splitbank$ --- splits a template bank file into several smaller
files

\item[Synopsis]
\verb$lalapps_splitbank$ 
[\verb$--help$]
[\verb$--verbose$]
[\verb$--version$]
[\verb$--debug-level$ \textsc{level}] \newline
[\verb$--user-tag$ \textsc{string}]
[\verb$--comment$ \textsc{string}] \newline
\verb$--bank-file$ \textsc{file}
\verb$--number-of-banks$ \textsc{n}
\verb$--minimal-match$ \textsc{m}

\item[Description] 
\verb$lalapps_splitbank$ splits a LIGO\_LW XML file containing inspiral
templates in a \texttt{sngl\_inspiral} table into several smaller bank
files. This allows a template bank to be split across several inspiral 
jobs and then recombined with \texttt{lalapps\_inca} or
\texttt{lalapps\_sire}.

The name of the output template bank files is derived from the name of 
the input bank file and the number of files that the bank should be split
into. For example, if the input bank file:\\
\begin{center}
\texttt{H1-TRIGBANK\_L1-729330491-2048.xml}\\
\end{center}
is split into 3 output files, then these will be named:\\
\begin{center}
\texttt{H1-TRIGBANK\_L1\_00-729330491-2048.xml}\\
\texttt{H1-TRIGBANK\_L1\_01-729330491-2048.xml}\\
\texttt{H1-TRIGBANK\_L1\_02-729330491-2048.xml}\\
\end{center}
The naming convention is to insert the bank file number after the usertag part
of the filename and before the GPS start time part of the file name.

In the case that the input file contains no templates, empty output bank files
are generated. This is done since DAGman does not implement decision rules
yet, so the nodes in the DAG must be identical regardless of the data flowing
through them.

\item[Options]\leavevmode
\begin{entry}
\item[\texttt{--help}] Optional. Print a usage summary and exit.

\item[\texttt{--verbose}] Optional. Print debugging information to the
standard output while executing.

\item[\texttt{--version}] Optional.  Print the CVS id and exit.

\item[\texttt{--debug-level} \textsc{level}] Optional. Set the LAL debug
level to \textsc{level}. If not specified the default is 1.

\item[\texttt{--user-tag} \textsc{string}] Optional. Set the user tag in
the \texttt{process\_params} table to \textsc{string}. Note that this does
not set the user tag in the output filename, as it does in other inspiral
pipeline tools. The output filename is derived strictly from the number of
output banks and the input bank file name.

\item[\texttt{--comment} \textsc{string}] Optional. Set the comment column in
the output \texttt{process} table to \textsc{string}.

\item[\texttt{--bank-file} \textsc{file}] Required. Read the templates from
the \texttt{sngl\_inspiral} table in \textsc{file}.

\item[\texttt{--number-of-banks} \textsc{n}] Required. Split the input
template banks into \textsc{n} output bank files.

\item[\texttt{--minimal-match} \textsc{m}] Required. Set the minimal match of
the output template bank file to \textsc{m}.
\end{entry}

\item[Example]
\begin{verbatim}
lalapps_splitbank --bank-file L1-TMPLTBANK-732488741-2048.xml \
--number-of-banks 3 --minimal-match 0.97
\end{verbatim}

\item[Algorithm]
\texttt{lalapps\_splitbank} counts the number of templates in the input file.
It increments this by one and divides by the number of template banks to
generate using standard integer division. This gives the upper limit on the
number of templates in a single output file.

\item[Author] 
Duncan Brown
\end{entry}

