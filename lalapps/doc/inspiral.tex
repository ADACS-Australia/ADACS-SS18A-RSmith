%
% API Documentation
% Module inspiral
%
% Generated by epydoc 2.0
% [Thu Sep 25 14:02:41 2003]
%

%%%%%%%%%%%%%%%%%%%%%%%%%%%%%%%%%%%%%%%%%%%%%%%%%%%%%%%%%%%%%%%%%%%%%%%%%%%
%%                          Module Description                           %%
%%%%%%%%%%%%%%%%%%%%%%%%%%%%%%%%%%%%%%%%%%%%%%%%%%%%%%%%%%%%%%%%%%%%%%%%%%%

    \index{inspiral \textit{(module)}|(}
\section{Python Module \texttt{inspiral}}

    \label{inspiral}
Classes needed for the inspiral analysis pipeline. This script produced 
the necessary condor submit and dag files to run the standalone inspiral 
code on LIGO data


%%%%%%%%%%%%%%%%%%%%%%%%%%%%%%%%%%%%%%%%%%%%%%%%%%%%%%%%%%%%%%%%%%%%%%%%%%%
%%                               Variables                               %%
%%%%%%%%%%%%%%%%%%%%%%%%%%%%%%%%%%%%%%%%%%%%%%%%%%%%%%%%%%%%%%%%%%%%%%%%%%%

  \subsection{Variables}

\begin{longtable}{|p{.30\textwidth}|p{.62\textwidth}|l}
\cline{1-2}
\cline{1-2} \centering \textbf{Name} & \centering \textbf{Description}& \\
\cline{1-2}
\endhead\cline{1-2}\multicolumn{3}{r}{\small\textit{continued on next page}}\\\endfoot\cline{1-2}
\endlastfoot\raggedright \_\-\_\-a\-u\-t\-h\-o\-r\-\_\-\_\- & \raggedright \textbf{Value:} 
{\tt '\-D\-u\-n\-c\-a\-n\-~\-B\-r\-o\-w\-n\-~\-{\textless}\-d\-u\-n\-c\-a\-n\-@\-g\-r\-a\-v\-i\-t\-y\-.\-p\-h\-y\-s\-.\-u\-w\-m\-.\-e\-d\-u\-{\textgreater}\-'\-}&\\
\cline{1-2}
\raggedright \_\-\_\-d\-a\-t\-e\-\_\-\_\- & \raggedright \textbf{Value:} 
{\tt '\-\$\-D\-a\-t\-e\-:\-~\-2\-0\-0\-3\-/\-1\-0\-/\-0\-1\-~\-0\-9\-:\-0\-5\-:\-0\-8\-~\-\$\-'\-}&\\
\cline{1-2}
\raggedright \_\-\_\-v\-e\-r\-s\-i\-o\-n\-\_\-\_\- & \raggedright \textbf{Value:} 
{\tt '\-1\-.\-1\-4\-'\-}&\\
\cline{1-2}
\end{longtable}

    \index{inspiral \textit{(module)}!DataFindJob \textit{(class)}|(}

%%%%%%%%%%%%%%%%%%%%%%%%%%%%%%%%%%%%%%%%%%%%%%%%%%%%%%%%%%%%%%%%%%%%%%%%%%%
%%                           Class Description                           %%
%%%%%%%%%%%%%%%%%%%%%%%%%%%%%%%%%%%%%%%%%%%%%%%%%%%%%%%%%%%%%%%%%%%%%%%%%%%

\subsection{Class DataFindJob}

    \label{inspiral:DataFindJob}
\begin{tabular}{cccccccc}
% Line for pipeline.AnalysisJob, linespec=[0]
\multicolumn{4}{r}{\settowidth{\BCL}{pipeline.AnalysisJob}\multirow{2}{\BCL}{pipeline.AnalysisJob}}
&&
  \\\cline{5-5}
  &&&&\multicolumn{1}{c|}{}
&&
  \\
% Line for pipeline.CondorJob, linespec=[0, 1]
\multicolumn{2}{r}{\settowidth{\BCL}{pipeline.CondorJob}\multirow{2}{\BCL}{pipeline.CondorJob}}
&&
&&\multicolumn{1}{|c}{}
  \\\cline{3-3}
  &&\multicolumn{1}{c|}{}
&&
&\multicolumn{1}{|c}{}&
  \\
% Line for pipeline.CondorDAGJob, linespec=[1]
\multicolumn{4}{r}{\settowidth{\BCL}{pipeline.CondorDAGJob}\multirow{2}{\BCL}{pipeline.CondorDAGJob}}
&&\multicolumn{1}{|c}{}
  \\\cline{5-5}
  &&&&\multicolumn{1}{c|}{}
&\multicolumn{1}{|c}{}&
  \\
&&&&\multicolumn{2}{l}{\textbf{DataFindJob}}
\end{tabular}

A LALdataFind job used by the inspiral pipeline. The static options are 
read from the section [datafind] in the ini file. The stdout from 
LALdataFind contains the paths to the frame files and is directed to a 
file in the cache directory named by site and GPS start and end times. 
The stderr is directed to the logs directory. The job always runs in the 
scheduler universe. The path to the executable is determined from the ini 
file.


%%%%%%%%%%%%%%%%%%%%%%%%%%%%%%%%%%%%%%%%%%%%%%%%%%%%%%%%%%%%%%%%%%%%%%%%%%%
%%                                Methods                                %%
%%%%%%%%%%%%%%%%%%%%%%%%%%%%%%%%%%%%%%%%%%%%%%%%%%%%%%%%%%%%%%%%%%%%%%%%%%%

  \subsubsection{Methods}

    \label{inspiral:DataFindJob:__init__}
    \index{inspiral \textit{(module)}!DataFindJob \textit{(class)}!\_\_init\_\_ \textit{(method)}}
    \vspace{0.5ex}

    \noindent\begin{boxedminipage}{\textwidth}

    \raggedright \textbf{\_\_init\_\_}(\textit{self}, \textit{cp})

    \vspace{-1.5ex}

    \rule{\textwidth}{0.5\fboxrule}
    cp = ConfigParser object from which options are read.

    \vspace{1ex}

      Overrides: pipeline.CondorDAGJob.\_\_init\_\_

    \end{boxedminipage}

  \noindent\textbf{Inherited from AnalysisJob:}
    calibration,
    channel,
    get\_config
    \\
  \noindent\textbf{Inherited from CondorDAGJob:}
    add\_var\_arg,
    add\_var\_opt
    \\
  \noindent\textbf{Inherited from CondorJob:}
    add\_arg,
    add\_condor\_cmd,
    add\_ini\_opts,
    add\_opt,
    get\_stderr\_file,
    get\_stdout\_file,
    get\_sub\_file,
    set\_log\_file,
    set\_notification,
    set\_stderr\_file,
    set\_stdout\_file,
    set\_sub\_file,
    write\_sub\_file
    \index{inspiral \textit{(module)}!DataFindJob \textit{(class)}|)}
    \index{inspiral \textit{(module)}!DataFindNode \textit{(class)}|(}

%%%%%%%%%%%%%%%%%%%%%%%%%%%%%%%%%%%%%%%%%%%%%%%%%%%%%%%%%%%%%%%%%%%%%%%%%%%
%%                           Class Description                           %%
%%%%%%%%%%%%%%%%%%%%%%%%%%%%%%%%%%%%%%%%%%%%%%%%%%%%%%%%%%%%%%%%%%%%%%%%%%%

\subsection{Class DataFindNode}

    \label{inspiral:DataFindNode}
\begin{tabular}{cccccccc}
% Line for pipeline.CondorDAGNode, linespec=[0, 0]
\multicolumn{2}{r}{\settowidth{\BCL}{pipeline.CondorDAGNode}\multirow{2}{\BCL}{pipeline.CondorDAGNode}}
&&
&&
  \\\cline{3-3}
  &&\multicolumn{1}{c|}{}
&&
&&
  \\
% Line for pipeline.AnalysisNode, linespec=[0]
\multicolumn{4}{r}{\settowidth{\BCL}{pipeline.AnalysisNode}\multirow{2}{\BCL}{pipeline.AnalysisNode}}
&&
  \\\cline{5-5}
  &&&&\multicolumn{1}{c|}{}
&&
  \\
% Line for pipeline.CondorDAGNode, linespec=[1]
\multicolumn{4}{r}{\settowidth{\BCL}{pipeline.CondorDAGNode}\multirow{2}{\BCL}{pipeline.CondorDAGNode}}
&&\multicolumn{1}{|c}{}
  \\\cline{5-5}
  &&&&\multicolumn{1}{c|}{}
&\multicolumn{1}{|c}{}&
  \\
&&&&\multicolumn{2}{l}{\textbf{DataFindNode}}
\end{tabular}

A DataFindNode runs an instance of datafind in a Condor DAG.


%%%%%%%%%%%%%%%%%%%%%%%%%%%%%%%%%%%%%%%%%%%%%%%%%%%%%%%%%%%%%%%%%%%%%%%%%%%
%%                                Methods                                %%
%%%%%%%%%%%%%%%%%%%%%%%%%%%%%%%%%%%%%%%%%%%%%%%%%%%%%%%%%%%%%%%%%%%%%%%%%%%

  \subsubsection{Methods}

    \label{inspiral:DataFindNode:__init__}
    \index{inspiral \textit{(module)}!DataFindNode \textit{(class)}!\_\_init\_\_ \textit{(method)}}
    \vspace{0.5ex}

    \noindent\begin{boxedminipage}{\textwidth}

    \raggedright \textbf{\_\_init\_\_}(\textit{self}, \textit{job})

    \vspace{-1.5ex}

    \rule{\textwidth}{0.5\fboxrule}
    job = A CondorDAGJob that can run an instance of LALdataFind.

    \vspace{1ex}

      Overrides: pipeline.AnalysisNode.\_\_init\_\_

    \end{boxedminipage}

    \label{inspiral:DataFindNode:get_output}
    \index{inspiral \textit{(module)}!DataFindNode \textit{(class)}!get\_output \textit{(method)}}
    \vspace{0.5ex}

    \noindent\begin{boxedminipage}{\textwidth}

    \raggedright \textbf{get\_output}(\textit{self})

    \vspace{-1.5ex}

    \rule{\textwidth}{0.5\fboxrule}
    Return the output file, i.e. the file containing the frame cache 
    data.

    \vspace{1ex}

      Overrides: pipeline.AnalysisNode.get\_output

    \end{boxedminipage}

    \label{inspiral:DataFindNode:set_end}
    \index{inspiral \textit{(module)}!DataFindNode \textit{(class)}!set\_end \textit{(method)}}
    \vspace{0.5ex}

    \noindent\begin{boxedminipage}{\textwidth}

    \raggedright \textbf{set\_end}(\textit{self}, \textit{time})

    \vspace{-1.5ex}

    \rule{\textwidth}{0.5\fboxrule}
    Set the end time of the datafind query. time = GPS end time of query.

    \vspace{1ex}

      Overrides: pipeline.AnalysisNode.set\_end

    \end{boxedminipage}

    \label{inspiral:DataFindNode:set_ifo}
    \index{inspiral \textit{(module)}!DataFindNode \textit{(class)}!set\_ifo \textit{(method)}}
    \vspace{0.5ex}

    \noindent\begin{boxedminipage}{\textwidth}

    \raggedright \textbf{set\_ifo}(\textit{self}, \textit{ifo})

    \vspace{-1.5ex}

    \rule{\textwidth}{0.5\fboxrule}
    Set the IFO to retrieve data for. Since the data from both Hanford 
    interferometers is stored in the same frame file, this takes the 
    first letter of the IFO (e.g. L or H) and passes it to the 
    --instrument option of LALdataFind. ifo = IFO to obtain data for.

    \vspace{1ex}

      Overrides: pipeline.AnalysisNode.set\_ifo

    \end{boxedminipage}

    \label{inspiral:DataFindNode:set_start}
    \index{inspiral \textit{(module)}!DataFindNode \textit{(class)}!set\_start \textit{(method)}}
    \vspace{0.5ex}

    \noindent\begin{boxedminipage}{\textwidth}

    \raggedright \textbf{set\_start}(\textit{self}, \textit{time})

    \vspace{-1.5ex}

    \rule{\textwidth}{0.5\fboxrule}
    Set the start time of the datafind query. time = GPS start time of 
    query.

    \vspace{1ex}

      Overrides: pipeline.AnalysisNode.set\_start

    \end{boxedminipage}

  \noindent\textbf{Inherited from AnalysisNode:}
    get\_end,
    get\_ifo,
    get\_input,
    get\_start,
    set\_cache,
    set\_input,
    set\_output
    \\
  \noindent\textbf{Inherited from CondorDAGNode:}
    \_\_repr\_\_,
    add\_parent,
    add\_var\_arg,
    add\_var\_opt,
    job,
    set\_log\_file,
    set\_name,
    set\_retry,
    write\_job,
    write\_parents,
    write\_vars
    \index{inspiral \textit{(module)}!DataFindNode \textit{(class)}|)}
    \index{inspiral \textit{(module)}!IncaJob \textit{(class)}|(}

%%%%%%%%%%%%%%%%%%%%%%%%%%%%%%%%%%%%%%%%%%%%%%%%%%%%%%%%%%%%%%%%%%%%%%%%%%%
%%                           Class Description                           %%
%%%%%%%%%%%%%%%%%%%%%%%%%%%%%%%%%%%%%%%%%%%%%%%%%%%%%%%%%%%%%%%%%%%%%%%%%%%

\subsection{Class IncaJob}

    \label{inspiral:IncaJob}
\begin{tabular}{cccccccc}
% Line for pipeline.AnalysisJob, linespec=[0]
\multicolumn{4}{r}{\settowidth{\BCL}{pipeline.AnalysisJob}\multirow{2}{\BCL}{pipeline.AnalysisJob}}
&&
  \\\cline{5-5}
  &&&&\multicolumn{1}{c|}{}
&&
  \\
% Line for pipeline.CondorJob, linespec=[0, 1]
\multicolumn{2}{r}{\settowidth{\BCL}{pipeline.CondorJob}\multirow{2}{\BCL}{pipeline.CondorJob}}
&&
&&\multicolumn{1}{|c}{}
  \\\cline{3-3}
  &&\multicolumn{1}{c|}{}
&&
&\multicolumn{1}{|c}{}&
  \\
% Line for pipeline.CondorDAGJob, linespec=[1]
\multicolumn{4}{r}{\settowidth{\BCL}{pipeline.CondorDAGJob}\multirow{2}{\BCL}{pipeline.CondorDAGJob}}
&&\multicolumn{1}{|c}{}
  \\\cline{5-5}
  &&&&\multicolumn{1}{c|}{}
&\multicolumn{1}{|c}{}&
  \\
&&&&\multicolumn{2}{l}{\textbf{IncaJob}}
\end{tabular}

A lalapps\_inca job used by the inspiral pipeline. The static options are 
read from the section [inca] in the ini file. The stdout and stderr from 
the job are directed to the logs directory. The job always runs in the 
scheduler universe. The path to the executable is determined from the ini 
file.


%%%%%%%%%%%%%%%%%%%%%%%%%%%%%%%%%%%%%%%%%%%%%%%%%%%%%%%%%%%%%%%%%%%%%%%%%%%
%%                                Methods                                %%
%%%%%%%%%%%%%%%%%%%%%%%%%%%%%%%%%%%%%%%%%%%%%%%%%%%%%%%%%%%%%%%%%%%%%%%%%%%

  \subsubsection{Methods}

    \label{inspiral:IncaJob:__init__}
    \index{inspiral \textit{(module)}!IncaJob \textit{(class)}!\_\_init\_\_ \textit{(method)}}
    \vspace{0.5ex}

    \noindent\begin{boxedminipage}{\textwidth}

    \raggedright \textbf{\_\_init\_\_}(\textit{self}, \textit{cp})

    \vspace{-1.5ex}

    \rule{\textwidth}{0.5\fboxrule}
    cp = ConfigParser object from which options are read.

    \vspace{1ex}

      Overrides: pipeline.CondorDAGJob.\_\_init\_\_

    \end{boxedminipage}

  \noindent\textbf{Inherited from AnalysisJob:}
    calibration,
    channel,
    get\_config
    \\
  \noindent\textbf{Inherited from CondorDAGJob:}
    add\_var\_arg,
    add\_var\_opt
    \\
  \noindent\textbf{Inherited from CondorJob:}
    add\_arg,
    add\_condor\_cmd,
    add\_ini\_opts,
    add\_opt,
    get\_stderr\_file,
    get\_stdout\_file,
    get\_sub\_file,
    set\_log\_file,
    set\_notification,
    set\_stderr\_file,
    set\_stdout\_file,
    set\_sub\_file,
    write\_sub\_file
    \index{inspiral \textit{(module)}!IncaJob \textit{(class)}|)}
    \index{inspiral \textit{(module)}!IncaNode \textit{(class)}|(}

%%%%%%%%%%%%%%%%%%%%%%%%%%%%%%%%%%%%%%%%%%%%%%%%%%%%%%%%%%%%%%%%%%%%%%%%%%%
%%                           Class Description                           %%
%%%%%%%%%%%%%%%%%%%%%%%%%%%%%%%%%%%%%%%%%%%%%%%%%%%%%%%%%%%%%%%%%%%%%%%%%%%

\subsection{Class IncaNode}

    \label{inspiral:IncaNode}
\begin{tabular}{cccccccc}
% Line for pipeline.CondorDAGNode, linespec=[0, 0]
\multicolumn{2}{r}{\settowidth{\BCL}{pipeline.CondorDAGNode}\multirow{2}{\BCL}{pipeline.CondorDAGNode}}
&&
&&
  \\\cline{3-3}
  &&\multicolumn{1}{c|}{}
&&
&&
  \\
% Line for pipeline.AnalysisNode, linespec=[0]
\multicolumn{4}{r}{\settowidth{\BCL}{pipeline.AnalysisNode}\multirow{2}{\BCL}{pipeline.AnalysisNode}}
&&
  \\\cline{5-5}
  &&&&\multicolumn{1}{c|}{}
&&
  \\
% Line for pipeline.CondorDAGNode, linespec=[1]
\multicolumn{4}{r}{\settowidth{\BCL}{pipeline.CondorDAGNode}\multirow{2}{\BCL}{pipeline.CondorDAGNode}}
&&\multicolumn{1}{|c}{}
  \\\cline{5-5}
  &&&&\multicolumn{1}{c|}{}
&\multicolumn{1}{|c}{}&
  \\
&&&&\multicolumn{2}{l}{\textbf{IncaNode}}
\end{tabular}

An IncaNode runs an instance of the inspiral coincidence code in a Condor 
DAG.


%%%%%%%%%%%%%%%%%%%%%%%%%%%%%%%%%%%%%%%%%%%%%%%%%%%%%%%%%%%%%%%%%%%%%%%%%%%
%%                                Methods                                %%
%%%%%%%%%%%%%%%%%%%%%%%%%%%%%%%%%%%%%%%%%%%%%%%%%%%%%%%%%%%%%%%%%%%%%%%%%%%

  \subsubsection{Methods}

    \label{inspiral:IncaNode:__init__}
    \index{inspiral \textit{(module)}!IncaNode \textit{(class)}!\_\_init\_\_ \textit{(method)}}
    \vspace{0.5ex}

    \noindent\begin{boxedminipage}{\textwidth}

    \raggedright \textbf{\_\_init\_\_}(\textit{self}, \textit{job})

    \vspace{-1.5ex}

    \rule{\textwidth}{0.5\fboxrule}
    job = A CondorDAGJob that can run an instance of lalapps\_inca.

    \vspace{1ex}

      Overrides: pipeline.AnalysisNode.\_\_init\_\_

    \end{boxedminipage}

    \label{inspiral:IncaNode:get_ifo_a}
    \index{inspiral \textit{(module)}!IncaNode \textit{(class)}!get\_ifo\_a \textit{(method)}}
    \vspace{0.5ex}

    \noindent\begin{boxedminipage}{\textwidth}

    \raggedright \textbf{get\_ifo\_a}(\textit{self})

    \vspace{-1.5ex}

    \rule{\textwidth}{0.5\fboxrule}
    Returns the IFO code of the primary interferometer.

    \vspace{1ex}

    \end{boxedminipage}

    \label{inspiral:IncaNode:get_ifo_b}
    \index{inspiral \textit{(module)}!IncaNode \textit{(class)}!get\_ifo\_b \textit{(method)}}
    \vspace{0.5ex}

    \noindent\begin{boxedminipage}{\textwidth}

    \raggedright \textbf{get\_ifo\_b}(\textit{self})

    \vspace{-1.5ex}

    \rule{\textwidth}{0.5\fboxrule}
    Returns the IFO code of the primary interferometer.

    \vspace{1ex}

    \end{boxedminipage}

    \label{inspiral:IncaNode:get_output}
    \index{inspiral \textit{(module)}!IncaNode \textit{(class)}!get\_output \textit{(method)}}
    \vspace{0.5ex}

    \noindent\begin{boxedminipage}{\textwidth}

    \raggedright \textbf{get\_output}(\textit{self})

    \vspace{-1.5ex}

    \rule{\textwidth}{0.5\fboxrule}
    Returns the file name of output from the inca code. This must be kept 
    synchronized with the name of the output file in inca.c.

    \vspace{1ex}

      Overrides: pipeline.AnalysisNode.get\_output

    \end{boxedminipage}

    \label{inspiral:IncaNode:set_ifo_a}
    \index{inspiral \textit{(module)}!IncaNode \textit{(class)}!set\_ifo\_a \textit{(method)}}
    \vspace{0.5ex}

    \noindent\begin{boxedminipage}{\textwidth}

    \raggedright \textbf{set\_ifo\_a}(\textit{self}, \textit{ifo})

    \vspace{-1.5ex}

    \rule{\textwidth}{0.5\fboxrule}
    Set the interferometer code to use as IFO A. ifo = IFO code (e.g. L1, 
    H1 or H2).

    \vspace{1ex}

    \end{boxedminipage}

    \label{inspiral:IncaNode:set_ifo_b}
    \index{inspiral \textit{(module)}!IncaNode \textit{(class)}!set\_ifo\_b \textit{(method)}}
    \vspace{0.5ex}

    \noindent\begin{boxedminipage}{\textwidth}

    \raggedright \textbf{set\_ifo\_b}(\textit{self}, \textit{ifo})

    \vspace{-1.5ex}

    \rule{\textwidth}{0.5\fboxrule}
    Set the interferometer code to use as IFO B. ifo = IFO code (e.g. L1, 
    H1 or H2).

    \vspace{1ex}

    \end{boxedminipage}

  \noindent\textbf{Inherited from AnalysisNode:}
    get\_end,
    get\_ifo,
    get\_input,
    get\_start,
    set\_cache,
    set\_end,
    set\_ifo,
    set\_input,
    get\_output,
    set\_start
    \\
  \noindent\textbf{Inherited from CondorDAGNode:}
    \_\_repr\_\_,
    add\_parent,
    add\_var\_arg,
    add\_var\_opt,
    job,
    set\_log\_file,
    set\_name,
    set\_retry,
    write\_job,
    write\_parents,
    write\_vars
    \index{inspiral \textit{(module)}!IncaNode \textit{(class)}|)}
    \index{inspiral \textit{(module)}!InspiralError \textit{(class)}|(}

%%%%%%%%%%%%%%%%%%%%%%%%%%%%%%%%%%%%%%%%%%%%%%%%%%%%%%%%%%%%%%%%%%%%%%%%%%%
%%                           Class Description                           %%
%%%%%%%%%%%%%%%%%%%%%%%%%%%%%%%%%%%%%%%%%%%%%%%%%%%%%%%%%%%%%%%%%%%%%%%%%%%

\subsection{Class InspiralError}

    \label{inspiral:InspiralError}
\begin{tabular}{cccccc}
% Line for exceptions.Exception, linespec=[0]
\multicolumn{2}{r}{\settowidth{\BCL}{exceptions.Exception}\multirow{2}{\BCL}{exceptions.Exception}}
&&
  \\\cline{3-3}
  &&\multicolumn{1}{c|}{}
&&
  \\
&&\multicolumn{2}{l}{\textbf{InspiralError}}
\end{tabular}


%%%%%%%%%%%%%%%%%%%%%%%%%%%%%%%%%%%%%%%%%%%%%%%%%%%%%%%%%%%%%%%%%%%%%%%%%%%
%%                                Methods                                %%
%%%%%%%%%%%%%%%%%%%%%%%%%%%%%%%%%%%%%%%%%%%%%%%%%%%%%%%%%%%%%%%%%%%%%%%%%%%

  \subsubsection{Methods}

    \label{inspiral:InspiralError:__init__}
    \index{inspiral \textit{(module)}!InspiralError \textit{(class)}!\_\_init\_\_ \textit{(method)}}
    \vspace{0.5ex}

    \noindent\begin{boxedminipage}{\textwidth}

    \raggedright \textbf{\_\_init\_\_}(\textit{self}, \textit{args}=\texttt{N\-o\-n\-e\-})

      Overrides: exceptions.Exception.\_\_init\_\_

    \end{boxedminipage}

  \noindent\textbf{Inherited from Exception:}
    \_\_getitem\_\_,
    \_\_str\_\_
    \index{inspiral \textit{(module)}!InspiralError \textit{(class)}|)}
    \index{inspiral \textit{(module)}!InspiralJob \textit{(class)}|(}

%%%%%%%%%%%%%%%%%%%%%%%%%%%%%%%%%%%%%%%%%%%%%%%%%%%%%%%%%%%%%%%%%%%%%%%%%%%
%%                           Class Description                           %%
%%%%%%%%%%%%%%%%%%%%%%%%%%%%%%%%%%%%%%%%%%%%%%%%%%%%%%%%%%%%%%%%%%%%%%%%%%%

\subsection{Class InspiralJob}

    \label{inspiral:InspiralJob}
\begin{tabular}{cccccccc}
% Line for pipeline.AnalysisJob, linespec=[0]
\multicolumn{4}{r}{\settowidth{\BCL}{pipeline.AnalysisJob}\multirow{2}{\BCL}{pipeline.AnalysisJob}}
&&
  \\\cline{5-5}
  &&&&\multicolumn{1}{c|}{}
&&
  \\
% Line for pipeline.CondorJob, linespec=[0, 1]
\multicolumn{2}{r}{\settowidth{\BCL}{pipeline.CondorJob}\multirow{2}{\BCL}{pipeline.CondorJob}}
&&
&&\multicolumn{1}{|c}{}
  \\\cline{3-3}
  &&\multicolumn{1}{c|}{}
&&
&\multicolumn{1}{|c}{}&
  \\
% Line for pipeline.CondorDAGJob, linespec=[1]
\multicolumn{4}{r}{\settowidth{\BCL}{pipeline.CondorDAGJob}\multirow{2}{\BCL}{pipeline.CondorDAGJob}}
&&\multicolumn{1}{|c}{}
  \\\cline{5-5}
  &&&&\multicolumn{1}{c|}{}
&\multicolumn{1}{|c}{}&
  \\
&&&&\multicolumn{2}{l}{\textbf{InspiralJob}}
\end{tabular}

A lalapps\_inspiral job used by the inspiral pipeline. The static options 
are read from the sections [data] and [inspiral] in the ini file. The 
stdout and stderr from the job are directed to the logs directory. The 
job runs in the universe specfied in the ini file. The path to the 
executable is determined from the ini file.


%%%%%%%%%%%%%%%%%%%%%%%%%%%%%%%%%%%%%%%%%%%%%%%%%%%%%%%%%%%%%%%%%%%%%%%%%%%
%%                                Methods                                %%
%%%%%%%%%%%%%%%%%%%%%%%%%%%%%%%%%%%%%%%%%%%%%%%%%%%%%%%%%%%%%%%%%%%%%%%%%%%

  \subsubsection{Methods}

    \label{inspiral:InspiralJob:__init__}
    \index{inspiral \textit{(module)}!InspiralJob \textit{(class)}!\_\_init\_\_ \textit{(method)}}
    \vspace{0.5ex}

    \noindent\begin{boxedminipage}{\textwidth}

    \raggedright \textbf{\_\_init\_\_}(\textit{self}, \textit{cp})

    \vspace{-1.5ex}

    \rule{\textwidth}{0.5\fboxrule}
    cp = ConfigParser object from which options are read.

    \vspace{1ex}

      Overrides: pipeline.CondorDAGJob.\_\_init\_\_

    \end{boxedminipage}

  \noindent\textbf{Inherited from AnalysisJob:}
    calibration,
    channel,
    get\_config
    \\
  \noindent\textbf{Inherited from CondorDAGJob:}
    add\_var\_arg,
    add\_var\_opt
    \\
  \noindent\textbf{Inherited from CondorJob:}
    add\_arg,
    add\_condor\_cmd,
    add\_ini\_opts,
    add\_opt,
    get\_stderr\_file,
    get\_stdout\_file,
    get\_sub\_file,
    set\_log\_file,
    set\_notification,
    set\_stderr\_file,
    set\_stdout\_file,
    set\_sub\_file,
    write\_sub\_file
    \index{inspiral \textit{(module)}!InspiralJob \textit{(class)}|)}
    \index{inspiral \textit{(module)}!InspiralNode \textit{(class)}|(}

%%%%%%%%%%%%%%%%%%%%%%%%%%%%%%%%%%%%%%%%%%%%%%%%%%%%%%%%%%%%%%%%%%%%%%%%%%%
%%                           Class Description                           %%
%%%%%%%%%%%%%%%%%%%%%%%%%%%%%%%%%%%%%%%%%%%%%%%%%%%%%%%%%%%%%%%%%%%%%%%%%%%

\subsection{Class InspiralNode}

    \label{inspiral:InspiralNode}
\begin{tabular}{cccccccc}
% Line for pipeline.CondorDAGNode, linespec=[0, 0]
\multicolumn{2}{r}{\settowidth{\BCL}{pipeline.CondorDAGNode}\multirow{2}{\BCL}{pipeline.CondorDAGNode}}
&&
&&
  \\\cline{3-3}
  &&\multicolumn{1}{c|}{}
&&
&&
  \\
% Line for pipeline.AnalysisNode, linespec=[0]
\multicolumn{4}{r}{\settowidth{\BCL}{pipeline.AnalysisNode}\multirow{2}{\BCL}{pipeline.AnalysisNode}}
&&
  \\\cline{5-5}
  &&&&\multicolumn{1}{c|}{}
&&
  \\
% Line for pipeline.CondorDAGNode, linespec=[1]
\multicolumn{4}{r}{\settowidth{\BCL}{pipeline.CondorDAGNode}\multirow{2}{\BCL}{pipeline.CondorDAGNode}}
&&\multicolumn{1}{|c}{}
  \\\cline{5-5}
  &&&&\multicolumn{1}{c|}{}
&\multicolumn{1}{|c}{}&
  \\
&&&&\multicolumn{2}{l}{\textbf{InspiralNode}}
\end{tabular}

An InspiralNode runs an instance of the inspiral code in a Condor DAG.


%%%%%%%%%%%%%%%%%%%%%%%%%%%%%%%%%%%%%%%%%%%%%%%%%%%%%%%%%%%%%%%%%%%%%%%%%%%
%%                                Methods                                %%
%%%%%%%%%%%%%%%%%%%%%%%%%%%%%%%%%%%%%%%%%%%%%%%%%%%%%%%%%%%%%%%%%%%%%%%%%%%

  \subsubsection{Methods}

    \label{inspiral:InspiralNode:__init__}
    \index{inspiral \textit{(module)}!InspiralNode \textit{(class)}!\_\_init\_\_ \textit{(method)}}
    \vspace{0.5ex}

    \noindent\begin{boxedminipage}{\textwidth}

    \raggedright \textbf{\_\_init\_\_}(\textit{self}, \textit{job})

    \vspace{-1.5ex}

    \rule{\textwidth}{0.5\fboxrule}
    job = A CondorDAGJob that can run an instance of lalapps\_inspiral.

    \vspace{1ex}

      Overrides: pipeline.AnalysisNode.\_\_init\_\_

    \end{boxedminipage}

    \label{inspiral:InspiralNode:get_output}
    \index{inspiral \textit{(module)}!InspiralNode \textit{(class)}!get\_output \textit{(method)}}
    \vspace{0.5ex}

    \noindent\begin{boxedminipage}{\textwidth}

    \raggedright \textbf{get\_output}(\textit{self})

    \vspace{-1.5ex}

    \rule{\textwidth}{0.5\fboxrule}
    Returns the file name of output from the inspiral code. This must be 
    kept synchronized with the name of the output file in inspiral.c.

    \vspace{1ex}

      Overrides: pipeline.AnalysisNode.get\_output

    \end{boxedminipage}

    \label{inspiral:InspiralNode:set_bank}
    \index{inspiral \textit{(module)}!InspiralNode \textit{(class)}!set\_bank \textit{(method)}}
    \vspace{0.5ex}

    \noindent\begin{boxedminipage}{\textwidth}

    \raggedright \textbf{set\_bank}(\textit{self}, \textit{bank})

    \end{boxedminipage}

  \noindent\textbf{Inherited from AnalysisNode:}
    get\_end,
    get\_ifo,
    get\_input,
    get\_start,
    set\_cache,
    set\_end,
    set\_ifo,
    set\_input,
    set\_output,
    set\_start
    \\
  \noindent\textbf{Inherited from CondorDAGNode:}
    \_\_repr\_\_,
    add\_parent,
    add\_var\_arg,
    add\_var\_opt,
    job,
    set\_log\_file,
    set\_name,
    set\_retry,
    write\_job,
    write\_parents,
    write\_vars
    \index{inspiral \textit{(module)}!InspiralNode \textit{(class)}|)}
    \index{inspiral \textit{(module)}!TmpltBankJob \textit{(class)}|(}

%%%%%%%%%%%%%%%%%%%%%%%%%%%%%%%%%%%%%%%%%%%%%%%%%%%%%%%%%%%%%%%%%%%%%%%%%%%
%%                           Class Description                           %%
%%%%%%%%%%%%%%%%%%%%%%%%%%%%%%%%%%%%%%%%%%%%%%%%%%%%%%%%%%%%%%%%%%%%%%%%%%%

\subsection{Class TmpltBankJob}

    \label{inspiral:TmpltBankJob}
\begin{tabular}{cccccccc}
% Line for pipeline.AnalysisJob, linespec=[0]
\multicolumn{4}{r}{\settowidth{\BCL}{pipeline.AnalysisJob}\multirow{2}{\BCL}{pipeline.AnalysisJob}}
&&
  \\\cline{5-5}
  &&&&\multicolumn{1}{c|}{}
&&
  \\
% Line for pipeline.CondorJob, linespec=[0, 1]
\multicolumn{2}{r}{\settowidth{\BCL}{pipeline.CondorJob}\multirow{2}{\BCL}{pipeline.CondorJob}}
&&
&&\multicolumn{1}{|c}{}
  \\\cline{3-3}
  &&\multicolumn{1}{c|}{}
&&
&\multicolumn{1}{|c}{}&
  \\
% Line for pipeline.CondorDAGJob, linespec=[1]
\multicolumn{4}{r}{\settowidth{\BCL}{pipeline.CondorDAGJob}\multirow{2}{\BCL}{pipeline.CondorDAGJob}}
&&\multicolumn{1}{|c}{}
  \\\cline{5-5}
  &&&&\multicolumn{1}{c|}{}
&\multicolumn{1}{|c}{}&
  \\
&&&&\multicolumn{2}{l}{\textbf{TmpltBankJob}}
\end{tabular}

A lalapps\_tmpltbank job used by the inspiral pipeline. The static 
options are read from the sections [data] and [tmpltbank] in the ini 
file. The stdout and stderr from the job are directed to the logs 
directory. The job runs in the universe specfied in the ini file. The 
path to the executable is determined from the ini file.


%%%%%%%%%%%%%%%%%%%%%%%%%%%%%%%%%%%%%%%%%%%%%%%%%%%%%%%%%%%%%%%%%%%%%%%%%%%
%%                                Methods                                %%
%%%%%%%%%%%%%%%%%%%%%%%%%%%%%%%%%%%%%%%%%%%%%%%%%%%%%%%%%%%%%%%%%%%%%%%%%%%

  \subsubsection{Methods}

    \label{inspiral:TmpltBankJob:__init__}
    \index{inspiral \textit{(module)}!TmpltBankJob \textit{(class)}!\_\_init\_\_ \textit{(method)}}
    \vspace{0.5ex}

    \noindent\begin{boxedminipage}{\textwidth}

    \raggedright \textbf{\_\_init\_\_}(\textit{self}, \textit{cp})

    \vspace{-1.5ex}

    \rule{\textwidth}{0.5\fboxrule}
    cp = ConfigParser object from which options are read.

    \vspace{1ex}

      Overrides: pipeline.CondorDAGJob.\_\_init\_\_

    \end{boxedminipage}

  \noindent\textbf{Inherited from AnalysisJob:}
    calibration,
    channel,
    get\_config
    \\
  \noindent\textbf{Inherited from CondorDAGJob:}
    add\_var\_arg,
    add\_var\_opt
    \\
  \noindent\textbf{Inherited from CondorJob:}
    add\_arg,
    add\_condor\_cmd,
    add\_ini\_opts,
    add\_opt,
    get\_stderr\_file,
    get\_stdout\_file,
    get\_sub\_file,
    set\_log\_file,
    set\_notification,
    set\_stderr\_file,
    set\_stdout\_file,
    set\_sub\_file,
    write\_sub\_file
    \index{inspiral \textit{(module)}!TmpltBankJob \textit{(class)}|)}
    \index{inspiral \textit{(module)}!TmpltBankNode \textit{(class)}|(}

%%%%%%%%%%%%%%%%%%%%%%%%%%%%%%%%%%%%%%%%%%%%%%%%%%%%%%%%%%%%%%%%%%%%%%%%%%%
%%                           Class Description                           %%
%%%%%%%%%%%%%%%%%%%%%%%%%%%%%%%%%%%%%%%%%%%%%%%%%%%%%%%%%%%%%%%%%%%%%%%%%%%

\subsection{Class TmpltBankNode}

    \label{inspiral:TmpltBankNode}
\begin{tabular}{cccccccc}
% Line for pipeline.CondorDAGNode, linespec=[0, 0]
\multicolumn{2}{r}{\settowidth{\BCL}{pipeline.CondorDAGNode}\multirow{2}{\BCL}{pipeline.CondorDAGNode}}
&&
&&
  \\\cline{3-3}
  &&\multicolumn{1}{c|}{}
&&
&&
  \\
% Line for pipeline.AnalysisNode, linespec=[0]
\multicolumn{4}{r}{\settowidth{\BCL}{pipeline.AnalysisNode}\multirow{2}{\BCL}{pipeline.AnalysisNode}}
&&
  \\\cline{5-5}
  &&&&\multicolumn{1}{c|}{}
&&
  \\
% Line for pipeline.CondorDAGNode, linespec=[1]
\multicolumn{4}{r}{\settowidth{\BCL}{pipeline.CondorDAGNode}\multirow{2}{\BCL}{pipeline.CondorDAGNode}}
&&\multicolumn{1}{|c}{}
  \\\cline{5-5}
  &&&&\multicolumn{1}{c|}{}
&\multicolumn{1}{|c}{}&
  \\
&&&&\multicolumn{2}{l}{\textbf{TmpltBankNode}}
\end{tabular}

A TmpltBankNode runs an instance of the template bank generation job in a 
Condor DAG.


%%%%%%%%%%%%%%%%%%%%%%%%%%%%%%%%%%%%%%%%%%%%%%%%%%%%%%%%%%%%%%%%%%%%%%%%%%%
%%                                Methods                                %%
%%%%%%%%%%%%%%%%%%%%%%%%%%%%%%%%%%%%%%%%%%%%%%%%%%%%%%%%%%%%%%%%%%%%%%%%%%%

  \subsubsection{Methods}

    \label{inspiral:TmpltBankNode:__init__}
    \index{inspiral \textit{(module)}!TmpltBankNode \textit{(class)}!\_\_init\_\_ \textit{(method)}}
    \vspace{0.5ex}

    \noindent\begin{boxedminipage}{\textwidth}

    \raggedright \textbf{\_\_init\_\_}(\textit{self}, \textit{job})

    \vspace{-1.5ex}

    \rule{\textwidth}{0.5\fboxrule}
    job = A CondorDAGJob that can run an instance of lalapps\_tmpltbank.

    \vspace{1ex}

      Overrides: pipeline.AnalysisNode.\_\_init\_\_

    \end{boxedminipage}

    \label{inspiral:TmpltBankNode:get_output}
    \index{inspiral \textit{(module)}!TmpltBankNode \textit{(class)}!get\_output \textit{(method)}}
    \vspace{0.5ex}

    \noindent\begin{boxedminipage}{\textwidth}

    \raggedright \textbf{get\_output}(\textit{self})

    \vspace{-1.5ex}

    \rule{\textwidth}{0.5\fboxrule}
    Returns the file name of output from the template bank code. This 
    must be kept synchronized with the name of the output file in 
    tmpltbank.c.

    \vspace{1ex}

      Overrides: pipeline.AnalysisNode.get\_output

    \end{boxedminipage}

  \noindent\textbf{Inherited from AnalysisNode:}
    get\_end,
    get\_ifo,
    get\_input,
    get\_start,
    set\_cache,
    set\_end,
    set\_ifo,
    set\_input,
    set\_output,
    set\_start
    \\
  \noindent\textbf{Inherited from CondorDAGNode:}
    \_\_repr\_\_,
    add\_parent,
    add\_var\_arg,
    add\_var\_opt,
    job,
    set\_log\_file,
    set\_name,
    set\_retry,
    write\_job,
    write\_parents,
    write\_vars
    \index{inspiral \textit{(module)}!TmpltBankNode \textit{(class)}|)}
    \index{inspiral \textit{(module)}!TrigToTmpltJob \textit{(class)}|(}

%%%%%%%%%%%%%%%%%%%%%%%%%%%%%%%%%%%%%%%%%%%%%%%%%%%%%%%%%%%%%%%%%%%%%%%%%%%
%%                           Class Description                           %%
%%%%%%%%%%%%%%%%%%%%%%%%%%%%%%%%%%%%%%%%%%%%%%%%%%%%%%%%%%%%%%%%%%%%%%%%%%%

\subsection{Class TrigToTmpltJob}

    \label{inspiral:TrigToTmpltJob}
\begin{tabular}{cccccccc}
% Line for pipeline.AnalysisJob, linespec=[0]
\multicolumn{4}{r}{\settowidth{\BCL}{pipeline.AnalysisJob}\multirow{2}{\BCL}{pipeline.AnalysisJob}}
&&
  \\\cline{5-5}
  &&&&\multicolumn{1}{c|}{}
&&
  \\
% Line for pipeline.CondorJob, linespec=[0, 1]
\multicolumn{2}{r}{\settowidth{\BCL}{pipeline.CondorJob}\multirow{2}{\BCL}{pipeline.CondorJob}}
&&
&&\multicolumn{1}{|c}{}
  \\\cline{3-3}
  &&\multicolumn{1}{c|}{}
&&
&\multicolumn{1}{|c}{}&
  \\
% Line for pipeline.CondorDAGJob, linespec=[1]
\multicolumn{4}{r}{\settowidth{\BCL}{pipeline.CondorDAGJob}\multirow{2}{\BCL}{pipeline.CondorDAGJob}}
&&\multicolumn{1}{|c}{}
  \\\cline{5-5}
  &&&&\multicolumn{1}{c|}{}
&\multicolumn{1}{|c}{}&
  \\
&&&&\multicolumn{2}{l}{\textbf{TrigToTmpltJob}}
\end{tabular}

A lalapps\_trigtotmplt job used by the inspiral pipeline. The static 
options are read from the section [trigtotmplt] in the ini file. The 
stdout and stderr from the job are directed to the logs directory. The 
job always runs in the scheduler universe. The path to the executable is 
determined from the ini file.


%%%%%%%%%%%%%%%%%%%%%%%%%%%%%%%%%%%%%%%%%%%%%%%%%%%%%%%%%%%%%%%%%%%%%%%%%%%
%%                                Methods                                %%
%%%%%%%%%%%%%%%%%%%%%%%%%%%%%%%%%%%%%%%%%%%%%%%%%%%%%%%%%%%%%%%%%%%%%%%%%%%

  \subsubsection{Methods}

    \label{inspiral:TrigToTmpltJob:__init__}
    \index{inspiral \textit{(module)}!TrigToTmpltJob \textit{(class)}!\_\_init\_\_ \textit{(method)}}
    \vspace{0.5ex}

    \noindent\begin{boxedminipage}{\textwidth}

    \raggedright \textbf{\_\_init\_\_}(\textit{self}, \textit{cp})

    \vspace{-1.5ex}

    \rule{\textwidth}{0.5\fboxrule}
    cp = ConfigParser object from which options are read.

    \vspace{1ex}

      Overrides: pipeline.CondorDAGJob.\_\_init\_\_

    \end{boxedminipage}

  \noindent\textbf{Inherited from AnalysisJob:}
    calibration,
    channel,
    get\_config
    \\
  \noindent\textbf{Inherited from CondorDAGJob:}
    add\_var\_arg,
    add\_var\_opt
    \\
  \noindent\textbf{Inherited from CondorJob:}
    add\_arg,
    add\_condor\_cmd,
    add\_ini\_opts,
    add\_opt,
    get\_stderr\_file,
    get\_stdout\_file,
    get\_sub\_file,
    set\_log\_file,
    set\_notification,
    set\_stderr\_file,
    set\_stdout\_file,
    set\_sub\_file,
    write\_sub\_file
    \index{inspiral \textit{(module)}!TrigToTmpltJob \textit{(class)}|)}
    \index{inspiral \textit{(module)}!TrigToTmpltNode \textit{(class)}|(}

%%%%%%%%%%%%%%%%%%%%%%%%%%%%%%%%%%%%%%%%%%%%%%%%%%%%%%%%%%%%%%%%%%%%%%%%%%%
%%                           Class Description                           %%
%%%%%%%%%%%%%%%%%%%%%%%%%%%%%%%%%%%%%%%%%%%%%%%%%%%%%%%%%%%%%%%%%%%%%%%%%%%

\subsection{Class TrigToTmpltNode}

    \label{inspiral:TrigToTmpltNode}
\begin{tabular}{cccccccc}
% Line for pipeline.CondorDAGNode, linespec=[0, 0]
\multicolumn{2}{r}{\settowidth{\BCL}{pipeline.CondorDAGNode}\multirow{2}{\BCL}{pipeline.CondorDAGNode}}
&&
&&
  \\\cline{3-3}
  &&\multicolumn{1}{c|}{}
&&
&&
  \\
% Line for pipeline.AnalysisNode, linespec=[0]
\multicolumn{4}{r}{\settowidth{\BCL}{pipeline.AnalysisNode}\multirow{2}{\BCL}{pipeline.AnalysisNode}}
&&
  \\\cline{5-5}
  &&&&\multicolumn{1}{c|}{}
&&
  \\
% Line for pipeline.CondorDAGNode, linespec=[1]
\multicolumn{4}{r}{\settowidth{\BCL}{pipeline.CondorDAGNode}\multirow{2}{\BCL}{pipeline.CondorDAGNode}}
&&\multicolumn{1}{|c}{}
  \\\cline{5-5}
  &&&&\multicolumn{1}{c|}{}
&\multicolumn{1}{|c}{}&
  \\
&&&&\multicolumn{2}{l}{\textbf{TrigToTmpltNode}}
\end{tabular}

A TrigToTmpltNode runs an instance of the triggered bank generator in a 
Condor DAG.


%%%%%%%%%%%%%%%%%%%%%%%%%%%%%%%%%%%%%%%%%%%%%%%%%%%%%%%%%%%%%%%%%%%%%%%%%%%
%%                                Methods                                %%
%%%%%%%%%%%%%%%%%%%%%%%%%%%%%%%%%%%%%%%%%%%%%%%%%%%%%%%%%%%%%%%%%%%%%%%%%%%

  \subsubsection{Methods}

    \label{inspiral:TrigToTmpltNode:__init__}
    \index{inspiral \textit{(module)}!TrigToTmpltNode \textit{(class)}!\_\_init\_\_ \textit{(method)}}
    \vspace{0.5ex}

    \noindent\begin{boxedminipage}{\textwidth}

    \raggedright \textbf{\_\_init\_\_}(\textit{self}, \textit{job})

    \vspace{-1.5ex}

    \rule{\textwidth}{0.5\fboxrule}
    job = A CondorDAGJob that can run an instance of 
    lalapps\_trigtotmplt.

    \vspace{1ex}

      Overrides: pipeline.AnalysisNode.\_\_init\_\_

    \end{boxedminipage}

  \noindent\textbf{Inherited from AnalysisNode:}
    get\_end,
    get\_ifo,
    get\_input,
    get\_output,
    get\_start,
    set\_cache,
    set\_end,
    set\_ifo,
    set\_input,
    set\_output,
    set\_start
    \\
  \noindent\textbf{Inherited from CondorDAGNode:}
    \_\_repr\_\_,
    add\_parent,
    add\_var\_arg,
    add\_var\_opt,
    job,
    set\_log\_file,
    set\_name,
    set\_retry,
    write\_job,
    write\_parents,
    write\_vars
    \index{inspiral \textit{(module)}!TrigToTmpltNode \textit{(class)}|)}
    \index{inspiral \textit{(module)}|)}

\clearpage

\section{Inspiral Search Programs}
\label{section:inspiral}

This section of \textsc{LALApps} contains programs that can be used to search
interferometer data for inspiral signals using templated matched filtering and
associated veto stratergies.

\clearpage
\subsection{Program \texttt{lalapps\_inspiral\_pipe}}
\label{program:inspiral-pipeline}
\idx[Program]{inspiral\_pipeline.py}

\begin{entry}
\item[Name]
\verb$lalapps_inspiral_pipe$ --- python script to generate Condor DAGs to
run the inspiral pipeline.

\item[Synopsis]
\begin{verbatim}
  -h, --help               display this message
  -v, --version            print version information and exit
  -u, --user-tag TAG       tag the job with TAG (overrides value in ini file)

  -d, --datafind           run LALdataFind to create frame cache files
  -t, --template-bank      run lalapps_tmpltbank to generate a template bank
  -i, --inspiral           run lalapps_inspiral on the first IFO
  -T, --triggered-bank     run lalapps_trigtotmplt to generate a triggered bank
  -I, --triggered-inspiral run lalapps_inspiral on the second IFO
  -C, --coincidence        run lalapps_inca on the triggers from both IFOs

  -j, --injections         add simulated inspirals from injection file

  -p, --playground-only    only create chunks that overlap with playground
  -P, --priority PRIO      run jobs with condor priority PRIO

  -f, --config-file FILE   use configuration file FILE
  -l, --log-path PATH      directory to write condor log file
\end{verbatim}

\item[Description] \verb$lalapps_inspiral_pipe$ generates a Condor DAG to run
the inspiral analysis pipeline. The configuration file should specify the
parameters needed to run the jobs and must be specified with the
\verb$--config-file$ option.  A file containing science segments to be
analyzed should be specified in the \verb$[input]$ section of the
configuration file with a line such as
\begin{verbatim}
segments = S2H1L1v03_selectedsegs.txt
\end{verbatim}
This should contain four whitespace separated coulumns:
\begin{verbatim}
  segment_id    gps_start_time  gps_end_time    duration
\end{verbatim}
that define the science segments to be used. Lines starting with an octothorpe
are ignored.

The analysis chunk size is determined from the number of data segments and
thier length and overlap specified in config file. A chunk length is typically
this is 1024 seconds for S2.  The chunks start and stop times are computed
from the science segment times and used to build the DAG.

The once the DAG file has been created it should be submitted to the Condor
pool with the \verb$condor_submit_dag$ command.

\item[Options]\leavevmode
\begin{entry}
\item[\texttt{--help}] Display a brief usage summary.
\end{entry}

\item[Example]
Generate a DAG to run an inspiral search on the first IFO. The generated
DAG is then submitted with \texttt{condor\_submit\_dag}
\begin{verbatim}
lalapps_inspiral_pipe --log-path /people/duncan/dag_logs \
--datafind --template-bank --inspiral --playground-only \
--config-file l1_s2.ini

condor_submit_dag l1_s2.dag
\end{verbatim}

\item[Author] 
Duncan Brown
\end{entry}

\clearpage
\subsection{Program \texttt{lalapps\_tmpltbank}}
\label{program:lalapps-tmpltbank}
\idx[Program]{lalapps\_tmpltbank}

\begin{entry}
\item[Name]
\verb$lalapps_tmpltbank$ --- program to generate inspiral template banks.

\item[Synopsis]
\begin{verbatim}
  --help                       display this message
  --verbose                    print progress information
  --debug-level LEVEL          set the LAL debug level to LEVEL
  --user-tag STRING            set the process_params usertag to STRING
  --comment STRING             set the process table comment to STRING

  --gps-start-time SEC         GPS second of data start time
  --gps-end-time SEC           GPS second of data end time
  --pad-data T                 pad the data start and end time by T seconds

  --frame-cache                obtain frame data from LAL frame cache FILE
  --calibration-cache FILE     obtain calibration from LAL frame cache FILE
  --channel-name CHAN          read data from interferometer channel CHAN

  --sample-rate F              filter data at F Hz, downsampling if necessary
  --resample-filter TYPE       set resample filter to TYPE [ldas|butterworth]

  --disable-high-pass          turn off the IIR highpass filter
  --enable-high-pass F         high pass data above F Hz using an IIR filter
  --spectrum-type TYPE         use PSD estimator TYPE [mean|median]

  --segment-length N           set data segment length to N points
  --number-of-segments N       set number of data segments to N

  --low-frequency-cutoff F     do not filter below F Hz
  --high-frequency-cutoff F    upper frequency cutoff in Hz

  --minimum-mass MASS          set minimum component mass of bank to MASS
  --maximum-mass MASS          set maximum component mass of bank to MASS
  --minimal-match M            generate bank with minimal match M

  --order ORDER                set post-Newtonian order of the waveform to ORDER
                                 (newtonian|oneHalfPN|onePN|onePointFivePN|
                                 twoPN|twoPointFive|threePN|threePointFivePN)
  --approximant APPROX         set approximant of the waveform to APPROX
                                 (TaylorT1|TaylorT2|TaylorT3|TaylorF1|TaylorF2|
                                 PadeT1|PadeT2|EOB|BCV|SpinTaylorT3)
  --space SPACE                grid up template bank with mass parameters SPACE
                                 (Tau0Tau2|Tau0Tau3)

  --write-raw-data             write raw data to a frame file
  --write-response             write the computed response function to a frame
  --write-spectrum             write the uncalibrated psd to a frame
  --write-strain-spectrum      write the calibrated strain psd to a text file
\end{verbatim}

\item[Description] 
\verb$lalapps_tmpltbank$ is a stand alone code for generating inspiral
template banks for LIGO data with the LAL bank package.  The code generates a
calibrated power spectrum at the specified time for the requested channel and
uses this to compute the template bank.  See the LAL bank package
documentation for detailed information on the algorithms used to generate the
banks.

\item[Options]\leavevmode
\begin{entry}
\item[\texttt{--help}] Display a brief usage summary.
\end{entry}

\item[Example]
\begin{verbatim}
lalapps_tmpltbank \
--gps-start-time 734357353 --gps-end-time 734358377 \
--frame-cache cache/L-734357345-734361107.cache \
--segment-length 1048576 --number-of-segments 7 \
--pad-data 7 --sample-rate 4096 --resample-filter ldas \
--enable-high-pass 5.000000e+01 --spectrum-type median
--low-frequency-cutoff 7.000000e+01 --high-frequency-cutoff 2.048000e+03 \
--minimum-mass 1.000000e+00  --maximum-mass 3.000000e+00 \
--minimal-match 9.700000e-01 --calibration-cache  \
/ldas_outgoing/calibration/cache_files/L1-CAL-V03-729273600-734367600.cache \
--space Tau0Tau3 --approximant TaylorT1 --order twoPN \
--channel-name L1:LSC-AS_Q --debug-level 33
\end{verbatim}

\item[Author] 
Duncan Brown
\end{entry}
\clearpage

\subsection{Program \texttt{lalapps\_inspiral}}
\label{program:lalapps-inspiral}
\idx[Program]{lalapps\_inspiral}

\begin{entry}
\item[Name]
\verb$lalapps_inspiral$ --- stand alone inspiral search code

\item[Synopsis]
\begin{verbatim}
  --help                       display this message
  --verbose                    print progress information
  --debug-level LEVEL          set the LAL debug level to LEVEL
  --user-tag STRING            set the process_params usertag to STRING
  --comment STRING             set the process table comment to STRING

  --gps-start-time SEC         GPS second of data start time
  --gps-start-time-ns NS       GPS nanosecond of data start time
  --gps-end-time SEC           GPS second of data end time
  --gps-end-time-ns NS         GPS nanosecond of data end time
  --pad-data T                 pad the data start and end time by T seconds

  --frame-cache                obtain frame data from LAL frame cache FILE
  --calibration-cache FILE     obtain calibration from LAL frame cache FILE
  --channel-name CHAN          read data from interferometer channel CHAN

  --injection-file FILE        inject simulated inspiral signals from FILE

  --bank-file FILE             read template bank parameters from FILE
  --minimal-match M            override bank minimal match with M (sets delta)
  --start-template N           start filtering at template number N in bank
  --stop-templateN             stop filtering at template number N in bank

  --sample-rate F              filter data at F Hz, downsampling if necessary
  --resample-filter TYPE       set resample filter to TYPE (ldas|butterworth)

  --disable-high-pass          turn off the IIR highpass filter
  --enable-high-pass F         high pass data above F Hz using an IIR filter
  --spectrum-type TYPE         use PSD estimator TYPE (mean|median)

  --segment-length N           set data segment length to N points
  --number-of-segments N       set number of data segments to N
  --segment-overlap N          overlap data segments by N points

  --low-frequency-cutoff F     do not filter below F Hz
  --inverse-spec-length T      set length of inverse spectrum to T seconds
  --dynamic-range-exponent X   set dynamic range scaling to 2^X

  --chisq-bins P               set number of chisq veto bins to P
  --snr-threshold RHO          set signal-to-noise threshold to RHO
  --chisq-threshold X          threshold on chi^2 < X * ( p + rho^2 * delta^2 )
  --enable-event-cluster       turn on maximization over chirp length
  --disable-event-cluster      turn off maximization over chirp length

  --enable-output              write the results to a LIGO LW XML file
  --disable-output             do not write LIGO LW XML output file

  --write-raw-data             write raw data to a frame file
  --write-filter-data          write data that is passed to filter to a frame
  --write-response             write the computed response function to a frame
  --write-spectrum             write the uncalibrated psd to a frame
  --write-snrsq                write the snr time series for each data segment
  --write-chisq                write the r^2 time series for each data segment
\end{verbatim}

\item[Description] 
\verb$lalapps_inspiral$ is a stand alone code for performing matched filtering
of LIGO data for graviational wave signals and Monte Carlo analysis.

\item[Options]\leavevmode
\begin{entry}
\item[\texttt{--help}] Display a brief usage summary.
\end{entry}

\item[Example]
\begin{verbatim}
lalapps_inspiral \
--enable-output --inverse-spec-length 16  --segment-length 1048576 \
--low-frequency-cutoff 7.000000e+01 --pad-data 8 \
--bank-file L1-TMPLTBANK-734357353-1024.xml \
--sample-rate 4096 --chisq-threshold 20.0 --resample-filter ldas \
--channel-name L1:LSC-AS_Q --calibration-cache \
/ldas_outgoing/calibration/cache_files/L1-CAL-V03-729273600-734367600.cache \
--segment-overlap 524288  --snr-threshold 8.0 \
--frame-cache cache/L-734357345-734361107.cache \
--number-of-segments 7 --dynamic-range-exponent 6.900000e+01 \
--enable-high-pass 5.000000e+01 --debug-level 33 \
--gps-start-time 734357353 --gps-end-time 734358377 \
--chisq-bins 8 --spectrum-type median --enable-event-cluster \
--minimal-match 9.700000e-01
\end{verbatim}

\item[Author] 
Duncan Brown
\end{entry}
\clearpage

\subsection{Program \texttt{lalapps\_inca}}
\label{program:lalapps-inca}
\idx[Program]{lalapps\_inca}

\begin{entry}
\item[Name]
\verb$lalapps_inca$ --- program does inspiral coincidence analysis.

\item[Synopsis]
\verb$lalapps_inca$ 
[\verb$--help$]
[\verb$--verbose$]
[\verb$--comment$ \textsc{comment}]
[\verb$--debug-level$ \textsc{level}] \newline
%
[\verb$--no-playground$]
[\verb$--playground-only$]
\verb$--ifo-a$ \textsc{IFOA} 
\verb$--ifo-b$ \textsc{IFOB} \newline
%
[\verb$--epsilon$ \textsc{$\epsilon$}]
[\verb$--kappa$ \textsc{$\kappa$}]
[\verb$--dm$ \textsc{$\delta m$}]
[\verb$--dt$ \textsc{$\delta t$}] \newline
%
\verb$--gps-start-time$ \textsc{seconds} 
\verb$--gps-end-time$ \textsc{seconds} 
[\verb$--write-uniq-triggers$] \newline
%
\textsc{(LIGO Lightweight XML files)}

\item[Description] 
\verb$lalapps_inca$ performs coincidence on triggers from the inspiral
search code.  At present it works for only two interferometers. The names
of the two interferometers must be given. Output is written to a LIGO
lightweight XML files. Two XML output files are written.  The output files
contain \texttt{process}, \texttt{process\_params} and
\texttt{search\_summary} tables that describe the search. The primary ifo
output file contains the triggers from \textsc{IFOA} that are found to be in
coincidence with triggers in \textsc{IFOB}. The secondary output file contains
the triggers from \textsc{IFOB} that are found to be in coincidence with the
triggers from \textsc{IFOA}.  Each trigger in the \textsc{IFOA} file
corresponds to the coincident trigger in the \textsc{IFOB} file, so there may
be duplicate  \textsc{IFOA} triggers.  To prevent this, specify the
\verb$--write-uniq-triggers$ option.

The output files are named in the standard way for inspiral pipeline output.
The primary triggers are in a file named\\
\begin{center}
\texttt{IFOA-INCA\_USERTAG-GPSSTARTTIME-DURATION.xml}\\
\end{center}
and the secondary triggers are in a file named\\
\begin{center}
\texttt{IFOB-INCA\_USERTAG-GPSSTARTTIME-DURATION.xml}\\
\end{center}
If a \texttt{--user-tag} is not specified on the command line, the
\texttt{\_USERTAG} part of the filename will be omitted.

The default behavior outputs triggers during playground times only. To obtain
those triggers that are not in the playground, use the \verb$--no-playground$
flag.     

\texttt{lalapps\_inca} calls the LAL function
\texttt{LALCompareSnglInspiral()} to test if two triggers are coincident. This
first tests that the time of the triggers is coincidence to within $\delta t$.
It then tests that both the template masses are coincident to within $\delta m$.
It then tests to that
\begin{equation}
\frac{\left|D_\mathrm{IFOA} - D_\mathrm{IFOA}\right|}{D_\mathrm{IFOA}}
< \frac{\epsilon}{\rho_\mathrm{IFOB}} + \kappa.
\end{equation}
This is equivalent to testing that
\begin{equation}
\left|\rho_\mathrm{IFOB} - \hat{\rho}_\mathrm{IFOB}\right| 
< \epsilon + \kappa\rho_\mathrm{IFOB}.
\end{equation}
If all three tests pass, the events are considered to be coincident and
written to the output file.

\item[Options]\leavevmode
\begin{entry}
\item[\texttt{--no-playground}] Optional.  Record all triggers that are
not in playground data.  The default behaviour returns only those triggers
which lie in the playground data set.  

\item[\texttt{--playground-only}] Optional.  Record only triggers that
occour in the playground times.  This is the default behaviour.

\item[\texttt{--ifo-a} \textsc{IFOA}] Required. This is the name of the
interfereometer to use as the interferometer A in the coincidence algorithm.
It must be a two letter IFO code e.g. \texttt{L1}, \texttt{H1}, etc.

\item[\texttt{--ifo-b} \textsc{IFOB}] Required. This is the name of the
interfereometer to use as the interferometer B in the coincidence algorithm.
It must be a two letter IFO code e.g. \texttt{L1}, \texttt{H1}, etc.

\item[\texttt{--epsilon} \textsc{$\epsilon$}] Optional. Set the value of
$\epsilon$ in the effective distance test. If not given the default of
$\epsilon = 2$ will be used.

\item[\texttt{--kappa} \textsc{$\kappa$}] Optional. Set the value of
$\kappa$ in the effective distance test. If not given the default of
$\kappa= 0.01$ will be used.

\item[\texttt{--dm} \textsc{$\delta m$}] Optional. Accept triggers as
coincident if both mass parameters agree within $\delta m$.  If not
supplied,  then $\delta m = 0$.

\item[\texttt{--dt} \textsc{$\delta t$}] Optional. Accept triggers as
coincident if their end times agree within $\delta t$ milliseconds.  If not
supplied,  then $\textsc{$\delta t$} = 0$.

\item[\texttt{--gps-start-time} \textsc{GPS seconds}] Required.  Look for
coincident triggers with end times after \textsc{GPS seconds}.

\item[\texttt{--gps-end-time} \textsc{GPS seconds}] Required.  Look for
coincident triggers with end times before \textsc{GPS seconds}.

\item[\texttt{--write-uniq-triggers}] Optional.  The default behaviour is to
only write all triggers from IFO A. However, a trigger from IFO A
may match two or more triggers from IFO B, so it may be duplicated in the 
output. Specifying this option causes only unique IFO A triggers to be
written.

\item[\texttt{--comment} \textsc{string}] Optional. Add \textsc{string} to the
comment field in the process table. If not specified, no comment is added. 

\item[\texttt{--user-tag} \textsc{string}] Optional. Set the user tag for this
job to be \textsc{string}. May also be specified on the command line as 
\texttt{-userTag} for LIGO database compatibility.

\item[\texttt{--help}] Optional.  Print a help message.
\end{entry}

\item[\texttt{--debug-level} \textsc{level}] Optional. Set the LAL debug
level to \textsc{level}. If not specified the default is 1.

\item[Arguments]\leavevmode
\begin{entry}
\item[\texttt{[LIGO Lightweight XML files]}] The arguments to the program
should be a list of LIGO Lightweight XML files containing the triggers from
the two interfereometers. The input files can be in any order and do not need
to be time ordered as \texttt{inca} will sort all the triggers once they are
read in. If the program encounters a LIGO Lightweight XML containing triggers
from an unknown interferometer (i.e. not IFO A or IFO B) it will exit with an
error.
\end{entry}

\item[Example]
\begin{verbatim}
lalapps_inca \
--playground-only  --gps-start-time 734357353 --drhominus 5.0 \
--dm 0.03 --gps-end-time 734358377 --ifo-b H1 --dt 20.0 \
--ifo-a L1 --drhoplus 5.0 --debug-level 33
\end{verbatim}

\item[Algorithm]
The code maintains two poniters to triggers from each ifo,
\texttt{currentTrigger[0]} and \texttt{currentTrigger[1]}, corresponding to
the current trigger from IFO A and B respectively.

\begin{enumerate}
\item An empty linked list of triggers from each interferometer is created.
Each input file is read in and the code determines which IFO the triggers in
the file correspond to. The triggers are appended to the linked list for the
corresponding interferometer.

\item If there are no triggers read in from either of the interferometers,
the code exits cleanly.

\item The triggers for each interferometer is sorted by the \texttt{end\_time}
of the trigger.

\item \texttt{currentTrigger[0]} is set to point to the first trigger from IFO
A that is after the specified GPS start time for coincidence. If no trigger is
found after the start time, the code exits cleanly.

\item Loop over each trigger from IFO A that occurs before the specified GPS
end time for coincidence:
\begin{enumerate}
\item \texttt{currentTrigger[1]} is set to point to the first trigger from IFO
B that is within the time coincidence window, $\delta t$, of
\texttt{currentTrigger[0]}. If no IFO B trigger exists within this window,
\texttt{currentTrigger[0]} is incremented to the next trigger from IFO A and
the loop over IFO A triggers restarts.

\item If the trigger \texttt{currentTrigger[0]} \emph{is, is not} in the
playground data, start looping over triggers from IFO B.
\begin{enumerate}
\item For each trigger from IFO B that is within $\delta t$ of
\texttt{currentTrigger[0]}
\item Call \texttt{LALCompareSnglInspiral()} to check if the triggers match as
determied by the options on the command line. If the trigger match, record
them for later output as coincident triggers.
\end{enumerate}

\item Increment \texttt{currentTrigger[0]} and continue loop over triggers
from IFO A.
\end{enumerate}
\end{enumerate}

\item[Author] 
Patrick Brady and Duncan Brown
\end{entry}

\clearpage
\subsection{Program \texttt{lalapps\_inspinj}}
\label{program:lalapps-inspinj}
\idx[Program]{lalapps\_inspinj}

\begin{entry}
\item[Name]
\verb$lalapps_inspinj$ --- produces inspiral injection data files.

\item[Synopsis]
\verb$lalapps_inspinj$ 
[\verb$--help$]
\verb$--source-file$ \textsc{sfile}
\verb$--mass-file$ \textsc{mfile}
\verb$--gps-start-time$ \textsc{tstart} 
\verb$--gps-end-time$ \textsc{tend} 
[\verb$--time-step$ \textsc{tstep}]
[\verb$--seed$ \textsc{seed}]
[\verb$--waveform$ \textsc{wave}]
[\verb$--usertag$ \textsc{tag}]
[\verb$--ilwd$]

\item[Description] 
\verb$lalapps_inspinj$
generates a number of inspiral  parameters suitable  for using in a Monte
Carlo injection to test the efficiency of a inspiral search.  The  various
parameters (detailed  below)  are randomly chosen and are appropriate for a
particular population of binary neutron stars  whose spatial  distribution
includes the Milky Way and a number of extragalactic objects that are  input
in  a  datafile.  The  possible  mass pairs for the binary neutron star com-
panions are also specified in a (different) datafile.

The output of this program  is  a  list  of  the  injected events,  starting
at  the specified start time, ending at the specified end time, and containing
one set  of  random inspiral parameters every specified time step.  The output
is written to a file name in the standard inspiral pipeline format:
\begin{center}
\begin{verbatim}
HL-INJECTIONS_USERTAG_SEED-GPSSTART-DURATION.xml
\end{verbatim}
\end{center}
where \verb$USERTAG$ is \textsc{tag} as specfied on the command line, 
\verb$SEED$ is the  value  of  the random number seed chosen and 
\verb$GPSSTART$ and \verb$DURATION$ describes the GPS time interval that
the file covers. The file is in the standarf LIGO lightweight XML format
containing a \texttt{sim\_inspiral} table that describes the injections.
In addition, an ascii log file called \verb$injlog.txt$ is also written.
If a \texttt{--user-tag} is not specified on the command line, the
\texttt{\_USERTAG} part of the filename will be omitted.

\item[Options]\leavevmode
\begin{entry}
\item[\texttt{--help}] Print a help message.

\item[\texttt{--source-file} \textsc{sfile}]
Optional. Data file containing spatial distribution of  extragalactic  objects.
Default  is  the file \verb+inspsrcs.dat+ provided by LALApps.

\item[\texttt{--mass-file} \textsc{mfile}]
Optional. Data file containing mass pairs  for  the binary  neutron  star
companions.   Default is the file \verb+BNSMasses.dat+ provided by LALApps.

\item[\texttt{--gps-start-time} \textsc{tstart}]
Optional.  Start time of the injection data to be created. Defaults to the
start of S2, Feb 14 2003 16:00:00 UTC (GPS time 729273613)

\item[\texttt{--gps-end-time} \textsc{tend}]
Optional. End time of the injection data to be created. Defaults to the end of
S2, Apr 14 2003 15:00:00 UTC (GPS time 734367613).

\item[\texttt{--time-step} \textsc{tstep}]
Optional. Sets the time step interval between injections. The injections will
occour at \textsc{tstep}$/\pi$ second intervals. Defaults to $2630/\pi$.

\item[\texttt{--seed} \textsc{seed}]
Optional. Seed the random number generator with the integer \textsc{seed}.
Defaults to $1$.

\item[\texttt{--waveform} \textsc{wave}]
Optional. The string \textsc{wave} will be written into the \texttt{waveform}
column of the \texttt{sim\_inspiral} table output. This is used by the
inspiral code to determine which type of waveforms it should inject into the
data. Defaults is \texttt{GeneratePPNtwoPN}.

\item[\texttt{--user-tag} \textsc{string}] Optional. Set the user tag for this
job to be \textsc{string}. May also be specified on the command line as 
\texttt{-userTag} for LIGO database compatibility.

\item[\texttt{--ilwd}] Optional. If this option is given,
\verb+lalapps_inspinj+ also produces two ILWD-format files, injepochs.ilwd and
injparams.ilwd, that contain, respectively, the  GPS  times  suitable for
inspiral injections, and the intrinsic inspiral signal parameters to be used
for  those injections.

The  file  injepochs.ilwd  contains  a sequence of integer pairs representing
the injection GPS time in  seconds  and residual  nano-seconds.   The file
injparams.ilwd contains the intrinsic binary parameters for each injection,
which is  a  sequence  of  eight  real  numbers representing (in order) (1) the
total mass of the binary system  (in  solar masses),  (2)  the  dimensionless
reduced mass --- reduced mass per unit total mass --- in the range from  0
(extreme mass  ratio)  to  0.25 (equal masses), (3) the distance to the system
in meters, (4) the inclination  of  the  binary system  orbit  to the plane of
the sky in radians, (5) the colaescence phase in radians, (6)  the  longitude
to  the direction  of  the  source in radians, (7) the latitude to the
direction of the source in radians, (8) and the polar- ization angle of the
source in radians.
\end{entry}

\item[Example]
\begin{verbatim}
lalapps_inspinj --seed 45\
--source-file inspsrcs.dat --mass-file BNSMasses.dat
\end{verbatim}

\item[Environment]\leavevmode
\begin{entry}
\item[LALAPPS\_DATA\_PATH] Directory to look for the default mass
file \verb+BNSMasses.dat+ and the default source file \verb+inspsrcs.dat+.
\end{entry}


\item[Author] 
Jolien Creighton, Patrick Brady, Duncan Brown
\end{entry}

%%%%%%%%%%%%%%%%%%%%%%%%%%%%%%%%%%%%%%%%%%%%%%%%%%%%%%%%%%%%%%%%%%%%%%%%%%%%%%%
%
% PROGRAM:   snglInspiralReader
% 
%%%%%%%%%%%%%%%%%%%%%%%%%%%%%%%%%%%%%%%%%%%%%%%%%%%%%%%%%%%%%%%%%%%%%%%%%%%%%%%
\clearpage
\subsection{Program \texttt{lalapps\_snglInspiralReader}}
\label{program:lalapps-snglInspiralReader}
\idx[Program]{lalapps\_snglInspiralReader}

\begin{entry}
\item[Name]
\verb$lalapps_snglInspiralReader$ --- manipulates LIGO lightweight XML
files of inspiral triggers allowing cuts and clustering.

\item[Synopsis]
\verb$lalapps_snglInspiralReader$ 
\verb$--input$ \textsc{infile} \verb$--table$ \textsc{tablename}
\newline 
\verb$--outfile$ \textsc{outfile}
[\verb$--snrstar$ \textsc{snrstar}] [\verb$--noplayground$] 
[\verb$--sort$] [\verb$--cluster$ \textsc{msec}] \newline
[\verb$--clusteralgorithm$ \textsc{clusterchoice}] [\verb$--help$]

\item[Description] 
\verb$lalapps_snglInspiralReader$ processes triggers from the inspiral
search code.   The \textsc{infile} should contain a list of the XML
files containing the triggers;  the format is one filename per line. 
The default behavior outputs triggers during playground
times to the file \textsc{outfile};  to obtain all triggers,  use the 
\verb$--noplayground$ flag.    To apply a cut on SNR,  use the flag 
\verb$--snrstar$ \textsc{snrstar}:  only triggers with $\texttt{SNR} 
> \textsc{snrstar}$ will be recorded.    Events can also be clustered
within \textsc{msec} msec,   in which case the \verb$--sort$ flag is
recommended unless you are certain that the triggers are time-ordered.  
There is a choice of several clustering algorithms, which can be selected
using \verb$--clusteralgorithm$.


\item[Options]\leavevmode
\begin{entry}
\item[\texttt{--input} \textsc{infile}] Required.  A file containing a
list of LIGO lightweight XML files with triggers to be processed.  The
format of \textsc{infile} is one file name per line.

\item[\texttt{--table} \textsc{tablename}] Required.  The name of the 
XML table containing the inspiral events, this will usually be 
\texttt{sngl\_inspiral}.

\item[\texttt{--outfile} \textsc{outfile}] Required.  Name of the file
to be used for output.  The output format is LIGO lightweight XML with
\texttt{sngl\_inspiral}, \texttt{process} and \texttt{process\_params}
tables. 

\item[\texttt{--snrstar} \textsc{snrstar}] Optional.  A threshold cut
on signal-to-noise.  Only triggers with $\textsc{snr} > \textsc{snrstar}$
are recorded in the output file.

\item[\texttt{--noplayground}] Optional.  Record all triggers.  The
default behaviour returns only those triggers which lie in the
playground data set.  

\item[\texttt{--sort}] Optional.   Sort the triggers in time (before
clustering).  

\item[\texttt{--cluster} \textsc{msec}] Optional.  Cluster triggers
within \textsc{msec} msec window.   The clustering algorithm
identifies the first trigger in a cluster, then displaces it if another 
trigger within the clustering window satisfies the appropriate condition 
(described below).

\item[\texttt{--clusteralgorithm} \textsc{choicenumber}]  Optional.  This
determines which condition will be used in clustering of the triggers.
The current choices are \texttt{snr\_and\_chisq} --- displace event if
its \textsc{snr} is exceeded by an event with a smaller
\textsc{chisq}; \texttt{snrsq\_over\_chisq} --- displace event if the
quantity $(\textsc{snr})^{2}/\textsc{chisq}$ is exceeded by a subsequent
event's.  The default is \texttt{snr\_and\_chisq}.

\item[\texttt{--help}] Optional.  Print a help message.
\end{entry}

\item[Example]
\begin{verbatim}
lalapps_snglInspiralReader --input xmlfilelist \
--table sngl_inspiral --outfile my.xml --snrstar 8.0 \
--sort --cluster 20 --clusteralgorithm snrsq_over_chisq
\end{verbatim}

\item[Author] 
Patrick Brady
\end{entry}
\clearpage



%%%%%%%%%%%%%%%%%%%%%%%%%%%%%%%%%%%%%%%%%%%%%%%%%%%%%%%%%%%%%%%%%%%%%%%%%%%%%%
%
% PROGRAM:   inspinj_find
% 
%%%%%%%%%%%%%%%%%%%%%%%%%%%%%%%%%%%%%%%%%%%%%%%%%%%%%%%%%%%%%%%%%%%%%%%%%%%%%%
\clearpage
\subsection{Program \texttt{lalapps\_inspinj\_find}}
\label{program:lalapps-inspinj-find}
\idx[Program]{lalapps\_inspinj\_find}

\begin{entry}
\item[Name]
\verb$lalapps_inspinj_find$ --- compares LIGO lightweight XML files 
containing inspiral triggers with an XML file containing injected
signals and tests for time coincidence.


\item[Synopsis]
\verb$lalapps_inspinj_find$ 
\verb$--input$ \textsc{infile} [\verb$--inject$ \textsc{injectfile}]
\verb$--outfile$ \textsc{outfile} \newline
[\verb$--snrstar$ \textsc{snrstar}]
[\verb$--sort$] [\verb$--noplayground$] [\verb$--deltat$
\textsc{dt}] \newline
[\verb$--cluster$ \textsc{clust}] 
[\verb$--clusteralgorithm$ \textsc{clusterchoice}] 
[\verb$--missedinjections$ \textsc{missedfile}]\newline
[\verb$--hardware$ \textsc{starttime}] [\verb$--help$]\newline
 


\item[Description] 
\verb$lalapps_inspinj_find$ compares triggers from the inspiral
search code with injections.  The \textsc{infile} should contain a
list of the XML files containing the triggers; the format is one
filename per line.  The triggers can be sorted, using the
\verb$--sort$ flag.  To apply a cut on SNR,  use the flag  
\verb$--snrstar$ \textsc{snrstar}:  only triggers with $\texttt{SNR} 
> \textsc{snrstar}$ will be recorded.  If the file \textsc{injectfile}
containing injections is provided then the program will compare the
triggers with the injections, in which case the \verb$--sort$ flag is
recommended unless you are certain that the triggers are time-ordered.
Any trigger occuring within \textsc{dt} msec of an injection
is retained.  Additionally, any injection which is coincident with one
or more triggers is also retained.  The tables of triggers and
injections are output to the file \textsc{outfile}.  Events can also
be clustered within \textsc{clust} msec, after coincidence has
been checked.  If \textsc{clust} is twice \textsc{dt}
then only one trigger per inspiral will survive.  There is a choice of
several clustering algorithms, which can be selected using
\verb$--clusteralgorithm$.  Finally, specifying \verb$--missedinjections$,
creates a file \textsc{missedfile} containing a table of those
injections which occured during the times of the input files, and in
the playground, which were not coincident with any triggers.


\item[Options]\leavevmode
\begin{entry}
\item[\texttt{--input} \textsc{infile}] Required.  A file containing a
list of LIGO lightweight XML files with triggers to be processed.  The
format of \textsc{infile} is one file name per line.

\item[\texttt{--inject} \textsc{injectfile}] Optional.  A file
containing a list of inspiral events which were injected into the
data.  The \textsc{injectfile} format is LIGO lightweight XML with a
\texttt{sim\_inspiral} table. If this file is not specified, the
program runs in the same way as \texttt{lalapps\_snglInspiralReader}.

\item[\texttt{--outfile} \textsc{outfile}] Required.  Name of the file
to be used for output.  The output format is LIGO lightweight XML with
\texttt{sngl\_inspiral}, \texttt{sim\_inspiral}, \texttt{process} and 
\texttt{process\_params} tables.  

\item[\texttt{--snrstar} \textsc{snrstar}] Optional.  A threshold cut
on signal-to-noise.  Only triggers with $\textsc{snr} > \textsc{snrstar}$
are checked for coincidence with the injections.

\item[\texttt{--sort}] Optional.   Sort the triggers in time (before
checking coincidence).  

\item[\texttt{--noplayground}] Optional.  Record all triggers.  The
default behaviour returns only those triggers which lie in the
playground data set.  

\item[\texttt{--deltat} \textsc{dt}] Optional.  This gives the maximum time
difference \textsc{dt} allowed between the injection and trigger.  If
not specified, the default is 20msec.  The time difference is calculated between
the trigger time and the end time for the injection at the detector. 

\item[\texttt{--cluster} \textsc{clust}] Optional.  Cluster triggers
within \textsc{clust} msec window.   The clustering algorithm
identifies the first trigger in a cluster, then displaces it if another 
trigger within the clustering window satisfies the appropriate condition 
(described below).

\item[\texttt{--clusteralgorithm} \textsc{choicenumber}]  Optional.  This
determines which condition will be used in clustering of the triggers.
The current choices are \texttt{snr\_and\_chisq} --- displace event if
its \textsc{snr} is exceeded by an event with a smaller
\textsc{chisq}; \texttt{snrsq\_over\_chisq} --- displace event if the
quantity $(\textsc{snr})^{2}/\textsc{chisq}$ is exceeded by a subsequent
event's.  The default is \texttt{snr\_and\_chisq}.

\item[\texttt{--missedinjections} \textsc{missedfile}]  Optional.
Output a \texttt{sim\_inspiral} table in \textsc{missedfile} containing
all injections which were not coincident with a trigger.  Only
injections occuring during during the times of the input files and
within the playground (unless \texttt{--noplayground} is specified)
are given.
 
\item[\texttt{--hardware} \textsc{starttime}]  Optional.  The
\textsc{starttime} should be the gps start time of the hardware
injections in seconds.  In this case, the \textsc{injectfile} is
expected to contain a list of the hardware injections, where the time
given is the time after the start of injections that the coalescence
occurs. 

\item[\texttt{--help}] Optional.  Print a help message.
\end{entry}

\item[Example]
\begin{verbatim}
lalapps_inspinj_find --input xmlfilelist \
--inject injections.xml --outfile my.xml --snrstar 8.0 \
--coincidence 20 --sort --cluster 20 --clusteralgorithm snrsq_over_chisq \
--missedinjections missedinj.xml
\end{verbatim}

\item[Author] 
Steve Fairhurst
\end{entry}
\clearpage


%%%%%%%%%%%%%%%%%%%%%%%%%%%%%%%%%%%%%%%%%%%%%%%%%%%%%%%%%%%%%%%%%%%%%%%%%%%%%%
%
% PROGRAM:   inspmultiawg
% 
%%%%%%%%%%%%%%%%%%%%%%%%%%%%%%%%%%%%%%%%%%%%%%%%%%%%%%%%%%%%%%%%%%%%%%%%%%%%%%
\clearpage
\subsection{Program \texttt{lalapps\_inspmultiawg}}
\label{program:lalapps-inspmultiawg}
\idx[Program]{lalapps\_inspmultiawg}

\begin{entry}
\item[Name]
\verb$lalapps_inspmultiawg$ --- injects specified inspiral chirps into zero
data.  Intended for producing the hardware injection data.

\item[Synopsis]
\begin{verbatim}
--help                  display this message
--source SFILE          source file containing details of injection
--response RESPFILE     file containing the response function
--summary SUMFILE       write found injection to file
--ifo IFO               name of interferomter (optional)
--flow FSTART           start frequency of injection (default 40 Hz)
--fhigh FSTOP           end frequency of injection (default: end at ISCO)
--length LENGTH         length of the data (default 64 seconds)
--samplerate FREQ       rate at which data is sampled (default (16384Hz)
--debug-level DEBUG     give the lal debug level
\end{verbatim}

\item[Description] 
\verb$lalapps_inspmultiawg$ injects inspiral chirps into zero data.  The details
of several chirps can be specified using the command \verb$--source$, otherwise,
a single inspiral of two 1.4 solar mass neutron stars will be injected.  Each
chirp is injected into a new file containing zero data of $\texttt{LENGTH}$
seconds, sampled at $\texttt{FREQ}$ Hz, and each injection begins at the
beginning of the data.  The response function can be provided in
$\texttt{RESPFILE}$.  The chirp is output to a file
named
\begin{verbatim}
IFO_inspiral_NUMBER.txt
\end{verbatim}
where $\texttt{IFO}$ is the name of the interferometer, and $\texttt{NUMBER}$ is
the injection number.  A summary of the injections performed can be saved in
$\texttt{SUMFILE}$.

\item[Options]\leavevmode
\begin{entry}
\item[\texttt{--sourcefile} \textsc{sfile}] Optional.  Reads source information
from the file \textsc{sfile}.  If absent, it injects a single 
1.4$M_\odot$--1.4$M_\odot$ inspiral, optimally oriented, at a distance
of $1.0Mpc$.

\item[\texttt{--response} \textsc{respfile}] Optional. Reads a detector response
function from the file \textsc{respfile}.  If absent, it generates raw
dimensionless strain.

\item[\texttt{--summary} \textsc{sumfile}] Optional. The \textsc{sumfile} format
is LIGO lightweight XML with \texttt{process}, \texttt{process\_params} and
\texttt{sim\_inspiral} tables.  The \texttt{sim\_inspiral} table contains
details of all the injections performed.  The details of the injection are
obtained from the source file.

\item[\texttt{--length} \textsc{sec}] Optional.  Specify the length of data into
which the signal will be injected.  The default is 64 seconds.

\item[\texttt{--samplerate} \textsc{freq}] Optional.  Specify the rate at which
the data is sampled.  The default is 16384 Hz.

\item[\texttt{--ifo} \textsc{ifo}] Optional.  Give the name of the
interferometer for which the injections are intended.  This is only used in
naming the output files.

\item[\texttt{--flow} \textsc{fstart}] Optional.  Give the start frequency
\textsc{fstart} for the inspiral.  The default is 40 Hz

\item[\texttt{--fhigh} \textsc{sftop}] Optional.  Give the end frequency
\textsc{fstop} for the inspiral.  The default behaviour is that the inspiral
will continue to ISCO.  If set to a negative number, the generator will use its
absolute value as the terminating frequency, but will ignore post-Newtonian
breakdown. 

\item[\texttt{--debug-level} \textsc{debug}]  Optional.  Set the LAL debug
level.  The default is 33.

\item[\texttt{--help}] Optional.  Print a help message.
\end{entry}

\paragraph{Format for \texttt{sourcefile}:} The source file consists
of any number of lines of data, each specifying a chirp waveform.
Each line must begin with a character code (\verb@CHAR@ equal to one
of \verb@'i'@, \verb@'f'@, or \verb@'c'@), followed by 6
whitespace-delimited numerical fields: the epoch of the chirp
(\verb@INT8@ seconds), the two binary masses (\verb@REAL4@
$M_\odot$), the distance to the source (\verb@REAL4@ Mpc), and the
source's inclination and phase at coalescence (\verb@REAL4@ degrees).
The character codes have the following meanings:
\begin{itemize}
\item[\texttt{'i'}] The epoch represents the GPS time of the start of
the chirp waveform.
\item[\texttt{'f'}] The epoch represents the GPS time of the end of
the chirp waveform.
\item[\texttt{'c'}] The epoch represents the GPS time when the
binaries would coalesce in the point-mass approximation.
\end{itemize}
Since the injection is started at time $t=0$, it is recommended that the
\texttt{'i'} option is used.

Thus a typical input line for two $1.4M_\odot$ objects at 1.1 Mpc
inclined $30^\circ$ with an initial phase of $45^\circ$, beginning at
70 seconds (after the start of the injections), will have the following line in the input
file:
\begin{verbatim}
i 70 1.4 1.4 1.1 30.0 45.0
\end{verbatim}
The time parameter (in this case 70 sec) does not affect the output data in any
way.  It is simply stored in the \texttt{sim\_inspiral} table of the
\textsc{sumfile}, in order to make analysis of the injections easier.

\paragraph{Format for \texttt{respfile}:} The response function $R(f)$
gives the real and imaginary components of the transformation
\emph{from} ADC output $o$ \emph{to} tidal strain $h$ via
$\tilde{h}(f)=R(f)\tilde{o}(f)$.  It is inverted internally to give
the detector \emph{transfer function} $T(f)=1/R(f)$.  The format
\verb@respfile@ is a header specifying the GPS epoch $t_0$ at which
the response was taken (\verb@INT8@ nanoseconds), the lowest frequency
$f_0$ at which the response is given (\verb@REAL8@ Hz), and the
frequency sampling interval $\Delta f$ (\verb@REAL8@ Hz):

\medskip
\begin{tabular}{l}
\verb@# epoch = @$t_0$ \\
\verb@# f0 = @$f_0$ \\
\verb@# deltaF = @$\Delta f$
\end{tabular}
\medskip

\noindent followed by two columns of \verb@REAL4@ data giving the real
and imaginary components of $R(f_0+k\Delta f)$.

\paragraph{Format for the data output:} The data output in the files
\verb$IFO_inspiral_NUMBER.txt$ is a single column of \verb$REAL4$ ADC data.

\item[Example]
\begin{verbatim}
lalapps_inspmultiawg --source s3.sources \
--response response_l1.txt --summary summ.xml --ifo l1
\end{verbatim}

\item[Author] 
Steve Fairhurst
\end{entry}
\clearpage
