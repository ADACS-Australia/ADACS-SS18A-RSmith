\section{Inspiral Search Programs}
\label{secion:inspiral}

This section of \textsc{LALApps} contains programs that can be used to search
interferometer data for inspiral signals using templated matched filtering and
associated veto stratergies.

\clearpage
\subsection{Program \texttt{lalapps\_tmpltbank}}
\label{program:lalapps-tmpltbank}
\idx[Program]{lalapps\_tmpltbank}

\begin{verbatim}
lalapps_tmpltbank is a stand alone code for generating inspiral template
banks for LIGO data with the LAL bank package.

The code generates a calibrated power spectrum at the specified time for
the requested channel and uses this to compute the template bank.

See the LAL bank package documentation for detailed information on the
algorithms used to generate the banks.

Usage: lalapps_tmpltbank [OPTIONS]

If the option takes an argument, the agument type is shown after the option.
Some of the options are required for operation. If a required option is
not specified, the program will exit with an error message reporting the
missing option. Sanity checking is performed on option arguments.


PROGRAM OPTIONS.

The following options control non-sceintific program behaviour, such as
enabling debugging information:

   --help                      display this message
   --debug-level lstring       set the LAL debug level to the specified
                                 value. Useful values are: NDEBUG, ERROR,
                                 WARNING, INFO, TRACE, MEMINFO and MEMDBG
   --verbose                   verbose operation


POWER SPECTRUM GENERATION OPTIONS.

The following options contol reading of the raw data and generation of 
the power spectum. The raw data is read from IGWD frame files.

   --gps-start-time                GPS start (seconds) of data in PSD
   --gps-stop-time                 GPS stop time (seconds) of data in PSD
   --channel-name lstring          name of the channel (e.g. H1:LSC-AS_Q)
   --segment-length int_4u         length of each individual segment
   --number-of-segments int_4u     numer of segments to break the data into
   --sample-rate int_4u            resample the raw data to the specified rate
   --enable-high-pass real_4       high pass the data above the specified
                                     frequency before computing PSD
   --disable-high-pass             do not high pass the data
   --spectrum-type lstring         method of psd computation (mean or median)
   --frame-cache lstring           name of the cache file containing input data
   --calibration-cache lstring     location of a frame cache file containing
                                     calibration information.


TEMPLATE BANK OPTIONS.

The following options are using to generate the bank. A brief overview of the
paramerers is given here. The LAL bank package contains should be consulted
for the full documentation on the parameters

   --minimum-mass real_4           minimum (component) mass in the bank
   --maximum-mass real_4           maximum (component) mass in the bank
   --minimum-match real_4          minimal match of the template bank
   --low-frequency-cutoff real_4   high frequency cutoff
   --high-frequency-cutoff real_4  upper frequency cutoff
   --order lstring                 post-Newtonian order of the waveform
                                    (newtonian|oneHalfPN|onePN|onePointFivePN|
                                    twoPN|twoPointFive|threePN|threePointFivePN)
   --approximant lstring           approximant of the waveform
                                    (TaylorT1|TaylorT2|TaylorT3|
                                    TaylorF1|TaylorF2|PadeT1|PadeT2|
                                    EOB|BCV|SpinTaylorT3)
   --space lstring                 space in which to lay down the bank
                                     (Tau0Tau2|Tau0Tau3)


OUTPUT OPTIONS.

The following options control writing any result data to disk. The bank is
written as LIGO lightweigh XML file (LIGOLw), data as IGWD frame files.

   --comment lstring           add a comment to the process params table
   --write-raw-data            write the raw data read in as a frame
   --write-response            write the response function used as a frame
   --write-spectrum            write the power spectrum used as a frame


EXAMPLE USEAGE.

Generates a typical template bank as used in the S1 analysis:

lalapps_tmpltbank --gps-start-time 715482185 
                  --gps-end-time 715482441 
                  --segment-length 262144 --number-of-segments 7 
                  --sample-rate 4096 
                  --enable-high-pass 30.0 --low-frequency-cutoff 40.0 
                  --spectrum-type mean 
                  --channel-name "L1:LSC-AS_Q" 
                  --frame-cache s1_bank_data.catalog 
                  --calibration-cache s1_bank_cal.catalog 
                  --minimum-mass 1.0 --maximum-mass 3.0 --minimum-match 0.97 
                  --high-frequency-cutoff 1024.0 
                  --order twoPN --approximant TaylorT1 --space Tau0Tau3 
                  --write-raw-data --write-spectrum 
                  --write-response --write-strain-spectrum
\end{verbatim}

%\begin{entry}
%
%\item[Name]
%\verb$lalapps_tmpltbank$ --- template bank generation from raw data.
%
%\item[Synopsis]
%\verb$lalapps_tmpltbank$ 
%[\verb$--help$]
%[\verb$--debug-level$ \textsc{level}]
%[\verb$--verbose$]
%\verb$--gps-start-time$ \textsc{gps} 
%\verb$--gps-stop-time$ \textsc{gps}
%\verb$--channel-name$ \textsc{name}
%\verb$--segment-length$ \textsc{points}
%\verb$--number-of-segments$ \textsc{segments} \\
%\verb$--sample-rate$ \textsc{rate}
%(\verb$--enable-high-pass$ \textsc{frequency}$|$\verb$--disable-high-pass$) \\
%\verb$--spectrum-type$ \textsc{(mean$|$median)}
%\verb$--frame-cache$ \textsc{file}
%\verb$--calibration-cache$ \textsc{file}
%\verb$--minimum-mass$ \textsc{mass}
%\verb$--maximum-mass$ \textsc{mass}
%\verb$--low-frequency-cutoff$ \textsc{freq}
%\verb$--high-frequency-cutoff$ \textsc{freq}
%\verb$--order$ 
%\textsc{(newtonian$|$oneHalfPN$|$onePN$|$onePointFivePN$|$twoPN$|$twoPointFive$|$threePN$|$threePointFivePN)}
%\verb$$ \textsc{}
%\verb$$ \textsc{}
%\verb$$ \textsc{}
%\verb$$ \textsc{}
%
%\item[Description]
%\verb$lalapps_tmpltbank$ generates inspiral template bank parameters from raw
%data for a specified time interval.
%
%The code generates a calibrated power spectrum at the specified time for
%the requested channel and uses this to compute the template bank.
%
%See the LAL bank package documentation for detailed information on the
%algorithms used to generate the banks.
%
%\item[Example]
%none
%
%\item[Author]
%Duncan Brown
%\end{entry}

\clearpage
\subsection{Program \texttt{lalapps\_inspiral}}
\label{program:lalapps-inspiral}
\idx[Program]{lalapps\_inspiral}

\begin{verbatim}
lalapps_inspiral is a stand alone code for performing matched filtering
of LIGO data for graviational wave signals and Monte Carlo analysis.

Usage: lalapps_inspiral [OPTIONS]

If the option takes an argument, the agument type is shown after the option.
Some of the options are required for operation. If a required option is
not specified, the program will exit with an error message reporting the
missing option. Sanity checking is performed on option arguments.


PROGRAM OPTIONS.

The following options control non-sceintific program behaviour, such as
enabling debugging information:

   --help                      display this message
   --debug-level lstring       set the LAL debug level to the specified
                                 value. Useful values are: NDEBUG, ERROR,
                                 WARNING, INFO, TRACE, MEMINFO and MEMDBG
   --verbose                   verbose operation


INPUT DATA OPTIONS.

The following options contol reading of the raw data that the code will filter.
The inspiral code reads in a block of data from the desired channel, startingat 
the specified GPS start time and ending at the last sample before the GPS
end time (i.e. start <= data < end). It breaks this into a specified number of 
segments of a given length and overlap for filtering. Sanity checking is 
performed to make sure that the specified block of data can be appropriately
segmented.

The raw data is read from IGWD frame files and the template bank is read from a
LIGO lightweight XML file containing the template parameters.

   --frame-cache lstring        name of the cache file containing input data
   --gps-start-time             GPS start (seconds) of input data
   --gps-start-time-ns          GPS start (nanoseconds) of input data
   --gps-stop-time              GPS stop time (seconds) of input data
   --gps-stop-time              GPS stop time (nanoseconds) of input data
   --channel-name lstring       name of the channel to filter (e.g. H1:LSC-AS_Q)
   --segment-length int_4u      length of each individual segment
   --number-of-segments int_4u  numer of segments to break the data into
   --segment-overlap int_4u     overlap between consecutive segments
   --start-template int_4u      start parsing templates at this row number
   --stop-template int_4u       stop parsing templates at this row number


DATA CONDITIONING OPTIONS.

The following options control any pre-conditioning applied to the raw data
before it is passed to the filtering code as well as parameters to generate
the one-sided power specral density of the data. It is possible to resample
and/or apply a time-domain high-pass filter to the data before the power
spectrum is computed

   --sample-rate int_4u            resample the raw data to the specified rate
   --enable-high-pass real_4       high pass the data above the specified
                                      frequency
   --disable-high-pass             do not high pass the data
   --low-frequency-cutoff real_4   specifiy a low frequency cutoff for the
                                      filter. All frequencies below this value
                                      are ignored. This takes effect after
                                      the high-pass filter.
   --spectrum-type lstring         method of psd computation (mean or median)
   --inverse-spec-length int_4u    truncate the inverse power spectal density in
                                   time domain. Specify desired length of the
                                   inverse specrum in seconds.
   --dynamic-range-exponent real_4 to keep the data in range a dynamic range
                                      scale factor is applied to the data as
                                      described in the findchirp documentation.
   --calibration-cache lstring     location of a frame cache file containing
                                      calibration information.


FILTERING OPTIONS.

The following options are using by the filtering routines to control the
parameters of the matched filter and chi squared veto. The thresholds for
signal to noise and chi squared are specified as a comma separated list
(e.g. 9.0,7.0,6.0) with the first value being used for the coarse search,
the second value for the first hierarchical search and so on. If the
numer of chi squared bins is set to 0 then the chi squared veto is disabled.

   --chisq-bins int_4u         number chi squared veto frequency bins
   --snr-thresholds real_4     list of signal to noise thresholds
   --chisq-thresholds real_4   list of chi squared thresholds
   --enable-event-cluster      enable the event clustering algorithm
   --disable-event-cluster     disable the event clustering algorithm


OUTPUT OPTIONS.

The following options control writing any result data to disk. Results are
written as LIGO lightweigh XML file (LIGOLw) or as IGWD frame files (frames)

   --enable-output             write inspiral triggers found as LIGOLw
   --disable-output            do not write triggers as LIGOLw 
   --comment lstring           add a comment to the process params table
   --write-raw-data            write the raw data read in as a frame
   --write-filter-data         write the pre-conditioned data as a frame
   --write-response            write the response function used as a frame
   --write-spectrum            write the power spectrum used as a frame
   --write-snrsq               write the last vector of snr^2 data as a frame
   --write-chisq               write the last vector of chi^2 data as a frame


EXAMPLE USEAGE.

Requires 256 seconds of LHO frame data in the current directory.

lalapps_inspiral --enable-output --enable-high-pass 50.0
                 --enable-event-cluster --gps-start-time 600000000
                 --gps-end-time 600000256 --segment-length 262144
                 --number-of-segments 7 --segment-overlap 131072
                 --sample-rate 4096 --chisq-bins 8
                 --low-frequency-cutoff 100.0 --spectrum-type mean
                 --inverse-spec-length 0 --dynamic-range-exponent 69.0
                 --channel-name H1:LSC-AS_Q --snr-threshold 1.0
                 --chisq-threshold 0.001 --verbose --debug-level NDEBUG
\end{verbatim}

\clearpage
\subsection{Program \texttt{inspiral\_pipeline.py}}
\label{program:inspiral-pipeline}
\idx[Program]{inspiral\_pipeline.py}

\begin{verbatim}
Usage: inspiral_pipeline.py [OPTIONS]

   -f, --config-file FILE   use configuration file FILE
   -p, --play               only create chunks that overlap with playground
   -v, --version            print version information and exit
   -h, --help               print help information
   -c, --cache              flag the frame cache query as done
   -b, --bank               flag the bank generation and cache query as done

This program generates a DAG to run the inspiral code. The configuration file 
should specify the parameters needed to run the jobs and must be specified 
with the --config-file (or -f) option. See the LALapps documentation for more
information on the syntax of the configuation file.

A file containing science segments to be analyzed should be specified in the 
[input] section of the configuration file with a line such as

segments = S2TripleCoincidetScienceSegments.txt

This should contain four whitespace separated coulumns:

  segment_id    gps_start_time  gps_end_time    duration

that define the science segments to be used. Lines starting with # are ignored.

The length of the number of inspiral segments, their overlap and length is 
determined from the config file and the length of an inspiral chunk is
computed (typically this is 1024 seconds for S2).

The chunks start and stop times are computed from the science segment times
and used to build the DAG.

If the --play (or -p) flag is specified, then only chunks that overlap the S2
playground defined by:

  ((t - 729273613) % 6370) < 600

are included in the DAG.

If the --cache (or -c) option is specifed, the generation of the frame cache 
files is marked as done in the DAG. The cache files are expected to exist by
the bank generation and inspiral codes.

If the --bank (or -b) option is specified, both the frame cache query and
generation of the template bank are marked as done. The cache files and
template banks are expected to exist by the inspiral code.
\end{verbatim}


\clearpage
\subsection{Program \texttt{lalapps\_inspinj}}
\label{program:lalapps-inspinj}
\idx[Program]{lalapps\_inspinj}

\begin{entry}
\item[Name]
\verb$lalapps_inspinj$ --- produces inspiral injection ILWD and LIGOLw files.

\item[Synopsis]
\verb$lalapps_inspinj$ 
[\verb$--help$]
\verb$--seed$ \textsc{seed}
\verb$--gps-start-time$ \textsc{tstart} 
\verb$--gps-end-time$ \textsc{tend} 
\verb$--time-step$ \textsc{tstep} 
[\verb$--datafile$ \textsc{file}]

\item[Description] 
\verb$lalapps_inspinj$
generates a number of inspiral  parameters suitable  for using in a Monte Carlo
injection to test the efficiency of a inspiral search.  The  various
parameters (detailed  below)  are randomly chosen and are appropriate for a
particular population of binary neutron stars  whose spatial  distribution
includes the Milky Way and a number of extragalactic objects that are  input
in  a  datafile.  The  possible  mass pairs for the binary neutron star com-
panions are also specified in a (different) datafile.

The output of this program  is  a  list  of  the  injected events,  starting
at  the specified start time, ending at the specified end time, and containing
one set  of  random inspiral parameters every specified time step.  The output
is written to a file called injections-X.xml, where  X  is the  value  of  the
random number seed chosen, which is a LIGOLw format file.  In addition, an
ascii log file called injlog.txt.

\verb+lalapps_inspinj+ also produces two ILWD-format files, injepochs.ilwd and
injparams.ilwd, that contain, respectively, the  GPS  times  suitable for
inspiral injections, and the intrinsic inspiral signal parameters to be used
for  those injections.

The  file  injepochs.ilwd  contains  a sequence of integer pairs representing
the injection GPS time in  seconds  and residual  nano-seconds.   The file
injparams.ilwd contains the intrinsic binary parameters for each injection,
which is  a  sequence  of  eight  real  numbers representing (in order) (1) the
total mass of the binary system  (in  solar masses),  (2)  the  dimensionless
reduced mass --- reduced mass per unit total mass --- in the range from  0
(extreme mass  ratio)  to  0.25 (equal masses), (3) the distance to the system
in meters, (4) the inclination  of  the  binary system  orbit  to the plane of
the sky in radians, (5) the colaescence phase in radians, (6)  the  longitude
to  the direction  of  the  source in radians, (7) the latitude to the
direction of the source in radians, (8) and the polar- ization angle of the
source in radians.


\item[Options]\leavevmode
\begin{entry}
\item[\texttt{--help}, -h] Print a help message.

\item[\texttt{--gps-start-time}=\textsc{seed}, \texttt{-s} \textsc{seed}]
Sets the random number generator seed.

\item[\texttt{--gps-start-time}=\textsc{tstart}, \texttt{-a} \textsc{tstart}]
Sets the start time in GPS seconds.

\item[\texttt{--gps-end-time}=\textsc{tend}, \texttt{-b} \textsc{tend}]
Sets the stop time in GPS seconds.

\item[\texttt{--time-step}=\textsc{tstep}, \texttt{-t} \textsc{tstep}]
Sets the time step interval in seconds.

\item[\texttt{--datafile}=\textsc{file}, \texttt{-f} \textsc{file}]
The seed used to initialize the random number generator.
\end{entry}

\item[Environment]\leavevmode

\begin{entry}
\item[\texttt{LALAPPS\_DATA\_PATH}]
Directory to look for required data file \verb+BNSMasses.dat+ and the
default data file \verb+inspsrcs.dat+.
\end{entry}

\item[Author] 
Jolien Creighton, Patrick Brady
\end{entry}



\clearpage
\subsection{Program \texttt{lalapps\_inspinj\_s1}}
\label{program:lalapps-inspinj-s1}
\idx[Program]{lalapps\_inspinj\_s1}

\begin{entry}
\item[Name]
\verb$lalapps_inspinj_s1$ --- generates a list of binary systems and
their parameters suitable for use with the inspiral search code.

\item[Synopsis]
\verb$lalapps_inspinj_s1$ 
[\verb$--gps-start-time$ \textsc{gpsstarttime}] 
[\verb$--gps-end-time$ \textsc{gpsendtime}] 
[\verb$time-step$ \textsc{tstep}] 
\verb$--seed$ \textsc{seed}

\item[Description] 
\verb$lalapps_inspinj_s1$ contructs a list of binary neutron star
inspiral signal parameters based on the mass distribution supplied in
the file \texttt{BNSMasses.dat},  a spatial distribution representing
the Milky Way,  LMC,  SMC and M31.   (The code should be extended to
include other galaxies out to LIGO design range.)   The internal
angles are randomly distributed,  the host galaxy is randomly selected
according to the blue-light luminosity.   The interval between
injections is constant and determined by $\textsc{tstep}/\pi$ to
insure that it is not commensurate with any other times in the
problem.   The injections span the time interval from \textsc{gpsstarttime} 
to \textsc{gpsendtime}.

\item[Options]\leavevmode
\begin{entry}
\item[\texttt{--gps-start-time} \textsc{gpsstarttime}] Optional.  The earliest
time for which an injection should be constructed.   Default is 
714150013 (Aug 23, 2002  08:00:00 PDT).

\item[\texttt{--gps-end-time} \textsc{gpsendtime}] Optional. The latest
time for which an injection should be constructed.  Default is
715618813 (Sep 09, 2002  08:00:00 PDT).

\item[\texttt{--time-step} \textsc{timestep}] Optional. The fixed interval
between injections is $\textsc{timestep}/\pi$ seconds.   Default is the fixed
interval $2630 / \pi$ seconds between injections.

\item[\texttt{--seed} \textsc{seed}] The seed used to initialize the
random number generator.
\end{entry}

\item[Example]

\item[Author] 
Jolien Creighton (Duncan Brown and Patrick Brady)
\end{entry}


%%%%%%%%%%%%%%%%%%%%%%%%%%%%%%%%%%%%%%%%%%%%%%%%%%%%%%%%%%%%%%%%%%%%%%%%%%%%%%%%%%%
%
% PROGRAM:   snglInspiralReader
% 
%%%%%%%%%%%%%%%%%%%%%%%%%%%%%%%%%%%%%%%%%%%%%%%%%%%%%%%%%%%%%%%%%%%%%%%%%%%%%%%%%%%
\clearpage
\subsection{Program \texttt{lalapps\_snglInspiralReader}}
\label{program:lalapps-snglInspiralReader}
\idx[Program]{lalapps\_snglInspiralReader}

\begin{entry}
\item[Name]
\verb$lalapps_snglInspiralReader$ --- manipulates LIGO lightweight XML
files of inspiral triggers allowing cuts and clustering.

\item[Synopsis]
\verb$lalapps_snglInspiralReader$ 
\verb$--input$ \textsc{infile} \verb$--table$ \textsc{tablename} 
\verb$--outfile$ \textsc{outfile}
[\verb$--snrstar$ \textsc{snrstar}] [\verb$--noplayground$] 
[\verb$--sort$] [\verb$--cluster$ \textsc{msec}] 
[\verb$--clusterchoice$ \textsc{choicenumber}] [\verb$--help$]

\item[Description] 
\verb$lalapps_snglInspiralReader$ processes triggers from the inspiral
search code.   The \textsc{infile} should contain a list of the XML
files containing the triggers;  the format is one filename per line. 
The default behavior outputs triggers during playground
times to the file \textsc{outfile};  to obtain all triggers,  use the 
\verb$--noplayground$ flag.    To apply a cut on SNR,  use the flag 
\verb$--snrstar$ \textsc{snrstar}:  only triggers with $\texttt{SNR} 
> \textsc{snrstar}$ will be recorded.    Events can also be clustered
within \textsc{msec} msec,   in which case the \verb$--sort$ flag is
recommended unless you are certain that the triggers are time-ordered.  
There is a choice of several clustering algorithms, which can be selected
using \textsc{choicenumber}.


\item[Options]\leavevmode
\begin{entry}
\item[\texttt{--input} \textsc{infile}] Required.  A file containing a
list of LIGO lightweight XML files with triggers to be processed.  The
format of \textsc{infile} is one file name per line.

\item[\texttt{--table} \textsc{tablename}] Required.  The name of the 
XML table containing the inspiral events, this will usually be 
\texttt{sngl\_inspiral}.

\item[\texttt{--outfile} \textsc{outfile}] Required.  Name of the file
to be used for output.  The output format is LIGO lightweight XML with
only a \texttt{sngl\_inspiral} table.

\item[\texttt{--snrstar} \textsc{snrstar}] Optional.  A threshold cut
on signal-to-noise.  Only triggers with $\textsc{snr} > \textsc{snrstar}$
are recorded in the output file.

\item[\texttt{--noplayground}] Optional.  Record all triggers.  The
default behaviour returns only those triggers which lie in the
playground data set.  

\item[\texttt{--sort}] Optional.   Sort the triggers in time (before
clustering).  

\item[\texttt{--cluster} \textsc{msec}] Optional.  Cluster triggers
within \textsc{msec} msec window.   The clustering algorithm
identifies the first trigger in a cluster, then displaces it if another 
trigger within the clustering window satisfies the appropriate condition 
(described below).

\item[\texttt{--clusterchoice} \textsc{choicenumber}]  Optional.  This
determines which condition will be used in clustering of the triggers.  
The default is 0.  The choices are \textsc{choicenumber} $=0$ --- displace
event if its \textsc{snr} is exceeded by an event with a smaller 
\textsc{chisq}; \textsc{choicenumber} $=1$ --- displace event if the quantity
$(\textsc{snr})^{2}/\textsc{chisq}$ is exceeded by a subsequent event's.

\item[\texttt{--help}] Optional.  Print a help message.
\end{entry}

\item[Example]
\begin{verbatim}
lalapps_snglInspiralReader --input xmlfilelist \
--table sngl_inspiral --outfile my.xml --snrstar 8.0 \
--sort --cluster 20 clusterchoice 1
\end{verbatim}

\item[Author] 
Patrick Brady
\end{entry}
\clearpage


\subsection{Program \texttt{lalapps\_inca}}
\label{program:lalapps-inca}
\idx[Program]{lalapps\_inca}

\begin{entry}
\item[Name]
\verb$lalapps_inca$ --- program does inspiral coincidence analysis.

\item[Synopsis]
\verb$lalapps_inca$ 
\verb$--ifo-a$ \textsc{trigfile.a} \verb$--ifo-b$ \textsc{trigfile.b} 
\verb$--start-time$ \textsc{startcoincidence} \verb$--stop-time$ \textsc{endcoincidence} 
\verb$--drho-plus$ \textsc{drhoplus} \verb$--drho-minus$ \textsc{drhominus} 
\verb$--dt$ \textsc{deltat} \verb$--dm$ \textsc{deltam} 
\verb$--outfile$ \textsc{outfile} \verb$--noplayground$ 
[\verb$--help$]

\item[Description] 
\verb$lalapps_inca$ performs coincidence on triggers from the inspiral
search code.  (At present it works for only two interferometers.) It
must be called with at least one input file from each instrument. The
default behavior outputs triggers during playground times to the file
\textsc{outfile};  to obtain all triggers,  use the
\verb$--noplayground$ flag.     
%Events can also be clustered
%within \textsc{msec} msec or a template bank for triggered searches
%can be constructed.

\item[Options]\leavevmode
\begin{entry}
\item[\texttt{--ifo-a} \textsc{trigfile.a}] Required.  A LIGO lightweight
XML file with triggers from interferometer A.  This argument can be
called multiple times.  Triggers are sorted \emph{after} all files
have been read in. 

\item[\texttt{--ifo-b} \textsc{trigfile.b}] Required.  A LIGO lightweight
XML file with triggers from interferometer B.  This argument can be
called multiple times.  Triggers are sorted \emph{after} all files
have been read in. 

\item[\texttt{--start-time} \textsc{startcoincidence}] Optional.  Look for
coincident triggers with end times after \textsc{startcoincidence}.
If not supplied,  the $\textsc{startcoincidence} = 0$.

\item[\texttt{--stop-time} \textsc{endcoincidence}]  Optional. Look for
coincident triggers with end times before \textsc{endcoincidence}.
If not supplied,  then $\textsc{endcoincidence} = 977788813$, i.e.
00:00 Dec 31, 2010 UTC.

\item[\texttt{--drhoplus} \textsc{drhoplus}] Optional. Accept triggers with
$\rho_B < (\sigma_B/\sigma_A) \rho_A + \textsc{drhoplus}$ as
possible coincidences.  If not supplied,  then $\textsc{drhoplus} =
0$.

\item[\texttt{--drhominus} \textsc{drhominus}] Optional. Accept triggers with
$\rho_B > (\sigma_B/\sigma_A) \rho_A - \textsc{drhominus}$ as
possible coincidences.  If not supplied,  then $\textsc{drhominus} =
0$.

\item[\texttt{--dm} \textsc{deltam}] Optional. Accept triggers as coincident if
both mass parameters agree within \textsc{deltam}.  If not supplied,  then 
$\textsc{deltam} = 0$.

\item[\texttt{--dt} \textsc{deltat}] Optional. Accept triggers as coincident if
their end times agree within \textsc{deltat} in msec.  If not supplied,  then 
$\textsc{deltat} = 0$.

\item[\texttt{--outfile} \textsc{outfile}] Required.  Name of the file
to be used for output.  The output format is LIGO lightweight XML with
only a \texttt{sngl\_inspiral} table.

\item[\texttt{--noplayground}] Optional.  Record all triggers.  The
default behaviour returns only those triggers which lie in the
playground data set.  

\item[\texttt{--cluster} \textsc{msec}] Optional.  Cluster triggers
within \textsc{msec} msec window.   The clustering algorithm
identifies the first trigger in a cluster,  then displaces it if its
\textsc{snr} is exceeded by an event with a smaller \textsc{chisq}.

\item[\texttt{--help}] Optional.  Print a help message.
\end{entry}

\item[Example]
\begin{verbatim}
lalapps_inca --ifo-a L-INSPIRAL-734357353-1024.xml \
--ifo-b L-INSPIRAL-734357353-1024.xml --drhoplus 1.0 --drhominus 1.0 \
--dm 1.0 --dt 0.01 --outfile my.xml --start-time 734357353 \
--stop-time 734358353 --noplayground
\end{verbatim}
\item[Author] 
Patrick Brady
\end{entry}
\clearpage

