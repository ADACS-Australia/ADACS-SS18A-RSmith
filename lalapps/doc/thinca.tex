\section{Program \texttt{lalapps\_thinca}}
\label{program:lalapps-thinca}
\idx[Program]{lalapps\_thinca}

\begin{entry}
\item[Name]
\verb$lalapps_thinca$ --- program does the inspiral coincidence analysis.

\item[Synopsis]
\prog{lalapps\_thinca} \newline \hspace*{0.5in}
[\option{--help}] \newline \hspace*{0.5in} 
[\option{--verbose}] \newline \hspace*{0.5in} 
[\option{--version}] \newline \hspace*{0.5in}
\option{--debug-level}~\parm{level} \newline \hspace*{0.5in} 
[\option{--user-tag}~\parm{usertag}] \newline \hspace*{0.5in} 
\option{--gps-start-time}~\parm{start\_time} \newline \hspace*{0.5in}           
\option{--gps-end-time}~\parm{end\_time} \newline \hspace*{0.5in}            
[\option{--g1-triggers}]  \newline \hspace*{0.5in}
[\option{--h1-triggers}]  \newline \hspace*{0.5in}
[\option{--h2-triggers}]  \newline \hspace*{0.5in}
[\option{--l1-triggers}]  \newline \hspace*{0.5in}
[\option{--t1-triggers}]  \newline \hspace*{0.5in}
[\option{--v1-triggers}]  \newline \hspace*{0.5in}
\option{--parameter-test}
\parm{(m1\_and\_m2|psi0\_and\_psi3|mchirp\_and\_eta)}
\newline \hspace*{0.5in}
[\option{--dm}]~\parm{dm} \newline \hspace*{0.5in}
[\option{--psi0}]~\parm{dpsi0} \newline \hspace*{0.5in}
[\option{--psi3}]~\parm{dpsi3} \newline \hspace*{0.5in}
[\option{--dmchirp}]~\parm{dmchirp} \newline \hspace*{0.5in}
[\option{--deta}]~\parm{deta} \newline \hspace*{0.5in}
\option{--dt}~\parm{dt} \newline \hspace*{0.5in}
\option{--data-type}~\parm{(playground\_only|exclude\_play|all\_data)} 
\newline \hspace*{0.5in}
\option{LIGOLW XML input files} 

\item[Description --- General] 

\verb$lalapps_thinca$ performs a coincidence test between triggers from
different interferometers.  It reads in triggers from up to four instruments
and returns coincident triggers.  At present, the code returns all double
coincident triggers --- it does not check for triple or quadruple coincidences
even if there are more than two active detectors.

The user specifies which instruments there will be triggers input from with
the \texttt{g1-triggers}, \texttt{h1-triggers} etc. options.  If less than two
of these are specified, the program exits as there cannot be coincidence.  The
triggers are then read in from the list of LIGO Lightweight XML files given
after the last command line argument.  The code only keeps triggers which
occur between the \textsc{start\_time} and the \textsc{end\_time}.  If the
\texttt{check-times} option is specified, then the input search summary tables
are checked to ensure that we have searched all data between the
\textsc{start\_time} and \textsc{end\_time} in all relevant ifos.  Following
this, we discard the non-playground triggers if \textsc{playground\_only} was
specified and any playground triggers if \textsc{exclude\_play} was specified.
At this stage, we check that there are triggers from at least two instruments,
if not, the code exits without testing for coincidences. 

The code now tests for any pairs of coincident triggers.  This is done in the
function \texttt{LALCreateTwoIFOCoincList()}.  Triggers are considered
coincident if their end times differ by less than \textsc{dt} and their mass
parameters pass coincidence.  We test on one of \texttt{(m1\_and\_m2 |
mchirp\_and\_eta | psi0\_and\_psi3)}, making use of \textsc{dm};
\textsc{dpsi0} and \textsc{dpsi3}; \textsc{dmchirp} and \textsc{deta}
respectively to test the relevant mass parameters.  Additionally, if demanding
coincidence over m1 and m2, it then tests that 
%
\begin{equation} \frac{\left|D_\mathrm{IFOA} -
  D_\mathrm{IFOA}\right|}{D_\mathrm{IFOA}} <
  \frac{\epsilon}{\rho_\mathrm{IFOB}} + \kappa.  \end{equation}
% 
Currently, the values of $\kappa$ and $\epsilon$ are hardwired to $1000$ and
$2$ respectively.  This essentially disables the effective distance test.
Additionally, the time and mass coincidence parameters are the same for all
interferometer pairs.  This should be changed as our timing resolution between
H1 and H2 will be better than between other sites, also different instruments
may have better parameter estimation accuaracy.  At the end of the process, we
have a list of pairs of triggers from different instruments which pass the
time and mass coincidence tests.  These need to be searched for triple and
quadruple coincidences, but this has not yet been implemented.

The coincident triggers are written into a single LIGO Lightweight XML file.
In order that the coincident triggers can be easily located, the
\textsc{event\_id} field is populated.  This is a \texttt{UINT8} which is
populated with \textsc{start\_time} $\times 10^{9} +$ an integer identifier.
The integer identifier is unique within the file, so the overall \textsc{id}
will be unique.  

The output file is named
\begin{center}
\texttt{IFOS-THINCA\_USERTAG-GPSSTARTTIME-DURATION.xml}\\
\end{center}
where \textsc{IFOS} is a list of the active ifos in alphabetical order.  The
file contains \texttt{process}, \texttt{process\_params},
\texttt{search\_summvars} and \texttt{search\_summary} tables that describe
the search.  Additionally there is a \texttt{summ\_value} table which contains
the summ values which were contained in the input files (in aniticipation of
performing a distance cut) as well as the \texttt{sngl\_inspiral} table
containing the coincident events.

\item[Options]\leavevmode
\begin{entry}

\item[\texttt{--data-type}(playground\_only|exclude\_play|all\_data)]
Required.  Specify whether the code should use only the playground, exclude
the playground or use all the data. 

\item[\texttt{--gps-start-time} \textsc{start\_time}] Required.  Look
for coincident triggers with end times after \textsc{start\_time}.

\item[\texttt{--gps-end-time} \textsc{end\_time}] Required.  Look for
coincident triggers with end times before \textsc{end\_time}.


\item[\texttt{--g1-triggers}] Optional.  Specify that triggers from G1 will be
provided.

\item[\texttt{--h1-triggers}] Optional.  Specify that triggers from H1 will be
provided.
\item[\texttt{--h2-triggers}] Optional.  Specify that triggers from H2 will be
provided.

\item[\texttt{--l1-triggers}] Optional.  Specify that triggers from L1 will be
provided.
\item[\texttt{--t1-triggers}] Optional.  Specify that triggers from T1 will be
provided.
\item[\texttt{--v1-triggers}] Optional.  Specify that triggers from V1 will be
provided.  Note: while having triggers from each of the instruments is
optional, the code requires triggers from at least two instruments, otherwise
it is impossible to do coincidence.

\item[\texttt{--check-times}] Optional.  If this flag is set, the code checks
the input search summary tables to verify that the data for each of the
requested interferometers was analyzed once and only once between the
\textsc{start\_time} and \textsc{end\_time}.  By default, the code will not
perform this check.

\item[\texttt{--parameter-test}
(m1\_and\_m2|psi0\_and\_psi3|mchirp\_and\_eta)]
Required. Choose which parameters to use when testing for coincidenc.
Depending on which test is chosen, the allowed windows on the appropriate
parameters should be set as described below.

\item[\texttt{--dm} \textsc{dm}] Optional. Accept triggers as coincident if
both m1 and m2 agree within \textsc{dm}.  If not supplied, then \textsc{dm}$=
0$.

\item[\texttt{--dpsi0} \textsc{dpsi0}] Optional. Accept
triggers as coincident if $\psi_{0}$ parameters agree within
\textsc{dpsi0}.  If not supplied,  then \textsc{dpsi0}$= 0$.

\item[\texttt{--dpsi3} \textsc{dpsi3}] Optional. Accept
triggers as coincident if $\psi_{3}$ parameters agree within
\textsc{dpsi3}.  If not supplied,  then \textsc{dpsi3}$= 0$.

\item[\texttt{--dmchirp} \textsc{dmchirp}] Optional. Accept
triggers as coincident if mchirp agrees within \textsc{dmchirp}.  If not
supplied, then \textsc{dmchirp}$ = 0$.

\item[\texttt{--deta} \textsc{deta}] Optional. Accept triggers
as coincident if $\eta$ agrees within \textsc{deta}.  If not supplied,
then \textsc{deta} $= 0$.

\item[\texttt{--dt} \textsc{dt}] Required. Accept triggers as
coincident if their end times agree within \textsc{dt} milliseconds.  


\item[\texttt{--comment} \textsc{string}] Optional. Add \textsc{string}
to the comment field in the process table. If not specified, no comment
is added. 

\item[\texttt{--user-tag} \textsc{usertag}] Optional. Set the user tag for
this job to be \textsc{usertag}. May also be specified on the command line as
\texttt{-userTag} for LIGO database compatibility.  This will affect the
naming of the output file.

\item[\texttt{--verbose}] Enable the output of informational messages.

\item[\texttt{--help}] Optional.  Print a help message and exit.

\item[\texttt{--version}] Optional.  Print out the author, CVS version and
tag information and exit.

\item[\texttt{--debug-level} \textsc{level}] Optional. Set the LAL debug
level to \textsc{level}. If not specified the default is 1.

\end{entry}

\item[Arguments]\leavevmode
\begin{entry}
\item[\texttt{[LIGO Lightweight XML files]}] The arguments to the program
should be a list of LIGO Lightweight XML files containing the triggers from
the two interferometers. The input files can be in any order and do not need
to be time ordered as \texttt{thinca} will sort all the triggers once they are
read in. If the program encounters a LIGO Lightweight XML containing triggers
from an unknown interferometer (i.e. not IFO A or IFO B) it will exit with an
error.
\end{entry}

\item[Example]
\begin{verbatim}
lalapps_thinca \
--playground-only  --dm 0.03 --kappa 1000.0 --ifo-b H1 --ifo-a L1 \
--user-tag SNR6_INJ --debug-level 33 --gps-start-time 734323079
--gps-end-time 734324999 --epsilon 2.0 --dt 11.0 \
L1-INSPIRAL_INJ-734323015-2048.xml H1-INSPIRAL_INJ-734323015-2048.xml
\end{verbatim}

\item[Algorithm]
The code maintains two pointers to triggers from each ifo,
\texttt{currentTrigger[0]} and \texttt{currentTrigger[1]}, corresponding to
the current trigger from IFO A and B respectively.

\begin{enumerate}
\item An empty linked list of triggers from each interferometer is created.
Each input file is read in and the code determines which IFO the triggers in
the file correspond to. The triggers are appended to the linked list for the
corresponding interferometer.

\item If there are no triggers read in from either of the interferometers,
the code exits cleanly.

\item The triggers for each interferometer is sorted by the \texttt{end\_time}
of the trigger.

\item \texttt{currentTrigger[0]} is set to point to the first trigger from IFO
A that is after the specified GPS start time for coincidence. If no trigger is
found after the start time, the code exits cleanly.

\item Loop over each trigger from IFO A that occurs before the specified GPS
end time for coincidence:
\begin{enumerate}
\item \texttt{currentTrigger[1]} is set to point to the first trigger from IFO
B that is within the time coincidence window, $\delta t$, of
\texttt{currentTrigger[0]}. If no IFO B trigger exists within this window,
\texttt{currentTrigger[0]} is incremented to the next trigger from IFO A and
the loop over IFO A triggers restarts.

\item If the trigger \texttt{currentTrigger[0]} \emph{is, is not} in the
playground data, start looping over triggers from IFO B.
\begin{enumerate}
\item For each trigger from IFO B that is within $\delta t$ of
\texttt{currentTrigger[0]}
\item Call \texttt{LALCompareSnglInspiral()} to check if the triggers match as
determined by the options on the command line. If the trigger match, record
them for later output as coincident triggers.
\end{enumerate}

\item Increment \texttt{currentTrigger[0]} and continue loop over triggers
from IFO A.
\end{enumerate}
\end{enumerate}

\item[Author] 
Patrick Brady, Duncan Brown and Steve Fairhurst
\end{entry}


