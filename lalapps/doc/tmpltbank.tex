\section{Program \texttt{lalapps\_tmpltbank}}
\label{program:lalapps-tmpltbank}
\idx[Program]{lalapps\_tmpltbank}

\item[Synopsis]
\noindent \verb$lalapps_tmpltbank$ --- program to generate inspiral template banks.
\begin{verbatim}

--help                          --verbose                   
--version                       --debug-level LEVEL      
--user-tag STRING               --comment STRING       
--gps-start-time SEC            --gps-end-time SEC  
--pad-data T                    --glob-frame-data         
--frame-type TAG                --frame-cache          
--calibration-cache FILE        --channel-name CHAN          
--calibrated-data TYPE          --geo-high-pass-freq F   
--geo-high-pass-order O         --geo-high-pass-atten A  
--sample-rate F                 --resample-filter TYPE      	
--disable-high-pass             --enable-high-pass F         
--high-pass-order O             --high-pass-attenuation A   
--spectrum-type TYPE            --segment-length N      
--number-of-segments N          --standard-candle           
--candle-snr SNR                --candle-mass1 M             
--candle-mass2 M                --low-frequency-cutoff F     
--high-frequency-cutoff F       --minimum-mass MASS        
--maximum-mass MASS             --minimum-psi0 PSI0          
--maximum-psi0 PSI0             --minimum-psi3 PSI3          
--maximum-psi3 PSI3             --maximum-fcut-tmplts N    
--alpha ALPHA                   --minimal-match M       
--order ORDER                   --approximant APPROX          
--space SPACE                   --write-raw-data           
--write-response                --dynamic-range-exponent X  

\end{verbatim}

\item[Options]
\noindent The following command line arguments are available when running tmpltbank.c
\\
\begin{entry}
\item[\texttt{--help}] display the help message which gives brief explanations
of the command arguments.  
\item[\option{--verbose}] print progress information as the code executes.
\item[\option{--version}] print version information and exit without running 
the tmpltbank code. 
\item[\option{--debug-level} \textsc{LEVEL}] Sets the LAL debug level to \texttt{LEVEL}.  
1 and 33 are the commonest choices for this code.  Debug level 1 is used for 
developing code because it enables memory and error checking.  Since memory 
checking is slow the correct choice for production code is 33.  
See the LAL documentation for more information.  
\item[\option{--user-tag} \textsc{STRING}] Set the user tag to the string \texttt{STRING}.  
This string must not contain spaces or dashes (``-'').  This string will appear 
in the name of the file to which output information is written, and is recorded 
in the various XML tables within the file.
\item[\option{--comment} \textsc{STRING}] set the process table comment to STRING.
\item[\option{--gps-start-time} \textsc{SEC}] Set the integer part of the GPS time from 
which you wish to begin reading data.
\item[\option{--gps-end-time} \textsc{SEC}] Set the integar part of the GPS time you want
to stop reading data. 
\item[\option{--pad-data} \textsc{T}] This flag specifies an amount of time to add to 
the beginning and end of the input time series data.  Padding the data is 
necessary because resampling and filtering corrupts these portions. 
8 seconds is the accepted choice for this paramenter.  See LAL documentation 
for a description of resampling and high pass filtering.  
\item[\option{--glob-frame-data}] This option along with \texttt{--frame-type}
can be used instead of \texttt{--frame-cache} to read data stored locally in 
the working directory.  It finds files of the specified frame type with a *.gwf 
extension. 
\item[\option{--frame-cache}] This option is used instead of 
\texttt{--glob-frame-data} to read frame data from a frame cache file. 
\item[\option{--calibration-cache} \textsc{FILE}] obtain calibration from LAL frame 
cache FILE
\item[\option{--channel-name} \textsc{CHAN}] read data from interferometer channel CHAN
\item[\option{--calibrated-data} \textsc{TYPE}] calibrated data of TYPE real\_4 or real\_8

\item[\option{--geo-high-pass-freq} \textsc{F}] This sets the high pass filter frequency
for GEO data above F Hz using an IIR filter.
\item[\option{--geo-high-pass-order} \textsc{O}] set the order of the GEO high pass 
filter to O
\item[\option{--geo-high-pass-atten} \textsc{A}] set the attenuation of the high pass 
filter to A
\item[\option{--sample-rate} \textsc{F}] Specifies the sampling frequency at which you
want to filter the data downsampling if necessary.
\item[\option{--resample-filter} \textsc{TYPE}] set resample filter to TYPE [ldas|butterworth]
\item[\option{--disable-high-pass}] turn off the IIR highpass filter.  This is an 
optimistc option.  Someday the data will be so good we won't need high pass filtering!  
\item[\option{--enable-high-pass} \textsc{F}] high pass data above F Hz using an IIR filter
\item[\option{--high-pass-order} \textsc{O}] set the order of the high pass filter to O
\item[\option{--high-pass-attenuation}] A set the attenuation of the high pass filter to A
\item[\option{--spectrum-type} \textsc{TYPE}] use PSD estimator TYPE [mean|median]
\item[\option{--dynamic-range-exponent} \textsc{X}] set dynamic range scaling to ${2}^X$
\item[\option{--segment-length} \textsc{N}] set data segment length to N points
\item[\option{--number-of-segments} \textsc{N}] set number of data segments to N
\item[\option{--standard-candle}] compute a standard candle from the PSD
\item[\option{--candle-snr} \textsc{SNR}] signal-to-noise ratio of standard candle
\item[\option{--candle-mass1} \textsc{M}] mass of first component in candle binary
\item[\option{--candle-mass2} \textsc{M}] mass of second component in candle binary
\item[\option{--low-frequency-cutoff} \textsc{F}] do not filter below F Hz
\item[\option{--high-frequency-cutoff} \textsc{F}] upper frequency cutoff in Hz
\item[\option{--minimum-mass} \textsc{MASS}] set minimum component mass of bank to MASS
\item[\option{--maximum-mass} \textsc{MASS}] set maximum component mass of bank to MASS
\item[\option{--minimum-psi0} \textsc{ PSI0}] set minimum range of BCV parameter psi0 to PSI0
\item[\option{--maximum-psi0} \textsc{ PSI0}] set maximum range of BCV parameter psi0 to PSI0
\item[\option{--minimum-psi3} \textsc{ PSI3}] set minimum range of BCV parameter psi3 to PSI3
\item[\option{--maximum-psi3} \textsc{ PSI3}] set maximum range of BCV parameter psi3 to PSI3
\item[\option{-maximum-fcut-tmplts} \textsc{ N}] maximum number of tmplts in fcut direction is N
\item[\option{--alpha} \textsc{ ALPHA}] set BCV amplitude correction to ALPHA
\item[\option{--minimal-match} \textsc{ M}] The minimal match M between templates in the 
bank and all possible signals in the parameter space. 
\item[\option{--order} \textsc{ ORDER}] This sets the post-Newtonian order of the waveform to
ORDER.   
\item[\option{--approximant} \textsc{ APPROX}] Sets the approximant of the waveform to APPROX
\item[\option{--space} \textsc{ SPACE}] In order to make the template bank coordinates nice
and friendly these parameters are used instead of masses.  
\item[\option{--write-raw-data}] write raw data to a frame file
\item[\option{--write-response}] write the computed response function to a frame
\item[\option{--write-spectrum}] write the uncalibrated psd to a frame
\item[\option{--write-strain-spectrum}] write the calibrated strain psd to a text file
\end{entry}

\item[Description]
\begin{entry} 
\noindent \verb$lalapps_tmpltbank$ is a stand alone code for generating inspiral
template banks for LIGO or GEO data with the LAL bank package.  The code 
generates a calibrated power spectrum at the specified time for the 
requested channel and uses this to compute the template bank.  
The number of templates and the
values of the bank parameters in the bank also depend on the minimal
match, the
minimum and maximum values of mass1 and mass2 (for the BNS search) or the
minimum and maximum values of psi0, psi3, the bank-alpha and the number of
fcut values (for the BCV search), which are all command-line arguments.
Other necessary pieces of information are the approximant and its order and
the space that the template bank will be laid on. The output of the code is
an xml file and the bank is contained in a \verb$sngl_inspiral$ table. The code has
also the capability of outputing the raw data, the response function and the
calibrated and unclibrated power spectra to frame files.
See the LAL bank package
documentation for detailed information on the algorithms used to generate the
template banks.
\end{entry}

\item[Example]

\begin{verbatim}
lalapps_tmpltbank \
--gps-start-time 734357353 --gps-end-time 734358377 \
--frame-cache cache/L-734357345-734361107.cache \
--segment-length 1048576 --number-of-segments 7 \
--pad-data 7 --sample-rate 4096 --resample-filter ldas \
--enable-high-pass 5.000000e+01 --spectrum-type median
--low-frequency-cutoff 7.000000e+01 --high-frequency-cutoff 2.048000e+03 \
--minimum-mass 1.000000e+00  --maximum-mass 3.000000e+00 \
--minimal-match 9.700000e-01 --calibration-cache  \
/ldas_outgoing/calibration/cache_files/L1-CAL-V03-729273600-734367600.cache \
--space Tau0Tau3 --approximant TaylorT1 --order twoPN \
--channel-name L1:LSC-AS_Q --debug-level 33

\end{verbatim}

\item[Author]
\noindent Duncan Brown

