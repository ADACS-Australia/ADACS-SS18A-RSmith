\section{Program \texttt{lalapps\_inspiral\_pipe}}
\label{program:inspiral-pipeline}
\idx[Program]{inspiral\_pipeline.py}

\begin{entry}
\item[Name]
\verb$lalapps_inspiral_pipe$ --- python script to generate Condor DAGs to
run the inspiral pipeline.

\item[Synopsis]
\begin{verbatim}
  -h, --help               display this message
  -v, --version            print version information and exit
  -u, --user-tag TAG       tag the job with TAG (overrides value in ini file)

  -d, --datafind           run LALdataFind to create frame cache files
  -t, --template-bank      run lalapps_tmpltbank to generate a template bank
  -i, --inspiral           run lalapps_inspiral on the first IFO
  -T, --triggered-bank     run lalapps_trigtotmplt to generate a triggered bank
  -I, --triggered-inspiral run lalapps_inspiral on the second IFO
  -C, --coincidence        run lalapps_inca on the triggers from both IFOs

  -j, --injections         add simulated inspirals from injection file

  -p, --playground-only    only create chunks that overlap with playground
  -P, --priority PRIO      run jobs with condor priority PRIO

  -f, --config-file FILE   use configuration file FILE
  -l, --log-path PATH      directory to write condor log file
\end{verbatim}

\item[Description] \verb$lalapps_inspiral_pipe$ generates a Condor DAG to run
the inspiral analysis pipeline. The configuration file should specify the
parameters needed to run the jobs and must be specified with the
\verb$--config-file$ option. A typical .ini file is the following:
\begin{verbatim}
; S2 inspiral pipeline configuration script.
; 
;
; this is the configuration file for the inspiral DAG generation program that
; creates a condor DAG to run the inspiral analysis pipeline.

[condor]
universe = standard
datafind  = LSCdataFind
tmpltbank = lalapps_tmpltbank
inspiral  = lalapps_inspiral
trigtotmplt = lalapps_inca
inca = lalapps_inca

[pipeline]
version = $Id$
user-tag = 
ifo1 = L1
ifo2 = H1
ifo1-snr-threshold = 6.0
ifo2-snr-threshold = 6.0
ifo1-chisq-threshold = 100.0
ifo2-chisq-threshold = 100.0

[input]
segments = S2H1L1v04_selectedsegs.txt
channel = LSC-AS_Q

[calibration]
path = /ldas_outgoing/calibration/cache_files
L1 = L1-CAL-V03-729273600-734367600.cache
H1 = H1-CAL-V03-729273600-734367600.cache
H2-1 = H2-CAL-V03-729296220-731849040.cache
H2-2 = H2-CAL-V03-731849076-734367576.cache
H2-cal-epoch-boundary = 731849076

[datafind]
type = RDS_R_L1
lal-cache = 

[data]
pad-data = 8
segment-length = 1048576
number-of-segments = 15
sample-rate = 4096
resample-filter = ldas
enable-high-pass = 100.0
spectrum-type = median
low-frequency-cutoff = 100.0
high-pass-order = 8
high-pass-attenuation = 0.1

[tmpltbank]
minimum-mass = 3.0 
maximum-mass = 20.0
minimal-match = 0.95
high-frequency-cutoff = 2048.0
order = twoPN
approximant = TaylorF2 {for BNS} BCV {for BCV}
space = Tau0Tau3 {for BNS} Psi0Psi3 {for BCV}
; the following are necessary for the BCV search
minimum-psi0 = 10.0
maximum-psi0 = 550000.0
minimum-psi3 = -4000.0
maximum-psi3 = -10.0
alpha = 0.0
maximum-fcut-tmplts = 3
; end of BCV-necessary tmpltbank arguments
`
[inspiral]
minimal-match = 0.9
segment-overlap = 524288
inverse-spec-length = 16
dynamic-range-exponent = 69.0
enable-output = 
enable-event-cluster = 
chisq-bins = 0
approximant = TaylorF2 {for BNS} BCV {for BCV}

[trigtotmplt]
minimal-match = 0.95
parameter-test = m1_and_m2 {for BNS} psi0_and_psi3 {for BCV}

[inca]
playground-only =
epsilon = 2.0
kappa = 5000.0
dt = 15.0
dm = 0.03
parameter-test =m1_and_m2 {for BNS} psi0_and_psi3 {for BCV}
; the following are necessary for the BCV search only
dpsi0 = 0.0
dpsi3 = 0.0
; end of BCV-necessary arguments
\end{verbatim}

A file containing science segments to be
analyzed should be specified in the \verb$[input]$ section of the
configuration file with a line such as
\begin{verbatim}
segments = S2H1L1v03_selectedsegs.txt
\end{verbatim}
This should contain four whitespace separated columns:
\begin{verbatim}
  segment_id    gps_start_time  gps_end_time    duration
\end{verbatim}
that define the science segments to be used. Lines starting with an octothorpe
are ignored.

The analysis chunk size is determined from the number of data segments and
their length and overlap specified in config file. A chunk length is 
is 2048 seconds for S2.  The chunks start and stop times are computed
from the science segment times and used to build the DAG.

Once the DAG file has been created it should be submitted to the Condor
pool with the \verb$condor_submit_dag$ command.

\item[Options]\leavevmode
\begin{entry}
\item[\texttt{--help}] Display a brief usage summary.
\end{entry}

\item[Example]
Generate a DAG to run an inspiral search on the first IFO. The generated
DAG is then submitted with \texttt{condor\_submit\_dag}
\begin{verbatim}
lalapps_inspiral_pipe --log-path /people/duncan/dag_logs \
--datafind --template-bank --inspiral --playground-only \
--config-file l1_s2.ini

condor_submit_dag l1_s2.dag
\end{verbatim}

\item[Author] 
Duncan Brown
\end{entry}

