\section{Program \texttt{lalapps\_inspfrinj}}
\label{program:lalapps-inspfrinj}
\idx[Program]{lalapps\_inspfrinj}

\begin{entry}
\item[Name]
\verb$lalapps_inspfrinj$ --- performs inspiral injections into frame data.

\item[Synopsis]
\begin{verbatim}
lalapps_inspfrinj [options]
 
 [--help ]       
 [--verbose ]   
 [--version ]  
 [--debug-level LEVEL ]   
 [--user-tag STRING ]     
 [--comment STRING ]    
 
  --gps-start-time SEC
 [--gps-start-time-ns NS ] 
  --gps-end-time SEC     
 [--gps-end-time-ns NS ] 
 
 [--frame-cache FRAME_CACHE    
  --channel-name CHAN ]  
 
 [--ifo  IFO       
  --sample-rate SAMPLE_RATE ] 

  --calibration-cache CAL_CACHE 
  --calibrated-data TYPE  
  
  --num-resp-points N    
 
  --injection-file FILE
 [--inject-overhead ]  
 [--inject-safety SEC ]
 
 [--write-raw-data ] 
 [--write-inj-only ]
 [--write-raw-plus-inj ]
 
  --output-frame-length OUTPUT_LENGTH
 [--output-file-name OUTPUT_NAME ]
 
\end{verbatim}

\item[Description] 

\texttt{lalapps\_frinspinj} generates injections and stores them as
frame files.  This code is essentially a copy of those parts of
\texttt{lalapps\_inspiral} responsible for reading in the data and
performing the injections.  The injection details are read in from the
\texttt{sim\_inspiral} table of a LIGO lightweight xml file.  The
injections are then generated using LALFindChirpInjectSignals.  The
injection data is output to frame files.  The length of the output
frames is specified by \texttt{output-frame-length}.

The code is capable of generating both calibrated and uncalibrated
injections.  To obtain calibrated h(t) injections the option
\texttt{calibrated-data} must be specified (at present the code only
works for real\_4 data).  Alternatively, if a \textsc{CAL\_CACHE} is
given, then uncalibrated injections will be produced using the a
response function generated from the calibration data in the cache.  The
calibration coefficients will by default be averaged over the duration
of the data.  When working with uncalibrated data, the number of points
used to determine the response function will have an effect, albeit
minor, on the injection signals.  By default, the value
\texttt{num-resp-points} is set to 4194304.  This matches the value used
by \texttt{lalapps\_inspiral} when working with 256 second segments and
a channel sampled at 16384 Hz.

If input data is given by a \textsc{FRAME\_CACHE}, the code will read in
this data.  In this case, the sample rate and ifo will be determined
from the data and channel name respectively.  The output frames can
contain any of the following: raw data, injection only and raw plus
injection.  

If no input data is provided, then the \textsc{IFO} and
\textsc{SAMPLE\_RATE} must be given on the command line.  Since no input
data is read in, the output data can only contain the injection only
channel.

The output of \verb$lalapps_frinspinj$ is an xml file and one or several
frame files.  The xml file contains
\verb$process$, \verb$process_params$, \verb$search_summary$ and
\verb$sim_inspiral$ tables and is named (unless set on the command line):
\begin{center}
\texttt{IFO-INSPFRINJ\_USERTAG-GPSSTARTTIME-DURATION.xml}\\
\end{center}
where \texttt{GPSSTARTTIME} and \texttt{DURATION} are the start time and
duration passed to the code.  If \verb$--output-file-name$ is set to
\verb$OUTPUT$ on the command line, the xml will be named
\begin{center}
\texttt{OUTPUT\_USERTAG-GPSSTARTTIME-DURATION.xml}.\\
\end{center}
The \verb$sim_inspiral$ table contains a list of all injections which
were performed into the specified data.  The length of output frame
files is specified by \verb$output-frame-length$.  Unless otherwise set
on the command line, output frames are named: 
\begin{center}
\texttt{IFO-INSPFRINJ\_USERTAG-FRAMESTARTTIME-FRAMELENGTH.gwf}\\
\end{center}
The channels stored in the output frame are determined by which of
\verb$--write-raw-data$, \verb$--write-inj-only$ and
\verb$--write-raw-plus-inj$ are selected.  The channel names are
\verb$CHAN$, \verb$CHAN_INSP_INJ_ONLY$ and
\verb$CHAN_RAW_PLUS_INSP_INJ$.  In the case where no input data is
given, the single output channel name is \verb$IFO:STRAIN_INSP_INJ_ONLY$.

\item[Options]\leavevmode
\begin{entry}

\item[\texttt{--help}]  Optional.  Print a help message and exit.

\item[\texttt{--verbose}] Optional.  Print out the author, CVS version
and tag information and exit.

\item[\texttt{--version}] Optional.  Print out the author, CVS version and
tag information and exit.
  
\item[\texttt{--debug-level} \textsc{LEVEL}] Optional. Set the LAL debug
level to \textsc{level}. If not specified the default is 1.

\item[\texttt{--user-tag} \textsc{USERTAG}] Optional. Set the user tag
for this job to be \textsc{USERTAG}. May also be specified on the command
line as \texttt{-userTag} for LIGO database compatibility.  This will
affect the naming of the output file.

\item[\texttt{--comment} \textsc{string}] Optional. Add \textsc{string}
to the comment field in the process table. If not specified, no comment
is added. 

\item[\texttt{--gps-start-time} \textsc{GPS seconds}] Required.  Start
time, GPS seconds.

\item[\texttt{--gps-start-time-ns} \textsc{GPS seconds}] Optional.
Start time, GPS nanoseconds.  If not specified, then set to zero.

\item[\texttt{--gps-end-time} \textsc{GPS seconds}] Required.  End
time, GPS seconds.

\item[\texttt{--gps-end-time-ns} \textsc{GPS seconds}] Optional.
Start time, GPS nanoseconds.  If not specified, then set to zero.

\item[\texttt{--frame-cache} \textsc{FRAME\_CACHE}] Optional.  Name of
the frame cache file.  If this is not specified then the code will not
read in any input data.  Hence it will be unable to produce the raw or
raw\_plus\_inj frames.

\item[\texttt{--channel-name} \textsc{CHAN}] Optional.  Input data
channel name.  This must be specified if the \texttt{frame-cache} file
is specified. 
 
\item[\texttt{--ifo} \textsc{IFO}] Optional.  Set input interferometer
name to \textsc{IFO}.  This information is required unless a
\texttt{channel-name} is specified (in this case the \textsc{IFO} is
obtained from the channel name).

\item[\texttt{--sample-rate} \textsc{SAMPLE\_RATE}] Optional.  Set the
sample rate (in Hz) at which the injections should be produced.  This is
required unless a frame cache is given (in this case, the sample rate of
the injections matches the input data).


\item[\texttt{--calibration-cache} \textsc{CAL\_CACHE}] Optional.
Specify the \textsc{CAL\_CACHE} to obtain the calibration information.

\item[\texttt{--calibrated-data }\textsc{TYPE}]  Optional.  Use
calibrated data as input/output.  This will produce strain data
injections.  This option must be specified if no calibration cache is
given.  
  
\item[\texttt{--num-resp-points} \textsc{N}] Required.  The number of
points used to generate the response function.  If generating
uncalibrated injections, this can actually change slightly the injection
data.   
 
\item[\texttt{--injection-file} \textsc{FILE}] Required. Read in the
injection data from \textsc{FILE}.  This should contain a
\texttt{sim\_inspiral} table with a list of injections.

\item[\texttt{--inject-overhead}] Optional.  Perform the injections
directly overhead the interferometer and optimally oriented.

\item[\texttt{--inject-safety} \textsc{SEC}] Optional.  Inject signals
whose end time is up to \texttt{SEC} after the gps end time.
 
\item[\texttt{--write-raw-data}]  Optional.  Write out a frame
containing the raw data.  This can only be done if the raw data has been
read in from frames.

\item[\texttt{--write-inj-only}]  Optional.  Write out a frame
containing only the injection data.

\item[\texttt{--write-raw-plus-inj}]  Optional.  Write out a frame
containing the raw data with the injections added.
  
\item[\texttt{--output-frame-length} \textsc{OUTPUT\_LENGTH}]  Required.
Specify the length of the output frames to be \textsc{output\_length}
seconds.

\item[\texttt{--output-file-name} \textsc{OUTPUT\_NAME} ] Optional.
Set the output file name.  This overrides the default file naming
convention given above.

\end{entry}


\item[Example]
\begin{verbatim}
lalapps_frinspinj --gps-start-time 732758030  --gps-end-time 732760078 \
  --frame-cache cache/L-732758022-732763688.cache --channel-name L1:LSC-AS_Q\	
  --calibration-cache cache_files/L1-CAL-V03-729273600-734367600.cache \
  --injection-file HL-INJECTIONS_4096-729273613-5094000.xml \
  --inject-safety 50 --output-frame-length 16 \
  --write-raw-data   --write-inj-only   --write-raw-plus-inj 
\end{verbatim}

\item[Author] 
Steve Fairhurst
\end{entry}

