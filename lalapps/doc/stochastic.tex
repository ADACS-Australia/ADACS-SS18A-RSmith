
\section{Program \prog{lalapps\_stochastic}}
\label{program:lalapps-stochastic}
\idx[Program]{lalapps\_stochastic}

\begin{entry}
\item[Name]
\prog{lalapps\_stochastic} --- standalone stochastic analysis code.

\item[Synopsis]
\prog{lalapps\_stochastic} \newline \hspace*{0.5in}
[\option{--help}] \newline \hspace*{0.5in}
[\option{--version}] \newline \hspace*{0.5in}
[\option{--verbose}] \newline \hspace*{0.5in}
[\option{--debug}] \newline \hspace*{0.5in}
[\option{--debug-level}~\parm{N}] \newline \hspace*{0.5in}
[\option{--user-tag}~\parm{STRING}] \newline \hspace*{0.5in}
[\option{--comment}~\parm{STRING}] \newline \hspace*{0.5in}
[\option{--output-dir}~\parm{DIR}] \newline \hspace*{0.5in}
[\option{--cc-spectra}] \newline \hspace*{0.5in}
\option{--gps-start-time}~\parm{N} \newline \hspace*{0.5in}
\option{--gps-end-time}~\parm{N} \newline \hspace*{0.5in}
\option{--interval-duration}~\parm{N} \newline \hspace*{0.5in}
\option{--segment-duration}~\parm{N} \newline \hspace*{0.5in}
\option{--resample-rate}~\parm{N} \newline \hspace*{0.5in}
\option{--f-min}~\parm{N} \newline \hspace*{0.5in}
\option{--f-max}~\parm{N} \newline \hspace*{0.5in}
\option{--ifo-one}~\parm{IFO} \newline \hspace*{0.5in}
\option{--ifo-two}~\parm{IFO} \newline \hspace*{0.5in}
\option{--channel-one}~\parm{CHANNEL} \newline \hspace*{0.5in}
\option{--channel-two}~\parm{CHANNEL} \newline \hspace*{0.5in}
\option{--frame-cache-one}~\parm{FILE} \newline \hspace*{0.5in}
\option{--frame-cache-two}~\parm{FILE} \newline \hspace*{0.5in}
\option{--calibration-cache-one}~\parm{FILE} \newline \hspace*{0.5in}
\option{--calibration-cache-two}~\parm{FILE} \newline \hspace*{0.5in}
\option{--calibration-offset}~\parm{N} \newline \hspace*{0.5in}
[\option{--apply-mask}~\parm{N} \newline \hspace*{0.5in}
\option{--mask-bin}~\parm{N}] \newline \hspace*{0.5in}
[\option{--overlap-hann} \newline \hspace*{0.5in}
\option{--hann-duration}~\parm{N}] \newline \hspace*{0.5in}
[\option{--high-pass-filter} \newline \hspace*{0.5in}
\option{--hpf-frequency}~\parm{N} \newline \hspace*{0.5in}
\option{--hpf-attenuation}~\parm{N} \newline \hspace*{0.5in}
\option{--hpf-order}~\parm{N}] \newline \hspace*{0.5in}
\option{--recentre} \newline \hspace*{0.5in}
\option{--middle-segment} \newline \hspace*{0.5in}
[\option{--geo-hpf-frequency}~\parm{N} \newline \hspace*{0.5in}
\option{--geo-hpf-attenuation}~\parm{N} \newline \hspace*{0.5in}
\option{--geo-hpf-order}~\parm{N}] \newline \hspace*{0.5in}
[\option{--alpha}~\parm{N}] \newline \hspace*{0.5in}
[\option{--f-ref}~\parm{N}] \newline \hspace*{0.5in}
[\option{--omega0}~\parm{N}]

\item[Description]
\prog{lalapps\_stochastic} runs the standalone stochastic analysis code.

\item[Options]\leavevmode
\begin{entry}
\item[\option{--help}]
Display usage information and exit.

\item[\option{--version}]
Display version information and exit.

\item[\option{--verbose}]
Enable the output of informational messages.

\item[\option{--debug}]
Run in debug mode, saves out all intermediate products as ASCII files.

\item[\option{--debug-level}~\parm{N}]
Sets the LAL debug level to \parm{N}. The default value is
\texttt{LALMSGLVL2}, displaying error and warning messages. A useful
setting is 65 which turns off memory padding, but keeps memory tracking
and error messages. If you want to turn off memory tracking completly,
then use 33.

\item[\option{--user-tag}~\parm{STRING}]
Set the user tag to the string \parm{STRING}. This string must not
contain spaces or dashes (``-''). This string will appear in the name of
the file to which output information is written, and is recorded in the
various XML tables within the file.

\item[\option{--comment}~\parm{STRING}]
Set the process table comment to \parm{STRING}

\item[\option{--output-dir}~\parm{DIR}]
Set the output directory for search results to \parm{DIR}

\item[\option{--cc-spectra}]
Save out cross correlation spectra as frame files.

\item[\option{--gps-start-time}~\parm{N}]
Sets the GPS time from which data should be read to \parm{N}

\item[\option{--gps-end-time}~\parm{N}]
Sets the GPS time to which data should be read to \parm{N}

\item[\option{--interval-duration}~\parm{N}]
Sets the interval duration to \parm{N}

\item[\option{--segment-duration}~\parm{N}]
Sets the segment duration to \parm{N}

\item[\option{--resample-rate}~\parm{N}]
Down-convert the input data stream to a sample rate of \parm{N} samples
per second prior to analysis.  This can be used to reduce the number of CPU
cycles required to analyze a given quantity of input data.

\item[\option{--f-min}~\parm{N}]
Sets the minimum frequency of the search band to \parm{N}

\item[\option{--f-max}~\parm{N}]
Sets the maximum frequency of the search band to \parm{N}

\item[\option{--ifo-one}~\parm{IFO}]
Sets the IFO for the first stream to be \parm{IFO}, currently supported
IFO's are H1, H2, L1 and G1

\item[\option{--ifo-two}~\parm{IFO}]
Sets the IFO for the second stream to be \parm{IFO}, currently supported
IFO's are H1, H2, L1 and G1

\item[\option{--channel-one}~\parm{CHANNEL}]
Sets the channel for the first stream to be \parm{CHANNEL}

\item[\option{--channel-two}~\parm{CHANNEL}]
Sets the channel for the second stream to be \parm{CHANNEL}

\item[\option{--frame-cache-one}~\parm{FILE}]
Obtain the locations of input \texttt{.gwf} frame files from the LAL frame
cache file \parm{FILE} for the first detector.  LAL frame cache files
are explained in the ``framedata'' package in LAL and can be constructed
by using \prog{LSCdataFind} on supported systems.

\item[\option{--frame-cache-two}~\parm{FILE}]
Obtain the locations of input \texttt{.gwf} frame files from the LAL frame
cache file \parm{FILE} for the second detector.  LAL frame cache files
are explained in the ``framedata'' package in LAL and can be constructed
by using \prog{LSCdataFind} on supported systems.

\item[\option{--calibration-cache-one}~\parm{FILE}]
Specify the location of calibration information for the first detector.
\parm{FILE} gives the path to a LAL-format frame cache file describing
locations of \texttt{.gwf} frame files that provide the calibration data
($\alpha$ and $\beta$ coefficients) for the analysis.  Frame cache files
are explained in the ``framedata'' package in LAL.

\item[\option{--calibration-cache-two}~\parm{FILE}]
Specify the location of calibration information for the second detector.
\parm{FILE} gives the path to a LAL-format frame cache file describing
locations of \texttt{.gwf} frame files that provide the calibration data
($\alpha$ and $\beta$ coefficients) for the analysis.  Frame cache files
are explained in the ``framedata'' package in LAL.

\item[\option{--calibration-offset}~\parm{N}]
Sets the calibration offset to \parm{N}

\item[\option{--apply-mask}]
Apply frequency masking

\item[\option{--mask-bin}~\parm{N}]
Set the number of bins to mask per frequency to \parm{N}

\item[\option{--overlap-hann}]
Use overlapping Hann windows for data segments

\item[\option{--hann-duration}~\parm{N}]
Set the Hann duration of the data segment window to \parm{N}, 0 for
Rectangular windowing, 1 for Tukey windowing and 60 for Hann windowing

\item[\option{--high-pass-filter}]
Apply a high pass filter to the input data

\item[\option{--hpf-frequency}~\parm{N}]
Set the knee frequency of the high pass filter to \parm{N}

\item[\option{--hpf-attenuation}~\parm{N}]
Set the attenuation coefficent for the high pass filter to \parm{N}

\item[\option{--hpf-order}~\parm{N}]
Sets the high pass filter order to \parm{N}

\item[\option{--recentre}]
Centre the data

\item[\option{--middle-segment}]
Include the middle segment in the power spectra estimation

\item[\option{--geo-hpf-frequency}~\parm{N}]
Set the knee frequency for the GEO high pass filter to \parm{N}

\item[\option{--geo-hpf-attenuation}~\parm{N}]
Set the attenuation coefficient for the GEO high pass filter to \parm{N}

\item[\option{--geo-hpf-order}~\parm{N}]
Set the GEO high pass filter order to \parm{N}

\item[\option{--alpha}~\parm{N}]
Exponent for $\Omega_{\mathrm{GW}}$ for construction of the optimal
filter.

\item[\option{--f-ref}~\parm{N}]
Reference frequency for $\Omega_{\mathrm{GW}}$ for the construction of
the optimal filter.

\item[\option{--omega0}~\parm{N}]
Reference $\Omega_0$ for $\Omega_{\mathrm{GW}}$ for the construction of
the optimal filter.
\end{entry}

\item[Example]
\prog{lalapps\_stochastic} is generally run as part of a DAG, as created
by the pipeline generation scripts, \prog{lalapps\_stochastic\_pipe} or
\prog{lalapps\_stochastic\_bayes}, however an example usage can be seen
below.

\begin{verbatim}
> lalapps_stochastic --debug-level 33 --verbose \
>   --gps-start-time 752242398 --gps-end-time 752242758 \
>   --interval-duration 180 --segment-duration 60 \
>   --resample-rate 1024 --f-min 50 --f-max 250 --ifo-one H1 \
>   --ifo-two H2 --channel-one LSC-AS_Q --channel-two LSC-AS_Q \
>   --frame-cache-one H1.cache --frame-cache-two H2.cache \
>   --calibration-cache-one H1-CAL-V02-751651244-757699245.cache \
>   --calibration-cache-two H2-CAL-V02-751651244-757699245.cache \
>   --calibration-offset 0 --hann-duration 1 --cc-spectra
\end{verbatim}

\item[Author] 
Adam Mercer, Tania Regimbau
\end{entry}
