%
% API Documentation
% Module stochastic
%
% Generated by epydoc 2.1
% [Tue Mar  1 11:29:31 2005]
%

\newenvironment{Ventry}[1]%
  {\begin{list}{}{%
    \renewcommand{\makelabel}[1]{\texttt{##1:}\hfil}%
    \settowidth{\labelwidth}{\texttt{#1:}}%
    \setlength{\leftmargin}{\labelsep}%
    \addtolength{\leftmargin}{\labelwidth}}}%
  {\end{list}}

%%%%%%%%%%%%%%%%%%%%%%%%%%%%%%%%%%%%%%%%%%%%%%%%%%%%%%%%%%%%%%%%%%%%%%%%%%%
%%                          Module Description                           %%
%%%%%%%%%%%%%%%%%%%%%%%%%%%%%%%%%%%%%%%%%%%%%%%%%%%%%%%%%%%%%%%%%%%%%%%%%%%

    \index{stochastic \textit{(module)}|(}
\section{Python Module \texttt{stochastic}}

    \label{stochastic}
Classes needed for the stochastic analysis pipeline. This script produces 
the necessary condor submit and dag files to run the standalone 
stochastic code on LIGO/GEO data.


%%%%%%%%%%%%%%%%%%%%%%%%%%%%%%%%%%%%%%%%%%%%%%%%%%%%%%%%%%%%%%%%%%%%%%%%%%%
%%                               Functions                               %%
%%%%%%%%%%%%%%%%%%%%%%%%%%%%%%%%%%%%%%%%%%%%%%%%%%%%%%%%%%%%%%%%%%%%%%%%%%%

  \subsection{Functions}

    \label{stochastic:version}
    \index{stochastic \textit{(module)}!version \textit{(function)}}
    \vspace{0.5ex}

    \begin{boxedminipage}{\textwidth}

    \raggedright \textbf{version}()

    \vspace{-1.5ex}

    \rule{\textwidth}{0.5\fboxrule}
    Return version

    \vspace{1ex}

    \end{boxedminipage}


%%%%%%%%%%%%%%%%%%%%%%%%%%%%%%%%%%%%%%%%%%%%%%%%%%%%%%%%%%%%%%%%%%%%%%%%%%%
%%                               Variables                               %%
%%%%%%%%%%%%%%%%%%%%%%%%%%%%%%%%%%%%%%%%%%%%%%%%%%%%%%%%%%%%%%%%%%%%%%%%%%%

  \subsection{Variables}

\begin{longtable}{|p{.30\textwidth}|p{.62\textwidth}|l}
\cline{1-2}
\cline{1-2} \centering \textbf{Name} & \centering \textbf{Description}& \\
\cline{1-2}
\endhead\cline{1-2}\multicolumn{3}{r}{\small\textit{continued on next page}}\\\endfoot\cline{1-2}
\endlastfoot\raggedright \_\-\_\-a\-u\-t\-h\-o\-r\-\_\-\_\- & \raggedright \textbf{Value:} 
{\tt '\-A\-d\-a\-m\-~\-M\-e\-r\-c\-e\-r\-~\-{\textless}\-r\-a\-m\-@\-s\-t\-a\-r\-.\-s\-r\-.\-b\-h\-a\-m\-.\-a\-c\-.\-u\-k\-{\textgreater}\-'\-}            \textit{(type=\texttt{str})}&\\
\cline{1-2}
\raggedright \_\-\_\-d\-a\-t\-e\-\_\-\_\- & \raggedright \textbf{Value:} 
{\tt '\-\$\-D\-a\-t\-e\-:\-~\-2\-0\-0\-5\-/\-0\-3\-/\-0\-1\-~\-1\-1\-:\-2\-7\-:\-3\-4\-~\-\$\-'\-}            \textit{(type=\texttt{str})}&\\
\cline{1-2}
\raggedright \_\-\_\-v\-e\-r\-s\-i\-o\-n\-\_\-\_\- & \raggedright \textbf{Value:} 
{\tt '\-1\-.\-3\-4\-'\-}            \textit{(type=\texttt{str})}&\\
\cline{1-2}
\end{longtable}

    \index{stochastic \textit{(module)}!LSCDataFindJob \textit{(class)}|(}

%%%%%%%%%%%%%%%%%%%%%%%%%%%%%%%%%%%%%%%%%%%%%%%%%%%%%%%%%%%%%%%%%%%%%%%%%%%
%%                           Class Description                           %%
%%%%%%%%%%%%%%%%%%%%%%%%%%%%%%%%%%%%%%%%%%%%%%%%%%%%%%%%%%%%%%%%%%%%%%%%%%%

\subsection{Class LSCDataFindJob}

    \label{stochastic:LSCDataFindJob}
\begin{tabular}{cccccccc}
% Line for glue.pipeline.AnalysisJob, linespec=[False]
\multicolumn{4}{r}{\settowidth{\BCL}{glue.pipeline.AnalysisJob}\multirow{2}{\BCL}{glue.pipeline.AnalysisJob}}
&&
  \\\cline{5-5}
  &&&&\multicolumn{1}{c|}{}
&&
  \\
% Line for glue.pipeline.CondorJob, linespec=[False, True]
\multicolumn{2}{r}{\settowidth{\BCL}{glue.pipeline.CondorJob}\multirow{2}{\BCL}{glue.pipeline.CondorJob}}
&&
&&\multicolumn{1}{|c}{}
  \\\cline{3-3}
  &&\multicolumn{1}{c|}{}
&&
&\multicolumn{1}{|c}{}&
  \\
% Line for glue.pipeline.CondorDAGJob, linespec=[True]
\multicolumn{4}{r}{\settowidth{\BCL}{glue.pipeline.CondorDAGJob}\multirow{2}{\BCL}{glue.pipeline.CondorDAGJob}}
&&\multicolumn{1}{|c}{}
  \\\cline{5-5}
  &&&&\multicolumn{1}{c|}{}
&\multicolumn{1}{|c}{}&
  \\
&&&&\multicolumn{2}{l}{\textbf{LSCDataFindJob}}
\end{tabular}

An LSCdataFind job used to locate data. The static options are read from 
the section [datafind] in the ini file. The stdout from LSCdataFind 
contains the paths to the frame files and is directed to a file in the 
cache directory named by site and GPS start and end times. The stderr is 
directed to the logs directory. The job always runs in the scheduler 
universe. The path to the executable is determined from the ini file.


%%%%%%%%%%%%%%%%%%%%%%%%%%%%%%%%%%%%%%%%%%%%%%%%%%%%%%%%%%%%%%%%%%%%%%%%%%%
%%                                Methods                                %%
%%%%%%%%%%%%%%%%%%%%%%%%%%%%%%%%%%%%%%%%%%%%%%%%%%%%%%%%%%%%%%%%%%%%%%%%%%%

  \subsubsection{Methods}

    \label{stochastic:LSCDataFindJob:__init__}
    \index{stochastic \textit{(module)}!LSCDataFindJob \textit{(class)}!\_\_init\_\_ \textit{(method)}}
    \vspace{0.5ex}

    \begin{boxedminipage}{\textwidth}

    \raggedright \textbf{\_\_init\_\_}(\textit{self}, \textit{cache\_dir}, \textit{log\_dir}, \textit{config\_file})

    \vspace{-1.5ex}

    \rule{\textwidth}{0.5\fboxrule}
    \vspace{1ex}

      \textbf{Parameters}
      \begin{quote}
        \begin{Ventry}{xxxxxxxxxxx}

          \item[cache\_dir]

          the directory to write the output lal cache files to.

          \item[log\_dir]

          the directory to write the stderr file to.

          \item[config\_file]

          ConfigParser object containing the path to the LSCdataFind 
          executable in the [condor] section and a [datafind] section 
          from which the LSCdataFind options are read.

        \end{Ventry}

      \end{quote}

    \vspace{1ex}

      Overrides: glue.pipeline.CondorDAGJob.\_\_init\_\_

    \end{boxedminipage}

    \label{stochastic:LSCDataFindJob:get_cache_dir}
    \index{stochastic \textit{(module)}!LSCDataFindJob \textit{(class)}!get\_cache\_dir \textit{(method)}}
    \vspace{0.5ex}

    \begin{boxedminipage}{\textwidth}

    \raggedright \textbf{get\_cache\_dir}(\textit{self})

    \vspace{-1.5ex}

    \rule{\textwidth}{0.5\fboxrule}
    returns the directroy that the cache files are written to.

    \vspace{1ex}

    \end{boxedminipage}

  \textbf{Inherited from AnalysisJob:}
    channel,
    get\_config
    \\
  \textbf{Inherited from CondorDAGJob:}
    add\_var\_arg,
    add\_var\_opt
    \\
  \textbf{Inherited from CondorJob:}
    add\_arg,
    add\_condor\_cmd,
    add\_ini\_opts,
    add\_opt,
    add\_short\_opt,
    get\_args,
    get\_opts,
    get\_short\_opts,
    get\_stderr\_file,
    get\_stdout\_file,
    get\_sub\_file,
    set\_log\_file,
    set\_notification,
    set\_stderr\_file,
    set\_stdout\_file,
    set\_sub\_file,
    write\_sub\_file
    \index{stochastic \textit{(module)}!LSCDataFindJob \textit{(class)}|)}
    \index{stochastic \textit{(module)}!LSCDataFindNode \textit{(class)}|(}

%%%%%%%%%%%%%%%%%%%%%%%%%%%%%%%%%%%%%%%%%%%%%%%%%%%%%%%%%%%%%%%%%%%%%%%%%%%
%%                           Class Description                           %%
%%%%%%%%%%%%%%%%%%%%%%%%%%%%%%%%%%%%%%%%%%%%%%%%%%%%%%%%%%%%%%%%%%%%%%%%%%%

\subsection{Class LSCDataFindNode}

    \label{stochastic:LSCDataFindNode}
\begin{tabular}{cccccccc}
% Line for glue.pipeline.CondorDAGNode, linespec=[False, False]
\multicolumn{2}{r}{\settowidth{\BCL}{glue.pipeline.CondorDAGNode}\multirow{2}{\BCL}{glue.pipeline.CondorDAGNode}}
&&
&&
  \\\cline{3-3}
  &&\multicolumn{1}{c|}{}
&&
&&
  \\
% Line for glue.pipeline.AnalysisNode, linespec=[False]
\multicolumn{4}{r}{\settowidth{\BCL}{glue.pipeline.AnalysisNode}\multirow{2}{\BCL}{glue.pipeline.AnalysisNode}}
&&
  \\\cline{5-5}
  &&&&\multicolumn{1}{c|}{}
&&
  \\
% Line for glue.pipeline.CondorDAGNode, linespec=[True]
\multicolumn{4}{r}{\settowidth{\BCL}{glue.pipeline.CondorDAGNode}\multirow{2}{\BCL}{glue.pipeline.CondorDAGNode}}
&&\multicolumn{1}{|c}{}
  \\\cline{5-5}
  &&&&\multicolumn{1}{c|}{}
&\multicolumn{1}{|c}{}&
  \\
&&&&\multicolumn{2}{l}{\textbf{LSCDataFindNode}}
\end{tabular}

A DataFindNode runs an instance of LSCdataFind in a Condor DAG.


%%%%%%%%%%%%%%%%%%%%%%%%%%%%%%%%%%%%%%%%%%%%%%%%%%%%%%%%%%%%%%%%%%%%%%%%%%%
%%                                Methods                                %%
%%%%%%%%%%%%%%%%%%%%%%%%%%%%%%%%%%%%%%%%%%%%%%%%%%%%%%%%%%%%%%%%%%%%%%%%%%%

  \subsubsection{Methods}

    \label{stochastic:LSCDataFindNode:__init__}
    \index{stochastic \textit{(module)}!LSCDataFindNode \textit{(class)}!\_\_init\_\_ \textit{(method)}}
    \vspace{0.5ex}

    \begin{boxedminipage}{\textwidth}

    \raggedright \textbf{\_\_init\_\_}(\textit{self}, \textit{job})

    \vspace{-1.5ex}

    \rule{\textwidth}{0.5\fboxrule}
    \vspace{1ex}

      \textbf{Parameters}
      \begin{quote}
        \begin{Ventry}{xxx}

          \item[job]

          A CondorDAGJob that can run an instance of LALdataFind.

        \end{Ventry}

      \end{quote}

    \vspace{1ex}

      Overrides: glue.pipeline.AnalysisNode.\_\_init\_\_

    \end{boxedminipage}

    \label{stochastic:LSCDataFindNode:get_output}
    \index{stochastic \textit{(module)}!LSCDataFindNode \textit{(class)}!get\_output \textit{(method)}}
    \vspace{0.5ex}

    \begin{boxedminipage}{\textwidth}

    \raggedright \textbf{get\_output}(\textit{self})

    \vspace{-1.5ex}

    \rule{\textwidth}{0.5\fboxrule}
    Return the output file, i.e. the file containing the frame cache 
    data.

    \vspace{1ex}

      Overrides: glue.pipeline.AnalysisNode.get\_output

    \end{boxedminipage}

    \label{stochastic:LSCDataFindNode:set_end}
    \index{stochastic \textit{(module)}!LSCDataFindNode \textit{(class)}!set\_end \textit{(method)}}
    \vspace{0.5ex}

    \begin{boxedminipage}{\textwidth}

    \raggedright \textbf{set\_end}(\textit{self}, \textit{time})

    \vspace{-1.5ex}

    \rule{\textwidth}{0.5\fboxrule}
    Set the end time of the datafind query.

    \vspace{1ex}

      \textbf{Parameters}
      \begin{quote}
        \begin{Ventry}{xxxx}

          \item[time]

          GPS end time of query.

        \end{Ventry}

      \end{quote}

    \vspace{1ex}

      Overrides: glue.pipeline.AnalysisNode.set\_end

    \end{boxedminipage}

    \label{stochastic:LSCDataFindNode:set_observatory}
    \index{stochastic \textit{(module)}!LSCDataFindNode \textit{(class)}!set\_observatory \textit{(method)}}
    \vspace{0.5ex}

    \begin{boxedminipage}{\textwidth}

    \raggedright \textbf{set\_observatory}(\textit{self}, \textit{obs})

    \vspace{-1.5ex}

    \rule{\textwidth}{0.5\fboxrule}
    Set the IFO to retrieve data for. Since the data from both Hanford 
    interferometers is stored in the same frame file, this takes the 
    first letter of the IFO (e.g. L or H) and passes it to the 
    --observatory option of LSCdataFind.

    \vspace{1ex}

      \textbf{Parameters}
      \begin{quote}
        \begin{Ventry}{xxx}

          \item[obs]

          IFO to obtain data for.

        \end{Ventry}

      \end{quote}

    \vspace{1ex}

    \end{boxedminipage}

    \label{stochastic:LSCDataFindNode:set_start}
    \index{stochastic \textit{(module)}!LSCDataFindNode \textit{(class)}!set\_start \textit{(method)}}
    \vspace{0.5ex}

    \begin{boxedminipage}{\textwidth}

    \raggedright \textbf{set\_start}(\textit{self}, \textit{time})

    \vspace{-1.5ex}

    \rule{\textwidth}{0.5\fboxrule}
    Set the start time of the datafind query.

    \vspace{1ex}

      \textbf{Parameters}
      \begin{quote}
        \begin{Ventry}{xxxx}

          \item[time]

          GPS start time of query.

        \end{Ventry}

      \end{quote}

    \vspace{1ex}

      Overrides: glue.pipeline.AnalysisNode.set\_start

    \end{boxedminipage}

    \label{stochastic:LSCDataFindNode:set_type}
    \index{stochastic \textit{(module)}!LSCDataFindNode \textit{(class)}!set\_type \textit{(method)}}
    \vspace{0.5ex}

    \begin{boxedminipage}{\textwidth}

    \raggedright \textbf{set\_type}(\textit{self}, \textit{type})

    \vspace{-1.5ex}

    \rule{\textwidth}{0.5\fboxrule}
    Set the frame type to retrieve data for

    \vspace{1ex}

      \textbf{Parameters}
      \begin{quote}
        \begin{Ventry}{xxxx}

          \item[type]

          Frame type to obtain data for.

        \end{Ventry}

      \end{quote}

    \vspace{1ex}

    \end{boxedminipage}

  \textbf{Inherited from AnalysisNode:}
    calibration,
    calibration\_cache\_path,
    get\_calibration,
    get\_end,
    get\_ifo,
    get\_ifo\_tag,
    get\_input,
    get\_start,
    set\_cache,
    set\_ifo,
    set\_ifo\_tag,
    set\_input,
    set\_output
    \\
  \textbf{Inherited from CondorDAGNode:}
    \_\_repr\_\_,
    add\_input\_file,
    add\_macro,
    add\_output\_file,
    add\_parent,
    add\_post\_script\_arg,
    add\_pre\_script\_arg,
    add\_var\_arg,
    add\_var\_opt,
    get\_args,
    get\_cmd\_line,
    get\_input\_files,
    get\_opts,
    get\_output\_files,
    get\_vds\_group,
    job,
    set\_log\_file,
    set\_name,
    set\_post\_script,
    set\_pre\_script,
    set\_retry,
    set\_vds\_group,
    write\_input\_files,
    write\_job,
    write\_output\_files,
    write\_parents,
    write\_post\_script,
    write\_pre\_script,
    write\_vars
    \index{stochastic \textit{(module)}!LSCDataFindNode \textit{(class)}|)}
    \index{stochastic \textit{(module)}!StochasticError \textit{(class)}|(}

%%%%%%%%%%%%%%%%%%%%%%%%%%%%%%%%%%%%%%%%%%%%%%%%%%%%%%%%%%%%%%%%%%%%%%%%%%%
%%                           Class Description                           %%
%%%%%%%%%%%%%%%%%%%%%%%%%%%%%%%%%%%%%%%%%%%%%%%%%%%%%%%%%%%%%%%%%%%%%%%%%%%

\subsection{Class StochasticError}

    \label{stochastic:StochasticError}
\begin{tabular}{cccccc}
% Line for exceptions.Exception, linespec=[False]
\multicolumn{2}{r}{\settowidth{\BCL}{exceptions.Exception}\multirow{2}{\BCL}{exceptions.Exception}}
&&
  \\\cline{3-3}
  &&\multicolumn{1}{c|}{}
&&
  \\
&&\multicolumn{2}{l}{\textbf{StochasticError}}
\end{tabular}


%%%%%%%%%%%%%%%%%%%%%%%%%%%%%%%%%%%%%%%%%%%%%%%%%%%%%%%%%%%%%%%%%%%%%%%%%%%
%%                                Methods                                %%
%%%%%%%%%%%%%%%%%%%%%%%%%%%%%%%%%%%%%%%%%%%%%%%%%%%%%%%%%%%%%%%%%%%%%%%%%%%

  \subsubsection{Methods}

    \label{stochastic:StochasticError:__init__}
    \index{stochastic \textit{(module)}!StochasticError \textit{(class)}!\_\_init\_\_ \textit{(method)}}
    \vspace{0.5ex}

    \begin{boxedminipage}{\textwidth}

    \raggedright \textbf{\_\_init\_\_}(\textit{self}, \textit{args}=\texttt{N\-o\-n\-e\-})

      Overrides: exceptions.Exception.\_\_init\_\_

    \end{boxedminipage}

  \textbf{Inherited from Exception:}
    \_\_getitem\_\_,
    \_\_str\_\_
    \index{stochastic \textit{(module)}!StochasticError \textit{(class)}|)}
    \index{stochastic \textit{(module)}!StochasticJob \textit{(class)}|(}

%%%%%%%%%%%%%%%%%%%%%%%%%%%%%%%%%%%%%%%%%%%%%%%%%%%%%%%%%%%%%%%%%%%%%%%%%%%
%%                           Class Description                           %%
%%%%%%%%%%%%%%%%%%%%%%%%%%%%%%%%%%%%%%%%%%%%%%%%%%%%%%%%%%%%%%%%%%%%%%%%%%%

\subsection{Class StochasticJob}

    \label{stochastic:StochasticJob}
\begin{tabular}{cccccccc}
% Line for glue.pipeline.AnalysisJob, linespec=[False]
\multicolumn{4}{r}{\settowidth{\BCL}{glue.pipeline.AnalysisJob}\multirow{2}{\BCL}{glue.pipeline.AnalysisJob}}
&&
  \\\cline{5-5}
  &&&&\multicolumn{1}{c|}{}
&&
  \\
% Line for glue.pipeline.CondorJob, linespec=[False, True]
\multicolumn{2}{r}{\settowidth{\BCL}{glue.pipeline.CondorJob}\multirow{2}{\BCL}{glue.pipeline.CondorJob}}
&&
&&\multicolumn{1}{|c}{}
  \\\cline{3-3}
  &&\multicolumn{1}{c|}{}
&&
&\multicolumn{1}{|c}{}&
  \\
% Line for glue.pipeline.CondorDAGJob, linespec=[True]
\multicolumn{4}{r}{\settowidth{\BCL}{glue.pipeline.CondorDAGJob}\multirow{2}{\BCL}{glue.pipeline.CondorDAGJob}}
&&\multicolumn{1}{|c}{}
  \\\cline{5-5}
  &&&&\multicolumn{1}{c|}{}
&\multicolumn{1}{|c}{}&
  \\
&&&&\multicolumn{2}{l}{\textbf{StochasticJob}}
\end{tabular}

A lalapps\_stochastic job used by the stochastic pipeline. The static 
options are read from the section [stochastic] in the ini file. The 
stdout and stderr from the job are directed to the logs directory. The 
path to the executable and the universe is determined from the ini file.


%%%%%%%%%%%%%%%%%%%%%%%%%%%%%%%%%%%%%%%%%%%%%%%%%%%%%%%%%%%%%%%%%%%%%%%%%%%
%%                                Methods                                %%
%%%%%%%%%%%%%%%%%%%%%%%%%%%%%%%%%%%%%%%%%%%%%%%%%%%%%%%%%%%%%%%%%%%%%%%%%%%

  \subsubsection{Methods}

    \label{stochastic:StochasticJob:__init__}
    \index{stochastic \textit{(module)}!StochasticJob \textit{(class)}!\_\_init\_\_ \textit{(method)}}
    \vspace{0.5ex}

    \begin{boxedminipage}{\textwidth}

    \raggedright \textbf{\_\_init\_\_}(\textit{self}, \textit{cp})

    \vspace{-1.5ex}

    \rule{\textwidth}{0.5\fboxrule}
    cp = ConfigParser object from which options are read.

    \vspace{1ex}

      Overrides: glue.pipeline.CondorDAGJob.\_\_init\_\_

    \end{boxedminipage}

  \textbf{Inherited from AnalysisJob:}
    channel,
    get\_config
    \\
  \textbf{Inherited from CondorDAGJob:}
    add\_var\_arg,
    add\_var\_opt
    \\
  \textbf{Inherited from CondorJob:}
    add\_arg,
    add\_condor\_cmd,
    add\_ini\_opts,
    add\_opt,
    add\_short\_opt,
    get\_args,
    get\_opts,
    get\_short\_opts,
    get\_stderr\_file,
    get\_stdout\_file,
    get\_sub\_file,
    set\_log\_file,
    set\_notification,
    set\_stderr\_file,
    set\_stdout\_file,
    set\_sub\_file,
    write\_sub\_file
    \index{stochastic \textit{(module)}!StochasticJob \textit{(class)}|)}
    \index{stochastic \textit{(module)}!StochasticNode \textit{(class)}|(}

%%%%%%%%%%%%%%%%%%%%%%%%%%%%%%%%%%%%%%%%%%%%%%%%%%%%%%%%%%%%%%%%%%%%%%%%%%%
%%                           Class Description                           %%
%%%%%%%%%%%%%%%%%%%%%%%%%%%%%%%%%%%%%%%%%%%%%%%%%%%%%%%%%%%%%%%%%%%%%%%%%%%

\subsection{Class StochasticNode}

    \label{stochastic:StochasticNode}
\begin{tabular}{cccccccc}
% Line for glue.pipeline.CondorDAGNode, linespec=[False, False]
\multicolumn{2}{r}{\settowidth{\BCL}{glue.pipeline.CondorDAGNode}\multirow{2}{\BCL}{glue.pipeline.CondorDAGNode}}
&&
&&
  \\\cline{3-3}
  &&\multicolumn{1}{c|}{}
&&
&&
  \\
% Line for glue.pipeline.AnalysisNode, linespec=[False]
\multicolumn{4}{r}{\settowidth{\BCL}{glue.pipeline.AnalysisNode}\multirow{2}{\BCL}{glue.pipeline.AnalysisNode}}
&&
  \\\cline{5-5}
  &&&&\multicolumn{1}{c|}{}
&&
  \\
% Line for glue.pipeline.CondorDAGNode, linespec=[True]
\multicolumn{4}{r}{\settowidth{\BCL}{glue.pipeline.CondorDAGNode}\multirow{2}{\BCL}{glue.pipeline.CondorDAGNode}}
&&\multicolumn{1}{|c}{}
  \\\cline{5-5}
  &&&&\multicolumn{1}{c|}{}
&\multicolumn{1}{|c}{}&
  \\
&&&&\multicolumn{2}{l}{\textbf{StochasticNode}}
\end{tabular}

An StochaticNode runs an instance of the stochastic code in a Condor DAG.


%%%%%%%%%%%%%%%%%%%%%%%%%%%%%%%%%%%%%%%%%%%%%%%%%%%%%%%%%%%%%%%%%%%%%%%%%%%
%%                                Methods                                %%
%%%%%%%%%%%%%%%%%%%%%%%%%%%%%%%%%%%%%%%%%%%%%%%%%%%%%%%%%%%%%%%%%%%%%%%%%%%

  \subsubsection{Methods}

    \label{stochastic:StochasticNode:__init__}
    \index{stochastic \textit{(module)}!StochasticNode \textit{(class)}!\_\_init\_\_ \textit{(method)}}
    \vspace{0.5ex}

    \begin{boxedminipage}{\textwidth}

    \raggedright \textbf{\_\_init\_\_}(\textit{self}, \textit{job})

    \vspace{-1.5ex}

    \rule{\textwidth}{0.5\fboxrule}
    job = A CondorDAGJob that can run an instance of lalapps\_stochastic.

    \vspace{1ex}

      Overrides: glue.pipeline.AnalysisNode.\_\_init\_\_

    \end{boxedminipage}

    \label{stochastic:StochasticNode:get_f_max}
    \index{stochastic \textit{(module)}!StochasticNode \textit{(class)}!get\_f\_max \textit{(method)}}
    \vspace{0.5ex}

    \begin{boxedminipage}{\textwidth}

    \raggedright \textbf{get\_f\_max}(\textit{self})

    \vspace{-1.5ex}

    \rule{\textwidth}{0.5\fboxrule}
    Return the maximum frequency

    \vspace{1ex}

    \end{boxedminipage}

    \label{stochastic:StochasticNode:get_f_min}
    \index{stochastic \textit{(module)}!StochasticNode \textit{(class)}!get\_f\_min \textit{(method)}}
    \vspace{0.5ex}

    \begin{boxedminipage}{\textwidth}

    \raggedright \textbf{get\_f\_min}(\textit{self})

    \vspace{-1.5ex}

    \rule{\textwidth}{0.5\fboxrule}
    Return the minimum frequency

    \vspace{1ex}

    \end{boxedminipage}

    \label{stochastic:StochasticNode:get_ifo_one}
    \index{stochastic \textit{(module)}!StochasticNode \textit{(class)}!get\_ifo\_one \textit{(method)}}
    \vspace{0.5ex}

    \begin{boxedminipage}{\textwidth}

    \raggedright \textbf{get\_ifo\_one}(\textit{self})

    \vspace{-1.5ex}

    \rule{\textwidth}{0.5\fboxrule}
    Returns the IFO code of the first interferometer.

    \vspace{1ex}

    \end{boxedminipage}

    \label{stochastic:StochasticNode:get_ifo_two}
    \index{stochastic \textit{(module)}!StochasticNode \textit{(class)}!get\_ifo\_two \textit{(method)}}
    \vspace{0.5ex}

    \begin{boxedminipage}{\textwidth}

    \raggedright \textbf{get\_ifo\_two}(\textit{self})

    \vspace{-1.5ex}

    \rule{\textwidth}{0.5\fboxrule}
    Returns the IFO code of the second interferometer.

    \vspace{1ex}

    \end{boxedminipage}

    \label{stochastic:StochasticNode:get_output}
    \index{stochastic \textit{(module)}!StochasticNode \textit{(class)}!get\_output \textit{(method)}}
    \vspace{0.5ex}

    \begin{boxedminipage}{\textwidth}

    \raggedright \textbf{get\_output}(\textit{self})

    \vspace{-1.5ex}

    \rule{\textwidth}{0.5\fboxrule}
    Returns the file name of output from the stochastic code. This must 
    be kept synchronized with the name of the output file in 
    stochastic.c.

    \vspace{1ex}

      Overrides: glue.pipeline.AnalysisNode.get\_output

    \end{boxedminipage}

    \label{stochastic:StochasticNode:set_cache_one}
    \index{stochastic \textit{(module)}!StochasticNode \textit{(class)}!set\_cache\_one \textit{(method)}}
    \vspace{0.5ex}

    \begin{boxedminipage}{\textwidth}

    \raggedright \textbf{set\_cache\_one}(\textit{self}, \textit{file})

    \vspace{-1.5ex}

    \rule{\textwidth}{0.5\fboxrule}
    Set the LAL frame cache to to use. The frame cache is passed to the 
    job with the --frame-cache-one argument. file = calibration file to 
    use.

    \vspace{1ex}

    \end{boxedminipage}

    \label{stochastic:StochasticNode:set_cache_two}
    \index{stochastic \textit{(module)}!StochasticNode \textit{(class)}!set\_cache\_two \textit{(method)}}
    \vspace{0.5ex}

    \begin{boxedminipage}{\textwidth}

    \raggedright \textbf{set\_cache\_two}(\textit{self}, \textit{file})

    \vspace{-1.5ex}

    \rule{\textwidth}{0.5\fboxrule}
    Set the LAL frame cache to to use. The frame cache is passed to the 
    job with the --frame-cache-two argument. file = calibration file to 
    use.

    \vspace{1ex}

    \end{boxedminipage}

    \label{stochastic:StochasticNode:set_calibration_one}
    \index{stochastic \textit{(module)}!StochasticNode \textit{(class)}!set\_calibration\_one \textit{(method)}}
    \vspace{0.5ex}

    \begin{boxedminipage}{\textwidth}

    \raggedright \textbf{set\_calibration\_one}(\textit{self}, \textit{ifo}, \textit{start})

    \vspace{-1.5ex}

    \rule{\textwidth}{0.5\fboxrule}
    Set the path to the calibration cache file for the given IFO. During 
    S2, the Hanford 2km IFO had two calibration epochs, so if the start 
    time is during S2, we use the correct cache file.

    \vspace{1ex}

    \end{boxedminipage}

    \label{stochastic:StochasticNode:set_calibration_two}
    \index{stochastic \textit{(module)}!StochasticNode \textit{(class)}!set\_calibration\_two \textit{(method)}}
    \vspace{0.5ex}

    \begin{boxedminipage}{\textwidth}

    \raggedright \textbf{set\_calibration\_two}(\textit{self}, \textit{ifo}, \textit{start})

    \vspace{-1.5ex}

    \rule{\textwidth}{0.5\fboxrule}
    Set the path to the calibration cache file for the given IFO. During 
    S2, the Hanford 2km IFO had two calibration epochs, so if the start 
    time is during S2, we use the correct cache file.

    \vspace{1ex}

    \end{boxedminipage}

    \label{stochastic:StochasticNode:set_f_max}
    \index{stochastic \textit{(module)}!StochasticNode \textit{(class)}!set\_f\_max \textit{(method)}}
    \vspace{0.5ex}

    \begin{boxedminipage}{\textwidth}

    \raggedright \textbf{set\_f\_max}(\textit{self}, \textit{f\_max})

    \vspace{-1.5ex}

    \rule{\textwidth}{0.5\fboxrule}
    Set the maximum frequency

    \vspace{1ex}

    \end{boxedminipage}

    \label{stochastic:StochasticNode:set_f_min}
    \index{stochastic \textit{(module)}!StochasticNode \textit{(class)}!set\_f\_min \textit{(method)}}
    \vspace{0.5ex}

    \begin{boxedminipage}{\textwidth}

    \raggedright \textbf{set\_f\_min}(\textit{self}, \textit{f\_min})

    \vspace{-1.5ex}

    \rule{\textwidth}{0.5\fboxrule}
    Set the minimum frequency

    \vspace{1ex}

    \end{boxedminipage}

    \label{stochastic:StochasticNode:set_f_ref}
    \index{stochastic \textit{(module)}!StochasticNode \textit{(class)}!set\_f\_ref \textit{(method)}}
    \vspace{0.5ex}

    \begin{boxedminipage}{\textwidth}

    \raggedright \textbf{set\_f\_ref}(\textit{self}, \textit{f\_ref})

    \vspace{-1.5ex}

    \rule{\textwidth}{0.5\fboxrule}
    Set the reference frequency

    \vspace{1ex}

    \end{boxedminipage}

    \label{stochastic:StochasticNode:set_ifo_one}
    \index{stochastic \textit{(module)}!StochasticNode \textit{(class)}!set\_ifo\_one \textit{(method)}}
    \vspace{0.5ex}

    \begin{boxedminipage}{\textwidth}

    \raggedright \textbf{set\_ifo\_one}(\textit{self}, \textit{ifo})

    \vspace{-1.5ex}

    \rule{\textwidth}{0.5\fboxrule}
    Set the interferometer code to use as IFO One. ifo = IFO code (e.g. 
    L1, H1 or H2).

    \vspace{1ex}

    \end{boxedminipage}

    \label{stochastic:StochasticNode:set_ifo_two}
    \index{stochastic \textit{(module)}!StochasticNode \textit{(class)}!set\_ifo\_two \textit{(method)}}
    \vspace{0.5ex}

    \begin{boxedminipage}{\textwidth}

    \raggedright \textbf{set\_ifo\_two}(\textit{self}, \textit{ifo})

    \vspace{-1.5ex}

    \rule{\textwidth}{0.5\fboxrule}
    Set the interferometer code to use as IFO Two. ifo = IFO code (e.g. 
    L1, H1 or H2).

    \vspace{1ex}

    \end{boxedminipage}

    \label{stochastic:StochasticNode:set_output_dir}
    \index{stochastic \textit{(module)}!StochasticNode \textit{(class)}!set\_output\_dir \textit{(method)}}
    \vspace{0.5ex}

    \begin{boxedminipage}{\textwidth}

    \raggedright \textbf{set\_output\_dir}(\textit{self}, \textit{dir})

    \vspace{-1.5ex}

    \rule{\textwidth}{0.5\fboxrule}
    Set the output directory

    \vspace{1ex}

    \end{boxedminipage}

    \label{stochastic:StochasticNode:set_user_tag}
    \index{stochastic \textit{(module)}!StochasticNode \textit{(class)}!set\_user\_tag \textit{(method)}}
    \vspace{0.5ex}

    \begin{boxedminipage}{\textwidth}

    \raggedright \textbf{set\_user\_tag}(\textit{self}, \textit{usertag})

    \vspace{-1.5ex}

    \rule{\textwidth}{0.5\fboxrule}
    Set the user tag

    \vspace{1ex}

    \end{boxedminipage}

  \textbf{Inherited from AnalysisNode:}
    calibration,
    calibration\_cache\_path,
    get\_calibration,
    get\_end,
    get\_ifo,
    get\_ifo\_tag,
    get\_input,
    get\_start,
    set\_cache,
    set\_end,
    set\_ifo,
    set\_ifo\_tag,
    set\_input,
    set\_output,
    set\_start
    \\
  \textbf{Inherited from CondorDAGNode:}
    \_\_repr\_\_,
    add\_input\_file,
    add\_macro,
    add\_output\_file,
    add\_parent,
    add\_post\_script\_arg,
    add\_pre\_script\_arg,
    add\_var\_arg,
    add\_var\_opt,
    get\_args,
    get\_cmd\_line,
    get\_input\_files,
    get\_opts,
    get\_output\_files,
    get\_vds\_group,
    job,
    set\_log\_file,
    set\_name,
    set\_post\_script,
    set\_pre\_script,
    set\_retry,
    set\_vds\_group,
    write\_input\_files,
    write\_job,
    write\_output\_files,
    write\_parents,
    write\_post\_script,
    write\_pre\_script,
    write\_vars
    \index{stochastic \textit{(module)}!StochasticNode \textit{(class)}|)}
    \index{stochastic \textit{(module)}!StoppJob \textit{(class)}|(}

%%%%%%%%%%%%%%%%%%%%%%%%%%%%%%%%%%%%%%%%%%%%%%%%%%%%%%%%%%%%%%%%%%%%%%%%%%%
%%                           Class Description                           %%
%%%%%%%%%%%%%%%%%%%%%%%%%%%%%%%%%%%%%%%%%%%%%%%%%%%%%%%%%%%%%%%%%%%%%%%%%%%

\subsection{Class StoppJob}

    \label{stochastic:StoppJob}
\begin{tabular}{cccccccc}
% Line for glue.pipeline.AnalysisJob, linespec=[False]
\multicolumn{4}{r}{\settowidth{\BCL}{glue.pipeline.AnalysisJob}\multirow{2}{\BCL}{glue.pipeline.AnalysisJob}}
&&
  \\\cline{5-5}
  &&&&\multicolumn{1}{c|}{}
&&
  \\
% Line for glue.pipeline.CondorJob, linespec=[False, True]
\multicolumn{2}{r}{\settowidth{\BCL}{glue.pipeline.CondorJob}\multirow{2}{\BCL}{glue.pipeline.CondorJob}}
&&
&&\multicolumn{1}{|c}{}
  \\\cline{3-3}
  &&\multicolumn{1}{c|}{}
&&
&\multicolumn{1}{|c}{}&
  \\
% Line for glue.pipeline.CondorDAGJob, linespec=[True]
\multicolumn{4}{r}{\settowidth{\BCL}{glue.pipeline.CondorDAGJob}\multirow{2}{\BCL}{glue.pipeline.CondorDAGJob}}
&&\multicolumn{1}{|c}{}
  \\\cline{5-5}
  &&&&\multicolumn{1}{c|}{}
&\multicolumn{1}{|c}{}&
  \\
&&&&\multicolumn{2}{l}{\textbf{StoppJob}}
\end{tabular}

A lalapps\_stopp job used by the stochastic pipeline. The static options 
are read from the section [stopp] in the ini file. The stdout and stderr 
from the job are directed to the logs directory. The path to the 
executable and the universe is determined from the ini file.


%%%%%%%%%%%%%%%%%%%%%%%%%%%%%%%%%%%%%%%%%%%%%%%%%%%%%%%%%%%%%%%%%%%%%%%%%%%
%%                                Methods                                %%
%%%%%%%%%%%%%%%%%%%%%%%%%%%%%%%%%%%%%%%%%%%%%%%%%%%%%%%%%%%%%%%%%%%%%%%%%%%

  \subsubsection{Methods}

    \label{stochastic:StoppJob:__init__}
    \index{stochastic \textit{(module)}!StoppJob \textit{(class)}!\_\_init\_\_ \textit{(method)}}
    \vspace{0.5ex}

    \begin{boxedminipage}{\textwidth}

    \raggedright \textbf{\_\_init\_\_}(\textit{self}, \textit{cp})

    \vspace{-1.5ex}

    \rule{\textwidth}{0.5\fboxrule}
    cp = ConfigParser object from which options are read.

    \vspace{1ex}

      Overrides: glue.pipeline.CondorDAGJob.\_\_init\_\_

    \end{boxedminipage}

  \textbf{Inherited from AnalysisJob:}
    channel,
    get\_config
    \\
  \textbf{Inherited from CondorDAGJob:}
    add\_var\_arg,
    add\_var\_opt
    \\
  \textbf{Inherited from CondorJob:}
    add\_arg,
    add\_condor\_cmd,
    add\_ini\_opts,
    add\_opt,
    add\_short\_opt,
    get\_args,
    get\_opts,
    get\_short\_opts,
    get\_stderr\_file,
    get\_stdout\_file,
    get\_sub\_file,
    set\_log\_file,
    set\_notification,
    set\_stderr\_file,
    set\_stdout\_file,
    set\_sub\_file,
    write\_sub\_file
    \index{stochastic \textit{(module)}!StoppJob \textit{(class)}|)}
    \index{stochastic \textit{(module)}!StoppNode \textit{(class)}|(}

%%%%%%%%%%%%%%%%%%%%%%%%%%%%%%%%%%%%%%%%%%%%%%%%%%%%%%%%%%%%%%%%%%%%%%%%%%%
%%                           Class Description                           %%
%%%%%%%%%%%%%%%%%%%%%%%%%%%%%%%%%%%%%%%%%%%%%%%%%%%%%%%%%%%%%%%%%%%%%%%%%%%

\subsection{Class StoppNode}

    \label{stochastic:StoppNode}
\begin{tabular}{cccccccc}
% Line for glue.pipeline.CondorDAGNode, linespec=[False, False]
\multicolumn{2}{r}{\settowidth{\BCL}{glue.pipeline.CondorDAGNode}\multirow{2}{\BCL}{glue.pipeline.CondorDAGNode}}
&&
&&
  \\\cline{3-3}
  &&\multicolumn{1}{c|}{}
&&
&&
  \\
% Line for glue.pipeline.AnalysisNode, linespec=[False]
\multicolumn{4}{r}{\settowidth{\BCL}{glue.pipeline.AnalysisNode}\multirow{2}{\BCL}{glue.pipeline.AnalysisNode}}
&&
  \\\cline{5-5}
  &&&&\multicolumn{1}{c|}{}
&&
  \\
% Line for glue.pipeline.CondorDAGNode, linespec=[True]
\multicolumn{4}{r}{\settowidth{\BCL}{glue.pipeline.CondorDAGNode}\multirow{2}{\BCL}{glue.pipeline.CondorDAGNode}}
&&\multicolumn{1}{|c}{}
  \\\cline{5-5}
  &&&&\multicolumn{1}{c|}{}
&\multicolumn{1}{|c}{}&
  \\
&&&&\multicolumn{2}{l}{\textbf{StoppNode}}
\end{tabular}

An StoppNode runs an instance of the stochastic stopp code in a Condor 
DAG.


%%%%%%%%%%%%%%%%%%%%%%%%%%%%%%%%%%%%%%%%%%%%%%%%%%%%%%%%%%%%%%%%%%%%%%%%%%%
%%                                Methods                                %%
%%%%%%%%%%%%%%%%%%%%%%%%%%%%%%%%%%%%%%%%%%%%%%%%%%%%%%%%%%%%%%%%%%%%%%%%%%%

  \subsubsection{Methods}

    \label{stochastic:StoppNode:__init__}
    \index{stochastic \textit{(module)}!StoppNode \textit{(class)}!\_\_init\_\_ \textit{(method)}}
    \vspace{0.5ex}

    \begin{boxedminipage}{\textwidth}

    \raggedright \textbf{\_\_init\_\_}(\textit{self}, \textit{job})

    \vspace{-1.5ex}

    \rule{\textwidth}{0.5\fboxrule}
    job = A CondorDagNode that can run an instance of lalapps\_stopp.

    \vspace{1ex}

      Overrides: glue.pipeline.AnalysisNode.\_\_init\_\_

    \end{boxedminipage}

    \label{stochastic:StoppNode:set_output_file}
    \index{stochastic \textit{(module)}!StoppNode \textit{(class)}!set\_output\_file \textit{(method)}}
    \vspace{0.5ex}

    \begin{boxedminipage}{\textwidth}

    \raggedright \textbf{set\_output\_file}(\textit{self}, \textit{output\_file})

    \vspace{-1.5ex}

    \rule{\textwidth}{0.5\fboxrule}
    Set the output file.

    \vspace{1ex}

    \end{boxedminipage}

  \textbf{Inherited from AnalysisNode:}
    calibration,
    calibration\_cache\_path,
    get\_calibration,
    get\_end,
    get\_ifo,
    get\_ifo\_tag,
    get\_input,
    get\_output,
    get\_start,
    set\_cache,
    set\_end,
    set\_ifo,
    set\_ifo\_tag,
    set\_input,
    set\_output,
    set\_start
    \\
  \textbf{Inherited from CondorDAGNode:}
    \_\_repr\_\_,
    add\_input\_file,
    add\_macro,
    add\_output\_file,
    add\_parent,
    add\_post\_script\_arg,
    add\_pre\_script\_arg,
    add\_var\_arg,
    add\_var\_opt,
    get\_args,
    get\_cmd\_line,
    get\_input\_files,
    get\_opts,
    get\_output\_files,
    get\_vds\_group,
    job,
    set\_log\_file,
    set\_name,
    set\_post\_script,
    set\_pre\_script,
    set\_retry,
    set\_vds\_group,
    write\_input\_files,
    write\_job,
    write\_output\_files,
    write\_parents,
    write\_post\_script,
    write\_pre\_script,
    write\_vars
    \index{stochastic \textit{(module)}!StoppNode \textit{(class)}|)}
    \index{stochastic \textit{(module)}|)}
