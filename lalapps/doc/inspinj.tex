\section{Program \texttt{lalapps\_inspinj}}
\label{program:lalapps-inspinj}
\idx[Program]{lalapps\_inspinj}

\begin{entry}

\item[Name]
\verb$lalapps_inspinj$ --- produces inspiral injection ILWD files

\item[Synopsis]
\verb$lalapps_inspinj$ [\verb$-h$] [\verb$-V$] [\verb$-v$]
[\verb$-d$ \textit{dbglvl}]
[\verb$-m$ \textit{mmin}\texttt{-}\textit{mmax}]
[\verb$-r$ \textit{range}]
[\verb$-s$ \textit{seed}]
[\verb$-t$ \textit{tstart}\texttt{-}\textit{tend}]
[\verb$-u$ \textit{tstep}]

\item[Description]
\verb$lalapps_inspinj$ produces  two  files,  injepochs.ilwd  and
injparams.ilwd,  that contain, respectively, the GPS times suitable for
inspiral injections, and the intrinsic inspiral signal parameters to be used
for those injections.

The  data  files  are  in  ILWD  format.   The  file injepochs.ilwd contains a
sequence of integer pairs representing  the  injection GPS time in seconds and
residual nano-seconds.  The file injparams.ilwd contains  the  intrinsic binary
parameters for each injection, which is a sequence of eight real numbers
representing  (in  order)  (1)  the total mass of the binary system (in solar
masses), (2) the dimensionless reduced mass --- reduced mass per unit  total
mass ---  in the range from 0 (extreme mass ratio) to 0.25 (equal masses), (3)
the distance to the system in  meters, (4)  the  inclination  of  the  binary
system orbit to the plane of the sky in radians, (5) the colaescence phase  in
radians,  (6) the longitude to the direction of the source in radians, (7) the
latitude  to  the  direction  of  the source  in  radians, and (8) the
polarization angle of the source in radians.


\item[Options]\leavevmode
\begin{entry}
\item[\texttt{-h}]
Print a help message.
\item[\texttt{-V}]
Print the version information.
\item[\texttt{-v}]
Verbose output.
\item[\texttt{-d} \textit{dbglvl}]
Set LAL debug level to \textit{dbglvl}.
\item[\texttt{-m} \textit{mmin}\texttt{-}\textit{mmax}]
Set minimum and maximum of the range of masses  for the  binary companions (in
solar masses).  (Default is 1.3--1.5.)
\item[\texttt{-r} \textit{range}]
Set the maximum distance range for the binary system (in kpc).  (Default is
1.0.)
\item[\texttt{-s} \textit{seed}]
Set  the random number generator seed.  (Default of 0 gets the seed from the
system clock.)
\item[\texttt{-t} \textit{tstart}\texttt{-}\textit{tend}]
Set the time interval for injection times  (in  GPS seconds).  (Default is
600000000--600010240.)
\item[\texttt{-u} \textit{tstep}]
Set  the mean time step size in seconds for uniform time steps if positive or
Poisson  time  steps  if negative.   (Default is $100/\pi = 31.8309886183791$.)
\end{entry}

\item[Debug levels]
The LAL debug level can be specified as an integer or as a string of flags:
\begin{entry}
\item[\texttt{NDEBUG}]
No debugging information is printed and memory debugging code is disabled.
\item[\texttt{ERROR}]
Error messages are printed.
\item[\texttt{WARNING}]
Warning messages are printed.
\item[\texttt{INFO}]
Information messages are printed.
\item[\texttt{TRACE}]
Function call tracing messages are printed.
\item[\texttt{MEMINFO}]
Memory  allocation  information messages are printed.
\item[\texttt{MEMDBG}]
Debugging of memory allocation routines is enabled but no messages are printed.
\end{entry}
The following composite levels are available:
\begin{entry}
\item[\texttt{MSGLVL1}]
Equivalent to \verb$ERROR$
\item[\texttt{MSGLVL2}]
Equivalent to \verb$ERROR | WARNING$
\item[\texttt{MSGLVL3}]
Equivalent to \verb$ERROR | WARNING | INFO$
\item[\texttt{ALLDBG}]
All debugging messages are printed.
\end{entry}

For example, the command
\begin{indented}
\verb$lalapps_inspinj -d "ERROR | INFO"$
\end{indented}
will set the debug level so that error and information messages are printed.

\item[Environment]\leavevmode

\begin{entry}
\item[\texttt{LAL\_DEBUG\_LEVEL}]
Default LAL debug level to use.
\end{entry}

\item[Author]
Jolien Creighton

\end{entry}
