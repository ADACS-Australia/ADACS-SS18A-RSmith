%%% $Id$
\section{Module \texttt{Julian.c}}

\subsection{Prototypes}
\begin{verbatim}
void
JulianDay (Status        *status,
           INT4          *jDay,
           const LALDate *date)

void
ModJulianDay (Status        *status,
              REAL8         *modJDay,
              const LALDate *date)

void
JulianDate (Status        *status,
            REAL8         *jDate,
            const LALDate *date)

void
ModJulianDate (Status        *status,
               REAL8         *modJDate,
               const LALDate *date)
\end{verbatim}


\subsection{Description}

These routines compute Julian Day, Modified Julian Day, Julian Date, and
Modified Julian Date for a given Gregorian date and time UTC.  Julian Day
and Modified Julian Day are integer number of days; Julian Date and
Modified Julian Date are decimal number of days.

\subsection{Operating Instructions}
See Section~\ref{sec:dateh:datatypes} for the definitions of the
data structures used.  Suppose we would like to get the Julian Date for
today.  The following program would accomplish this:

\begin{verbatim}
#include "LALStdlib.h"
#include "Date.h"

INT4 debuglevel = 2;

NRCSID (TESTJULIANDAYC, "Id");

int
main(int argc, char *argv[])
{
    time_t        now;
    LALUnixDate  *ltime;
    Status        status = {0};
    LALDate       date;
    REAL8         jDate;

    INITSTATUS (&status, "TestJulianDay", TESTJULIANDAYC);

    /*
     * Get current local time
     */
    time(&now);
    ltime = localtime(&now);

    /*
     * Assign local time to LALDate structure 
     */
    date.unixDate.tm_sec  = ltime->tm_sec;
    date.unixDate.tm_min  = ltime->tm_min;
    date.unixDate.tm_hour = ltime->tm_hour;
    date.unixDate.tm_mday = ltime->tm_mday;
    date.unixDate.tm_mon  = ltime->tm_mon;
    date.unixDate.tm_year = ltime->tm_year;
    date.unixDate.tm_wday = ltime->tm_wday;
    date.unixDate.tm_yday = ltime->tm_yday;
    date.unixDate.tm_isdst = ltime->tm_isdst;

    /*
     * Convert to Julian Date
     */
    JulianDate(&status, &jDate, &date);
    printf("\tJulian Date                = %10.1f\n", jDate);

    return 0;
}


\end{verbatim}

\subsubsection{Arguments}

\paragraph{\texttt{JulianDay}}

\subparagraph{Inputs}

\begin{itemize}
    \item \texttt{date}: pointer to \texttt{LALDate} structure containing the
    Gregorian date for which the Julian day number is to be computed.
\end{itemize}

\subparagraph{Outputs}

\begin{itemize}
    \item \texttt{jDay}: pointer to \texttt{INT4} variable containing the
    Julian Day number.
\end{itemize}


\paragraph{\texttt{ModJulianDay}}

\subparagraph{Inputs}

\begin{itemize}
    \item \texttt{date}: pointer to \texttt{LALDate} structure containing
    the Gregorian date for which the Modified Julian Day number is to be
    computed.
\end{itemize}

\subparagraph{Outputs}

\begin{itemize}
    \item \texttt{modJDay}: pointer to \texttt{INT4} variable containing
    the Modified Julian Day number.
\end{itemize}

\paragraph{\texttt{JulianDate}}

\subparagraph{Inputs}

\begin{itemize}
    \item \texttt{date}: A pointer to \texttt{LALDate} structure containing
    the Gregorian date and UTC time for which the Julian Date is to be
    computed.
\end{itemize}

\subparagraph{Outputs}

\begin{itemize}
    \item \texttt{jDate}: A pointer to \texttt{REAL8} variable containing the
    Julian Date.
\end{itemize}

\paragraph{\texttt{ModJulianDate}}

\subparagraph{Inputs}

\begin{itemize}
    \item \texttt{date}: A pointer to \texttt{LALDate} structure containing
    the Gregorian date and UTC time for which the Modified Julian Date is
    to be computed.
\end{itemize}

\subparagraph{Outputs}

\begin{itemize}
    \item \texttt{modJDate}: A pointer to \texttt{REAL8} variable containing
    the Julian Date.
\end{itemize}

\subsubsection{Error Conditions}

\subsubsection{JulianDay}
\begin{tabular}{|c|l|l|}
  \hline
  Status & Status       & Explanation \\
  code   & description  &             \\
  \hline
  \tt 1  & \tt Null pointer & Pointer to input data is NULL. \\
  \tt 2  & \tt Date too early & Julian Day can only be computed \\
         &                    & for dates after 1900-Mar-01 \\
  \hline
\end{tabular}

\subsubsection{ModJulianDay}
\begin{tabular}{|c|l|l|}
  \hline
  Status & Status       & Explanation \\
  code   & description  &             \\
  \hline
  \tt 1  & \tt Null pointer & Pointer to input data is NULL. \\
  \hline
\end{tabular}

\subsubsection{JulianDate}
\begin{tabular}{|c|l|l|}
  \hline
  Status & Status       & Explanation \\
  code   & description  &             \\
  \hline
  \tt 1  & \tt Null pointer & Pointer to input data is NULL. \\
  \hline
\end{tabular}

\subsubsection{ModJulianDate}
\begin{tabular}{|c|l|l|}
  \hline
  Status & Status       & Explanation \\
  code   & description  &             \\
  \hline
  \tt 1  & \tt Null pointer & Pointer to input data is NULL. \\
  \hline
\end{tabular}

\subsection{Algorithm}

Julian Days and Dates are measured from the Julian Epoch: 1 Jan 4713BCE
12:00 GMT = 4713 BC Jan 01d.5 GMT.  Julian Date is the time elapsed since
the Julian epoch, measured in days and fractions of a day.  There are some
complications, including the length of a Tropical Year, the Gregorian
correction to the calendar, and leap years. See \cite{green:1985} and
\cite{esaa:1992} for details.

The algorithm for computing the Julian Day number used here is due to
Van~Flandern and Pulkkinen \cite{vfp:1979}.  It is valid only for dates
since 1900-Mar.
%
\begin{equation}
  jd = 367 * y - 7 * (y + (m + 9)/12)/4 + 275 * m/9 + d + 1721014
\end{equation}
%
where $y$ is the year, $m$ is the month (1-12), and $d$ is the day of the
month (1-31).  The Modified Julian Day is given by:
%
\begin{equation}
  mjd = jd - 2400000.5
\end{equation}
%
To compute the Julian Date and the Modified Julian Date, the time of day is
converted into fractions of a day, and added to the Julian Day.


\subsection{Accuracy}

\subsection{Tests}

\subsection{Uses}

% Dependencies, LAL or otherwise

\subsection{Notes}


  




%%% Local Variables: 
%%% mode: latex
%%% TeX-master: "Date"
%%% End: 

