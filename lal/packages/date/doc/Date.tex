%%% $Id$
% Is this file is meant to be standalone or included in another?
% Should arrange that it is included with the wrapper setting headers,
% etc. 
\documentclass{article}
\usepackage{amssymb}
\usepackage{amstext}
\usepackage{amsfonts}
\begin{document}

\title{Package \texttt{Date}}
\author{David Chin \texttt{<dwchin@umich.edu>}}
\date{2000-May-23}
\maketitle

This package covers LAL routines for manipulating date and time
information.  It is organized under a single header file, \verb@Date.h@,
which provides all the necessary declarations for using the date and time
routines. 

%%% $Id$

\section{Header \texttt{Date.h}}
\label{sec:dateh:header}
Provides routines for manipulating date and time information.

\subsection{Synopsis}
\label{sec:dateh:synopsis}
\begin{verbatim}
#include "Date.h"
\end{verbatim}

\subsubsection{Datatypes}
\label{sec:dateh:datatypes}
\begin{verbatim}
/* The standard Unix tm structure */
typedef struct
tm
LALUnixDate;

/* This time object is exactly like LIGOTimeGPS, except for the name */
typedef struct
tagLIGOTimeUTC
{
    INT4 utcSeconds;     /* no. of seconds since Unix epoch */
    INT4 utcNanoSeconds; /* residual no. of nanoseconds */
}
LIGOTimeUTC;

/* Encode timezone information */
typedef struct
tagLALTimezone
{
    INT4 secondsWest; /* seconds West of UTC */
    INT4 dst;         /* Daylight Savings Time correction to apply */
}
LALTimezone;    

/* Date and time structure */
typedef struct
tagLALDate
{
    LALUnixDate unixDate;
    INT4        residualNanoSeconds; /* residual nanoseconds */
    LALTimezone timezone;            /* timezone information */
}
LALDate;
\end{verbatim}

\subsubsection{Routine prototypes}
\begin{verbatim}

void JulianDay (Status*, INT4*, const LALDate*);
void ModJulianDay(Status*, REAL8*, const LALDate*);
void JulianDate(Status*, REAL8*, const LALDate*);
void ModJulianDate(Status*, REAL8*, const LALDate*);
void UTCtoGPS(Status*, LIGOTimeGPS*, const LIGOTimeUTC*);
void GPStoUTC(Status*, LIGOTimeUTC*, const LIGOTimeGPS*);
void UTCtime(Status*, LALDate*, const LIGOTimeUTC*);
void DateString(Status*, CHARVector*, const LALDate*);
void GMST1(Status*, REAL8*, const LALDate*, INT4);
void LMST1(Status*, REAL8*, const LALDate*, REAL8, INT4);
void SecsToLALDate(Status*, LALDate*, REAL8);

\end{verbatim}



\subsection{Error conditions}

%%% Local Variables: 
%%% mode: latex
%%% TeX-master: "Date"
%%% End: 


%%% $Id$
\section{Module \texttt{Julian.c}}

\subsection{Prototypes}
\begin{verbatim}
void
JulianDay (Status        *status,
           INT4          *jDay,
           const LALDate *date)

void
ModJulianDay (Status        *status,
              REAL8         *modJDay,
              const LALDate *date)

void
JulianDate (Status        *status,
            REAL8         *jDate,
            const LALDate *date)

void
ModJulianDate (Status        *status,
               REAL8         *modJDate,
               const LALDate *date)
\end{verbatim}


\subsection{Description}

These routines compute Julian Day, Modified Julian Day, Julian Date, and
Modified Julian Date for a given Gregorian date and time UTC.  Julian Day
and Modified Julian Day are integer number of days; Julian Date and
Modified Julian Date are decimal number of days.

\subsection{Operating Instructions}
See Section~\ref{sec:dateh:datatypes} for the definitions of the
data structures used.  Suppose we would like to get the Julian Date for
today.  The following program would accomplish this:

\begin{verbatim}
#include "LALStdlib.h"
#include "Date.h"

INT4 debuglevel = 2;

NRCSID (TESTJULIANDAYC, "Id");

int
main(int argc, char *argv[])
{
    time_t        now;
    LALUnixDate  *ltime;
    Status        status = {0};
    LALDate       date;
    REAL8         jDate;

    INITSTATUS (&status, "TestJulianDay", TESTJULIANDAYC);

    /*
     * Get current local time
     */
    time(&now);
    ltime = localtime(&now);

    /*
     * Assign local time to LALDate structure 
     */
    date.unixDate.tm_sec  = ltime->tm_sec;
    date.unixDate.tm_min  = ltime->tm_min;
    date.unixDate.tm_hour = ltime->tm_hour;
    date.unixDate.tm_mday = ltime->tm_mday;
    date.unixDate.tm_mon  = ltime->tm_mon;
    date.unixDate.tm_year = ltime->tm_year;
    date.unixDate.tm_wday = ltime->tm_wday;
    date.unixDate.tm_yday = ltime->tm_yday;
    date.unixDate.tm_isdst = ltime->tm_isdst;

    /*
     * Convert to Julian Date
     */
    JulianDate(&status, &jDate, &date);
    printf("\tJulian Date                = %10.1f\n", jDate);

    return 0;
}


\end{verbatim}

\subsubsection{Arguments}

\paragraph{\texttt{JulianDay}}

\subparagraph{Inputs}

\begin{itemize}
    \item \texttt{date}: pointer to \texttt{LALDate} structure containing the
    Gregorian date for which the Julian day number is to be computed.
\end{itemize}

\subparagraph{Outputs}

\begin{itemize}
    \item \texttt{jDay}: pointer to \texttt{INT4} variable containing the
    Julian Day number.
\end{itemize}


\paragraph{\texttt{ModJulianDay}}

\subparagraph{Inputs}

\begin{itemize}
    \item \texttt{date}: pointer to \texttt{LALDate} structure containing
    the Gregorian date for which the Modified Julian Day number is to be
    computed.
\end{itemize}

\subparagraph{Outputs}

\begin{itemize}
    \item \texttt{modJDay}: pointer to \texttt{INT4} variable containing
    the Modified Julian Day number.
\end{itemize}

\paragraph{\texttt{JulianDate}}

\subparagraph{Inputs}

\begin{itemize}
    \item \texttt{date}: A pointer to \texttt{LALDate} structure containing
    the Gregorian date and UTC time for which the Julian Date is to be
    computed.
\end{itemize}

\subparagraph{Outputs}

\begin{itemize}
    \item \texttt{jDate}: A pointer to \texttt{REAL8} variable containing the
    Julian Date.
\end{itemize}

\paragraph{\texttt{ModJulianDate}}

\subparagraph{Inputs}

\begin{itemize}
    \item \texttt{date}: A pointer to \texttt{LALDate} structure containing
    the Gregorian date and UTC time for which the Modified Julian Date is
    to be computed.
\end{itemize}

\subparagraph{Outputs}

\begin{itemize}
    \item \texttt{modJDate}: A pointer to \texttt{REAL8} variable containing
    the Julian Date.
\end{itemize}

\subsubsection{Error Conditions}

\subsubsection{JulianDay}
\begin{tabular}{|c|l|l|}
  \hline
  Status & Status       & Explanation \\
  code   & description  &             \\
  \hline
  \tt 1  & \tt Null pointer & Pointer to input data is NULL. \\
  \tt 2  & \tt Date too early & Julian Day can only be computed \\
         &                    & for dates after 1900-Mar-01 \\
  \hline
\end{tabular}

\subsubsection{ModJulianDay}
\begin{tabular}{|c|l|l|}
  \hline
  Status & Status       & Explanation \\
  code   & description  &             \\
  \hline
  \tt 1  & \tt Null pointer & Pointer to input data is NULL. \\
  \hline
\end{tabular}

\subsubsection{JulianDate}
\begin{tabular}{|c|l|l|}
  \hline
  Status & Status       & Explanation \\
  code   & description  &             \\
  \hline
  \tt 1  & \tt Null pointer & Pointer to input data is NULL. \\
  \hline
\end{tabular}

\subsubsection{ModJulianDate}
\begin{tabular}{|c|l|l|}
  \hline
  Status & Status       & Explanation \\
  code   & description  &             \\
  \hline
  \tt 1  & \tt Null pointer & Pointer to input data is NULL. \\
  \hline
\end{tabular}

\subsection{Algorithm}

Julian Days and Dates are measured from the Julian Epoch: 1 Jan 4713BCE
12:00 GMT = 4713 BC Jan 01d.5 GMT.  Julian Date is the time elapsed since
the Julian epoch, measured in days and fractions of a day.  There are some
complications, including the length of a Tropical Year, the Gregorian
correction to the calendar, and leap years. See \cite{green:1985} and
\cite{esaa:1992} for details.

The algorithm for computing the Julian Day number used here is due to
Van~Flandern and Pulkkinen \cite{vfp:1979}.  It is valid only for dates
since 1900-Mar.
%
\begin{equation}
  jd = 367 * y - 7 * (y + (m + 9)/12)/4 + 275 * m/9 + d + 1721014
\end{equation}
%
where $y$ is the year, $m$ is the month (1-12), and $d$ is the day of the
month (1-31).  The Modified Julian Day is given by:
%
\begin{equation}
  mjd = jd - 2400000.5
\end{equation}
%
To compute the Julian Date and the Modified Julian Date, the time of day is
converted into fractions of a day, and added to the Julian Day.


\subsection{Accuracy}

\subsection{Tests}

\subsection{Uses}

% Dependencies, LAL or otherwise

\subsection{Notes}


  




%%% Local Variables: 
%%% mode: latex
%%% TeX-master: "Date"
%%% End: 



%%% $Id$

\section{Module \texttt{UtoGPS.c}}

\subsection{Prototypes}
\begin{verbatim}
void
UtoGPS (Status             *status,
        LIGOTimeGPS        *gpstime,
        const LIGOTimeUnix *unixtime)

void
GPStoU (Status            *status,
        LIGOTimeUnix      *unixtime,
        const LIGOTimeGPS *gpstime)
\end{verbatim}

\subsection{Description}

These routines convert between time measured from the Unix epoch and time
measured from the GPS epoch.

\subsection{Operating Instructions}
For an example, see Section~\ref{sec:datestringc}.

\subsubsection{Arguments}

\paragraph{\texttt{UtoGPS}}

\subparagraph{Inputs}

\begin{itemize}
    \item \texttt{unixtime}: pointer to a \texttt{LIGOTimeUnix} structure
      containing the time in the Unix epoch.
\end{itemize}

\subparagraph{Outputs}

\begin{itemize}
  \item \texttt{gpstime}: pointer to a \texttt{LIGOTimeGPS} structure containing
    the time in the GPS epoch.
\end{itemize}

\paragraph{\texttt{GPStoU}}

\subparagraph{Inputs}

\begin{itemize}
  \item \texttt{gpstime}: pointer to a \texttt{LIGOTimeGPS} structure containing
    the time in the GPS epoch.
\end{itemize}

\subparagraph{Outputs}

\begin{itemize}
  \item \texttt{unixtime}: pointer to a \texttt{LIGOTimeUnix} structure
    containing the time in the Unix epoch.
\end{itemize}


\subsubsection{Error Conditions}
\begin{tabular}{|c|l|l|}
  \hline
  Status & Status       & Explanation \\
  code   & description  &             \\
  \hline
  \tt 1  & \tt Null pointer & Pointer to input data is NULL. \\
  \hline
\end{tabular}


\subsection{Algorithm}

The algorithm used here is from the GRASP distribution \cite{grasp:194}.
The Unix epoch is 1970-Jan-01 00:00:00 UTC, and the GPS epoch is 1980-Jan-06
00:00:00 UTC.  According to the US Naval Observatory GPS time is {\em not}
adjusted for leap seconds.  The USNO also gives this table, true as of
1999-Jan-01:

\begin{center}
  \begin{tabular}{|l|l|}
    \hline 
    TAI is ahead of UTC by & 32 seconds \\
    TAI is ahead of GPS by & 19 seconds \\
    GPS is ahead of UTC by & 13 seconds \\
    \hline
  \end{tabular}
\end{center}

For more information, see \texttt{http://tycho.usno.navy.mil/leapsec.html}
and also the comments in \texttt{UTCtime.c}.

The integer number of seconds between the Unix epoch and the GPS epoch is
315964811: 8 years, 2 leap years, 5 days, and $11 = (19.0 - 8.0)$ leap
seconds between 1970-01-01 00:00:00 and 1980-01-06 00:00:00.  So,
converting from GPS to Unix epoch and vice versa requires adding or
subtracting 315964811 seconds.

\subsection{Accuracy}

Strictly speaking, the amount of time between the Unix epoch and the GPS
epoch is 315964810.999918 seconds.

\subsection{Tests}
\label{sec:utogpsc:tests}

\subsection{Uses}
\label{sec:utogpsc:uses}


\subsection{Notes}
\label{sec:utogpsc:notes}


%%% Local Variables: 
%%% mode: latex
%%% TeX-master: "Date"
%%% End: 


%%% $Id$
\section{Module \texttt{Utime.c}}

\subsection{Prototypes}
\begin{verbatim}
void
Utime (Status              *status,
       LALDate             *utc,
       const LIGOTimeUnix  *unixtime)
\end{verbatim}

\subsection{Description}

This routine converts (UTC) time in a \texttt{LIGOTimeUnix} structure
measured in the Unix epoch to the same time in a \texttt{LALDate}
structure, corrected for leap seconds.

\subsection{Operating Instructions}

Here is a simple example (adapted from GRASP) to form a timestamp string:

\begin{verbatim}
/* Taken from GRASP */
void printone(Status *status, const LIGOTimeUnix *time1)
{
    LIGOTimeGPS  gpstime;
    LALDate      laldate;
    LIGOTimeUnix tmp;
    LALUnixDate  utimestruct;
    CHARVector *utc = NULL;

    INITSTATUS (status, "printone", TESTUTOGPSC);

    /*
     * Allocate space for CHARVectors
     */
    CHARCreateVector(status, &utc, (UINT4)64);

    /* compute GPS time */
    UtoGPS(status, &gpstime, time1);

    /*
     * construct strings with appropriate time stamps
     */
    /* Utime() */
    Utime(status, &laldate, time1);
    DateString(status, utc, &laldate);

    printf("%s\n", utc->data);

    /*
     * House cleaning
     */
    CHARDestroyVector(status, &utc);
    
    RETURN (status);
}
\end{verbatim}

\subsubsection{Arguments}

\paragraph{\texttt{Utime}}

\subparagraph{Inputs}

\begin{itemize}
    \item \texttt{unixtime}: A pointer to a \texttt{LIGOTimeUnix} structure
    containing the time elapsed since the Unix epoch (1970-Jan-01 0h).
\end{itemize}

\subparagraph{Outputs}

\begin{itemize}
    \item \texttt{utc}: A pointer to an \texttt{LALDate} structure that
    contains the date and time UTC.
\end{itemize}

\subsubsection{Error Conditions}
\begin{tabular}{|c|l|l|}
  \hline
  Status & Status       & Explanation \\
  code   & description  &             \\
  \hline
  \tt 1  & \tt Null pointer & Pointer to input data is NULL. \\
  \tt 2  & \tt Input time out of range & $0 \leqslant \textrm{input} \leqslant 946684823$ \\
  \hline
\end{tabular}

\subsection{Algorithm}


\subsection{Accuracy}

\subsection{Tests}

\subsection{Uses}


\subsection{Notes}
\label{sec:utimec:notes}



%%% Local Variables: 
%%% mode: latex
%%% TeX-master: "Date"
%%% End: 


%%% $Id$
\section{Module \texttt{LMST1.c}}

\subsection{Prototypes}
\begin{verbatim}
void
GMST1 (Status        *status,
       REAL8         *gmst,
       const LALDate *date,
       INT4           outunits)

void
LMST1 (Status        *status,
       REAL8         *lmst,
       const LALDate *date,
       REAL8          longitude,
       INT4           outunits)
\end{verbatim}

\subsection{Description}

The routines in this module compute Mean Sidereal Time in a choice of
units: seconds, hours, degrees, or radians. \texttt{GMST1} computes GMST1,
and \texttt{LMST1} computes LMST1.  LMST1 is offset from GMST1 by
the longitude of the observing post.

\subsection{Operating Instructions}
Here is a simple example:

\begin{verbatim}
#include <stdlib.h>
#include "LALStdlib.h"
#include "Date.h"

INT4 debuglevel = 2;

NRCSID (TESTLMSTC, "Id");

int
main(int argc, char *argv[])
{
    LALDate date;
    LALDate mstdate;
    REAL8   gmsthours, lmsthours;
    REAL8   gmstsecs;
    REAL8   longitude;
    time_t  timer;
    CHAR    timestamp[64], tmpstr[64];
    Status  status = {0};

    if (argc == 1)
    {
        /*
         * Print help message and exit
         */
        printf("Usage: TestUTCtoGPS debug_level -- debug_level = [0,1,2]\n");
        return 0;
    }

    if (argc == 2)
        debuglevel = atoi(argv[1]);

    INITSTATUS(&status, "TestLMST", TESTLMSTC);

    printf("TEST of GMST1 routine\n");
    printf("=====================\n");

    /*
     * Check against the Astronomical Almanac:
     * For 1994-11-16 0h UT - Julian Date 2449672.5, GMST 03h 39m 21.2738s
     */
    date.unixDate.tm_sec  = 0;
    date.unixDate.tm_min  = 0;
    date.unixDate.tm_hour = 0;
    date.unixDate.tm_mday = 16;
    date.unixDate.tm_mon  = 10;
    date.unixDate.tm_year = 94;

    longitude = 0.; /* Greenwich */
    GMST1(&status, &gmsthours, &date, MST_HRS);
    LMST1(&status, &lmsthours, &date, longitude, MST_HRS);

    GMST1(&status, &gmstsecs, &date, MST_SEC);
    SecsToLALDate(&status, &mstdate, gmstsecs);
    strftime(timestamp, 64, "%Hh %Mm %S", &(mstdate.unixDate));
    sprintf(tmpstr, "%fs", mstdate.residualNanoSeconds * 1.e-9);
    strcat(timestamp, tmpstr+1); /* remove leading 0 */
    
    printf("gmsthours = %f = %s\n", gmsthours, timestamp);
    printf("    expect: 3.655728 = 03h 39m 20.6222s \n");
    /* printf("lmsthours = %f\n", lmsthours); */

    return(0);
}
\end{verbatim}


\subsubsection{Arguments}

\paragraph{\texttt{GMST1}}

\subparagraph{Inputs}

\begin{itemize}

   \item \texttt{date}: A pointer to an \texttt{LALDate} structure
     containing Gregorian date and UTC time for which to compute GMST1.

   \item \texttt{outunits}: An integer defining the units in which GMST1
     are to be expressed.  Preprocessor macros are defined in \texttt{Date.h}
     for selecting the units for MST1, which are passed in the argument
     \texttt{outunits}.  The following table lists the macros:
%
    \begin{center}
         \begin{tabular}{|c|l|}
           \hline 
           \verb MST_SEC & seconds \\
           \verb MST_HRS & hours   \\
           \verb MST_DEG & degrees \\
           \verb MST_RAD & radians \\
           \hline
         \end{tabular}
       \end{center}
%
\end{itemize}

\subparagraph{Outputs}

\begin{itemize}
  \item \texttt{gmst}: A pointer to a \texttt{REAL8} containing GMST1 in
    requested units.
\end{itemize}

\paragraph{\texttt{LMST1}}

\subparagraph{Inputs}

\begin{itemize}

   \item \texttt{date}: A pointer to an \texttt{LALDate} structure
     containing Gregorian date and UTC time for which to compute LMST1.

   \item \texttt{longitude}: A \texttt{REAL8} with the longitude of the
     observing station in decimal degrees (\textit{i.e.} as stored in a
     \texttt{GeodeticCoords} structure (see module \texttt{AM}).

   \item \texttt{outunits}: An integer defining the units in which LMST1
     are to be expressed.  Preprocessor macros are defined in \texttt{Date.h}
     for selecting the units for MST1, which are passed in the argument
     \texttt{outunits}.  The following table lists the macros:
%
    \begin{center}
         \begin{tabular}{|c|l|}
           \hline 
           \verb MST_SEC & seconds \\
           \verb MST_HRS & hours   \\
           \verb MST_DEG & degrees \\
           \verb MST_RAD & radians \\
           \hline
         \end{tabular}
       \end{center}
%
\end{itemize}

\subparagraph{Outputs}

\begin{itemize}
  \item \texttt{lmst}: A pointer to a \texttt{REAL8} containing the LMST1 in
    requested units.
\end{itemize}


\subsubsection{Error Conditions}

\paragraph{\texttt{LMST1}}

\subparagraph{}

\begin{tabular}{|c|l|l|}
  \hline
  Status & Status       & Explanation \\
  code   & description  &             \\
  \hline
  \tt 1  & \tt Null pointer & Pointer to input data is NULL. \\
  \hline
\end{tabular}

\paragraph{\texttt{GMST1}}

\subparagraph{}

\begin{tabular}{|c|l|l|}
  \hline
  Status & Status       & Explanation \\
  code   & description  &             \\
  \hline
  \tt 1  & \tt Null pointer & Pointer to input data is NULL. \\
  \hline
\end{tabular}

\subsection{Algorithm}

The formula used to compute GMST1 at 0h UT1 is from the \textit{Explanatory
Supplement to the Astronomical Almanac}, Ch.~2, Sec.~24 \cite{esaa:1992},
and also in Section B of the \textit{Astronomical Almanac}.  The formula is
%
\begin{eqnarray*}
    {\text{GMST at $0^{h}$ UT}} & = & 24\ 110^{s}\!.548\ 41 + 8\ 640\ 184^{s}\!.812\ 866 T_{U} \\
 & & + 0^{s}\!.093\ 104 T_{U}^{2} - 6^{s}\!.2\times10^{-6} T_{U}^{3}
\end{eqnarray*}
%
where $T_{U} = (\mathrm{JD} - 2\ 451\ 545.0)/36\ 525$.  To compute GMST at
some other time UT, the Mean Sidereal Time interval is added to the GMST
computed above.  The Mean Sidereal Time interval is obtained by multiplying
the UT time interval by $1.002737909350795 + 5.9006\times10^{-11}T_{U} +
5.9\times10^{-15}T_{U}^{2}$.

The implementation used in \texttt{GMST1()} is copied from NOVAS
(\texttt{http://aa.usno.navy.mil/AA/}).  It gives better precision than a
naive translation of the above algorithm.

\subsection{Accuracy}
This implementation for computing GMST1 is accurate to about 1 sidereal
second.  To compute accurate locations of celestial objects, what is needed
is the Apparent Sidereal Time, which differs from Mean Sidereal Time by the
Equation of the Equinoxes:
%
\begin{eqnarray*}
  GAST = GMST + \textrm{equation of equinoxes}
\end{eqnarray*}
%
where the equation of equinoxes $= \frac{1}{15}(\Delta\psi \cos\epsilon +
0^{"}\!.002\ 64 \sin\Omega + 0^{"}\!.000\ 063 \sin
2\Omega$, and $\Delta\psi$ is the total nutation in longitude, $\epsilon$
is the mean obliquity of th eecliptic and $\Omega$ is the mean longitude of
the ascending node of the Moon.  The Equation of the Equinoxes is tabulated
in the \textit{Astronomical Almanac}, and can be interpolated.  It can also
be computed using the NOVAS software from the US Naval Observatory
(\texttt{http://aa.usno.navy.mil/AA/}). For the year 2000, the Equation of
the Equinoxes goes to an absolute maximum of about 1 sidereal second.

\subsection{Tests}

\subsection{Uses}

\subsection{Notes}


%%% Local Variables: 
%%% mode: latex
%%% TeX-master: "Date"
%%% End: 


%%% $Id$


\section{Module \texttt{SecsToLALDate.c}}

\subsection{Prototypes}
\begin{verbatim}
void
SecsToLALDate(Status  *status,
              LALDate *date,
              REAL8    seconds)
\end{verbatim}

\subsection{Description}

This routine converts a time of day in decimal seconds since 0h (midnight)
to an LALDate structure.  Of course, the date information is not present.

\subsection{Operating Instructions}

A simple example:

\begin{verbatim}
#include <stdlib.h>
#include "LALStdlib.h"
#include "Date.h"

INT4 debuglevel = 2;

NRCSID (TESTLMSTC, "Id");

int
main(int argc, char *argv[])
{
    LALDate date;
    LALDate mstdate;
    REAL8   gmstsecs;
    CHAR    timestamp[64], tmpstr[64];
    Status  status = {0};

    
    INITSTATUS(&status, "TestLMST", TESTLMSTC);

    printf("TEST of GMST1 routine\n");
    printf("=====================\n");

    /*
     * Check against the Astronomical Almanac:
     * For 1994-11-16 0h UT - Julian Date 2449672.5, GMST 03h 39m 21.2738s
     */
    date.unixDate.tm_sec  = 0;
    date.unixDate.tm_min  = 0;
    date.unixDate.tm_hour = 0;
    date.unixDate.tm_mday = 16;
    date.unixDate.tm_mon  = 10;
    date.unixDate.tm_year = 94;

    GMST1(&status, &gmstsecs, &date, MST_SEC);
    SecsToLALDate(&status, &mstdate, gmstsecs);
    strftime(timestamp, 64, "%Hh %Mm %S", &(mstdate.unixDate));
    sprintf(tmpstr, "%fs", mstdate.residualNanoSeconds * 1.e-9);
    strcat(timestamp, tmpstr+1); /* remove leading 0 */
    
    printf("gmst = %s\n", timestamp);
    printf("    expect: 03h 39m 20.6222s \n");

    return(0);
}

\end{verbatim}

\subsubsection{Arguments}

\paragraph{\texttt{SecsToLALDate}}

\subparagraph{Inputs}

\begin{itemize}
    \item \texttt{seconds}: A \texttt{REAL8} containing decimal number of
    seconds since 0h (midnight) to be converted to an \texttt{LALDate}
    structure.
\end{itemize}

\subparagraph{Outputs}

\begin{itemize}
    \item \texttt{date}: A pointer to an \texttt{LALDate} structure
    containing the converted time.
\end{itemize}

\subsubsection{Error Conditions}
\begin{tabular}{|c|l|l|}
  \hline
  Status & Status       & Explanation \\
  code   & description  &             \\
  \hline
  \tt 1  & \tt Null pointer & Pointer to output data is NULL. \\
  \hline
\end{tabular}

\subsection{Algorithm}

\subsection{Accuracy}

\subsection{Tests}

\subsection{Uses}

\subsection{Notes}


%%% Local Variables: 
%%% mode: latex
%%% TeX-master: "Date"
%%% End: 


%%% $Id$
\section{Module \texttt{DateString.c}}
\label{sec:datestringc}

\subsection{Prototypes}
\begin{verbatim}
void
DateString (Status        *status,
            CHARVector    *timestamp,
            const LALDate *date)
\end{verbatim}


\subsection{Description}

Forms a timestamp string in ISO 8601 format, stored in
\texttt{timestamp->data}.  See \cite{iso8601} available via the Web
\texttt{http://www.iso.ch/markete/8601.pdf}.

\subsection{Operating Instructions}
See Section~\ref{sec:dateh:datatypes} for the definitions of the
data structures used.  Suppose we would like to form a timestamp string for
the current time.  The following program would accomplish this:

\begin{verbatim}
/* Taken from GRASP */
void printone(Status *status, const LIGOTimeUnix *time1)
{
    LIGOTimeGPS  gpstime;
    LALDate      laldate;
    LIGOTimeUnix tmp;
    LALUnixDate  utimestruct;
    CHARVector *utc = NULL;

    INITSTATUS (status, "printone", TESTUTOGPSC);

    /*
     * Allocate space for CHARVectors
     */
    CHARCreateVector(status, &utc, (UINT4)64);

    /* compute GPS time */
    UtoGPS(status, &gpstime, time1);

    /*
     * construct strings with appropriate time stamps
     */
    /* Utime() */
    Utime(status, &laldate, time1);
    DateString(status, utc, &laldate);

    printf("%s\n", utc->data);

    /*
     * House cleaning
     */
    CHARDestroyVector(status, &utc);
    
    RETURN (status);
}
\end{verbatim}

\subsubsection{Arguments}

\paragraph{\texttt{DateString}}

\subparagraph{Inputs}

\begin{itemize}
    \item \texttt{date}: A pointer to an \texttt{LALDate} structure
        containing the date and time for which a timestamp string is to be
        generated.
\end{itemize}

\subparagraph{Outputs}

\begin{itemize}
    \item \texttt{timestamp}: A pointer to a \texttt{CHARVector} structure
    which will contain the timestamp string.  The length of this vector
    should be at least 26.
\end{itemize}

\subsubsection{Error Conditions}
\begin{tabular}{|c|l|l|}
  \hline
  Status & Status       & Explanation \\
  code   & description  &             \\
  \hline
  \tt 1  & \tt Null pointer & Pointer to input data is NULL. \\
  \hline
\end{tabular}

\subsection{Algorithm}

\texttt{DateString()} uses \texttt{strftime (3)} to format the timestamp
string. 

\subsection{Accuracy}

\subsection{Tests}

\subsection{Uses}

% Dependencies, LAL or otherwise

\subsection{Notes}



%%% Local Variables: 
%%% mode: latex
%%% TeX-master: "Date"
%%% End: 


\begin{thebibliography}{0}
\bibitem{ptvf:1992}
  W. H. Press, S. A. Teukolsky, W. T. Vetterling, and B. P. Flannery,
  \textit{Numerical Recipes in C: The Art of Scientific Computing}, 2nd ed.
  (Cambridge University Press, Cambridge, 1992).

\bibitem{vfp:1979}
  T. C. Van Flandern, and K. F. Pulkkinen, 
  \textit{Astrophysical Journal  Supplement Series}, \textbf{41},
  391-411, 1979 Nov.
  
\bibitem{esaa:1992}
  \textit{Explanatory Supplement to the Astronomical Almanac} (University
  Science Books, Mill Valley, 1992)

\bibitem{novas:1999}
  \textit{Naval Observatory Vector Astrometry Subroutines, C Version 2.0.1}
(U.~S.~Naval Observatory, Astronomical Applications Dept., Dec. 1999)
  
\bibitem{green:1985}
  R.M. Green, \textit{Spherical Astronomy} (Cambridge University Press,
  Cambridge, 1985)
  
\bibitem{grasp:194}
  B. Allen, \textit{et al.}, GRASP Release 1.9.4 (University of Wisconsin
  - Milwaukee, Milwaukee, 1999)
  
\bibitem{iso8601}
  \textit{International Standard ISO 8601} (International Organization for
  Standardization, Switzerland, 1988)
  
\end{thebibliography}

\end{document}

%%% Local Variables: 
%%% mode: latex
%%% TeX-master: t
%%% End: 

