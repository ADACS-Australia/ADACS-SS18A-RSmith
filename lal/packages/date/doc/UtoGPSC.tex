%%% $Id$

\section{Module \texttt{UtoGPS.c}}

\subsection{Prototypes}
\begin{verbatim}
void
UtoGPS (Status             *status,
        LIGOTimeGPS        *gpstime,
        const LIGOTimeUnix *unixtime)

void
GPStoU (Status            *status,
        LIGOTimeUnix      *unixtime,
        const LIGOTimeGPS *gpstime)
\end{verbatim}

\subsection{Description}

These routines convert between time measured from the Unix epoch and time
measured from the GPS epoch.

\subsection{Operating Instructions}
For an example, see Section~\ref{sec:datestringc}.

\subsubsection{Arguments}

\paragraph{\texttt{UtoGPS}}

\subparagraph{Inputs}

\begin{itemize}
    \item \texttt{unixtime}: pointer to a \texttt{LIGOTimeUnix} structure
      containing the time in the Unix epoch.
\end{itemize}

\subparagraph{Outputs}

\begin{itemize}
  \item \texttt{gpstime}: pointer to a \texttt{LIGOTimeGPS} structure containing
    the time in the GPS epoch.
\end{itemize}

\paragraph{\texttt{GPStoU}}

\subparagraph{Inputs}

\begin{itemize}
  \item \texttt{gpstime}: pointer to a \texttt{LIGOTimeGPS} structure containing
    the time in the GPS epoch.
\end{itemize}

\subparagraph{Outputs}

\begin{itemize}
  \item \texttt{unixtime}: pointer to a \texttt{LIGOTimeUnix} structure
    containing the time in the Unix epoch.
\end{itemize}


\subsubsection{Error Conditions}
\begin{tabular}{|c|l|l|}
  \hline
  Status & Status       & Explanation \\
  code   & description  &             \\
  \hline
  \tt 1  & \tt Null pointer & Pointer to input data is NULL. \\
  \hline
\end{tabular}


\subsection{Algorithm}

The algorithm used here is from the GRASP distribution \cite{grasp:194}.
The Unix epoch is 1970-Jan-01 00:00:00 UTC, and the GPS epoch is 1980-Jan-06
00:00:00 UTC.  According to the US Naval Observatory GPS time is {\em not}
adjusted for leap seconds.  The USNO also gives this table, true as of
1999-Jan-01:

\begin{center}
  \begin{tabular}{|l|l|}
    \hline 
    TAI is ahead of UTC by & 32 seconds \\
    TAI is ahead of GPS by & 19 seconds \\
    GPS is ahead of UTC by & 13 seconds \\
    \hline
  \end{tabular}
\end{center}

For more information, see \texttt{http://tycho.usno.navy.mil/leapsec.html}
and also the comments in \texttt{UTCtime.c}.

The integer number of seconds between the Unix epoch and the GPS epoch is
315964811: 8 years, 2 leap years, 5 days, and $11 = (19.0 - 8.0)$ leap
seconds between 1970-01-01 00:00:00 and 1980-01-06 00:00:00.  So,
converting from GPS to Unix epoch and vice versa requires adding or
subtracting 315964811 seconds.

\subsection{Accuracy}

Strictly speaking, the amount of time between the Unix epoch and the GPS
epoch is 315964810.999918 seconds.

\subsection{Tests}
\label{sec:utogpsc:tests}

\subsection{Uses}
\label{sec:utogpsc:uses}


\subsection{Notes}
\label{sec:utogpsc:notes}


%%% Local Variables: 
%%% mode: latex
%%% TeX-master: "Date"
%%% End: 
