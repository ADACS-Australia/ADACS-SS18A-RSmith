%%% $Id$
\section{Module \texttt{Utime.c}}

\subsection{Prototypes}
\begin{verbatim}
void
Utime (Status              *status,
       LALDate             *utc,
       const LIGOTimeUnix  *unixtime)
\end{verbatim}

\subsection{Description}

This routine converts (UTC) time in a \texttt{LIGOTimeUnix} structure
measured in the Unix epoch to the same time in a \texttt{LALDate}
structure, corrected for leap seconds.

\subsection{Operating Instructions}

Here is a simple example (adapted from GRASP) to form a timestamp string:

\begin{verbatim}
/* Taken from GRASP */
void printone(Status *status, const LIGOTimeUnix *time1)
{
    LIGOTimeGPS  gpstime;
    LALDate      laldate;
    LIGOTimeUnix tmp;
    LALUnixDate  utimestruct;
    CHARVector *utc = NULL;

    INITSTATUS (status, "printone", TESTUTOGPSC);

    /*
     * Allocate space for CHARVectors
     */
    CHARCreateVector(status, &utc, (UINT4)64);

    /* compute GPS time */
    UtoGPS(status, &gpstime, time1);

    /*
     * construct strings with appropriate time stamps
     */
    /* Utime() */
    Utime(status, &laldate, time1);
    DateString(status, utc, &laldate);

    printf("%s\n", utc->data);

    /*
     * House cleaning
     */
    CHARDestroyVector(status, &utc);
    
    RETURN (status);
}
\end{verbatim}

\subsubsection{Arguments}

\paragraph{\texttt{Utime}}

\subparagraph{Inputs}

\begin{itemize}
    \item \texttt{unixtime}: A pointer to a \texttt{LIGOTimeUnix} structure
    containing the time elapsed since the Unix epoch (1970-Jan-01 0h).
\end{itemize}

\subparagraph{Outputs}

\begin{itemize}
    \item \texttt{utc}: A pointer to an \texttt{LALDate} structure that
    contains the date and time UTC.
\end{itemize}

\subsubsection{Error Conditions}
\begin{tabular}{|c|l|l|}
  \hline
  Status & Status       & Explanation \\
  code   & description  &             \\
  \hline
  \tt 1  & \tt Null pointer & Pointer to input data is NULL. \\
  \tt 2  & \tt Input time out of range & $0 \leqslant \textrm{input} \leqslant 946684823$ \\
  \hline
\end{tabular}

\subsection{Algorithm}


\subsection{Accuracy}

\subsection{Tests}

\subsection{Uses}


\subsection{Notes}
\label{sec:utimec:notes}



%%% Local Variables: 
%%% mode: latex
%%% TeX-master: "Date"
%%% End: 
