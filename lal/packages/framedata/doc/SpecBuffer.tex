\documentclass{article}
\begin{document}
\section{SpecBuffer}

\subsection{Purpose}

Generates and stores a buffer of spectra and returns an average spectrum.

\subsection{Synopsis}

\begin{verbatim}
typedef struct
tagComputeSpectrumPar
{
  WindowType   windowType;
  REAL4Vector *window;
  REAL4        wss;
  RealFFTPlan *plan;
}
ComputeSpectrumPar;
typedef struct

tagSpectrumBufferPar
{
  INT4         numSpec;
  INT4         numPoints;
  WindowType   windowType;
  RealFFTPlan *plan;
}
SpectrumBufferPar;

typedef struct
tagSpectrumBuffer
{
  /* ... */
}
SpectrumBuffer;


void
ComputeSpectrum (
    Status               *status,
    REAL4FrequencySeries *spectrum,
    INT2TimeSeries       *timeSeries,
    ComputeSpectrumPar   *parameters
    );

void
CreateSpectrumBuffer (
    Status             *status,
    SpectrumBuffer    **buffer,
    SpectrumBufferPar  *params
    );

void
DestroySpectrumBuffer (
    Status          *status,
    SpectrumBuffer **buffer
    );

void
AddSpectrum (
    Status         *status,
    SpectrumBuffer *specBuffer,
    INT2TimeSeries *timeSeries
    );

void
AverageSpectrum (
    Status               *status,
    REAL4FrequencySeries *spectrum,
    SpectrumBuffer       *buffer
    );
\end{verbatim}

\subsection{Description}

The routine \texttt{ComputeSpectrum()} computes the spectrum of a
\texttt{INT2TimeSeries} to produce an \texttt{REAL4FrequencySeries} power
spectrum.  It requires parameters including the forward real FFT plan and an
appropriate window.  This routine is used by the routine
\texttt{AddSpectrum()}.

The routine \texttt{CreateSpectrumBuffer()} creates a spectrum buffer.  This
requires a parameter input that specifies the number of points in each time
series that will be used to construct the spectra, the number of spectra to
store, the window type to be used when constructing the spectra, and the real
FFT plan to be used.  The spectrum buffer is destroyed using the routine
\texttt{DestroySpectrumBuffer()}.

The routine \texttt{AddSpectrum()} adds the spectrum of time series data to a
spectrum buffer.  The routine \texttt{AverageSpectrum()} outputs the power
spectrum frequency series consisting of the average of the (most recent)
spectra that have been stored so far.

\subsection{Operating Instructions}

\begin{verbatim}
const  INT4                  numSpec   = 8;
const  INT4                  numPoints = 1024;
static Status                status;
static SpectrumBufferPar     buffPar;
static SpectrumBuffer        buff;
static INT2TimeSeries        data;
static REAL4FrequencySeries  spec;

INT i;

buffPar.numSpec    = numSpec;
buffPar.numPoints  = numPoints;
buffPar.windowType = Welch;
EstimateFwdRealFFTPlan (&status, &buffPar.plan, numPoints);
CreateSpectrumBuffer   (&status, &buff, &buffPar);
I2CreateVector         (&status, &data.data, numPoints);
SCreateVector          (&status, &spec.data, numPoints/2 + 1);

for (i = 0; i < numSpec; ++i)
{

  /* get data here; break out if end of data */

  AddSpectrum (&status, buff, &data);
}

AverageSpectrum (&status, &spec, buff);

SDestroyVector        (&status, &spec.data);
I2DestroyVector       (&status, &data.data);
DestroySpectrumBuffer (&status, &buff);
DestroyRealFFTPlan    (&status, &buffPar.plan);
\end{verbatim}

\subsubsection{Arguments}

\begin{itemize}
\item \texttt{status} is a universal status structure.  Its contents are
assigned by the functions.
\item \texttt{buffPar} are the parameters required for creating the spectrum
buffer.
\item \texttt{buff} is the spectrum buffer.
\item \texttt{data} is the data time series.
\item \texttt{spec} is the power spectrum frequency series.
\end{itemize}

\subsubsection{Options}

\subsubsection{Error conditions}

These functions all set the universal status structure on return.
Error conditions are described in the following table.

\begin{table}
\begin{tabular}{|r|l|p{2in}|}\hline
status  & status          & Description\\
code    & description     & \\\hline
SPECBUFFER\_ENULL 1  & Null pointer & \\
SPECBUFFER\_ENNUL 2  & Non-null pointer & \\
SPECBUFFER\_ESIZE 4  & Invalid input size & \\
SPECBUFFER\_ESZMM 8  & Size mismatch & \\
SPECBUFFER\_EZERO 16 & Zero divide & \\
SPECBUFFER\_ENONE 32 & No stored spectra & \\
\hline
\end{tabular}
\caption{Error conditions for SpecBuffer functions}\label{tbl:CV}
\end{table}

\subsection{Algorithms}

\subsection{Accuracy}

\subsection{Tests}

The program \texttt{SpecBufferTest} reads and (optionally) outputs frame data
and computes an average spectrum using these routines.  The environment
variable \texttt{LAL\_FRAME\_PATH} should be set to the directory path
containing the frame data.  The option \texttt{-h} gives a list of options.

\subsection{Uses}

\begin{itemize}
\item\texttt{CreateVector()}
\item\texttt{RealPowerSpectrum()}
\item\texttt{DestroyVector()}
\item\texttt{Window()}
\end{itemize}

\subsection{Notes}

\subsection{References}

\end{document}


