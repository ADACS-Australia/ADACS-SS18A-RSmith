\chapter{Package \texttt{burstsearch}}

Functions for the implementation of the standard burst searches:

\begin{itemize}

\item A standard interface for burst event trigger generators.

\item A function for the estimation of burst parameters.

\item A wrapper for the TFCLUSTERS algorithm.

\item A wrapper for the SLOPE algorithm.

\item A set of functions to implement the excess power search technique which was suggested in Ref.~\cite{fh:1998} and later independently invented in Ref.~\cite{acdhp:1999}.  The implementation here is described in detail in Ref.~\cite{abcf:2000}. 

\end{itemize}

\newpage
%%%%%%%%%%%%%%%%%%%%%%%%%%%%%%%%%%%%%%%%%%%%%%%%%%%%%%%%%%%%%%%%%%%%%%%%%%%
\input{StdBurstSearchH}
\newpage
\input{StdBurstSearchC}
\newpage
\input{TFClustersETGC}
\newpage
\input{SlopeETGC}
%%%%%%%%%%%%%%%%%%%%%%%%%%%%%%%%%%%%%%%%%%%%%%%%%%%%%%%%%%%%%%%%%%%%%%%%%%%


\newpage
%%%%%%%%%%%%%%%%%%%%%%%%%%%%%%%%%%%%%%%%%%%%%%%%%%%%%%%%%%%%%%%%%%%%%%%%%%%
\section{Header \texttt{TFTransform.h}}
\label{s:TFTransform.h}
%%%%%%%%%%%%%%%%%%%%%%%%%%%%%%%%%%%%%%%%%%%%%%%%%%%%%%%%%%%%%%%%%%%%%%%%%%%

\noindent Provides routines to to compute time-frequency planes from either
time-domain or frequency-domain data, for use in the excess
power search technique.

\subsection*{Synopsis}
\begin{verbatim}
#include "TFTransform.h"
\end{verbatim}

\noindent This package provides a suite for functions for computing time-frequency
representations of data using stacked Fourier transforms.

The first few functions simply combine functionality from the packages
\verb+fft+ and \verb+Window+ in a convenient way.  They are designed to
streamline the task of setting up structures to prepare for taking many
discrete Fourier transforms (DFTs), including windows and plans for FFTW.

A general description of the time-frequency (TF) transform provided by
TFTransform is as follows.  Suppose one starts with some data $h_j$, $0 \le j
< n$ in the time domain, with sampling time $\Delta t$, so that the data point
$h_j$ corresponds to a time $t_j = t_{\rm start} + j \Delta t$.  Taking the
standard DFT yields complex data
\begin{equation}
{\tilde h}_\gamma = \sum_{j=0}^{n-1} \, e^{-2 \pi i j \gamma / n} \, h_j
\label{standarddft}
\end{equation}
in the Fourier domain, for $0 \le \gamma \le [n/2]+1$.  Here the data point
${\tilde h}_\gamma$ corresponds to a frequency $f_\gamma = \gamma \Delta f$,
where $\Delta f= 1/(n \Delta t)$ is the frequency resolution.  


Now suppose that we can factorize the number $n$ of data points as
\begin{equation}
n = 2 N_T N_F.
\end{equation}
Then, by a time-frequency plane we shall mean a set of $N_T N_F$ complex
numbers $H_{I\Gamma}$ with $0 \le I < N_T$ and $0 \le \Gamma < N_F$, obtained
by an invertible linear transformation from the original data, such  that the
data point $H_{I\Gamma}$ corresponds approximately to a time $t_I = t_{\rm
start} + I {\overline {\Delta t}}$ and to a frequency $f_\Gamma = \Gamma
{\overline {\Delta f}}$.  Here $N_F$ is the number of frequency bins in the TF
plane, and $N_T$ is the number of time bins.  The time resolution ${\overline
{\Delta t}}$ and frequency resolution ${\overline {\Delta f}}$ are related by
${\overline {\Delta t}} \ {\overline {\Delta f}} =1$, and are given by
${\overline {\Delta t}} = 2 N_F \Delta t$ and ${\overline {\Delta f}} = N_T
\Delta f$.  Note that there are many other time-frequency representations
of data that are not of this type; see \cite{ab:1999}.


There are many possible choices of linear transformations from the data $h_j$
to data $H_{J\Gamma}$ satisfying the above properties.  Here we have
implemented two simple choices.  The first choice consists of dividing the
time-domain data $h_j$ into $N_T$ equal-sized chunks, each of length $n/N_T$,
and then taking the forward DFT of each chunk.  Then, $H_{J\Gamma}$ is just
the $\Gamma$th element of the $J$th chunk.  In terms of formulae this
corresponds to
\begin{equation}
H_{J\Sigma} = \sum_{k=0}^{2 N_F-1} \, \exp \left[ 2 \pi i k \Sigma / (2
N_F) \right] \, h_{2 N_F J + k},
\label{verticalTFP}
\end{equation}
for $0 \le J < N_T$ and $0 \le \Sigma < N_F$.  We call this first type
of TF plane a vertical TF plane, since it corresponds to a series of
vertical lines if the time axis is horizontal and the frequency axis
vertical. 

The second type of TF plane is obtained by first taking a DFT of all the time
domain data to obtain frequency domain data, then dividing the frequency
domain data into $N_F$ equal-sized chunks, then taking the inverse DFT of each
chunk.  We call the resulting TF plane a horizontal TF plane. In terms of
formulae the TF plane elements are \begin{equation} H_{J\Sigma} =
\sum_{\gamma=0}^{N_T-1} \, \exp \left[ -2 \pi i J \gamma / N_T \right] \,
{\tilde h}_{N_T \Sigma + \gamma}, \label{horizontalTFP} \end{equation} for $0
\le J < N_T$ and $0 \le \Sigma < N_F$, where ${\tilde h}_\gamma$ is given by
Eq.\ (\ref{standarddft}).


\subsection*{Structures}

%%%%%%%%%%%%%%%%%%%%%%%%%%%%%%%%%%%%%%%%%%%%%%%%%%%%%%%%%%%%%%%%%%%%%%%%%%
\subsubsection*{struct \texttt{TFPlaneParams}}
\idx[Type]{TFPlaneParams}
%%%%%%%%%%%%%%%%%%%%%%%%%%%%%%%%%%%%%%%%%%%%%%%%%%%%%%%%%%%%%%%%%%%%%%%%%%

\noindent Parameters needed to describe a particular TF plane.

\begin{description}
\item[\texttt{INT4 timeBins}] Number of time bins $N_T$ in TF plane. 

\item[\texttt{INT4 freqBins}] Number of freq bins $N_F$ in TF plane.

\item[\texttt{REAL8 deltaT}] The time resolution ${\overline {\Delta t}}$ 
of the TF plane in seconds, \texttt{deltaF} will always be 1/\texttt{deltaT}.

\item[\texttt{REAL8 flow}] The lowest frequency $f_{\rm low}$ in the TF plane
in Hertz [such that the data point $H_{J\Gamma}$ corresponds to a time $t_J =
t_{\rm start} + J {\overline {\Delta t}}$ and to a frequency $f_\Gamma =
f_{\rm low} + \Gamma {\overline {\Delta f}}$, in a slight generalization of
the above correspondence].
\end{description}

%%%%%%%%%%%%%%%%%%%%%%%%%%%%%%%%%%%%%%%%%%%%%%%%%%%%%%%%%%%%%%%%%%%%%%%%%%
\subsubsection*{struct \texttt{RealDFTParams}}
\idx[Type]{RealDFTParams}
%%%%%%%%%%%%%%%%%%%%%%%%%%%%%%%%%%%%%%%%%%%%%%%%%%%%%%%%%%%%%%%%%%%%%%%%%%

\noindent 

\begin{description}
\item[\texttt{WindowType windowType}]
\item[\texttt{REAL4Vector *window}]
\item[\texttt{REAL4 sumofsquares}]
\item[\texttt{RealFFTPlan *plan}]
\end{description}

%%%%%%%%%%%%%%%%%%%%%%%%%%%%%%%%%%%%%%%%%%%%%%%%%%%%%%%%%%%%%%%%%%%%%%%%%%
\subsubsection*{struct \texttt{ComplexDFTParams}}
\idx[Type]{ComplexDFTParams}
%%%%%%%%%%%%%%%%%%%%%%%%%%%%%%%%%%%%%%%%%%%%%%%%%%%%%%%%%%%%%%%%%%%%%%%%%%

\noindent 

\begin{description}
\item[\texttt{WindowType  windowType}]
\item[\texttt{REAL4Vector   *window}]
\item[\texttt{REAL4  sumofsquares}]
\item[\texttt{ComplexFFTPlan  *plan}]
\end{description}

%%%%%%%%%%%%%%%%%%%%%%%%%%%%%%%%%%%%%%%%%%%%%%%%%%%%%%%%%%%%%%%%%%%%%%%%%%
\subsubsection*{struct \texttt{COMPLEX8TimeFrequencyPlane}}
\idx[Type]{COMPLEX8TimeFrequencyPlane}
%%%%%%%%%%%%%%%%%%%%%%%%%%%%%%%%%%%%%%%%%%%%%%%%%%%%%%%%%%%%%%%%%%%%%%%%%%

\noindent 
This structure has some fields that also appear in the structures
\verb+REAL4TimeSeries+ and \verb+COMPLEX8FrequencySeries+.

\begin{description}
\item[\texttt{CHAR  *name}] The name of the TF plane.

\item[\texttt{LIGOTimeGPS   epoch}] The initial time $t_{\rm start}$ of the
data used to generate the TF plane.

\item[\texttt{CHARVector  *sampleUnits}] The units of the quantities $H_{J\Gamma}$.

\item[\texttt{TFPlaneParams   *params}]  Parameters needed to generate the
plane from input data.  (See above.)

\item[\texttt{TFPlaneType   planeType}]  This is an enumerated type that can
be either \verb+verticalPlane+ or \verb+horizontalPlane+, corresponding to the
two types of TF plane. 

\item[\texttt{COMPLEX8   *data}]  The $N_T \times N_F$ array of complex
numbers $H_{J\Sigma}$.

\end{description}

%%%%%%%%%%%%%%%%%%%%%%%%%%%%%%%%%%%%%%%%%%%%%%%%%%%%%%%%%%%%%%%%%%%%%%%%%%
\subsubsection*{struct \texttt{VerticalTFTransformIn}}
\idx[Type]{VerticalTFTransformIn}
%%%%%%%%%%%%%%%%%%%%%%%%%%%%%%%%%%%%%%%%%%%%%%%%%%%%%%%%%%%%%%%%%%%%%%%%%%

\noindent 

\begin{description}
\item[\texttt{RealDFTParams  *dftParams}]
\item[\texttt{INT4  startT}]
\end{description}

%%%%%%%%%%%%%%%%%%%%%%%%%%%%%%%%%%%%%%%%%%%%%%%%%%%%%%%%%%%%%%%%%%%%%%%%%%
\subsubsection*{struct \texttt{HorizontalTFTransformIn}}
\idx[Type]{HorizontalTFTransformIn}
%%%%%%%%%%%%%%%%%%%%%%%%%%%%%%%%%%%%%%%%%%%%%%%%%%%%%%%%%%%%%%%%%%%%%%%%%%

\noindent 

\begin{description}
\item[\texttt{ComplexDFTParams  *dftParams}]
\item[\texttt{INT4  startT}]
\end{description}

\newpage
%%%%%%%%%%%%%%%%%%%%%%%%%%%%%%%%%%%%%%%%%%%%%%%%%%%%%%%%%%%%%%%%%%%%%%%%%%%%%
\subsection{Module \texttt{CreateRealDFTParams.c}}
\label{ss:CreateRealDFTParams.c}
%%%%%%%%%%%%%%%%%%%%%%%%%%%%%%%%%%%%%%%%%%%%%%%%%%%%%%%%%%%%%%%%%%%%%%%%%%%%%

Creates a structure of type \verb+RealDFTParams+,

\subsubsection*{Prototypes}
\vspace{0.1in}
%\input{CreateRealDFTParamsCP}
\idx{LALCreateRealDFTParams()}

\subsubsection*{Description}

The inputs to \verb+CreateDFTParams()+ consist of (i) a parameter
\verb+winParams+ of type \verb+WindowParams*+ giving the length of the vectors
to be Fourier  transformed and the type of windowing to be used, (ii) a
pointer \verb+dftParams+ to a pointer to a \verb+RealDFTParams+ structure, and
(iii) an integer \verb+sign+ specifying the direction of the transform, with
$+1$ indicating forward transform and $-1$ indicating inverse transform.  On
exit, \verb+*dftParams+ will point to the newly created structure.

\subsubsection*{Uses}
\begin{verbatim}
LALCreateForwardRealFFTPlan
LALCreateReverseRealFFTPlan
LALSCreateVector
LALWindow 
\end{verbatim}

\subsubsection*{Notes}

%\vfill{\footnotesize\input{CreateRealDFTParamsCV}}

\newpage
%%%%%%%%%%%%%%%%%%%%%%%%%%%%%%%%%%%%%%%%%%%%%%%%%%%%%%%%%%%%%%%%%%%%%%%%%%%%%
\subsection{Module \texttt{DestroyRealDFTParams.c}}
\label{ss:DestroyRealDFTParams.c}
%%%%%%%%%%%%%%%%%%%%%%%%%%%%%%%%%%%%%%%%%%%%%%%%%%%%%%%%%%%%%%%%%%%%%%%%%%%%%

Destroys a structure of type \verb+RealDFTParams+,

\subsubsection*{Prototypes}
\vspace{0.1in}
%\input{DestroyRealDFTParamsCP}
\idx{LALDestroyRealDFTParams()}

\subsubsection*{Description}

On exit, \verb+*dftParams+ will be NULL.

\subsubsection*{Uses}
\begin{verbatim}
LALSDestroyVector
LALDestroyRealFFTPlan
LALFree
\end{verbatim}

\subsubsection*{Notes}

%\vfill{\footnotesize\input{DestroyRealDFTParamsCV}}

\newpage
%%%%%%%%%%%%%%%%%%%%%%%%%%%%%%%%%%%%%%%%%%%%%%%%%%%%%%%%%%%%%%%%%%%%%%%%%%%%%
\subsection{Module \texttt{ComputeFrequencySeries.c}}
\label{ss:ComputeFrequencySeries.c}
%%%%%%%%%%%%%%%%%%%%%%%%%%%%%%%%%%%%%%%%%%%%%%%%%%%%%%%%%%%%%%%%%%%%%%%%%%%%%

It Fourier transforms the time series.

\subsubsection*{Prototypes}
\vspace{0.1in}
\input{ComputeFrequencySeriesCP}
\idx{LALComputeFrequencySeries()}

\subsubsection*{Description}

The routine \verb+ComputeFrequencySeries()+ takes as input a
\verb+REAL4TimeSeries+ and a structure \verb+RealDFTParams+.  It Fourier
transforms the time series according to the information (window and FFT plan)
in the \verb+RealDFTParams+ structure, using \verb+FwdRealFFT()+.  It then
multiplies the frequency domain data by a normalization factor which is
$1/\sqrt{n}$ for rectangular windowing, where $n$ is the length of the time
series, and returns the data as a \verb+COMPLEX8FrequencySeries+.  The length
of the frequency series is $[n/2]+1$, where $[n/2]$ means $n/2$ rounded down
to the nearest integer.  Appropriate values for the various fields in the
\verb+COMPLEX8FrequencySeries+ structure (\verb+deltaF+ etc.~) are computed
from the corresponding fields in the input \verb+REAL4TimeSeries+. 

\subsubsection*{Uses}
\begin{verbatim}
LALCreateVector
LALForwardRealFFT
LALDestroyVector
\end{verbatim}

\subsubsection*{Notes}

\vfill{\footnotesize\input{ComputeFrequencySeriesCV}}

\newpage
%%%%%%%%%%%%%%%%%%%%%%%%%%%%%%%%%%%%%%%%%%%%%%%%%%%%%%%%%%%%%%%%%%%%%%%%%%%%%
\subsection{Module \texttt{CreateTFPlane.c}}
\label{ss:CreateTFPlane.c}
%%%%%%%%%%%%%%%%%%%%%%%%%%%%%%%%%%%%%%%%%%%%%%%%%%%%%%%%%%%%%%%%%%%%%%%%%%%%%

Allocates the memory for a TF plane.

\subsubsection*{Prototypes}
\vspace{0.1in}
\input{CreateTFPlaneCP}
\idx{LALCreateTFPlane()}

\subsubsection*{Description}

The routine \verb+CreateTFPlane()+ allocates the memory for a TF plane.  It
takes as input a pointer to a structure \verb+TFPlaneParams+.  The
\verb+CreateTFPlane()+ routine returns a pointer to a structure
\verb+COMPLEX8TimeFrequencyPlane+.    This routine allocates memory for all of
these fields, and copies into the \verb+params+ structure the values of $N_F$,
$N_T$, ${\overline {\Delta t}}$ and $f_{\rm low}$ from its input structure.
It assigns nominal values to all the other fields in the
\verb+COMPLEX8TimeFrequencyPlane+ structure. 

\subsubsection*{Uses}
\begin{verbatim}
LALMalloc
LALFree
\end{verbatim}

\subsubsection*{Notes}

\vfill{\footnotesize\input{CreateTFPlaneCV}}

\newpage
%%%%%%%%%%%%%%%%%%%%%%%%%%%%%%%%%%%%%%%%%%%%%%%%%%%%%%%%%%%%%%%%%%%%%%%%%%%%%
\subsection{Module \texttt{DestroyTFPlane.c}}
\label{ss:DestroyTFPlane.c}
%%%%%%%%%%%%%%%%%%%%%%%%%%%%%%%%%%%%%%%%%%%%%%%%%%%%%%%%%%%%%%%%%%%%%%%%%%%%%

Destroys a TF plane created with \texttt{CreateTFPlane()}.

\subsubsection*{Prototypes}
\vspace{0.1in}
\input{DestroyTFPlaneCP}
\idx{LALDestroyTFPlane()}

\subsubsection*{Description}

The routine \verb+DestroyTFPlane()+ frees all the memory associated with a
\verb+COMPLEX8FrequencyPlane+ and destroys the TF plane.

\subsubsection*{Uses}
\begin{verbatim}
LALFree
\end{verbatim}

\subsubsection*{Notes}

\vfill{\footnotesize\input{DestroyTFPlaneCV}}

\newpage
%%%%%%%%%%%%%%%%%%%%%%%%%%%%%%%%%%%%%%%%%%%%%%%%%%%%%%%%%%%%%%%%%%%%%%%%%%%%%
\subsection{Module \texttt{FreqSeriesToTFPlane.c}}
\label{ss:FreqSeriesToTFPlane.c}
%%%%%%%%%%%%%%%%%%%%%%%%%%%%%%%%%%%%%%%%%%%%%%%%%%%%%%%%%%%%%%%%%%%%%%%%%%%%%

Computes a TF plane from an input \verb+COMPLEX8FrequencySeries+.

\subsubsection*{Prototypes}
\vspace{0.1in}
\input{FreqSeriesToTFPlaneCP}
\idx{LALFreqSeriesToTFPlane()}

\subsubsection*{Description}

The routine \verb+FreqSeriesToTFPlane()+ computes a TF plane from an input
\verb+COMPLEX8FrequencySeries+.  It takes as input a structure \verb+input+ of
type \verb+HorizontalTFTransformIn+ which consists of a structure
\verb+dftParams+ of type \verb+ComplexDFTParams+ and an integer \verb+startT+,
which is not used in the current implementation.  Another argument is a
pointer \verb+freqSeries+ to the \verb+COMPLEX8FrequencySeries+.  Lastly there
is a pointer \verb+tfp+ to a \verb+COMPLEX8FrequencyPlane+ which as before is
both an input and an output; the fields $N_T$, $N_F$, ${\overline {\Delta t}}$
and $f_{\rm low}$ are inputs and the remaining fields are outputs.


The input parameters to the routine \verb+FreqSeriesToTFPlane()+ are the
parameters $N_T$, $N_F$, ${\overline {\Delta t}}$ and $f_{\rm low}$ of the TF
plane (from \verb+tfp->params+), and also the parameters $\Delta f$ and $f_0$
of the input frequency series.  The routine takes the data in the frequency
domain starting at frequency $f_{\rm low} - f_0$, breaks it up into $N_F$
chunks of length ${\tilde N}_T$, and performs an inverse DFT on each chunk.
The chunk size ${\tilde N}_T$ is given by
\begin{equation}
{\tilde N}_T =  1 /( \Delta f  {\overline {\Delta t}}),
\end{equation}
where the right hand side is rounded off to the nearest integer.  This choice
of chunk-size is necessary so that the time resolution of the TF plane is
${\overline {\Delta t}}$.  After computing the inverse DFT of each chunk, the
first $N_T$ time bins out of the ${\tilde N}_T$ time bins are retained and
stored, and the remainder are discarded.  The input parameters must be such
that $N_T \le {\tilde N}_T$.  The input frequency series can extend beyond the
highest frequency used which is $f_{\rm low} - f_0 + (N_F {\tilde N}_T
-1){\Delta f}$.  The TF plane elements are computed according to the formula
(\ref{horizontalTFP}), except that they are multiplied by a normalization
factor of $1/\sqrt{{\tilde N}_T}$.  [This normalization convention insures
that if the input data is white noise with rms power 1 in the real and
imaginary parts of each frequency bin, then the real and imaginary parts of
the data points in the TF plane are also independent Gaussian random variables
with rms value 1.]  

\subsubsection*{Uses}
\begin{verbatim}
LALCCreateVector
LALCOMPLEX8VectorFFT
LALCDestroyVector
\end{verbatim}

\subsubsection*{Notes}

\vfill{\footnotesize\input{FreqSeriesToTFPlaneCV}}
















\newpage
%%%%%%%%%%%%%%%%%%%%%%%%%%%%%%%%%%%%%%%%%%%%%%%%%%%%%%%%%%%%%%%%%%%%%%%%%%%
\section{Header \texttt{ExcessPower.h}}
\label{s:ExcessPower.h}
%%%%%%%%%%%%%%%%%%%%%%%%%%%%%%%%%%%%%%%%%%%%%%%%%%%%%%%%%%%%%%%%%%%%%%%%%%%

Provides routines to implement the excess power search method 
described in \cite{abcf:2000}. 

\subsection*{Synopsis}
\begin{verbatim}
#include "ExcessPower.h"
\end{verbatim}

This package provides a set of functions for implementing the excess power
search method described in \cite{abcf:2000}.  We refer to that paper for the
derivation of and motivation for the search method.  Here we give a brief
description of the method.

One starts with $N$ complex data points in the frequency domain, so the input
to the search method is a \verb+COMPLEX8FrequencySeries+.  The current
implementation requires that the input noise be white, and that the real and
imaginary parts of each frequency bin have rms value 1, so frequency domain
data should be multiplied by an appropriately normalized pre-whitening filter
before being input to the routines in the \verb+ExcessPower+ package.  The
implementation also requires that $N$ be a power of 2, so that $N =2^p$ for
some $p$.  For each integer $i$ with $0 \le i \le p$, one performs a
time-frequency transform of the data (see the package \verb+TFTransform+) to
obtain a time-frequency plane with $N_T$ time bins and $N_F$ frequency bins,
where
\begin{equation}
  N_T = 2^i \ \ \ , \ \ N_F = 2^{p-i}.
  \label{tfplanes}
\end{equation}
One thus constructs of order $\log_2 N$ different time-frequency planes. 

The $i$th time-frequency (TF) plane will consist of a $N_T \times N_F$ matrix
of complex numbers $H_{J\Gamma}$.  The time resolution ${\overline {\Delta
t}}^{(i)}$ and frequency resolution ${\overline {\Delta f}}^{(i)}$ of the
$i$th plane will be given by
\begin{equation}
  {\overline {\Delta t}}^{(i)} = 2^{-i} {\overline {\Delta t}}^{(0)} \ \
  \ , \ \ \ 
  {\overline {\Delta f}}^{(i)} = 2^{i} {\overline {\Delta f}}^{(0)}.
  \label{resolutions}
\end{equation}
We will call a submatrix of any one of these matrices a {\it time-frequency
tile}.  A tile will be specified by giving the TF plane $i$, and by giving
the initial and final times $J_1, J_2$ and initial and final frequencies
$\Sigma_1$, $\Sigma_2$.  We will label tiles by ${\cal T}$, thus 
\begin{equation}
  {\cal T} \equiv (i,J_1,J_2,\Sigma_1,\Sigma_2).
\end{equation}
The tile ${\cal T}$ will consist
of the complex numbers $H_{J\Sigma}$ with $J_1 \le J \le J_2$ and
$\Sigma_1 \le \Sigma \le \Sigma_2$, so there will be 
\begin{equation}
  N_{\cal T} = (\Delta J+1) (\Delta \Sigma+1)
\end{equation}
complex numbers in all, where $\Delta J = J_2 - J_1$, $\Delta \Sigma =
\Sigma_2 - \Sigma_1$. 


For any given time-frequency tile ${\cal T}$ one computes the total power
\begin{equation}
  P_{\cal T} = \sum_{J = J_1}^{J_2} \ \sum_{\Sigma=\Sigma_1}^{\Sigma_2}
  \ | H_{J\Sigma} |^2.
  \label{power}
\end{equation}
In the absence of a signal, $P_{\cal T}$ will be distributed a chi-squared
distribution with $2 N_{\cal T}$ degrees of freedom \cite{abcf:2000}, so we
can compute from $P_{\cal T}$ a probability
\begin{equation}
  \alpha_{\cal T} = 1 - p(P_{\cal T},2 N_{\cal T}).
  \label{alphadef}
\end{equation}
Here $p(\chi^2,n)$ is the cumulative distribution function for a chisquared
distribution, given by
\begin{equation}
  p(\chi^2,n) = {1 \over \Gamma(n/2)} \int_0^{\chi^2/2} dx \, x^{n/2-1} e^{-x};
\end{equation}
see the package \verb+Thresholds+.  We compute the right hand side of Eq.\
(\ref{alphadef}) using the routine \verb+OneMinusChisqCdf()+.  The expected
power for any tile is $\langle P_{\cal T} \rangle = 2 N_{\cal T}$, so the {\it
excess power} is
\begin{equation}
  \Delta P_{\cal T} = P_{\cal T} - 2 N_{\cal T}.
  \label{ep}
\end{equation}

The essential idea of the search method is to search through time frequency
tiles ${\cal T}$, and for each tile to compare the probability $\alpha_{\cal
T}$ to some threshold value $\alpha_*$; tiles with $\alpha_{\cal T} \le
\alpha_*$ will be selected as possible events.  Of course, the method will
zero in on non-Gaussian noise events in the data as well as on genuine
gravitational wave events; it is only when one coincidences between several
detectors (using future versions of this package) that one can hope to
detect genuine signals.  Thus, the current version of the package is useful
mostly for characterizing the non-Gaussian noise in detector data.

\subsubsection{Tricks for saving computation time}

It would be computationally prohibitive as well as unnecessary to compute
the probability $\alpha_{\cal T}$ for all possible TF tiles ${\cal
T}$.  The strategy adopted to reduce the computational burden is
twofold:  
\begin{itemize}
\item We compute the total power $P_{\cal T}$ only for a discrete set
${\cal T}_j$ of all possible tiles that samples sufficiently finely
the space of all possible tiles.
\item For each tile ${\cal T}_j$, we compute the probability
$\alpha_{\cal T}$ only if the quantity
\begin{equation}
n_{\cal T} \equiv {\Delta P_{\cal T} \over \sqrt{2 N_{\cal T}}}
\label{nsigma}
\end{equation}
satisfies 
\begin{equation}
n_{\cal T} \ge N_{\rm min},
\label{firstcut}
\end{equation}
where $N_{\rm min}$ is a fixed constant (eg 2 or 3).  The quantity $n_{\cal
T}$ is roughly the ``number of sigma'' for the detection, and is much less
costly to compute that $\alpha_{\cal T}$.  Since in realistic searches the
threshold $\alpha_*$ will be very small, all tiles that qualify as events
will have $n_{\cal T} \ge$ (a few), so computing $\alpha_{\cal T}$ only for
tiles with $n_{\cal T} \ge N_{\rm min}$ saves a lot of computation time.
[Note though that we cannot in general threshold on $n_{\cal T}$ as the
precise value of the threshold would depend on $N_{\cal T}$].
\end{itemize}

The algorithm used to select the set of tiles ${\cal T}_j$ is as
follows.  For each TF plane $i$, one loops through values of $J_1$
and $\Delta J$ with
\begin{equation}
0 \le J_1 \le J_1 + \Delta J < N_T,
\end{equation}
where the step size for both $J_1$ and $\Delta J$ is
\begin{equation}
\delta J = 1 + \left[\Delta J / N_{\rm ov} \right].
\end{equation}
Here the brackets mean ``round down to the nearest integer'', and
$N_{\rm ov}$ is a fixed integer of order a few.  The idea is that
successive tiles that differ only in their value of $J_1$ and have
identical values of $\Delta J$ (for example) can have a fractional
overlap as small as $N_{\rm ov} / (N_{\rm ov} + 1)$.  For large $N_T$,
the spacing of the tiles is exhaustive (all possible tiles covered)
for $\Delta J \le $ a few, while the spacing becomes uniform in $\log
(\Delta J)$ for large $\Delta J$.  Similarly, the set of frequencies
$\Sigma_1$ and $\Delta \Sigma$ is determined by 
looping through values of $\Sigma_1$
and $\Delta \Sigma$ with
\begin{equation}
0 \le \Sigma_1 \le \Sigma_1 + \Delta \Sigma < N_F,
\end{equation}
where the step size for both $\Sigma_1$ and $\Delta \Sigma$ is
\begin{equation}
\delta \Sigma = 1 + \left[\Delta \Sigma / N_{\rm ov} \right].
\end{equation}
More detailed explanation of and justification for this method of
sampling the space of TF tiles can be found in \cite{abcf:2000}.
The number $N_{\rm tiles}$ of tiles generated with $N_{\rm ov}=3$ (for
example) is
approximately 
\begin{equation}
N_{\rm tiles} \sim 100 N
\end{equation}
for large $N$, where $N$ is the original number of data points in the
frequency domain.  [By
contrast, an exhaustive search over all possible tiles would give
$N_{\rm tiles} \sim N^2$ for large $N$, neglecting logarithmic factors.]


\subsubsection{The two search methods provided}
\label{twomethods}

This package provides two types of search routines:
\begin{itemize}
\item Routines for which the user specifies the threshold $\alpha_*$.
Then, the routine searches through all tiles ${\cal T}_j$ in the
discrete set discussed above, and returns a list of tiles (candidate
events) for which $\alpha_{\cal T} \le \alpha_*$, sorted in order of
probability.  A drawback of this
approach is that the appropriate value of the threshold $\alpha_*$
depends on how 
statistically independent the individual statistics $\alpha_{\cal T}$
are, and also on how many tiles are being computed.  The correct way
to determine the value of the threshold $\alpha_*$ would be to perform
Monte Carlo simulations.  However, one would need to repeat the Monte
Carlo simulations whenever one changed the search parameters (for
example, the total number of points in the frequency domain, the
number of tiles generated etc.)

\item A more powerful search method is to compute an average ${\bar
\Lambda}$ of the quantity 
\begin{equation}
\Lambda_{\cal T} \equiv {1 \over \alpha_{\cal T}}
\end{equation}
over the space of all TF tiles, and to threshold on the average value
${\bar \Lambda}$.  We will refer to ${\bar \Lambda}$ as the mean likelihood.
Motivation for this can be found in Ref.\ \cite{abcf:2000}.
Data segments for which ${\bar \Lambda} \le {\bar
\Lambda}_*$ are then rejected, while data segments for which
${\bar \Lambda} \ge {\bar \Lambda}_*$ are tagged for further
analysis.  Here ${\bar \Lambda}_*$ is a threshold.  An approximate but
sufficiently accurate formula for the average is \cite{abcf:2000}
\begin{equation}
{\bar \Lambda} = {1 \over N_{\rm tiles}} \, \sum_j \left[{2 N_{{\cal
T}_j} \over (\Delta P_{{\cal T}_j})^2 } \right] \, {1 \over
\alpha_{{\cal T}_j}},
\label{meanlikelihood}
\end{equation}
where $N_{\rm tiles}$ is the total number of tiles in the sum.  Under
suitable assumed prior probability for signals (roughly speaking,
``uniform'' in each tile and ``uniformly distributed'' over all tiles)  
, it can be shown that the probability $p$ that a signal is present in
the data is related to the prior probability $p_0$ that a signal is
present by \cite{abcf:2000}
\begin{equation}
{p \over 1 - p} = {\bar \Lambda} \ {p_0 \over 1 - p_0}.
\label{ff1}
\end{equation}
Thus the user can specify a desired value of $p$ for a given $p_0$, or
equivalently can specify ${\bar \Lambda}$.  

\end{itemize}

In the second method, {\it if} the noise were purely Gaussian and
stationary, the threshold ${\bar \Lambda}_*$ could be prescribed 
without any Monte Carlo simulations, with the (Bayesian)
interpretation being given by Eq.\ (\ref{ff1}).  However, due to the
ubiquitous presence of nonGaussian noise, it will be necessary to
resort to Monte Carlo simulations to determine a (frequentist)
threshold ${\bar \Lambda}_*$.  Even so, the second method will have
significant advantages over the first method \cite{abcf:2000}. 
 


%%%%%%%%%%%%%%%%%%%%%%%%%%%%%%%%%%%%%%%%%%%%%%%%%%%%%%%%%%%%%%%%%%%%%%%%%%
\subsection*{Error Conditions}
\input{ExcessPowerHErrTab}
%%%%%%%%%%%%%%%%%%%%%%%%%%%%%%%%%%%%%%%%%%%%%%%%%%%%%%%%%%%%%%%%%%%%%%%%%%

\subsection*{Structures}

%%%%%%%%%%%%%%%%%%%%%%%%%%%%%%%%%%%%%%%%%%%%%%%%%%%%%%%%%%%%%%%%%%%%%%%%%%
\subsubsection*{struct \texttt{TFTile}}
\idx[Type]{TFTile}
%%%%%%%%%%%%%%%%%%%%%%%%%%%%%%%%%%%%%%%%%%%%%%%%%%%%%%%%%%%%%%%%%%%%%%%%%%

\noindent This structure contains the parameters needed to describe a
particular tile over the time-frequency volume to be searched.  Each
\verb+TFTile+ will be the element of a linked list. 

\begin{description}
\item[\texttt{INT4 fstart}] First frequency bin in the tile denoted by
$J_1$ above.

\item[\texttt{INT4 fend}] Last frequency bin in the tile denoted by $J_2$
above.

\item[\texttt{INT4 tstart}] First time bin in the tile denoted by $\Sigma_1$
above.

\item[\texttt{INT4 tend}] Last time bin in the tile denoted by $\Sigma_2$
above.

\item[\texttt{INT4 whichPlane}] The time-frequency plane to which the tile
     belongs.

\item[\texttt{REAL8 excessPower}] The excess power $\Delta P_{\cal T}$ computed 
for this time-frequency tile.

\item[\texttt{REAL8 alpha}] The measure $\alpha_{\cal T}$ of the confidence
associated with the power in this tile.

\item[\texttt{REAL8 weight}] A real number which will be used in future
versions of \verb+ExcessPower+ to compute the average (\ref{meanlikelihood}) 
over all TF tiles more accurately.

\item[\texttt{BOOLEAN firstCutFlag}] A boolean flag \verb+firstCutFlag+ which 
is set to \verb+TRUE+ if the tile satisfies the criterion (\ref{firstcut}) 
and thus passes the ``first cut'' in the data analysis.  

\item[\texttt{struct tagTFTile *nextTile}] Pointer to the next tile in the
     linked list.
\end{description}

%%%%%%%%%%%%%%%%%%%%%%%%%%%%%%%%%%%%%%%%%%%%%%%%%%%%%%%%%%%%%%%%%%%%%%%%%%
\subsubsection*{struct \texttt{TFTiling}}
\idx[Type]{TFTiling}
%%%%%%%%%%%%%%%%%%%%%%%%%%%%%%%%%%%%%%%%%%%%%%%%%%%%%%%%%%%%%%%%%%%%%%%%%%

\noindent This structure contains all the information needed to perform an
excess power search:  a list of all tiles,  pointers to the time-frequency
planes contructed from the input data stream,  and flags to indicate what
stage of the search has been reached.

\begin{description}
\item[\texttt{TFTile *firstTile}] Pointer to the first tile in the linked list
of tiles to be searched.  The \verb+TFTile+ structure describing
time-frequency tiles is described above.

\item[\texttt{INT4 numTiles}] Total number of tiles in the linked list pointed
to by \texttt{*firstTile}.

\item[\texttt{COMPLEX8TimeFrequencyPlane **tfp}] The quantities *(tfp+i) for $0 \le 
i < p$ constitute a vector of pointers to the $p+1$ frequency planes.

\item[\texttt{ComplexDFTParams **dftParams}] The quantities *(dftParams+i) for $0
\le i < p$ constitute a vector of pointers to the $p+1$
\verb+ComplexDFTParams+ structures (see package \verb+TFTransform+) used to
compute the corresponding TF planes.  The idea is that these structures are
set up once initially and then used to analyze many segments of data.

\item[\texttt{INT4 numPlanes}] An integer \verb+numPlanes+ giving the total 
number $p+1$ of time-frequency planes.

\item[\texttt{BOOLEAN planesComputed}] Flag set to 1 if the time-frequency
planes have been computed.

\item[\texttt{BOOLEAN excessPowerComputed}] Flag set to 1 if the excess-power
has been computed for each of the tiles and time-frequency planes.

\item[\texttt{BOOLEAN tilesSorted}] Flag set to 1 if the tiles have been
sorted from most significant to least significant.
\end{description}

%%%%%%%%%%%%%%%%%%%%%%%%%%%%%%%%%%%%%%%%%%%%%%%%%%%%%%%%%%%%%%%%%%%%%%%%%%
\subsubsection*{struct \texttt{CreateTFTilingIn}}
\idx[Type]{CreateTFTilingIn}
%%%%%%%%%%%%%%%%%%%%%%%%%%%%%%%%%%%%%%%%%%%%%%%%%%%%%%%%%%%%%%%%%%%%%%%%%%

\begin{description}
\item[\texttt{INT4 overlapFactor}] The number of bins to overlap neighboring
time-frequency tiles to reduce computational cost which was denoted $N_{\rm
ov}$ above.

\item[\texttt{INT4 minFreqBins}] The smallest extent in frequency of TF tiles
to be searched over.  Only tiles with $\Delta \Sigma +1$ larger than this
parameter will be included in the set of tiles generated.

\item[\texttt{INT4 minTimeBins}] The smallest extent in time of TF tiles
to be searched over.  Only tiles with $\Delta J +1$ larger than this
parameter will be included in the set of tiles generated.

\item[\texttt{REAL8 flow}] The lowest frequency in Hz to be searched.

\item[\texttt{REAL8 deltaF}] A real number giving the frequency resolution of
the first (trivial) TF plane, for which the number $N_T$ of time bins is $1$.

\item[\texttt{INT4 length}] An integer \verb+length+ giving the length $N_F$
of the first (trivial) TF plane, for which the number $N_T$ of time bins is
$1$.
\end{description}

%%%%%%%%%%%%%%%%%%%%%%%%%%%%%%%%%%%%%%%%%%%%%%%%%%%%%%%%%%%%%%%%%%%%%%%%%%
\subsubsection*{struct \texttt{ComputeExcessPowerIn}}
\idx[Type]{ComputeExcessPowerIn}
%%%%%%%%%%%%%%%%%%%%%%%%%%%%%%%%%%%%%%%%%%%%%%%%%%%%%%%%%%%%%%%%%%%%%%%%%%

\begin{description}
\item[\texttt{REAL8 numSigmaMin}] An integer $N_{\rm min}$,
the minimum ``number of sigma'' in the first cut criterion
(\ref{firstcut}). 

\item[\texttt{REAL8 alphaDefault}] The value assigned to $\alpha_{\rm
default}$. It should be of order unity, since the tiles will eventually be
sorted (see below) in order of their $\alpha_{\cal T}$ values, and the
interesting tiles will have very small values of $\alpha_{\cal T}$.  Aside
from this, the exact value assigned to $\alpha_{\rm default}$  is unimportant.  
\end{description}

\newpage
%%%%%%%%%%%%%%%%%%%%%%%%%%%%%%%%%%%%%%%%%%%%%%%%%%%%%%%%%%%%%%%%%%%%%%%%%%%%%
\subsection{Module \texttt{LALAddWhiteNoise.c}}
\label{ss:LALAddWhiteNoise.c}
%%%%%%%%%%%%%%%%%%%%%%%%%%%%%%%%%%%%%%%%%%%%%%%%%%%%%%%%%%%%%%%%%%%%%%%%%%%%%

Adds white Gaussian noise to a \verb+COMPLEX8Vector+.  

\subsubsection*{Prototypes}
\vspace{0.1in}
\input{AddWhiteNoiseCP}
\idx{LALAddWhiteNoise()}

\subsubsection*{Description}
It takes as an input parameter a real number \verb+noiseLevel+, and an
input/output parameter \verb+v+ which is pointer to a \verb+COMPLEX8Vector+.
On exit from the routine, the vector has been augmented by a random complex
vector of which the real and imaginary parts of each element are independent,
zero-mean Gaussian deviates with rms value \verb+noiseLevel+.

\subsubsection*{Uses}
\begin{verbatim}
LALCreateRandomParams
LALSCreateVector
LALNormalDeviates
LALSDestroyVector
LALDestroyRandomParams
\end{verbatim}

\subsubsection*{Notes}

\vfill{\footnotesize\input{AddWhiteNoiseCV}}

\newpage
%%%%%%%%%%%%%%%%%%%%%%%%%%%%%%%%%%%%%%%%%%%%%%%%%%%%%%%%%%%%%%%%%%%%%%%%%%%%%
\subsection{Module \texttt{LALCreateTFTiling.c}}
\label{ss:LALCreateTFTiling.c}
%%%%%%%%%%%%%%%%%%%%%%%%%%%%%%%%%%%%%%%%%%%%%%%%%%%%%%%%%%%%%%%%%%%%%%%%%%%%%

Create a data structure of type \verb+TFTiling+ (time-frequency tiling).

\subsubsection*{Prototypes}
\vspace{0.1in}
\input{CreateTFTilingCP}
\idx{LALCreateTFTiling()}

\subsubsection*{Description}

The routine \verb+LALCreateTFTiling()+ takes as input a pointer \verb+input+
to a data structure of type \verb+CreateTFTilingIn+, which consists of the
following elements.

The routine \verb+CreateTFTiling()+ has an input/output parameter
\verb+tfTiling+ of type \verb+TFTiling**+.  On exit from
\verb+CreateTFTiling+, the pointer \verb+*tfTiling+ will point to the newly
created \verb+TFTiling+ structure.  The routine \verb+CreateTFTiling+ also
assigns memory for and creates the $p+1$ time frequency planes and the $p+1$
\verb+ComplexDFTParams+ structures, and the linked list of TF tiles.

\subsubsection*{Uses}
\begin{verbatim}
LALMalloc
LALFree
LALCreateTFPlane
LALCreateComplexDFTParams
\end{verbatim}

\subsubsection*{Notes}
A failure to allocate memory for some element of the linked list of tiles
causes \verb+LALCreateTFTiling+ to abort with error code
\verb+EXCESSPOWERH_ETILES+.  A developer should call \verb+LALDestroyTFTiling+
to free the partially allocated list.   

\vfill{\footnotesize\input{CreateTFTilingCV}}

\newpage
%%%%%%%%%%%%%%%%%%%%%%%%%%%%%%%%%%%%%%%%%%%%%%%%%%%%%%%%%%%%%%%%%%%%%%%%%%%%%
\subsection{Module \texttt{LALDestroyTFTiling.c}}
\label{ss:LALDestroyTFTiling.c}
%%%%%%%%%%%%%%%%%%%%%%%%%%%%%%%%%%%%%%%%%%%%%%%%%%%%%%%%%%%%%%%%%%%%%%%%%%%%%

Destroy a data structure of type \verb+TFTiling+ (time-frequency tiling).

\subsubsection*{Prototypes}
\vspace{0.1in}
\input{DestroyTFTilingCP}
\idx{LALDestroyTFTiling()}

\subsubsection*{Description}

When \verb+DestroyTFTiling()+ is called, \verb+*tfTiling+ should point to the
structure to be destroyed, and on exit \verb+*tfTiling+ will be \verb+NULL+.

\subsubsection*{Uses}
\begin{verbatim}
LALMalloc
LALFree
LALDestroyTFPlane
LALDestroyComplexDFTParams
\end{verbatim}
\subsubsection*{Notes}

\vfill{\footnotesize\input{DestroyTFTilingCV}}

\newpage
%%%%%%%%%%%%%%%%%%%%%%%%%%%%%%%%%%%%%%%%%%%%%%%%%%%%%%%%%%%%%%%%%%%%%%%%%%%%%
\subsection{Module \texttt{LALComputeTFPlanes.c}}
\label{ss:LALComputeTFPlanes.c}
%%%%%%%%%%%%%%%%%%%%%%%%%%%%%%%%%%%%%%%%%%%%%%%%%%%%%%%%%%%%%%%%%%%%%%%%%%%%%

Computes the $p+1$ time frequency transforms according to the parameters
(\ref{tfplanes}) and (\ref{resolutions})

\subsubsection*{Prototypes}
\vspace{0.1in}
\input{ComputeTFPlanesCP}
\idx{LALComputeTFPlanes()}

\subsubsection*{Description}

It takes as input a pointer \verb+freqSeries+ to a
\verb+COMPLEX8FrequencySeries+, and as input/output a pointer \verb+tfTiling+
to a \verb+TFTiling+ structure.  It computes the $p+1$ time frequency
transforms according to the parameters (\ref{tfplanes}) and
(\ref{resolutions}), using the routine \verb+FreqSeriesToTFPlane()+, and
stores the resulting TF planes in the structure \verb+*tfTiling+.

\subsubsection*{Uses}
\begin{verbatim}
LALFreqSeriesToTFPlane
\end{verbatim}
\subsubsection*{Notes}

\vfill{\footnotesize\input{ComputeTFPlanesCV}}

\newpage
%%%%%%%%%%%%%%%%%%%%%%%%%%%%%%%%%%%%%%%%%%%%%%%%%%%%%%%%%%%%%%%%%%%%%%%%%%%%%
\subsection{Module \texttt{LALComputeExcessPower.c}}
\label{ss:LALComputeExcessPower.c}
%%%%%%%%%%%%%%%%%%%%%%%%%%%%%%%%%%%%%%%%%%%%%%%%%%%%%%%%%%%%%%%%%%%%%%%%%%%%%

Computes the excess power $\Delta P_{\cal T}$ for each tile.

\subsubsection*{Prototypes}
\vspace{0.1in}
\input{ComputeExcessPowerCP}
\idx{LALComputeExcessPower()}

\subsubsection*{Description}

This routine computes the excess power $\Delta
P_{\cal T}$ for each tile, according to Eqs.\ (\ref{power}) and (\ref{ep}).
It also computes for each tile the ``number of sigma'' parameter $n_{\cal T}$
given by Eq.\ (\ref{nsigma}).  For those tiles which satisfy the first cut
criterion (\ref{firstcut}), the routine computes the probability $\alpha_{\cal
T}$ given by Eq.\ (\ref{alphadef}) and stores it in the field \verb+alpha+ of
the TF tile.  For tiles which fail the first cut criterion, the field
\verb+alpha+ is assigned a default value $\alpha_{\rm default}$.  The
arguments of the routine are (i) an input/output argument consisting of a
pointer \verb+tfTiling+ to a \verb+TFTiling+ structure, and (ii) a pointer
\verb+input+ to a data structure of type \verb+ComputeExcessPowerIn+.  The
routine \verb+ComputeExcessPower()+ must be called after calling
\verb+ComputeTFPlanes()+; calling the routines out of sequence will give rise
to an error.

\subsubsection*{Uses}
\begin{verbatim}
LALOneMinusChisqCdf
\end{verbatim}

\subsubsection*{Notes}

\vfill{\footnotesize\input{ComputeExcessPowerCV}}

\newpage
%%%%%%%%%%%%%%%%%%%%%%%%%%%%%%%%%%%%%%%%%%%%%%%%%%%%%%%%%%%%%%%%%%%%%%%%%%%%%
\subsection{Module \texttt{LALSortTFTiling.c}}
\label{ss:LALSortTFTiling.c}
%%%%%%%%%%%%%%%%%%%%%%%%%%%%%%%%%%%%%%%%%%%%%%%%%%%%%%%%%%%%%%%%%%%%%%%%%%%%%

Sorts the linked list of TF tiles in a \verb+TFTiling+ structure is
sorted in order of increasing $\alpha_{\cal T}$.

\subsubsection*{Prototypes}
\vspace{0.1in}
\input{SortTFTilingCP}
\idx{LALSortTFTiling()}

\subsubsection*{Description}

The linked list of TF tiles in a \verb+TFTiling+ structure is sorted in order
of increasing $\alpha_{\cal T}$ by the routine \verb+SortTFTiling()+, which
has as an input/output argument a pointer \verb+tfTiling+ to the
\verb+TFTiling+ structure.  After the routine \verb+SortTFTiling()+ is called,
the linked list has as its first element the tile with the lowest value of
$\alpha_{\cal T}$, and consequently the highest probability of being an event.
The routine \verb+SortTFTiling()+ must be called after calling
\verb+ComputeExcessPower()+; calling the routines out of sequence will give
rise to an error.

\subsubsection*{Uses}
\begin{verbatim}
LALMalloc
qsort
LALFree
\end{verbatim}

\subsubsection*{Notes}

\vfill{\footnotesize\input{SortTFTilingCV}}

\newpage
%%%%%%%%%%%%%%%%%%%%%%%%%%%%%%%%%%%%%%%%%%%%%%%%%%%%%%%%%%%%%%%%%%%%%%%%%%%%%
\subsection{Module \texttt{LALPrintTFTileList.c}}
\label{ss:LALPrintTFTileList.c}
%%%%%%%%%%%%%%%%%%%%%%%%%%%%%%%%%%%%%%%%%%%%%%%%%%%%%%%%%%%%%%%%%%%%%%%%%%%%%

Outputs the details of a specified number of TF tiles to a file.

\subsubsection*{Prototypes}
\vspace{0.1in}
\input{PrintTFTileListCP}
\idx{LALPrintTFTileList()}

\subsubsection*{Description}

The routine \verb+PrintTFTileList()+ outputs the details of a specified number
of TF tiles to a file.  Its input parameters are (i) an integer
\verb+maxTiles+ giving the maximum number of tiles to be output, (ii) a
pointer \verb+tfTiling+ to a \verb+TFTiling+ structure, and (iii) a pointer
\verb+fp+ to a \verb+FILE+.  Information on each of the first \verb+maxTiles+
tiles in the linked list are then output to the file in a text format.  The
information includes:
\begin{itemize}

\item The index $i$ of the time-frequency plane
to which the tile belongs.

\item The tile's frequency and time intervals, both in absolute units
(Hertz and seconds) and in bins on the time frequency plane.

\item The number of degrees of freedom $2 N_{\cal T}$.

\item The excess power $\Delta P_{\cal T}$.

\item The probability $\alpha_{\cal T}$.  

\item The ``effective number of sigma'', $z$, corresponding to the
probability $\alpha_{\cal T}$, defined by ${\rm erfc}(z/\sqrt{2}) =
\alpha_{\cal T}$.  This is useful for comparisons with matched
filtering signal-to-noise ratios.

\end{itemize}

By calling this routine after calling \verb+SortTFTiling()+, and by
passing to it the parameter $n_{\rm events}$ obtained from the routine
\verb+CountEPEvents()+, one can obtain an ordered list of all tiles
passing a specified threshold $\alpha_*$.

\subsubsection*{Uses}
\begin{verbatim}
LALChi2Threshold 
\end{verbatim}

\subsubsection*{Notes}

\vfill{\footnotesize\input{PrintTFTileListCV}}

\newpage
%%%%%%%%%%%%%%%%%%%%%%%%%%%%%%%%%%%%%%%%%%%%%%%%%%%%%%%%%%%%%%%%%%%%%%%%%%%%%
\subsection{Module \texttt{LALComputeLikelihood.c}}
\label{ss:LALComputeLikelihood.c}
%%%%%%%%%%%%%%%%%%%%%%%%%%%%%%%%%%%%%%%%%%%%%%%%%%%%%%%%%%%%%%%%%%%%%%%%%%%%%

Computes the mean likelihood given in Eq.\ (\ref{meanlikelihood}).

\subsubsection*{Prototypes}
\vspace{0.1in}
\input{ComputeLikelihoodCP}
\idx{LALComputeLikelihood()}

\subsubsection*{Description}

The last routine \verb+ComputeLikelihood()+ computes the mean likelihood given
in Eq.\ (\ref{meanlikelihood}).  It takes as input a pointer \verb+tfTiling+
to a \verb+TFTiling+ structure, and returns the mean likelihood ${\bar
\Lambda}$ in the output parameter \verb+lambda+ of type \verb+REAL8*+.  It can
only be called after calling \verb+ComputeExcessPower()+.  The routine
actually returns not the value (\ref{meanlikelihood}) but instead the more
approximate quantity  
\begin{equation}
{\bar \Lambda} = {1 \over N_{\rm tiles}} \, \sum_j^\prime \left[{2 N_{{\cal
T}_j} \over (\Delta P_{{\cal T}_j})^2 } \right] \, {1 \over
\alpha_{{\cal T}_j}},
\label{meanlikelihood1}
\end{equation}
Here the notation $\sum^\prime$ means that the sum is taken only over those
tiles that satisfy the first cut condition (\ref{firstcut}).  The reason that
this approximation is justified is that reasonable thresholds ${\bar
\Lambda}_*$ will be very large compared to unity, and the contribution of
tiles with $n_{\cal T} \le N_{\rm min}$ to ${\bar \Lambda}$ is bounded above
by a number of order unity.  In other words, tiles for which the probability
$\alpha_{\cal T}$ is of order unity will contribute negligibly to the final
value of the mean likelihood ${\bar \Lambda}$ for those data segments which
exceed the threshold.  So, we can omit such tiles in computing the sum.
[Clearly it does not matter if we compute ${\bar \Lambda}$ somewhat
inaccurately for ${\bar \Lambda} \ll {\bar \Lambda}_*$.] 

The routine \verb+ComputeLikelihood()+ must be called after calling
\verb+ComputeExcessPower()+; calling the routines out of sequence will give
rise to an error.  However, it does not require a prior execution of
\verb+SortTFTiling()+ or of \verb+CountEPEvents()+.  

The routine \verb+ComputeLikelihood()+ allows the user to the second type of
search discussed in Sec.\ \ref{twomethods} above.  Namely, one can compute
${\bar \Lambda}$ for successive segments of data in the time domain [whose
fractional overlap could be taken to be $N_{\rm ov} / (N_{\rm ov} +1)$ say].
Only those data segments with ${\bar \Lambda}$ above threshold are tagged for
further analysis.  In this further analysis, one can examine the first few
elements of the sorted linked list to find the most probable time-frequency
windows for a putative event.  Or, in the likely scenario where one is using
the \verb+ExcessPower+ routines to help characterize the non-Gaussian noise in
detector data, one can amass statistics of the ``loudest'' TF tiles over many
data segments, which could yield information as to the typical time-frequency
characteristics of the most common/loudest non-Gaussian noise events.

\subsubsection*{Uses}

\subsubsection*{Notes}

\vfill{\footnotesize\input{ComputeLikelihoodCV}}


%
%
%
% EPSEARCH
%
%
%


\newpage
%%%%%%%%%%%%%%%%%%%%%%%%%%%%%%%%%%%%%%%%%%%%%%%%%%%%%%%%%%%%%%%%%%%%%%%%%%%
\section{Header \texttt{EPSearch.h}}
\label{s:EPSearch.h}
%%%%%%%%%%%%%%%%%%%%%%%%%%%%%%%%%%%%%%%%%%%%%%%%%%%%%%%%%%%%%%%%%%%%%%%%%%%

\noindent Provides routines which put the power search code together.

\subsection*{Synopsis}
\begin{verbatim}
#include "EPSearch.h"
\end{verbatim}

\noindent This package provides a suite for functions to perform a
power code search of the \texttt{REAL4} time series data.

\subsection*{Structures}

%%%%%%%%%%%%%%%%%%%%%%%%%%%%%%%%%%%%%%%%%%%%%%%%%%%%%%%%%%%%%%%%%%%%%%%%%%
\subsubsection*{struct \texttt{EPSearchParams}}
\idx[Type]{EPSearchParams}
%%%%%%%%%%%%%%%%%%%%%%%%%%%%%%%%%%%%%%%%%%%%%%%%%%%%%%%%%%%%%%%%%%%%%%%%%%

\noindent Parameters needed to set perform the search

\begin{verbatim}
typedef struct
tagEPSearchParams
{
        CHAR                 *channelName;
        BOOLEAN               printSpectrum;
        INT4                  eventLimit;
        UINT4                 windowLength;
        UINT4                 windowShift;
        REAL8                 alphaThreshold;
        AvgSpecMethod         method;
        CreateTFTilingIn      tfTilingInput;
        ComputeExcessPowerIn  compEPInput;
        WindowType            windowType;
}
EPSearchParams;
\end{verbatim}

\newpage
%%%%%%%%%%%%%%%%%%%%%%%%%%%%%%%%%%%%%%%%%%%%%%%%%%%%%%%%%%%%%%%%%%%%%%%%%%%%%
\subsection{Module \texttt{EPSearch.c}}
\label{ss:EPSearch.c}
%%%%%%%%%%%%%%%%%%%%%%%%%%%%%%%%%%%%%%%%%%%%%%%%%%%%%%%%%%%%%%%%%%%%%%%%%%%%%

This package provides a suite for functions for implementing the
excess power code. 

\subsubsection*{Prototypes}
\vspace{0.1in}
\input{EPConditionDataCP}
\idx{EPConditionData()}
\input{EPSearchCP}
\idx{EPSearch()}

\subsubsection*{Description}

The function \texttt{EPConditionData()} takes a frequency series as input
and performs some data conditioning operations on it including optionally
down-sampling the data, and high-pass filtering it.  This function is
typically always called prior to the call(s) to \texttt{EPSearch()}.

The routine \texttt{EPSearch()} is the powerhouse of the excess-power code.


\subsubsection*{Notes}

\begin{enumerate}
\item These functions were originally in-line code in LALWrapper.  This
made for very inconvenient use of the EPSearch in LALApps.  The
implementation in LAL,  which leads to yet another layer of translation in
LALWrapper,  is much more convenient to maintain and test.
\end{enumerate}

\subsubsection*{Uses}
\begin{verbatim}
Lots o' Stuff:  coming soon
\end{verbatim}
\subsubsection*{Notes}

\vfill{\footnotesize\input{EPSearchCV}}



%
%
%
%
%  THRESHOLDSH
%
%
%
%




\newpage
%%%%%%%%%%%%%%%%%%%%%%%%%%%%%%%%%%%%%%%%%%%%%%%%%%%%%%%%%%%%%%%%%%%%%%%%%%%
\section{Header \texttt{Thresholds.h}}
\label{s:Thresholds.h}
%%%%%%%%%%%%%%%%%%%%%%%%%%%%%%%%%%%%%%%%%%%%%%%%%%%%%%%%%%%%%%%%%%%%%%%%%%%

\noindent Provides routines to compute thresholds and quantities related to
thresholds for chi-squared and non-central chi-squared distributions
\cite{ptvf:1992}, for use in the excess power search technique.

\subsection*{Synopsis}
\begin{verbatim}
#include "Thresholds.h"
\end{verbatim}

\noindent This package provides a suite for functions for computing cumulative
probability functions and thresholds for the chi-squared and non-central
chi-squared distributions.

\subsection*{Structures}

%%%%%%%%%%%%%%%%%%%%%%%%%%%%%%%%%%%%%%%%%%%%%%%%%%%%%%%%%%%%%%%%%%%%%%%%%%
\subsubsection*{struct \texttt{ChisqCdfIn}}
\idx[Type]{ChisqCdfIn}
%%%%%%%%%%%%%%%%%%%%%%%%%%%%%%%%%%%%%%%%%%%%%%%%%%%%%%%%%%%%%%%%%%%%%%%%%%

\noindent Parameters needed to described a particular TF plane.

\begin{description}
\item[\texttt{REAL8     chi2}] value of $\chi^2$
\item[\texttt{REAL8     dof}]  number of degrees of freedom
\item[\texttt{REAL8     nonCentral}]  non-centrality parameter
\end{description}

%%%%%%%%%%%%%%%%%%%%%%%%%%%%%%%%%%%%%%%%%%%%%%%%%%%%%%%%%%%%%%%%%%%%%%%%%%
\subsubsection*{struct \texttt{Chi2ThresholdIn}}
\idx[Type]{Chi2ThresholdIn}
%%%%%%%%%%%%%%%%%%%%%%%%%%%%%%%%%%%%%%%%%%%%%%%%%%%%%%%%%%%%%%%%%%%%%%%%%%

\noindent Parameters needed to described a particular TF plane.

\begin{description}
\item[\texttt{REAL8     dof}]   number of degrees of freedom 
\item[\texttt{REAL8     falseAlarm}]  false alarm probability 
\end{description}

%%%%%%%%%%%%%%%%%%%%%%%%%%%%%%%%%%%%%%%%%%%%%%%%%%%%%%%%%%%%%%%%%%%%%%%%%%
\subsubsection*{struct \texttt{RhoThresholdIn}}
\idx[Type]{RhoThresholdIn}
%%%%%%%%%%%%%%%%%%%%%%%%%%%%%%%%%%%%%%%%%%%%%%%%%%%%%%%%%%%%%%%%%%%%%%%%%%

\noindent Parameters needed to described a particular TF plane.

\begin{description}
\item[\texttt{REAL8     chi2}]   value of chi squared  
\item[\texttt{REAL8     dof}]  number of degrees of freedom
\item[\texttt{REAL8     falseDismissal}] false dismissal probability 
\end{description}

\newpage
%%%%%%%%%%%%%%%%%%%%%%%%%%%%%%%%%%%%%%%%%%%%%%%%%%%%%%%%%%%%%%%%%%%%%%%%%%%%%
\subsection{Module \texttt{Thresholds.c}}
\label{ss:Thresholds.c}
%%%%%%%%%%%%%%%%%%%%%%%%%%%%%%%%%%%%%%%%%%%%%%%%%%%%%%%%%%%%%%%%%%%%%%%%%%%%%

This package provides a suite for functions for computing cumulative
probability functions and thresholds for the chi-squared and
non-central chi-squared distributions.

\subsubsection*{Prototypes}
\vspace{0.1in}
\input{ChisqCdfP}
\idx{LALChisqCdf()}
\input{OneMinusChisqCdfP}
\idx{LALOneMinusChisqCdf()}
\input{NoncChisqCdfP}
\idx{LALNoncChisqCdf()}
\input{Chi2ThresholdP}
\idx{LALChi2Threshold()}
\input{RhoThresholdP}
\idx{LALRhoThreshold()}

\subsubsection*{Description}

The routine \verb+ChisqCdf()+ computes as a function of $\chi^2$ the
probability $p = p(\chi^2,n)$ that $x_1^2 + x_2^2 + \ldots + x_n^2 \le \chi^2$,
where $x_1$, $x_2$, \ldots $x_n$ are independent Gaussian random
variables of zero mean and unit variance, and $n$ is the number of
degrees of freedom.  An integral expression for $p$ is
$$
p(\chi^2,n) = {1 \over \Gamma(n/2)} \int_0^{\chi^2/2} dx \, x^{n/2-1} e^{-x},
$$
where 
$$
\Gamma(n/2) \equiv \int_0^\infty dx \, x^{n/2-1} e^{-x}
$$
is the usual $\Gamma$ function.  \verb+ChisqCdf()+ takes as input a
structure \verb+ChisqCdfIn+ which has elements \verb+chi2+ or
$\chi^2$, \verb+dof+ or $n$, and \verb+nonCentral+ which is not used
by \verb+ChisqCdf()+.  The probability $p$ is returned by the routine 
in the output variable \verb+*prob+.


The routine \verb+OneMinusChisqCdf()+ takes the same arguments as
\verb+ChisqCdf()+ and computes $1-p$.  It is more accurate than
\verb+ChisqCdf()+ when $p$ is very close to $1$.  The quantity $1-p$
is returned in the output variable \verb+*prob+.


The routine \verb+NoncChisqCdf()+ takes the same arguments
\verb+ChisqCdf()+ and computes the non-central chi-squared
distribution.  It computes as a function of $\chi^2$ the
probability $p = p(\chi^2,n,\rho)$ that $(x_1+\rho)^2 + x_2^2 + \ldots
+ x_n^2 \le \chi^2$, 
where $x_1$, $x_2$, \ldots $x_n$ are independent Gaussian random
variables of zero mean and unit variance, $n$ is the number of
degrees of freedom, and $\rho^2$ is called the non-centrality
parameter.  It uses the series formula
$$
p(\chi^2,n,\rho) = \sum_{j=0}^\infty \, { \exp\left[ -{1 \over 2}
\rho^2 \right] \over j!} \ \left( {\rho^2 \over 2} \right)^j \
p(\chi^2,n+2j)
$$
and uses \verb+ChisqCdf()+.  The non-centrality parameter
$\rho^2$ is passed to \verb+NoncChisqCdf()+ through the element
\verb+nonCentral+ of the input structure \verb+ChisqCdfIn+.
Again $p$ is returned in the output variable \verb+*prob+.



The routine \verb+Chi2Threshold()+ returns the value $\chi^2 =
\chi^2(\alpha,n)$ of $\chi^2$ such that 
$$
\alpha = 1 - p(\chi^2,n),
$$
where $\alpha$ is a false alarm probability and $n$ is the number of
degrees of freedom.  \verb+Chi2Threshold()+ takes as input a
structure \verb+Chi2ThresholdIn+ which has elements \verb+dof+ or
$n$ and \verb+falseAlarm+ or $\alpha$.  The square $\chi^2$ of the
threshold $\chi$ is returned by the routine in the output variable
\verb+*chi2+. 




The routine \verb+RhoThreshold()+ returns the value $\rho =
\rho(\chi^2,n,\beta)$ of $\rho$ such that 
$$
\beta = p(\chi^2,n,\rho),
$$
where $\beta$ is a false dismissal probability and $n$ is the number
of degrees of freedom.  \verb+RhoThreshold()+ takes as input a
structure \verb+RhoThresholdIn+ which has elements \verb+chi2+ or
$\chi^2$, \verb+dof+ or $n$, and \verb+falseDismissal+ or $\beta$.
The result $\rho(\chi^2,n,\beta)$ is returned by the routine in the
output variable \verb+*rho+. 


Notes:
\begin{enumerate}
\item These are all double precision (REAL8) functions of double
precision variables.
\item The functions \verb+Chi2Threshold()+ and \verb+RhoThreshold()+
use the routines \verb+BracketRootReal8()+ and
\verb+BisectionFindRootReal8()+. 
\item The definitions given above for $p(\chi^2,n)$ and
$p(\chi^2,n,\rho)$ are for integral $n$ only.  However,
the integral formulae serve to define these quantities for all
positive real $n$, and the routines work for non-integral $n$.
\end{enumerate}

\subsubsection*{Uses}
\begin{verbatim}
LALDBracketRoot
LALDBisectionFindRoot

\end{verbatim}
\subsubsection*{Notes}

\vfill{\footnotesize\input{ThresholdsCV}}







\newpage
%%%%%%%%%%%%%%%%%%%%%%%%%%%%%%%%%%%%%%%%%%%%%%%%%%%%%%%%%%%%%%%%%%%%%%%%%%%%%
\begin{thebibliography}{0}
%%%%%%%%%%%%%%%%%%%%%%%%%%%%%%%%%%%%%%%%%%%%%%%%%%%%%%%%%%%%%%%%%%%%%%%%%%%%%
\bibitem{ab:1999}
Warren G. Anderson and R. Balasubramanian, \textit{Time-frequency
detection of gravitational waves}, gr-qc/9905023, and references
therein. 

\bibitem{fh:1998}
\'Eanna~\'E.~Flanagan and S. A. Hughes, Phys. Rev. D {\bf 57},
4535-4565 (1998) (also gr-qc/9701039).  

\bibitem{acdhp:1999}
N. Arnaud, F. Cavalier, M. Davier, P. Hello, and T. Pradier,
\textit{Triggers for the Detection of Gravitational Wave Bursts},
gr-qc/9903035.  

\bibitem{abcf:2000}
Warren G. Anderson, Patrick R. Brady, Jolien D. E. Creighton and 
{\'E}anna {\'E}. Flanagan, 
\textit{An excess power statistic for detection of burst sources of
gravitational radiation}, PRD 63, 042003 (2001) (gr-qc/0008066).
For a short version see {\it ibid}, \textit{A power filter for the
detection of burst sources of gravitational radiation in
interferometric detectors}, gr-qc/0001044, submitted to
Int. J. Modern. Phys. D.

\end{thebibliography}
