\chapter{Package \texttt{burstsearch}}

Functions for the implementation of the standard burst searches:

\begin{itemize}

\item A standard interface for burst event trigger generators.

\item A function for the estimation of burst parameters.

\item A wrapper for the TFCLUSTERS algorithm.

\item A wrapper for the SLOPE algorithm.

\item A set of functions to implement the excess power search technique which was suggested in Ref.~\cite{fh:1998} and later independently invented in Ref.~\cite{acdhp:1999}.  The implementation here is described in detail in Ref.~\cite{abcf:2000}.

\end{itemize}

\newpage
%%%%%%%%%%%%%%%%%%%%%%%%%%%%%%%%%%%%%%%%%%%%%%%%%%%%%%%%%%%%%%%%%%%%%%%%%%%
\section{Header \texttt{TFTransform.h}}
\label{s:TFTransform.h}
%%%%%%%%%%%%%%%%%%%%%%%%%%%%%%%%%%%%%%%%%%%%%%%%%%%%%%%%%%%%%%%%%%%%%%%%%%%

\noindent Provides routines to to compute time-frequency planes from either
time-domain or frequency-domain data, for use in the excess
power search technique.

\subsection*{Synopsis}
\begin{verbatim}
#include "TFTransform.h"
\end{verbatim}

\noindent This package provides a suite for functions for computing time-frequency
representations of data using stacked Fourier transforms.

The first few functions simply combine functionality from the packages
\verb+fft+ and \verb+Window+ in a convenient way.  They are designed to
streamline the task of setting up structures to prepare for taking many
discrete Fourier transforms (DFTs), including windows and plans for FFTW.

A general description of the time-frequency (TF) transform provided by
TFTransform is as follows.  Suppose one starts with some data $h_j$, $0 \le j
< n$ in the time domain, with sampling time $\Delta t$, so that the data point
$h_j$ corresponds to a time $t_j = t_{\rm start} + j \Delta t$.  Taking the
standard DFT yields complex data
\begin{equation}
{\tilde h}_\gamma = \sum_{j=0}^{n-1} \, e^{-2 \pi i j \gamma / n} \, h_j
\label{standarddft}
\end{equation}
in the Fourier domain, for $0 \le \gamma \le [n/2]+1$.  Here the data point
${\tilde h}_\gamma$ corresponds to a frequency $f_\gamma = \gamma \Delta f$,
where $\Delta f= 1/(n \Delta t)$ is the frequency resolution.


Now suppose that we can factorize the number $n$ of data points as
\begin{equation}
n = 2 N_T N_F.
\end{equation}
Then, by a time-frequency plane we shall mean a set of $N_T N_F$ complex
numbers $H_{I\Gamma}$ with $0 \le I < N_T$ and $0 \le \Gamma < N_F$, obtained
by an invertible linear transformation from the original data, such  that the
data point $H_{I\Gamma}$ corresponds approximately to a time $t_I = t_{\rm
start} + I {\overline {\Delta t}}$ and to a frequency $f_\Gamma = \Gamma
{\overline {\Delta f}}$.  Here $N_F$ is the number of frequency bins in the TF
plane, and $N_T$ is the number of time bins.  The time resolution ${\overline
{\Delta t}}$ and frequency resolution ${\overline {\Delta f}}$ are related by
${\overline {\Delta t}} \ {\overline {\Delta f}} =1$, and are given by
${\overline {\Delta t}} = 2 N_F \Delta t$ and ${\overline {\Delta f}} = N_T
\Delta f$.  Note that there are many other time-frequency representations
of data that are not of this type; see \cite{ab:1999}.


There are many possible choices of linear transformations from the data $h_j$
to data $H_{J\Gamma}$ satisfying the above properties.  Here we have
implemented two simple choices.  The first choice consists of dividing the
time-domain data $h_j$ into $N_T$ equal-sized chunks, each of length $n/N_T$,
and then taking the forward DFT of each chunk.  Then, $H_{J\Gamma}$ is just
the $\Gamma$th element of the $J$th chunk.  In terms of formulae this
corresponds to
\begin{equation}
H_{J\Sigma} = \sum_{k=0}^{2 N_F-1} \, \exp \left[ 2 \pi i k \Sigma / (2
N_F) \right] \, h_{2 N_F J + k},
\label{verticalTFP}
\end{equation}
for $0 \le J < N_T$ and $0 \le \Sigma < N_F$.  We call this first type
of TF plane a vertical TF plane, since it corresponds to a series of
vertical lines if the time axis is horizontal and the frequency axis
vertical.

The second type of TF plane is obtained by first taking a DFT of all the time
domain data to obtain frequency domain data, then dividing the frequency
domain data into $N_F$ equal-sized chunks, then taking the inverse DFT of each
chunk.  We call the resulting TF plane a horizontal TF plane. In terms of
formulae the TF plane elements are \begin{equation} H_{J\Sigma} =
\sum_{\gamma=0}^{N_T-1} \, \exp \left[ -2 \pi i J \gamma / N_T \right] \,
{\tilde h}_{N_T \Sigma + \gamma}, \label{horizontalTFP} \end{equation} for $0
\le J < N_T$ and $0 \le \Sigma < N_F$, where ${\tilde h}_\gamma$ is given by
Eq.\ (\ref{standarddft}).


\subsection*{Structures}

%%%%%%%%%%%%%%%%%%%%%%%%%%%%%%%%%%%%%%%%%%%%%%%%%%%%%%%%%%%%%%%%%%%%%%%%%%
\subsubsection*{struct \texttt{TFPlaneParams}}
\idx[Type]{TFPlaneParams}
%%%%%%%%%%%%%%%%%%%%%%%%%%%%%%%%%%%%%%%%%%%%%%%%%%%%%%%%%%%%%%%%%%%%%%%%%%

\noindent Parameters needed to describe a particular TF plane.

\begin{description}
\item[\texttt{INT4 timeBins}] Number of time bins $N_T$ in TF plane.

\item[\texttt{INT4 freqBins}] Number of freq bins $N_F$ in TF plane.

\item[\texttt{REAL8 deltaT}] The time resolution ${\overline {\Delta t}}$
of the TF plane in seconds, \texttt{deltaF} will always be 1/\texttt{deltaT}.

\item[\texttt{REAL8 flow}] The lowest frequency $f_{\rm low}$ in the TF plane
in Hertz [such that the data point $H_{J\Gamma}$ corresponds to a time $t_J =
t_{\rm start} + J {\overline {\Delta t}}$ and to a frequency $f_\Gamma =
f_{\rm low} + \Gamma {\overline {\Delta f}}$, in a slight generalization of
the above correspondence].
\end{description}

%%%%%%%%%%%%%%%%%%%%%%%%%%%%%%%%%%%%%%%%%%%%%%%%%%%%%%%%%%%%%%%%%%%%%%%%%%
\subsubsection*{struct \texttt{RealDFTParams}}
\idx[Type]{RealDFTParams}
%%%%%%%%%%%%%%%%%%%%%%%%%%%%%%%%%%%%%%%%%%%%%%%%%%%%%%%%%%%%%%%%%%%%%%%%%%

\noindent

\begin{description}
\item[\texttt{WindowType windowType}]
\item[\texttt{REAL4Vector *window}]
\item[\texttt{REAL4 sumofsquares}]
\item[\texttt{RealFFTPlan *plan}]
\end{description}

%%%%%%%%%%%%%%%%%%%%%%%%%%%%%%%%%%%%%%%%%%%%%%%%%%%%%%%%%%%%%%%%%%%%%%%%%%
\subsubsection*{struct \texttt{ComplexDFTParams}}
\idx[Type]{ComplexDFTParams}
%%%%%%%%%%%%%%%%%%%%%%%%%%%%%%%%%%%%%%%%%%%%%%%%%%%%%%%%%%%%%%%%%%%%%%%%%%

\noindent

\begin{description}
\item[\texttt{WindowType  windowType}]
\item[\texttt{REAL4Vector   *window}]
\item[\texttt{REAL4  sumofsquares}]
\item[\texttt{ComplexFFTPlan  *plan}]
\end{description}

%%%%%%%%%%%%%%%%%%%%%%%%%%%%%%%%%%%%%%%%%%%%%%%%%%%%%%%%%%%%%%%%%%%%%%%%%%
\subsubsection*{struct \texttt{COMPLEX8TimeFrequencyPlane}}
\idx[Type]{COMPLEX8TimeFrequencyPlane}
%%%%%%%%%%%%%%%%%%%%%%%%%%%%%%%%%%%%%%%%%%%%%%%%%%%%%%%%%%%%%%%%%%%%%%%%%%

\noindent
This structure has some fields that also appear in the structures
\verb+REAL4TimeSeries+ and \verb+COMPLEX8FrequencySeries+.

\begin{description}
\item[\texttt{CHAR  *name}] The name of the TF plane.

\item[\texttt{LIGOTimeGPS   epoch}] The initial time $t_{\rm start}$ of the
data used to generate the TF plane.

\item[\texttt{CHARVector  *sampleUnits}] The units of the quantities $H_{J\Gamma}$.

\item[\texttt{TFPlaneParams   *params}]  Parameters needed to generate the
plane from input data.  (See above.)

\item[\texttt{TFPlaneType   planeType}]  This is an enumerated type that can
be either \verb+verticalPlane+ or \verb+horizontalPlane+, corresponding to the
two types of TF plane.

\item[\texttt{COMPLEX8   *data}]  The $N_T \times N_F$ array of complex
numbers $H_{J\Sigma}$.

\end{description}

%%%%%%%%%%%%%%%%%%%%%%%%%%%%%%%%%%%%%%%%%%%%%%%%%%%%%%%%%%%%%%%%%%%%%%%%%%
\subsubsection*{struct \texttt{VerticalTFTransformIn}}
\idx[Type]{VerticalTFTransformIn}
%%%%%%%%%%%%%%%%%%%%%%%%%%%%%%%%%%%%%%%%%%%%%%%%%%%%%%%%%%%%%%%%%%%%%%%%%%

\noindent

\begin{description}
\item[\texttt{RealDFTParams  *dftParams}]
\item[\texttt{INT4  startT}]
\end{description}

%%%%%%%%%%%%%%%%%%%%%%%%%%%%%%%%%%%%%%%%%%%%%%%%%%%%%%%%%%%%%%%%%%%%%%%%%%
\subsubsection*{struct \texttt{HorizontalTFTransformIn}}
\idx[Type]{HorizontalTFTransformIn}
%%%%%%%%%%%%%%%%%%%%%%%%%%%%%%%%%%%%%%%%%%%%%%%%%%%%%%%%%%%%%%%%%%%%%%%%%%

\noindent

\begin{description}
\item[\texttt{ComplexDFTParams  *dftParams}]
\item[\texttt{INT4  startT}]
\end{description}

\newpage
%%%%%%%%%%%%%%%%%%%%%%%%%%%%%%%%%%%%%%%%%%%%%%%%%%%%%%%%%%%%%%%%%%%%%%%%%%%%%
\subsection{Module \texttt{CreateRealDFTParams.c}}
\label{ss:CreateRealDFTParams.c}
%%%%%%%%%%%%%%%%%%%%%%%%%%%%%%%%%%%%%%%%%%%%%%%%%%%%%%%%%%%%%%%%%%%%%%%%%%%%%

Creates a structure of type \verb+RealDFTParams+,

\subsubsection*{Prototypes}
\vspace{0.1in}
%\input{CreateRealDFTParamsCP}
\idx{LALCreateRealDFTParams()}

\subsubsection*{Description}

The inputs to \verb+CreateDFTParams()+ consist of (i) a parameter
\verb+winParams+ of type \verb+WindowParams*+ giving the length of the vectors
to be Fourier  transformed and the type of windowing to be used, (ii) a
pointer \verb+dftParams+ to a pointer to a \verb+RealDFTParams+ structure, and
(iii) an integer \verb+sign+ specifying the direction of the transform, with
$+1$ indicating forward transform and $-1$ indicating inverse transform.  On
exit, \verb+*dftParams+ will point to the newly created structure.

\subsubsection*{Uses}
\begin{verbatim}
LALCreateForwardRealFFTPlan
LALCreateReverseRealFFTPlan
LALSCreateVector
LALWindow
\end{verbatim}

\subsubsection*{Notes}

\newpage
%%%%%%%%%%%%%%%%%%%%%%%%%%%%%%%%%%%%%%%%%%%%%%%%%%%%%%%%%%%%%%%%%%%%%%%%%%%%%
\begin{thebibliography}{0}
%%%%%%%%%%%%%%%%%%%%%%%%%%%%%%%%%%%%%%%%%%%%%%%%%%%%%%%%%%%%%%%%%%%%%%%%%%%%%
\bibitem{ab:1999}
Warren G. Anderson and R. Balasubramanian, \textit{Time-frequency
detection of gravitational waves}, gr-qc/9905023, and references
therein.

\bibitem{fh:1998}
\'Eanna~\'E.~Flanagan and S. A. Hughes, Phys. Rev. D {\bf 57},
4535-4565 (1998) (also gr-qc/9701039).

\bibitem{acdhp:1999}
N. Arnaud, F. Cavalier, M. Davier, P. Hello, and T. Pradier,
\textit{Triggers for the Detection of Gravitational Wave Bursts},
gr-qc/9903035.

\bibitem{abcf:2000}
Warren G. Anderson, Patrick R. Brady, Jolien D. E. Creighton and
{\'E}anna {\'E}. Flanagan,
\textit{An excess power statistic for detection of burst sources of
gravitational radiation}, PRD 63, 042003 (2001) (gr-qc/0008066).
For a short version see {\it ibid}, \textit{A power filter for the
detection of burst sources of gravitational radiation in
interferometric detectors}, gr-qc/0001044, submitted to
Int. J. Modern. Phys. D.

\end{thebibliography}
