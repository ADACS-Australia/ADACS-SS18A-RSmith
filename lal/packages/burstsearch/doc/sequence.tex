\clearpage
\section{Sequence Manipulation}
\label{s:SequenceManipulation}

This is a suite of functions for creating, destroying, and manipulating LAL
sequences.  One pair of functions (the XLAL version and its LAL
counterpart) is available for each method and sequence type.  For example
\texttt{XLALCreateREAL4Sequence()} is available for creating sequences of
\texttt{REAL4} data, and the LAL-stype wrapper
\texttt{LALCreateREAL4Sequence()} is provided which is equivalent to the
XLAL version in all respects except that it adheres to the LAL calling
conventions (eg.\ it takes a \texttt{LALStatus} pointer as its first
argument, its return type is \texttt{void}, etc.).

\subsection{Creation Functions}

\subsubsection{Name}

\texttt{XLALCreate}\textit{sequencetype}\texttt{()},
\texttt{LALCreate}\textit{sequencetype}\texttt{()}

\subsubsection{Synopsis}

\begin{verbatim}
#include <lal/Sequence.h>
\end{verbatim}
\input{SequenceCreateP}

\subsubsection{Description}

These functions create LAL sequences.  The parameter \texttt{length}
specifies the length of the desired sequence.  An XLAL function returns a
pointer to the newly created sequence or \texttt{NULL} on failure.  The LAL
counterpart accepts the address of a pointer which it fills with the
address of the newly created sequence or \texttt{NULL} on failure.
Additionally, the LAL wrapper provides standard LAL-style error checking
via a \texttt{LALStatus} pointer.

\subsubsection{Author}

\verb|Kipp Cannon <kipp@gravity.phys.uwm.edu>|


\subsection{Destruction Functions}

\subsubsection{Name}

\texttt{XLALDestroy}\textit{sequencetype}\texttt{()},
\texttt{LALDestroy}\textit{sequencetype}\texttt{()}

\subsubsection{Synopsis}

\begin{verbatim}
#include <lal/Sequence.h>
\end{verbatim}
\input{SequenceDestroyP}

\subsubsection{Description}

These functions free all memory associated with a LAL sequence.  It is safe
to pass \texttt{NULL} to these functions.

\subsubsection{Author}

\verb|Kipp Cannon <kipp@gravity.phys.uwm.edu>|


\subsection{Cutting Functions}

\subsubsection{Name}

\texttt{XLALCut}\textit{sequencetype}\texttt{()},
\texttt{LALCut}\textit{sequencetype}\texttt{()},
\texttt{XLALCopy}\textit{sequencetype}\texttt{()},
\texttt{LALCopy}\textit{sequencetype}\texttt{()}

\subsubsection{Synopsis}

\subsubsection{Synopsis}

\begin{verbatim}
#include <lal/Sequence.h>
\end{verbatim}
\input{SequenceCutP}

\subsubsection{Description}

These functions create a new sequence by extracting a section of an
existing sequence.

\subsubsection{Author}

\verb|Kipp Cannon <kipp@gravity.phys.uwm.edu>|


\subsection{Shifting Functions}

\subsubsection{Name}

\texttt{XLALShift}\textit{sequencetype}\texttt{()},
\texttt{LALShift}\textit{sequencetype}\texttt{()}

\subsubsection{Synopsis}

\begin{verbatim}
#include <lal/Sequence.h>
\end{verbatim}
\input{SequenceShiftP}

\subsubsection{Description}

These functions shift the samples in a sequence, with zeros being placed in
the space that is freed.

\subsubsection{Author}

\verb|Kipp Cannon <kipp@gravity.phys.uwm.edu>|


\subsection{Resizing Functions}

\subsubsection{Name}

\texttt{XLALResize}\textit{sequencetype}\texttt{()},
\texttt{LALResize}\textit{sequencetype}\texttt{()},
\texttt{XLALShrink}\textit{sequencetype}\texttt{()},
\texttt{LALShrink}\textit{sequencetype}\texttt{()}

\subsubsection{Synopsis}

\begin{verbatim}
#include <lal/Sequence.h>
\end{verbatim}
\input{SequenceResizeP}

\subsubsection{Description}

The resize functions alter the size of an existing sequence.  The sequence
is adjust to have the specified length, and that part of the original
sequence starting at sample first is used to fill the new sequence.  If
first is negative, then the start of the new sequence is padded by that
many samples.  If part of the new sequence does not correspond to some part
of the original sequence, then those samples are uninitialized (this
behaviour may change in the future).

The shrink functions, originally, could only handle the special case in
which the new sequence is wholly contained in the original sequence.  Now
the shrink functions are wrappers for the resize functions and are only
retained for backwards compatibility.

\subsubsection{Author}

\verb|Kipp Cannon <kipp@gravity.phys.uwm.edu>|
