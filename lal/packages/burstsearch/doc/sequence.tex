\clearpage
\section{Sequence Manipulation}
\label{s:SequenceManipulation}

This is a suite of functions for creating, destroying, and manipulating LAL
sequences.  One pair of functions (the XLAL version and its LAL
counterpart) is available for each method and sequence type.  For example
\texttt{XLALCreateREAL4Sequence()} is available for creating sequences of
\texttt{REAL4} data, and the LAL-stype wrapper
\texttt{LALCreateREAL4Sequence()} is provided which is equivalent to the
XLAL version in all respects except that it adheres to the LAL calling
conventions (eg.\ it takes a \texttt{LALStatus} pointer as its first
argument, its return type is \texttt{void}, etc.).

\subsection{Creation Functions}

\subsubsection{Name}

(\texttt{X})\texttt{LALCreate}\textit{sequencetype}\texttt{()}

\subsubsection{Synopsis}

\begin{verbatim}
#include <lal/Sequence.h>
\end{verbatim}
\input{SequenceCreateP}

\subsubsection{Description}

These functions create LAL sequences.  The parameter \texttt{length}
specifies the length of the desired sequence.  An XLAL function returns a
pointer to the newly created sequence or \texttt{NULL} on failure.  The LAL
counterpart accepts the address of a pointer which it fills with the
address of the newly created sequence or \texttt{NULL} on failure.
Additionally, the LAL wrapper provides standard LAL-style error checking
via a \texttt{LALStatus} pointer.

\subsubsection{Author}

\verb|Kipp Cannon <kipp@gravity.phys.uwm.edu>|


\subsection{Destruction Functions}

\subsubsection{Name}

(\texttt{X})\texttt{LALDestroy}\textit{sequencetype}\texttt{()}

\subsubsection{Synopsis}

\begin{verbatim}
#include <lal/Sequence.h>
\end{verbatim}
\input{SequenceDestroyP}

\subsubsection{Description}

These functions free all memory associated with a LAL sequence.  It is safe
to pass \texttt{NULL} to these functions.

\subsubsection{Author}

\verb|Kipp Cannon <kipp@gravity.phys.uwm.edu>|


\subsection{Cutting Functions}

\subsubsection{Name}

(\texttt{X})\texttt{LALCut}\textit{sequencetype}\texttt{()}

\subsubsection{Synopsis}

\begin{verbatim}
#include <lal/Sequence.h>
\end{verbatim}
\input{SequenceCutP}

\subsubsection{Description}

These functions create a new sequence by extracting a section of an
existing sequence.

\subsubsection{Author}

\verb|Kipp Cannon <kipp@gravity.phys.uwm.edu>|


\subsection{Shrinking Functions}

\subsubsection{Name}

(\texttt{X})\texttt{LALShrink}\textit{sequencetype}\texttt{()}

\subsubsection{Synopsis}

\begin{verbatim}
#include <lal/Sequence.h>
\end{verbatim}
\input{SequenceShrinkP}

\subsubsection{Description}

These functions reduce the size of an existing sequence.

\subsubsection{Author}

\verb|Kipp Cannon <kipp@gravity.phys.uwm.edu>|
