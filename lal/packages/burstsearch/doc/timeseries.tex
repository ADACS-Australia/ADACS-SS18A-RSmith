\clearpage
\section{Time Series Manipulation}
\label{s:TimeSeriesManipulation}

This is a suite of functions for creating, destroying, and manipulating LAL
time series.  One pair of functions (the XLAL version and its LAL
counterpart) is available for each method and series type.  For example
\texttt{XLALCreateREAL4TimeSeries()} is available for creating time series
of \texttt{REAL4} data, and the LAL-stype wrapper
\texttt{LALCreateREAL4TimeSeries()} is provided which is equivalent to the
XLAL version in all respects except that it adheres to the LAL calling
conventions (eg.\ it takes a \texttt{LALStatus} pointer as its first
argument, its return type is \texttt{void}, etc.).

\subsection{Creation Functions}

\subsubsection{Name}

(\texttt{X})\texttt{LALCreate}\textit{seriestype}\texttt{()}

\subsubsection{Synopsis}

\begin{verbatim}
#include <lal/TimeSeries.h>
\end{verbatim}
\input{TimeSeriesCreateP}

\subsubsection{Description}

These functions create LAL frequency series.  An XLAL function returns a
pointer to the newly created series or \texttt{NULL} on failure.  The LAL
counterpart accepts the address of a pointer which it fills with the
address of the newly created series or \texttt{NULL} on failure.
Additionally, the LAL wrapper provides standard LAL-style error checking
via a \texttt{LALStatus} pointer.

\subsubsection{Author}

\verb|Kipp Cannon <kipp@gravity.phys.uwm.edu>|


\subsection{Destruction Functions}

\subsubsection{Name}

(\texttt{X})\texttt{LALDestroy}\textit{seriestype}\texttt{()}

\subsubsection{Synopsis}

\begin{verbatim}
#include <lal/TimeSeries.h>
\end{verbatim}
\input{TimeSeriesDestroyP}

\subsubsection{Description}

These functions free all memory associated with a LAL time series.  It is
safe to pass \texttt{NULL} to these functions.

\subsubsection{Author}

\verb|Kipp Cannon <kipp@gravity.phys.uwm.edu>|


\subsection{Cutting Functions}

\subsubsection{Name}

(\texttt{X})\texttt{LALCut}\textit{seriestype}\texttt{()}

\subsubsection{Synopsis}

\begin{verbatim}
#include <lal/TimeSeries.h>
\end{verbatim}
\input{TimeSeriesCutP}

\subsubsection{Description}

These functions create a new time series by extracting a section of an
existing time series.

\subsubsection{Author}

\verb|Kipp Cannon <kipp@gravity.phys.uwm.edu>|


\subsection{Shrinking Functions}

\subsubsection{Name}

(\texttt{X})\texttt{LALShrink}\textit{seriestype}\texttt{()}

\subsubsection{Synopsis}

\begin{verbatim}
#include <lal/TimeSeries.h>
\end{verbatim}
\input{TimeSeriesShrinkP}

\subsubsection{Description}

These functions reduce an existing time series to a section of itself.

\subsubsection{Author}

\verb|Kipp Cannon <kipp@gravity.phys.uwm.edu>|
