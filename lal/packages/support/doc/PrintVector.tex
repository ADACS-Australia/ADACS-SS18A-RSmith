% This file is meant to be included in another 
\documentclass{article}
\begin{document}
\section{PrintVector()}

\subsection{Purpose}

This is a simple utility to print vectors into a file.
 
\subsection{Synopsis}

% Syntax: argument definitions, calling signature

\begin{verbatim}
#include "PrintVector.h"
void PrintVector(REAL4Vector *vector)
\end{verbatim}


\subsection{Description}

This function prints the elements of {\tt vector} into a file.
Note: the file names are {\tt PrintVector.000}, {\tt PrintVector.001},
and so on.  The file numbers are incremented with each additional call.
This function is for debugging use only: it uses a static internal
variable to keep track of the file number so it should not
be used in any real analysis codes.


\subsection{Operating Instructions}

% Detailed usage 

\begin{verbatim}
REAL4Vector *vector;
... 
PrintVector(vector)
\end{verbatim}


\subsubsection{Arguments}

% Describe meaning of each argument

\begin{itemize}
\item {\tt vector\/} is a pointer to a vector.
\end{itemize}

\subsubsection{Options}

None. 

\subsubsection{Error conditions}

% What constitutes an error condition? What do the error codes mean?

If {\tt vector} is {\tt NULL} then PrintVector() simply returns.

                                
\subsection{Algorithms}

Not applicable.

% Describe algorithm by which work is done

\subsection{Accuracy}

% For numerical routines address issues related to accuracy:
% approximations, argument ranges, etc.

Not applicable.

\subsection{Tests}

% Describe the tests that are part of the test suite

None.

\subsection{Uses}

% What LAL, other routines does this one call?

None.

\subsection{Notes}

This function uses an internal static variable to keep track of
file numbers.  For this reason it should only be used for debugging
purposes in test functions, not in any production code.

\subsection{References}

% Any references for algorithms, tests, etc.
None.

\end{document}


