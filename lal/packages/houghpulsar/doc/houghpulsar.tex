% -----------------------------------------------------
%  hough.tex
%  
%  Documentation of the Hough transform code
%
%   Authors: Alicia M. Sintes, Maria Alessandra Papa, et...
%
%  11-10-2001
%
%   $Id$
% ------------------------------------------------------

\chapter{Package \texttt{houghpulsar}: The Hough transform}


Alicia M. Sintes, M. Alessandra Papa., Krishnan, B.\\

The  hierarchical Hough transform search strategy is 
an efficient and highly parallel computer algorithm 
\cite{Schutz:1998} \cite{Papa:2001}.
This package provides the necessary routines for the Hough incoherent search
(the second stage of the search) 
  to track the frequency evolution of peaks in the spectra, which
  is known in advance for some choice of the source parameters.\\


{\it The Hough transform} is  a transformation
between the data and the space of parameters that describe the signal.
\begin{description}
\item[ Input:] Set of points in time-frequency plane that have been obtained
from the demodulated FT.
The demodulation  has been performed with respect to a certain sky
location and certain spin-down parameters,
in such a way  that, if a source was located in
the center of the patch and having the same spin-down parameters 
for which it has been
demodulated, we will observe a set of points forming a horizontal 
line at the intrinsic frequency  of the source $f_0$.
 Due to the mismatch between the source  parameters
 and the demodulated parameters we will observe a certain pattern in the time
  frequency diagram following the 
   {\sl Hough transform master equation}.
\item[ Output:] Histograms in the parameter space:
  for each $f_0$, residual spin-down and refined sky location.
\end{description}
The principal is the following: for every point in the {\it  time-frequency}
 plane, we enhance the number count in the 
histogram in the pixels that are {\it consistent}. At the end of this procedure,
significant clustering
 in a pixel in parameter space indicates
 likelihood of the data containing a signal 
having the parameters
of that pixel.
\\

The package  is organized under the headers
\verb@LUT.h@ \verb@PHMD.h@ \verb@HoughMap.h@  \verb@LALHough.h@ and 
the modules   \verb@PatchGrid.c@,  \verb@Stereographic.c@,  
\verb@ParamPLUT.c@, \verb@NDParamPLUT.c@,\verb@ConstructPLUT.c@, 
 \verb@Peak2PHMD.c@,  \verb@HoughMap.c@,  and \verb@DriveHough.c@.

\subsection*{Acknowledgment}
The authors thank S. Frasca and  C. Palomba for helpful discussions,
F. Massaioli for helping in the initial stages of code
development, and B. Allen and J. Creighton for their valuable 
implementation advice.

\newpage\input{LUTH}
\newpage\input{PHMDH}
\newpage\input{HoughMapH}
\newpage\input{LALHoughH}
\newpage\input{StatisticsH}
\newpage\input{VelocityH}


\newpage\begin{thebibliography}{0}

\bibitem{Schutz:1998}
 B.~F. Schutz, M.~A. Papa, 
{\it End-to-end algorithm for hierarchical area searches for 
 long-duration GW sources for GEO 600} in 
  {\it \lq\lq Gravitational waves and experimental gravity"}.
 Edts. J. Tran Thanh Van, J. Dumarchez, S. Reynaud, C. Salomon, S. Thorsett,
J.Y. Vinet; World Publishers,  (2000), Hanoi-Vietnam. gr-qc/9905018
 
 
 \bibitem{Papa:2001}
 M.~A. Papa, B.~F. Schutz, and A.~M. Sintes, 
{\it Searching for continuous gravitational wave signals. The hierarchical Hough
transform algorithm} in 
{\it \lq\lq Gravitational waves: A challenge to theoretical astrophysics''}.
Edts. V. Ferrari, J.C. Miller, L. Rezzolla, 
ICTP Lecture Notes Series, Volume III  (ISBN 92-95003-05-5), p. 431-442 (2001), 
Italy. AEI-2000-077, gr-qc/0011034

 %A.~M. Sintes, B.~F. Schutz, Phys. Rev. D\textbf{58}, 122003 (1998)

\end{thebibliography}
