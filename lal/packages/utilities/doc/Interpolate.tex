\documentclass{article}
\begin{document}

\section{Interpolate}

Interpolate a set of data.

\subsection{Synopsis}
\begin{verbatim}
#include "Interpolate.h"

typedef struct
tagInterpolateOut
{
  REAL4  y;
  REAL4 dy;
}
InterpolateOut;

typedef struct
tagInterpolatePar
{
  INT4   n;
  REAL4 *x;
  REAL4 *y;
}
InterpolatePar;

void
PolynomialInterpolation (
    Status         *status,
    InterpolateOut *output,
    REAL4           target,
    InterpolatePar *params
    );
\end{verbatim}

\subsection{Description}

The structure \verb+InterpolatePar+ contains the parameters of the
interpolation.  These are the arrays of \verb+n+ domain values
\verb+x[0]+\ldots\verb+x[n-1]+ and their corresponding values
\verb+y[0]+\ldots\verb+y[n-1]+.  The routine \verb+PolynomialInterpolation()+
computes the interpolated $y$ value \verb+output+ at the $x$ value
\verb+target+ by fitting a polynomial of order \verb+n-1+ to the data.  The
result \verb+output+ is of type \verb+InterpolateOut+, which contains the
value \verb+y+ as well as an estimate of the error \verb+dy+.

\subsection{Operating Instructions}

The following program fits a fourth-order polynomial to the five data points
$\{(0,0),(1,1),(2,3),(3,4),(4,3)\}$, and interpolates the value at $x=2.4$.

\begin{verbatim}
#include "LALStdlib.h"
#include "Interpolate.h"

int main ()
{
  enum { ArraySize = 5 };
  static Status  status;
  REAL4          x[ArraySize] = {0,1,2,3,4};
  REAL4          y[ArraySize] = {0,1,3,4,3};
  REAL4          target       = 2.4;
  InterpolatePar intpar       = {ArraySize, x, y};
  InterpolateOut intout;

  PolynomialInterpolation (&status, &intout, target, &intpar);

  return 0;
}
\end{verbatim}

\subsection{Options}

\subsection{Error conditions}

This function sets the universal status structure on return.
Error conditions are described in the following table.

\begin{table}
\begin{tabular}{|r|l|p{2in}|}\hline
status  & status          & Description\\
code    & description     & \\\hline
\verb+INTERPOLATE_ENULL 1+ & Null pointer & \\
\verb+INTERPOLATE_ESIZE 2+ & Invalid size  & \\
\verb+INTERPOLATE_EZERO 4+ & Zero divide & \\
\hline
\end{tabular}
\caption{Error conditions for RombergIntegrate function}\label{tbl:CV}
\end{table}

\subsection{Algorithms}

This is an implementation of Neville's algroithm, see \verb+polint+ in
Numerical Recipes~\cite{ptvf:1992}.

\subsection{Accuracy}

\subsection{Uses}

\subsection{Notes}

\subsection{References}
\begin{thebibliography}{0}
\bibitem{ptvf:1992}
  W. H. Press, S. A. Teukolsky, W. T. Vetterling, and B. P. Flannery,
  \textit{Numerical Recipes in C: The Art of Scientific Computing}, 2nd ed.
  (Cambridge University Press, Cambridge, England, 1992).
\end{thebibliography}

\end{document}
