\documentclass{article}
\begin{document}

\section{Random}

Generates random numbers.

\subsection{Synopsis}
\begin{verbatim}
#include "Random.h"

void CreateRandomParams  (Status *status, RandomParams **params, INT4 seed);
void DestroyRandomParams (Status *status, RandomParams **params);

void UniformDeviate (Status *status, REAL4       *deviate,  RandomParams *params);
void NormalDeviates (Status *status, REAL4Vector *deviates, RandomParams *params);
\end{verbatim}

\subsection{Description}

The routines \verb+CreateRandomParams()+ and \verb+DestroyRandomParams+ create
and destroy a parameter structure for the generation of random variables.  The
creation routine requires a random number seed \verb+seed+.  If the seed is
zero then a seed is generated using the current time.

The routine \verb+UniformDeviate()+ returns a single random deviate
distributed uniformly between zero and unity.  

The routine \verb+NormalDeviates()+ fills a vector with normal (Gaussian)
deviates with zero mean and unit variance.

\subsection{Operating Instructions}

\begin{verbatim}
static Status        status;
static RandomParams *params;
static REAL4Vector  *vector;

INT4 seed = 0;
INT4 i;

CreateVector       (&status, &vector, 9999);
CreateRandomParams (&status, &params, seed);

/* fill vector with uniform deviates */
for (i = 0; i < vector->length; ++i)
{
  UniformDeviate (&status, vector->data + i, params);
}

/* fill vector with normal deviates */
NormalDeviates (&status, vector, params);

DestroyRandomParams (&status, &params);
DestroyVector       (&status, &vector);
\end{verbatim}

\begin{itemize}
\item\verb+status+ is a universal status structure.  Its contents are
assigned by the functions.
\item\verb+params+ is a parameter structure holding current random number
generation seeds.
\item\verb+seed+ is an initialization seed; set to zero to take the seed from
the current time.
\item\verb+vector+ is a real vector that will be filled with random numbers.
\end{itemize}

\subsubsection{Options}

\subsubsection{Error conditions}

These functions all set the universal status structure on return.
Error conditions are described in the following table.

\begin{table}
\begin{tabular}{|r|l|p{2in}|}\hline
status  & status          & Description\\
code    & description     & \\\hline
\verb+RANDOM_ENULL 1+ & Null pointer & \\
\verb+RANDOM_ENNUL 2+ & Non-null pointer & \\
\verb+RANDOM_ESIZE 4+ & Invalid size & \\
\hline
\end{tabular}
\caption{Error conditions for all Random functions}\label{tbl:CV}
\end{table}

\subsection{Algorithms}

This is an implementation of the random number generators \verb+ran1+ and
\verb+gasdev+ described in Numerical Recipes~\cite{ptvf:1992}.

\subsection{Accuracy}

\subsection{Tests}

The program \verb+RandomTest+ generates output files containing uniform
deviates and normal deviates.

\subsection{Uses}

\subsection{Notes}

\subsection{References}
\begin{thebibliography}{0}
\bibitem{ptvf:1992}
  W. H. Press, S. A. Teukolsky, W. T. Vetterling, and B. P. Flannery,
  \textit{Numerical Recipes in C: The Art of Scientific Computing}, 2nd ed.
  (Cambridge University Press, Cambridge, England, 1992).
\end{thebibliography}

\end{document}
