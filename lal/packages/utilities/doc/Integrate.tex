\documentclass{article}
\begin{document}

\section{Integrate}

Integrates a function.

\subsection{Synopsis}
\begin{verbatim}
#include "Random.h"

typedef void (REAL4LALFunction) (Status *s, REAL4 *y, REAL4 x, void *p);

typedef enum
{
  ClosedInterval,     /* evaluate integral on a closed interval             */
  OpenInterval,       /* evaluate integral on an open interval              */
  SingularLowerLimit, /* integrate an inv sqrt singularity at lower limit   */
  SingularUpperLimit, /* integrate an inv sqrt singularity at upper limit   */
  InfiniteDomainPow,  /* integrate infinite domain with power-law falloff   */
  InfiniteDomainExp   /* integrate infinite domain with exponential falloff */
}
IntegralType;

typedef struct
tagIntegrateIn
{
  REAL4LALFunction *function;
  REAL4             xmax;
  REAL4             xmin;
  IntegralType      type;
}
IntegrateIn;

void
RombergIntegrate (
    Status      *status,
    REAL4       *result,
    IntegrateIn *input,
    void        *params
    );
\end{verbatim}

\subsection{Description}

The structure \verb+IntegrateIn+ contains the function, the range of
integration, and the type of integral, i.e., the method to grid the domain.
The types of integration are the following: I.\@ \verb+ClosedInterval+
indicates that the integral should be computed on equal-spaced domain
intervals including the boundary.  II.\@ \verb+OpenInterval+ indicates that
the integral should be computed on intervals of the domain not including the
boundary.  III.\@ \verb+SingularLowerLimit+ indicates that the integral should
be evaluated on an open interval with a transformation so that a
inverse-square-root singularity at the lower limit can be integrated.
IV.\@ \verb+SingularUpperLimit+ is the same as above but for a singularity at
the upper limit.  V.\@ \verb+InfiniteDomainPow+ indicates that the integral
should be evaluated over an semi-infinite domain---appropriate when both
limits have the same sign (though one is very large) and when the integrand
vanishes faster than $x^{-1}$ at infinity.  VI.\@ \verb+InfiniteDomainExp+
indicates that the integral should be evaluated over an infinite domain
starting at \verb+xmin+ and going to infinity (\verb+xmax+ is ignored)---the
integrand should vanish exponentially for large $x$.

The routine \verb+RombergIntegrate+ performs the integral specified by the
structure \verb+input+ and the result is returned as \verb+result+.  Any
additional parameters (other than the integration variable $x$) can be passed
as \verb+params+.  The routine \verb+RombergIntegrate+ does not use
\verb+params+ but just passes it to the integrand.

\subsection{Operating Instructions}

The following program performs the integral $\int_0^2F(x)dx$ where
$F(x)=x^4\log(x+\sqrt{x^2+1})$.

\begin{verbatim}
#include <math.h>
#include "LALStdlib.h"
#include "Integrate.h"

void F (Status *s, REAL4 *y, REAL4 x, void *p)
{
  REAL4  x2 = x*x;
  REAL4  x4 = x2*x2;
  INITSTATUS (s, "Function F()");
  ASSERT (!p, s, 1, "Non-null pointer");
  *y = x4*log(x + sqrt(x2 + 1));
  RETURN (s);
}

int main ()
{
  const REAL4   epsilon = 1e-6;
  long double   expect  = 8.153364119811650205L;
  static Status status;
  IntegrateIn   intinp;
  REAL4         result;

  intinp.function = F;
  intinp.xmin     = 0;
  intinp.xmax     = 2;
  intinp.type     = ClosedInterval;

  RombergIntegrate (&status, &result, &intinp, NULL);
  if (fabs(result - expect) > epsilon*fabs(expect))
  {
    /* integration did not achieve desired accuracy --- exit failure */
    return 1;
  }

  return 0;
}
\end{verbatim}

\subsection{Options}

\subsection{Error conditions}

This function sets the universal status structure on return.
Error conditions are described in the following table.

\begin{table}
\begin{tabular}{|r|l|p{2in}|}\hline
status  & status          & Description\\
code    & description     & \\\hline
\verb+INTEGRATE_ENULL 1+ & Null pointer & \\
\verb+INTEGRATE_ETYPE 2+ & Unknown integral type & \\
\verb+INTEGRATE_EIDOM 4+ & Invalid domain & \\
\verb+INTEGRATE_EMXIT 8+ & Maximum iterations exceeded & \\
\hline
\end{tabular}
\caption{Error conditions for RombergIntegrate function}\label{tbl:CV}
\end{table}

\subsection{Algorithms}

This is an implementation of the Romberg integrating function \verb+qromb+ in
Numerical Recipes~\cite{ptvf:1992}.

\subsection{Accuracy}

The fractional accuracy of the integral returned should be $1\times10^{-6}$.
However, this is highly dependent on the nature of the function.

\subsection{Uses}

This routine uses the function \verb+PolynomialInterpolation()+.

\subsection{Notes}

\subsection{References}
\begin{thebibliography}{0}
\bibitem{ptvf:1992}
  W. H. Press, S. A. Teukolsky, W. T. Vetterling, and B. P. Flannery,
  \textit{Numerical Recipes in C: The Art of Scientific Computing}, 2nd ed.
  (Cambridge University Press, Cambridge, England, 1992).
\end{thebibliography}

\end{document}

