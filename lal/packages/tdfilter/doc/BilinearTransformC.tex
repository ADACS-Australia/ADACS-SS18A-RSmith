
\subsection{Module \texttt{BilinearTransform.c}}

Transforms the complex frequency coordinate of a ZPG filter.

\subsubsection{Prototypes}
\vspace{0.1in}
\marginpar{\texttt{\tiny l.151}\\{\tiny BilinearTransform.c}}
\vspace{-0.1in}
\begin{verbatim}
void WToZCOMPLEX8ZPGFilter(Status            *stat,
			   COMPLEX8ZPGFilter *filter)
\end{verbatim}
\marginpar{\texttt{\tiny l.386}\\{\tiny BilinearTransform.c}}
\vspace{-0.1in}
\begin{verbatim}
void WToZCOMPLEX16ZPGFilter(Status             *stat,
			    COMPLEX16ZPGFilter *filter)
\end{verbatim}


\subsubsection{Description}

These functions perform an in-place bilinear transformation on an
object \verb@*filter@ of type \verb@<datatype>ZPGFilter@, transforming
from $w$ to $z=(1+iw)/(1-iw)$.  Care is taken to ensure that zeros and
poles at $w=\infty$ are correctly transformed to $z=-1$, and zeros and
poles at $w=-i$ are correctly transformed to $z=\infty$.  In addition
to simply relocating the zeros and poles, residual factors are also
incorporated into the gain of the filter (i.e.\ the leading
coefficient of the rational function).

\subsubsection{Algorithm}

The vectors \verb@filter->zeros@ and \verb@filter->poles@ only record
those zeros and poles that have finite value.  If one includes the
point $\infty$ on the complex plane, then a rational function always
has the same number of zeros and poles: a number \verb@num@ that is
the larger of \verb@z->zeros->length@ or \verb@z->poles->length@.  If
one or the other vector has a smaller length, then after the
transformation that vector will receive additional elements, with a
complex value of $z=-1$, to bring its length up to \verb@num@.
However, each vector will then \emph{lose} those elements that
previously had values $w=-i$, (which are sent to $z=\infty$,) thus
possibly decreasing the length of the vector.  These routines handle
this by simply allocating a new vector for the transformed data, and
freeing the old vector after the transformation.

When transforming a zero $w_k$ on the complex plane, one makes use of
the identity:
$$
(w - w_k) = -(w_k + i)\times\frac{z-z_k}{z+1} \; ,
$$
and similarly, when transforming a pole at $w_k$,
$$
(w - w_k)^{-1} = -(w_k + i)^{-1}\times\frac{z+1}{z-z_k} \; ,
$$
where $z=(1+iw)/(1-iw)$ and $z_k=(1+iw_k)/(1-iw_k)$.  If there are an
equal number of poles and zeros being transformed, then the factors of
$z+1$ will cancel; otherwise, the remaining factors correspond to the
zeros or poles at $z=-1$ brought in from $w=\infty$.  The factor
$(z-z_k)$ represents the transformation of the zero or pole at $w_k$.
The important factor to note, though, is the factor $-(w_k+i)^{\pm1}$.
This factor represents the change in the gain \verb@filter->gain@.
When $w_k=-i$, the transformation is slightly different:
$$
(w + i) = \frac{2i}{z+1} \; ;
$$
thus the gain correction factor is $2i$ (rather than 0) in this case.

The algorithms in this module computes and stores all the gain
correction factors before applying them to the gain.  The correction
factors are sorted in order of absolute magnitude, and are multiplied
together in small- and large-magnitude pairs.  In this way one reduces
the risk of overrunning the floating-point dynamical range during
intermediate calculations.

As a similar precaution, the routines in this module use the algorithm
discussed in the \verb@VectorOps@ package whenever they perform
complex division, to avoid intermediate results that mey be the
product of two large numbers.  When transforming $z=(1+iw)/(1-iw)$,
these routines also test for special cases (such as $w$ purely
imaginary) that have qualitatively significant results ($z$ purely
real), so that one doesn't end up with, for instance, an imaginary
part of $10^{-12}$ instead of 0.

\subsubsection{Uses}
\begin{verbatim}
I4CreateVector()
SCreateVector()
DCreateVector()
CCreateVector()
ZCreateVector()
I4DestroyVector()
SDestroyVector()
DDestroyVector()
CDestroyVector()
ZDestroyVector()
SHeapIndex()
DHeapIndex()
\end{verbatim}

\subsubsection{Notes}

