
\subsection{Program \texttt{BandPassTest.c}}

Tests time-domain high- and low-pass filters.

\subsubsection{Usage}
\begin{verbatim}
BandPassTest [-o [outfile]] [-d [debug-level]]
\end{verbatim}

\subsubsection{Description}

This program generates a time series with an impulse in it, and passes
it through a time-domain low-pass or high-pass filter.  By default,
running this program with no arguments simply tests the subroutines,
producing no output.  All filter parameters are set from
\verb@#define@d constants.  The program returns a value of 0 upon
successful completion, 1 if any of its function calls failed, or 2 if
output file creation failed.

The \verb@-o@ flag tells the program to print the impulse response to
a data file; if \verb@outfile@ is not specified, it will write to the
file \verb@out.dat@.  The \verb@-d@ option increases the default debug
level from 0 to 1, or sets it to the specified value
\verb@debug-level@.

\subsubsection{Algorithm}

\subsubsection{Uses}
\begin{verbatim}
debuglevel
SCreateVector()
SDestroyVector()
ButterworthREAL4TimeSeries()
LALPrintError()
\end{verbatim}

\subsubsection{Notes}

