
\section{Header \texttt{BandPassTimeSeries.h}}

Provides routines to low- or high-pass filter a time series.

\subsection{Synopsis}
\begin{verbatim}
#include "BandPassTimeSeries.h"
\end{verbatim}

\noindent This header covers routines that apply a time-domain low- or
high-pass filter to a data series of type \verb@<datatype>TimeSeries@.
Further documentation is given in the individual routines' modules.


\subsection{Error conditions}
\begin{tabular}{|c|l|l|}
\hline
status & status                    & Explanation                           \\
 code  & description               &                                       \\
\hline
\tt 1  & \tt Null pointer          & Missing a required pointer.           \\
\tt 2  & \tt Bad filter parameters & Filter creation parameters outside of \\
       &                           & acceptable ranges.                    \\
\hline
\end{tabular}


\subsection{Structures}

\begin{verbatim}
struct PassBandParamStruc
\end{verbatim}

\noindent This structure stores data used for constructing a low- or
high-pass filter: either the order and characteristic frequency of the
filter, or the frequencies and desired attenuations at the ends of
some transition band.  In the latter case, a nonzero filter order
parameter n indicates a maximum allowed order.  The fields are:

\begin{description}
\item[\texttt{CHAR *name}] A user-assigned name.

\item[\texttt{INT4 n}] The maximum desired filter order (actual order
  may be less if specified attenuations do not require a high order).

\item[\texttt{REAL8 f1}, \texttt{f2}] The reference frequencies of the
  transition band.

\item[\texttt{REAL8 a1}, \texttt{a2}] The minimal desired attenuation
  factors at the reference frequencies.
\end{description}

\newpage
\subsection{Module \texttt{ButterworthTimeSeries.c}}

Applies a low- or high-pass Butterworth filter to a time series.

\subsubsection{Prototypes}
\vspace{0.1in}
\marginpar{\texttt{\tiny l.196}\\{\tiny ButterworthTimeSeries.c}}
\vspace{-0.1in}
\begin{verbatim}
void ButterworthREAL4TimeSeries(Status             *stat,
				REAL4TimeSeries    *series,
				PassBandParamStruc *params)
\end{verbatim}
\marginpar{\texttt{\tiny l.322}\\{\tiny ButterworthTimeSeries.c}}
\vspace{-0.1in}
\begin{verbatim}
void ButterworthREAL8TimeSeries(Status             *stat,
				REAL8TimeSeries    *series,
				PassBandParamStruc *params)
\end{verbatim}


\subsubsection{Description}

These routines perform an in-place time-domain band-pass filtering of
a data sequence \verb@*series@, using a Butterworth filter generated
from parameters \verb@*params@.  The routines construct a filter with
the square root of the desired amplitude response, which it then
applied to the data once forward and once in reverse.  This gives the
full amplitude response with little or no frequency-dependent phase
shift.

\subsubsection{Algorithm}

The frequency response of a Butterworth low-pass filter is easiest to
express in terms of the transformed frequency variable $w=\tan(\pi
f\Delta t)$, where $\Delta t$ is the sampling interval (i.e.\
\verb@series->deltaT@).  In this parameter, then, the \emph{power}
response (attenuation) of the filter is:
$$
|R|^2 = a = \frac{1}{1+(w/w_c)^{2n}} \; ,
$$
where $n$ is the filter order and $w_c$ is the characteristic
frequency.  Similarly, a Butterworth high-pass filter is given by
$$
|R|^2 = a = \frac{1}{1+(w_c/w)^{2n}} \; .
$$
If one is given a filter order $n$, then the characteristic frequency
can be determined from the attenuation at some any given frequency.
Alternatively, $n$ and $w_c$ can both be computed given attenuations
at two different frequencies.

Frequencies in \verb@*params@ are assumed to be real frequencies $f$
given in the inverse of the units used for the sampling interval
\verb@series->deltaT@.  In order to be used, the pass band parameters
must lie in the ranges given below; if a parameter lies outside of its
range, then it is ignored and the filter is calculated from the
remaining parameters.  If too many parameters are missing, the routine
will fail.  The acceptable parameter ranges are:

\begin{description}
\item[\texttt{params->nMax}]   = 1, 2, $\ldots$
\item[\texttt{params->f1}, \texttt{f2}] $\in
  (0,\{2\times\verb@series->deltaT@\}^{-1}) $
\item[\texttt{params->a1}, \texttt{a2}] $\in (0,1) $
\end{description}

If both pairs of frequencies and amplitudes are given, then \verb@a1@,
\verb@a2@ specify the minimal requirements on the attenuation of the
filter at frequencies \verb@f1@, \verb@f2@.  Whether the filter is a
low- or high-pass filter is determined from the relative sizes of
these parameters.  In this case the \verb@nMax@ parameter is optional;
if given, it specifies an upper limit on the filter order.  If the
desired attenuations would require a higher order, then the routine
will sacrifiece performance in the stop band in order to remain within
the specified \verb@nMax@.

If one of the frequency/attenuation pairs is missing, then the filter
is computed using the remaining pair and \verb@nMax@ (which must be
given).  The filter is taken to be a low-pass filter if \verb@f1@,
\verb@a1@ are given, and high-pass if \verb@f2@, \verb@a2@ are given.
If only one frequency and no corresponding attenuation is specified,
then it is taken to be the characteristic frequency (i.e. the
corresponding attenuation is assumed to be 1/2).  If none of these
conditions are met, the routine will return an error.

Once an order $n$ and characteristic frequency $w_c$ are known, the
zeros and poles of a ZPG filter are readily determined.  A stable,
physically realizable Butterworth filter will have $n$ poles evenly
spaced on the upper half of a circle of radius $w_c$; that is,
$$
R = \frac{(-iw_c)^n}{\prod_{k=0}^{n-1}(w - w_c e^{2\pi i(k+1/2)/n})}
$$
for a low-pass filter, and
$$
R = \frac{w^n}{\prod_{k=0}^{n-1}(w - w_c e^{2\pi i(k+1/2)/n})}
$$
for a high-pass filter.  By choosing only poles on the upper-half
plane, one ensures that after transforming to $z$ the poles will have
$|z|<1$.

Although higher orders $n$ would appear to produce better (i.e.\
sharper) filter responses, one rapidly runs into numerical errors, as
one ends up adding and subtracting $n$ large numbers to obtain small
filter responses.  One way around this is to break the filter up into
several lower-order filters.  The routines in this module do just
that.  Poles are paired up across the imaginary axis, (and combined
with pairs of zeros at $w=0$ for high-pass filters,) to form $[n/2]$
second-order filters.  If $n$ is odd, there will be an additional
first-order filter, with one pole at $w=iw_c$ (and one zero at $w=0$
for a high-pass filter).

Each ZPG filter in the $w$-plane is first transformed to the $z$-plane
by a bilinear transformation, and is then used to construct a
time-domain IIR filter.  Each filter is then applied to the time
series.  As mentioned in the description above, the filters are
designed to give an overall amplitude response that is the square root
of the desired attenuation; however, each time-domain filter is
applied to the data stream twice: once in the normal sense, and once
in the time-reversed sense.  This gives the full attenuation with very
little frequency-dependent phase shift.

\subsubsection{Uses}
\begin{verbatim}
debuglevel
LALPrintError()
CreateREAL4IIRFilter()
CreateREAL8IIRFilter()
DestroyREAL4IIRFilter()
DestroyREAL8IIRFilter()
CreateCOMPLEX8ZPGFilter()
CreateCOMPLEX16ZPGFilter()
DestroyCOMPLEX8ZPGFilter()
DestroyCOMPLEX16ZPGFilter()
IIRFilterREAL4Vector()
IIRFilterREAL8Vector()
IIRFilterREAL4VectorR()
IIRFilterREAL8VectorR()
\end{verbatim}

\subsubsection{Notes}


\newpage
\subsection{Program \texttt{BandPassTest.c}}

Tests time-domain high- and low-pass filters.

\subsubsection{Usage}
\begin{verbatim}
BandPassTest [-o [outfile]] [-d [debug-level]]
\end{verbatim}

\subsubsection{Description}

This program generates a time series with an impulse in it, and passes
it through a time-domain low-pass or high-pass filter.  By default,
running this program with no arguments simply tests the subroutines,
producing no output.  All filter parameters are set from
\verb@#define@d constants.  The program returns a value of 0 upon
successful completion, 1 if any of its function calls failed, or 2 if
output file creation failed.

The \verb@-o@ flag tells the program to print the impulse response to
a data file; if \verb@outfile@ is not specified, it will write to the
file \verb@out.dat@.  The \verb@-d@ option increases the default debug
level from 0 to 1, or sets it to the specified value
\verb@debug-level@.

\subsubsection{Algorithm}

\subsubsection{Uses}
\begin{verbatim}
debuglevel
SCreateVector()
SDestroyVector()
ButterworthREAL4TimeSeries()
LALPrintError()
\end{verbatim}

\subsubsection{Notes}


