\chapter{Package \texttt{tdfilter}}

This package covers LAL routines for constructing and applying digital
time-domain filters.  It is organized under three headers.  The
\verb@ZPGFilter.h@ header provides routines for manipulating filters
in the ``zeros, poles, gain'' representation, which is typically the
simplest way of representing a filter response.  These routines create
and destroy ZPG filters, and can transform the complex variable used
to represent them.  The \verb@IIRFilter.h@ header provides routines
for creating actual time-domain filters from the ZPG representation,
and applying these filters to data.  The \verb@BandPassTimeSeries.h@
header provides routines an actual implementation of these utilities
to the specific task of high- or low-pass filtering of a data stream.
These routines also serve as an example for the more general task of
designing time-domain filters with any desired frequency response.

The module \verb@ButterworthTimeSeries.c@ provides specific advice and
guidelines for building a numerically stable time-domain filter, but
the general procedure is as follows.  (1) Decide on the desired filter
response, and express it as a rational function of the frequency
variable $w=\tan(\pi f\Delta t)$ (which maps the Nyquist interval onto
the positive real axis).  (2) Factorize this rational function into
zeros and poles, restricting oneself to the upper half of the $w$
complex plane.  Assign these to one or more objects of type
\verb@<datatype>ZPGFilter@.  (3) Use \verb@WToZ<datatype>ZPGFilter()@
to transform the filter to the more conventional $z=\exp(2\pi if\Delta
t)$ frequency variable.  (4) Use the routines in \verb@IIRFilter.h@ to
create IIR filters from the ZPG filters, and to apply them to data.

\newpage
\section{Header \texttt{ZPGFilter.h}}

Provides routines to manipulate ZPG filters.

\subsection{Synopsis}
\begin{verbatim}
#include "ZPGFilter.h"
\end{verbatim}

\noindent This header covers routines that create, destroy, and
transform objects of type \verb@<datatype>ZPGFilter@, where
\verb@<datatype>@ is either \verb@COMPLEX8@ or \verb@COMPLEX16@.
Generically, these data types can be used to store any rational
complex function in a factored form.  Normally this function is a
filter response, or ``transfer function'' $T(z)$, expressed in terms
of a complex frequency parameter $z=\exp(2\pi if\Delta t)$, where
$\Delta t$ is the sampling interval.  The rational function is
factored as follows:
$$
T(f) = g\times\frac{\prod_k (z-a_k)}{\prod_l (z-b_l)}
$$
where $g$ is the gain, $a_k$ are the (finite) zeros, and $b_l$ are the
(finite) poles.  It should be noted that rational functions always
have the same number of zeros as poles if one includes the point
$z=\infty$; any excess in the number of finite zeros or poles in the
rational expression simply indicates that there is a corresponding
pole or zero of that order at infinity.  It is also worth pointing out
that the ``gain'' is just the overall prefactor of this rational
function, and is not necessarily equal to the actual gain of the
transfer function at any particular frequency.

Another common complex frequency space is the $w$-space, obtained
from the $z$-space by the bilinear transformation:
$$
w = i\left(\frac{1-z}{1+z}\right) = \tan(\pi f\Delta t) , \quad
z = \frac{1+iw}{1-iw} \; .
$$
Other variables can also be used to represent the complex frequency
plane.  The \verb@<datatype>ZPGFilter@ structure can be used to
represent the transfer function in any of these spaces by transforming
the coordinates of the zeros and poles, and incorporating any residual
factors into the gain.  Care must be taken to include any zeros or
poles that are brought in from infinity by the transformation, and to
remove any zeros or poles which were sent to infinity.  Thus the
number of zeros and poles of the \verb@<datatype>ZPGFilter@ is not
necessarily constant under transformations!  Routines invoking the
\verb@<datatype>ZPGFilter@ data types should document which complex
variable is assumed.


\subsection{Error conditions}
\begin{tabular}{|c|l|l|}
\hline
status & status      & Explanation \\
 code  & description &             \\
\hline
\tt 1  & \tt Null pointer            & Missing a required pointer.           \\
\tt 2  & \tt Output already exists   & Can't allocate to a non-null pointer. \\
\tt 3  & \tt Memory allocation error & Could not allocate memory.            \\
\tt 4  & \tt Bad filter parameters   & Filter creation parameters outside of \\
       &                             & acceptable ranges.                    \\
\hline
\end{tabular}

\subsection{Structures}
\newpage
\subsection{Module \texttt{CreateZPGFilter.c}}

Creates ZPG filter objects.

\subsubsection{Prototypes}
\vspace{0.1in}
\marginpar{\texttt{\tiny l.68}\\{\tiny CreateZPGFilter.c}}
\vspace{-0.1in}
\begin{verbatim}
void CreateCOMPLEX8ZPGFilter(Status            *stat,
			     COMPLEX8ZPGFilter **output,
			     INT4              numZeros,
			     INT4              numPoles)
\end{verbatim}
\marginpar{\texttt{\tiny l.106}\\{\tiny CreateZPGFilter.c}}
\vspace{-0.1in}
\begin{verbatim}
void CreateCOMPLEX16ZPGFilter(Status             *stat,
			      COMPLEX16ZPGFilter **output,
			      INT4               numZeros,
			      INT4               numPoles)
\end{verbatim}


\subsubsection{Description}

These functions create an object \verb@**output@, of type
\verb@COMPLEX8ZPGFilter@ or \verb@COMPLEX16ZPGFilter@, having
\verb@numZeros@ zeros and \verb@numPoles@ poles.  The values of those
zeros and poles are not set by these routines (in general they will
start out as garbage).  The handle passed into the functions must be a
valid handle (i.e.\ \verb@output@$\neq$\verb@NULL@), but must not
point to an existing object (\i.e.\ \verb@*output@=\verb@NULL@).

\subsubsection{Algorithm}

\subsubsection{Uses}
\begin{verbatim}
LALMalloc()
CCreateVector()
ZCreateVector()
\end{verbatim}

\subsubsection{Notes}


\newpage
\subsection{Module \texttt{DestroyZPGFilter.c}}

Destroys ZPG filter objects.

\subsubsection{Prototypes}
\vspace{0.1in}
\marginpar{\texttt{\tiny l.63}\\{\tiny DestroyZPGFilter.c}}
\vspace{-0.1in}
\begin{verbatim}
void DestroyCOMPLEX8ZPGFilter(Status            *stat,
			      COMPLEX8ZPGFilter **input)
\end{verbatim}
\marginpar{\texttt{\tiny l.90}\\{\tiny DestroyZPGFilter.c}}
\vspace{-0.1in}
\begin{verbatim}
void DestroyCOMPLEX16ZPGFilter(Status             *stat,
			       COMPLEX16ZPGFilter **input)
\end{verbatim}


\subsubsection{Description}

These functions destroy an object \verb@**output@ of type
\verb@COMPLEX8ZPGFilter@ or \verb@COMPLEX16ZPGFilter@, and set
\verb@*output@ to \verb@NULL@.

\subsubsection{Algorithm}

\subsubsection{Uses}
\begin{verbatim}
LALFree()
CDestroyVector()
ZDestroyVector()
\end{verbatim}

\subsubsection{Notes}


\newpage
\subsection{Module \texttt{BilinearTransform.c}}

Transforms the complex frequency coordinate of a ZPG filter.

\subsubsection{Prototypes}
\vspace{0.1in}
\marginpar{\texttt{\tiny l.151}\\{\tiny BilinearTransform.c}}
\vspace{-0.1in}
\begin{verbatim}
void WToZCOMPLEX8ZPGFilter(Status            *stat,
			   COMPLEX8ZPGFilter *filter)
\end{verbatim}
\marginpar{\texttt{\tiny l.386}\\{\tiny BilinearTransform.c}}
\vspace{-0.1in}
\begin{verbatim}
void WToZCOMPLEX16ZPGFilter(Status             *stat,
			    COMPLEX16ZPGFilter *filter)
\end{verbatim}


\subsubsection{Description}

These functions perform an in-place bilinear transformation on an
object \verb@*filter@ of type \verb@<datatype>ZPGFilter@, transforming
from $w$ to $z=(1+iw)/(1-iw)$.  Care is taken to ensure that zeros and
poles at $w=\infty$ are correctly transformed to $z=-1$, and zeros and
poles at $w=-i$ are correctly transformed to $z=\infty$.  In addition
to simply relocating the zeros and poles, residual factors are also
incorporated into the gain of the filter (i.e.\ the leading
coefficient of the rational function).

\subsubsection{Algorithm}

The vectors \verb@filter->zeros@ and \verb@filter->poles@ only record
those zeros and poles that have finite value.  If one includes the
point $\infty$ on the complex plane, then a rational function always
has the same number of zeros and poles: a number \verb@num@ that is
the larger of \verb@z->zeros->length@ or \verb@z->poles->length@.  If
one or the other vector has a smaller length, then after the
transformation that vector will receive additional elements, with a
complex value of $z=-1$, to bring its length up to \verb@num@.
However, each vector will then \emph{lose} those elements that
previously had values $w=-i$, (which are sent to $z=\infty$,) thus
possibly decreasing the length of the vector.  These routines handle
this by simply allocating a new vector for the transformed data, and
freeing the old vector after the transformation.

When transforming a zero $w_k$ on the complex plane, one makes use of
the identity:
$$
(w - w_k) = -(w_k + i)\times\frac{z-z_k}{z+1} \; ,
$$
and similarly, when transforming a pole at $w_k$,
$$
(w - w_k)^{-1} = -(w_k + i)^{-1}\times\frac{z+1}{z-z_k} \; ,
$$
where $z=(1+iw)/(1-iw)$ and $z_k=(1+iw_k)/(1-iw_k)$.  If there are an
equal number of poles and zeros being transformed, then the factors of
$z+1$ will cancel; otherwise, the remaining factors correspond to the
zeros or poles at $z=-1$ brought in from $w=\infty$.  The factor
$(z-z_k)$ represents the transformation of the zero or pole at $w_k$.
The important factor to note, though, is the factor $-(w_k+i)^{\pm1}$.
This factor represents the change in the gain \verb@filter->gain@.
When $w_k=-i$, the transformation is slightly different:
$$
(w + i) = \frac{2i}{z+1} \; ;
$$
thus the gain correction factor is $2i$ (rather than 0) in this case.

The algorithms in this module computes and stores all the gain
correction factors before applying them to the gain.  The correction
factors are sorted in order of absolute magnitude, and are multiplied
together in small- and large-magnitude pairs.  In this way one reduces
the risk of overrunning the floating-point dynamical range during
intermediate calculations.

As a similar precaution, the routines in this module use the algorithm
discussed in the \verb@VectorOps@ package whenever they perform
complex division, to avoid intermediate results that mey be the
product of two large numbers.  When transforming $z=(1+iw)/(1-iw)$,
these routines also test for special cases (such as $w$ purely
imaginary) that have qualitatively significant results ($z$ purely
real), so that one doesn't end up with, for instance, an imaginary
part of $10^{-12}$ instead of 0.

\subsubsection{Uses}
\begin{verbatim}
I4CreateVector()
SCreateVector()
DCreateVector()
CCreateVector()
ZCreateVector()
I4DestroyVector()
SDestroyVector()
DDestroyVector()
CDestroyVector()
ZDestroyVector()
SHeapIndex()
DHeapIndex()
\end{verbatim}

\subsubsection{Notes}



\newpage
\section{Header \texttt{IIRFilter.h}}

Provides routines to make and apply IIR filters.

\subsection{Synopsis}
\begin{verbatim}
#include "IIRFilter.h"
\end{verbatim}

\noindent This header covers routines that create, destroy, and apply
generic time-domain filters, given by objects of type
\verb@<datatype>IIRFilter@, where \verb@<datatype>@ is either
\verb@REAL4@ or \verb@REAL8@.

An IIR (Infinite Impulse Response) filter is a generalized linear
causal time-domain filter, in which the filter output $y_n=y(t_n)$ at
any sampled time $t_n=t_0+n\Delta t$ is a linear combination of the
input $x$ \emph{and} output $y$ at previous sampled times:
$$
y_n = \sum_{k=0}^M c_k x_{n-k} + \sum_{l=1}^N d_l y_{n-l} \; .
$$
The coefficients $c_k$ are called the direct filter coefficients, and
the coefficients $d_l$ are the recursive filter coefficients.  The
filter order is the larger of $M$ or $N$, and determines how far back
in time the filter must look to determine its next output.  However,
the recursive nature of the filter means that the output can depend on
input arbitrarily far in the past; hence the name ``infinite impulse
response''.  Nonetheless, for a well-designed, stable filter, the
actual filter response to an impulse should diminish rapidly beyond
some characteristic timescale.

Note that nonrecursive FIR (Finite Impulse Response) filters are
considered a subset of IIR filters, having $N=0$.

For practical implementation, it is convenient to express the bilinear
equation above as two linear equations involving an auxiliary sequence
$w$:
$$
w_n = x_n + \sum_{l=1}^N d_l w_{n-l} \; ,
$$
$$
y_n = \sum_{k=0}^M c_k w_{n-k} \; .
$$
The equivalence of this to the first expression is not obvious, but
can be proven by mathematical induction.  The advantage of the
auxiliary variable representation is twofold.  First, when one is
feeding data point by point to the filter, the filter needs only
``remember'' the previous $M$ or $N$ (whichever is larger) values of
$w$, rather than remembering the previous $M$ values of $x$ \emph{and}
the previous $N$ values of $y$.  Second, when filtering a large stored
data vector, the filter response can be computed in place: one first
runs forward through the vector replacing $x$ with $w$, and then
backward replacing $w$ with $y$.

Although the IIR filters in these routines are explicitly real, one
can consider formally their complex response.  A sinusoidal input can
thus be written as $x_n=X\exp(2\pi ifn\Delta t)=z^n$, where $X$ is a
complex amplitude and $z=\exp(2\pi if\Delta t)$ is a complex
parametrization of the frequency.  By linearity, the output must also
be sinusoidal: $y_m=Y\exp(2\pi ifm\Delta t)=z^m$.  Putting these into
the bilinear equation, one can easily compute the filter's complex
transfer function:
$$
T(z) = \frac{Y}{X} = \frac{\sum_{k=0}^M c_k z^{-k}}
                      {1 - \sum_{l=1}^N d_l z^{-l}}
$$
This can be readily converted to and from the ``zeros, poles, gain''
representation of a filter, which expresses $T(z)$ as a factored
rational function of $z$.

It should also be noted that, in the routines covered by this header,
I have adopted the convention of including a redundant recursive
coefficient $d_0$, in order to make the indexing more intuitive.  For
formal correctness $d_0$ should be set to $-1$, although the filtering
routines never actually use this coefficient.


\subsection{Error conditions}
\begin{tabular}{|c|l|l|}
\hline
status & status      & Explanation \\
 code  & description & \\
\hline
\tt 1  & \tt Null pointer            & Missing a required pointer.           \\
\tt 2  & \tt Output already exists   & Can't allocate to a non-null pointer. \\
\tt 3  & \tt Memory allocation error & Could not allocate memory.            \\
\tt 4  & \tt Input has unpaired      & For real filters, complex poles or    \\
       & \tt nonreal poles or zeros  & zeros must come in conjugate pairs.   \\
\hline
\end{tabular}


\subsection{Structures}
\begin{verbatim}
<datatype>IIRFilter
\end{verbatim}

\noindent This structure stores the direct and recursive filter
coefficients, as well as the history of the auxiliary sequence $w$.
The length of the history vector gives the order of the filter.  The
fields are:

\begin{description}
\item[\texttt{CHAR *name}] A user-assigned name.

\item[\texttt{<datatype>Vector *directCoef}] The direct filter
  coefficients.

\item[\texttt{<datatype>Vector *recursCoef}] The recursive filter
  coefficients.

\item[\texttt{<datatype>Vector *history}] The previous values of $w$.
\end{description}

\newpage
\subsection{Module \texttt{CreateIIRFilter.c}}

Creates IIR filter objects.

\subsubsection{Prototypes}
\vspace{0.1in}
\marginpar{\texttt{\tiny l.127}\\{\tiny CreateIIRFilter.c}}
\vspace{-0.1in}
\begin{verbatim}
void CreateREAL4IIRFilter(Status            *stat,
			  REAL4IIRFilter    **output,
			  COMPLEX8ZPGFilter *input)
\end{verbatim}
\marginpar{\texttt{\tiny l.368}\\{\tiny CreateIIRFilter.c}}
\vspace{-0.1in}
\begin{verbatim}
void CreateREAL8IIRFilter(Status             *stat,
			  REAL8IIRFilter     **output,
			  COMPLEX16ZPGFilter *input)
\end{verbatim}


\subsubsection{Description}

These functions create an object \verb@**output@ of type
\verb@<datatype>IIRFilter@, where \verb@<datatype>@ is \verb@REAL4@ or
\verb@REAL8@.  The filter coefficients are computed from the zeroes,
poles, and gain of an input object \verb@*input@ of type
\verb@COMPLEX8ZPGFilter@ or \verb@COMPLEX16ZPGFilter@, respectively.
The ZPG filter should express the factored transfer function in the
$z=\exp(2\pi if)$ plane.  Initially the output handle must be a valid
handle (\verb@output@$\neq$\verb@NULL@) but should not point to an
existing object (\verb@*output@=\verb@NULL@)

\subsubsection{Algorithm}

An IIR filter is a real time-domain filter, which imposes certain
constraints on the zeros, poles, and gain of the filter transfer
function.  The function \verb@Create<datatype>IIRFilter()@ deals with
the constraints either by aborting if they are not met, or by
adjusting the filter response so that they are met.  In the latter
case, warning messages will be issued if the external parameter
\verb@debuglevel@ is 1 or more.  The specific constraints, and how
they are dealt with, are as follows:

First, the filter must be \emph{causal}; that is, the output at any
time can be generated entirely from the input at previous times.  In
practice this means that the number of (finite) poles in the $z$ plane
must equal or exceed the number of (finite) zeros.  If this is not the
case, \verb@Create<datatype>IIRFilter()@ will add additional poles at
$z=0$.  Effectively this is just adding a delay to the filter response
in order to make it causal.

Second, the filter should be \emph{stable}, which means that all poles
should be located on or within the circle $|z|=1$.  This is not
enforced by \verb@Create<datatype>IIRFilter()@, which can be used to
make unstable filters; however, warnings will be issued if
\verb@debuglevel@ is 1 or more.  (In some sense the first condition is
a special case of this one, since a transfer function with more zeros
than poles actually has corresponding poles at infinity.)

Third, the filter must be \emph{physically realizable}; that is, the
transfer function should expand to a rational function of $z$ with
real coefficients.  Necessary and sufficient conditions for this are
that the gain be real, and that all zeros and poles either be real or
come in complex conjugate pairs.  The routine
\verb@Create<datatype>IIRFilter()@ enforces this by using only the
real part of the gain, and only the real or positive-imaginary zeros
and poles; it assumes that the latter are paired with
negative-imaginary conjugates.  The routine will abort if this
assumption results in a change in the given number of zeros or poles,
but will otherwise simply modify the filter response.  This allows
\verb@debuglevel@=0 runs to proceed without lengthy and usually
unnecessary error trapping; when \verb@debuglevel@ is 1 or more, the
routine checks to make sure that each nonreal zero or pole does in
fact have a complex-conjugate partner.

\subsubsection{Uses}
\begin{verbatim}
debuglevel
LALMalloc()
SCreateVector()
DCreateVector()
\end{verbatim}

\subsubsection{Notes}


\newpage
\subsection{Module \texttt{DestroyIIRFilter.c}}

Destroys IIR filter objects.

\subsubsection{Prototypes}
\vspace{0.1in}
\marginpar{\texttt{\tiny l.64}\\{\tiny DestroyIIRFilter.c}}
\vspace{-0.1in}
\begin{verbatim}
void DestroyREAL4IIRFilter(Status         *stat,
			   REAL4IIRFilter **input)
\end{verbatim}
\marginpar{\texttt{\tiny l.92}\\{\tiny DestroyIIRFilter.c}}
\vspace{-0.1in}
\begin{verbatim}
void DestroyREAL8IIRFilter(Status         *stat,
			   REAL8IIRFilter **input)
\end{verbatim}


\subsubsection{Description}

These functions destroy an object \verb@**input@ of type
\texttt{REAL4IIRFilter} or \texttt{REAL8IIRFilter}, and set
\verb@*input@ to \verb@NULL@.

\subsubsection{Algorithm}

\subsubsection{Uses}
\begin{verbatim}
void LALFree()
void SDestroyVector()
void DDestroyVector()
\end{verbatim}

\subsubsection{Notes}


\newpage
\subsection{Module \texttt{IIRFilter.c}}

Computes an instant-by-instant IIR filter response.

\subsubsection{Prototypes}
\vspace{0.1in}
\marginpar{\texttt{\tiny l.64}\\{\tiny IIRFilter.c}}
\vspace{-0.1in}
\begin{verbatim}
void IIRFilterREAL4(Status         *stat,
		    REAL4          *output,
		    REAL4          input,
		    REAL4IIRFilter *filter)
\end{verbatim}
\marginpar{\texttt{\tiny l.117}\\{\tiny IIRFilter.c}}
\vspace{-0.1in}
\begin{verbatim}
void IIRFilterREAL8(Status         *stat,
		    REAL8          *output,
		    REAL8          input,
		    REAL8IIRFilter *filter)
\end{verbatim}
\marginpar{\texttt{\tiny l.170}\\{\tiny IIRFilter.c}}
\vspace{-0.1in}
\begin{verbatim}
REAL4 SIIRFilter(REAL4          x,
		 REAL4IIRFilter *filter)
\end{verbatim}
\marginpar{\texttt{\tiny l.210}\\{\tiny IIRFilter.c}}
\vspace{-0.1in}
\begin{verbatim}
REAL8 DIIRFilter(REAL8          x,
		 REAL8IIRFilter *filter)
\end{verbatim}


\subsubsection{Description}

These functions pass a time-domain datum to an object \verb@*filter@
of type \verb@REAL4IIRFilter@ or \verb@REAL8IIRFilter@, and return the
filter response.  This is done using the auxiliary data series
formalism described in the header \verb@IIRFilter.h@.

There are two pairs of routines in this module.  The functions
\verb@IIRFilterReal4()@ and \verb@IIRFilterREAL8()@ conform to the LAL
standard, with status handling and error trapping; the input datum is
passed in as \verb@input@ and the response is returned in
\verb@*output@.  The functions \verb@SIIRFilter()@ and
\verb@DIIRFilter()@ are non-standard lightweight routines, which may
be more suitable for multiple callings from the inner loops of
programs; they have no status handling or error trapping.  The input
datum is passed in by the variable \verb@x@, and the response is
returned through the function's return statement.

\subsubsection{Algorithm}

\subsubsection{Uses}

\subsubsection{Notes}


\newpage
\subsection{Module \texttt{IIRFilterVector.c}}

Applies an IIR filter to a data stream.

\subsubsection{Prototypes}
\vspace{0.1in}
\marginpar{\texttt{\tiny l.61}\\{\tiny IIRFilterVector.c}}
\vspace{-0.1in}
\begin{verbatim}
void IIRFilterREAL4Vector(Status         *stat,
			  REAL4Vector    *vector,
			  REAL4IIRFilter *filter)
\end{verbatim}
\marginpar{\texttt{\tiny l.147}\\{\tiny IIRFilterVector.c}}
\vspace{-0.1in}
\begin{verbatim}
void IIRFilterREAL8Vector(Status         *stat,
			  REAL8Vector    *vector,
			  REAL8IIRFilter *filter)
\end{verbatim}


\subsubsection{Description}

These functions apply a generic time-domain filter given by an object
\verb@*filter@ of type \verb@REAL4IIRFilter@ or \verb@REAL8IIRFilter@
to a list \verb@*vector@ of data representing a time series.  This is
done in place using the auxiliary data series formalism described in
\verb@IIRFilter.h@, so as to accomodate potentially large data series.
To filter a piece of a larger dataset, the calling routine may pass a
vector structure whose data pointer and length fields specify a subset
of a larger vector.

\subsubsection{Algorithm}

\subsubsection{Uses}
\begin{verbatim}
LALMalloc()
LALFree()
\end{verbatim}

\subsubsection{Notes}


\newpage
\subsection{Module \texttt{IIRFilterVectorR.c}}

Applies a time-reversed IIR filter to a data stream.

\subsubsection{Prototypes}
\vspace{0.1in}
\marginpar{\texttt{\tiny l.61}\\{\tiny IIRFilterVectorR.c}}
\vspace{-0.1in}
\begin{verbatim}
void IIRFilterREAL4VectorR(Status         *stat,
			   REAL4Vector    *vector,
			   REAL4IIRFilter *filter)
\end{verbatim}
\marginpar{\texttt{\tiny l.120}\\{\tiny IIRFilterVectorR.c}}
\vspace{-0.1in}
\begin{verbatim}
void IIRFilterREAL8VectorR(Status         *stat,
			   REAL8Vector    *vector,
			   REAL8IIRFilter *filter)
\end{verbatim}


\subsubsection{Description}

These functions apply a generic time-domain filter \verb@*filter@ to a
time series \verb@*vector@, as with the routines
\verb@IIRFilterREAL4Vector()@ and \verb@IIRFilterREAL8Vector()@, but
do so in a time-reversed manner.  By successively applying normal and
time-reversed IIR filters to the same data, one squares the magnitude
of the frequency response while canceling the phase shift.  This can
be significant when one wishes to preserve phase correlations across
wide frequency bands.

\subsubsection{Algorithm}

Because these filter routines are inherently acausal, the
\verb@filter->history@ vector is meaningless and unnecessary.  These
routines neither use nor modify this data array.

\subsubsection{Uses}

\subsubsection{Notes}



\newpage
\section{Header \texttt{BandPassTimeSeries.h}}

Provides routines to low- or high-pass filter a time series.

\subsection{Synopsis}
\begin{verbatim}
#include "BandPassTimeSeries.h"
\end{verbatim}

\noindent This header covers routines that apply a time-domain low- or
high-pass filter to a data series of type \verb@<datatype>TimeSeries@.
Further documentation is given in the individual routines' modules.


\subsection{Error conditions}
\begin{tabular}{|c|l|l|}
\hline
status & status                    & Explanation                           \\
 code  & description               &                                       \\
\hline
\tt 1  & \tt Null pointer          & Missing a required pointer.           \\
\tt 2  & \tt Bad filter parameters & Filter creation parameters outside of \\
       &                           & acceptable ranges.                    \\
\hline
\end{tabular}


\subsection{Structures}

\begin{verbatim}
struct PassBandParamStruc
\end{verbatim}

\noindent This structure stores data used for constructing a low- or
high-pass filter: either the order and characteristic frequency of the
filter, or the frequencies and desired attenuations at the ends of
some transition band.  In the latter case, a nonzero filter order
parameter n indicates a maximum allowed order.  The fields are:

\begin{description}
\item[\texttt{CHAR *name}] A user-assigned name.

\item[\texttt{INT4 n}] The maximum desired filter order (actual order
  may be less if specified attenuations do not require a high order).

\item[\texttt{REAL8 f1}, \texttt{f2}] The reference frequencies of the
  transition band.

\item[\texttt{REAL8 a1}, \texttt{a2}] The minimal desired attenuation
  factors at the reference frequencies.
\end{description}

\newpage
\subsection{Module \texttt{ButterworthTimeSeries.c}}

Applies a low- or high-pass Butterworth filter to a time series.

\subsubsection{Prototypes}
\vspace{0.1in}
\marginpar{\texttt{\tiny l.196}\\{\tiny ButterworthTimeSeries.c}}
\vspace{-0.1in}
\begin{verbatim}
void ButterworthREAL4TimeSeries(Status             *stat,
				REAL4TimeSeries    *series,
				PassBandParamStruc *params)
\end{verbatim}
\marginpar{\texttt{\tiny l.322}\\{\tiny ButterworthTimeSeries.c}}
\vspace{-0.1in}
\begin{verbatim}
void ButterworthREAL8TimeSeries(Status             *stat,
				REAL8TimeSeries    *series,
				PassBandParamStruc *params)
\end{verbatim}


\subsubsection{Description}

These routines perform an in-place time-domain band-pass filtering of
a data sequence \verb@*series@, using a Butterworth filter generated
from parameters \verb@*params@.  The routines construct a filter with
the square root of the desired amplitude response, which it then
applied to the data once forward and once in reverse.  This gives the
full amplitude response with little or no frequency-dependent phase
shift.

\subsubsection{Algorithm}

The frequency response of a Butterworth low-pass filter is easiest to
express in terms of the transformed frequency variable $w=\tan(\pi
f\Delta t)$, where $\Delta t$ is the sampling interval (i.e.\
\verb@series->deltaT@).  In this parameter, then, the \emph{power}
response (attenuation) of the filter is:
$$
|R|^2 = a = \frac{1}{1+(w/w_c)^{2n}} \; ,
$$
where $n$ is the filter order and $w_c$ is the characteristic
frequency.  Similarly, a Butterworth high-pass filter is given by
$$
|R|^2 = a = \frac{1}{1+(w_c/w)^{2n}} \; .
$$
If one is given a filter order $n$, then the characteristic frequency
can be determined from the attenuation at some any given frequency.
Alternatively, $n$ and $w_c$ can both be computed given attenuations
at two different frequencies.

Frequencies in \verb@*params@ are assumed to be real frequencies $f$
given in the inverse of the units used for the sampling interval
\verb@series->deltaT@.  In order to be used, the pass band parameters
must lie in the ranges given below; if a parameter lies outside of its
range, then it is ignored and the filter is calculated from the
remaining parameters.  If too many parameters are missing, the routine
will fail.  The acceptable parameter ranges are:

\begin{description}
\item[\texttt{params->nMax}]   = 1, 2, $\ldots$
\item[\texttt{params->f1}, \texttt{f2}] $\in
  (0,\{2\times\verb@series->deltaT@\}^{-1}) $
\item[\texttt{params->a1}, \texttt{a2}] $\in (0,1) $
\end{description}

If both pairs of frequencies and amplitudes are given, then \verb@a1@,
\verb@a2@ specify the minimal requirements on the attenuation of the
filter at frequencies \verb@f1@, \verb@f2@.  Whether the filter is a
low- or high-pass filter is determined from the relative sizes of
these parameters.  In this case the \verb@nMax@ parameter is optional;
if given, it specifies an upper limit on the filter order.  If the
desired attenuations would require a higher order, then the routine
will sacrifiece performance in the stop band in order to remain within
the specified \verb@nMax@.

If one of the frequency/attenuation pairs is missing, then the filter
is computed using the remaining pair and \verb@nMax@ (which must be
given).  The filter is taken to be a low-pass filter if \verb@f1@,
\verb@a1@ are given, and high-pass if \verb@f2@, \verb@a2@ are given.
If only one frequency and no corresponding attenuation is specified,
then it is taken to be the characteristic frequency (i.e. the
corresponding attenuation is assumed to be 1/2).  If none of these
conditions are met, the routine will return an error.

Once an order $n$ and characteristic frequency $w_c$ are known, the
zeros and poles of a ZPG filter are readily determined.  A stable,
physically realizable Butterworth filter will have $n$ poles evenly
spaced on the upper half of a circle of radius $w_c$; that is,
$$
R = \frac{(-iw_c)^n}{\prod_{k=0}^{n-1}(w - w_c e^{2\pi i(k+1/2)/n})}
$$
for a low-pass filter, and
$$
R = \frac{w^n}{\prod_{k=0}^{n-1}(w - w_c e^{2\pi i(k+1/2)/n})}
$$
for a high-pass filter.  By choosing only poles on the upper-half
plane, one ensures that after transforming to $z$ the poles will have
$|z|<1$.

Although higher orders $n$ would appear to produce better (i.e.\
sharper) filter responses, one rapidly runs into numerical errors, as
one ends up adding and subtracting $n$ large numbers to obtain small
filter responses.  One way around this is to break the filter up into
several lower-order filters.  The routines in this module do just
that.  Poles are paired up across the imaginary axis, (and combined
with pairs of zeros at $w=0$ for high-pass filters,) to form $[n/2]$
second-order filters.  If $n$ is odd, there will be an additional
first-order filter, with one pole at $w=iw_c$ (and one zero at $w=0$
for a high-pass filter).

Each ZPG filter in the $w$-plane is first transformed to the $z$-plane
by a bilinear transformation, and is then used to construct a
time-domain IIR filter.  Each filter is then applied to the time
series.  As mentioned in the description above, the filters are
designed to give an overall amplitude response that is the square root
of the desired attenuation; however, each time-domain filter is
applied to the data stream twice: once in the normal sense, and once
in the time-reversed sense.  This gives the full attenuation with very
little frequency-dependent phase shift.

\subsubsection{Uses}
\begin{verbatim}
debuglevel
LALPrintError()
CreateREAL4IIRFilter()
CreateREAL8IIRFilter()
DestroyREAL4IIRFilter()
DestroyREAL8IIRFilter()
CreateCOMPLEX8ZPGFilter()
CreateCOMPLEX16ZPGFilter()
DestroyCOMPLEX8ZPGFilter()
DestroyCOMPLEX16ZPGFilter()
IIRFilterREAL4Vector()
IIRFilterREAL8Vector()
IIRFilterREAL4VectorR()
IIRFilterREAL8VectorR()
\end{verbatim}

\subsubsection{Notes}


\newpage
\subsection{Program \texttt{BandPassTest.c}}

Tests time-domain high- and low-pass filters.

\subsubsection{Usage}
\begin{verbatim}
BandPassTest [-o [outfile]] [-d [debug-level]]
\end{verbatim}

\subsubsection{Description}

This program generates a time series with an impulse in it, and passes
it through a time-domain low-pass or high-pass filter.  By default,
running this program with no arguments simply tests the subroutines,
producing no output.  All filter parameters are set from
\verb@#define@d constants.  The program returns a value of 0 upon
successful completion, 1 if any of its function calls failed, or 2 if
output file creation failed.

The \verb@-o@ flag tells the program to print the impulse response to
a data file; if \verb@outfile@ is not specified, it will write to the
file \verb@out.dat@.  The \verb@-d@ option increases the default debug
level from 0 to 1, or sets it to the specified value
\verb@debug-level@.

\subsubsection{Algorithm}

\subsubsection{Uses}
\begin{verbatim}
debuglevel
SCreateVector()
SDestroyVector()
ButterworthREAL4TimeSeries()
LALPrintError()
\end{verbatim}

\subsubsection{Notes}




\newpage\input{LPCH}
\newpage\input{LPCC}
