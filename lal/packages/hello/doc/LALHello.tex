\documentclass{article}
\begin{document}

\section{LALHello}

\subsection{Purpose}

A simple example routine: prints ``hello, LSC!''

\subsection{Synopsis}

\begin{verbatim}
void LALHello (Status *status, const CHAR *fileName);
\end{verbatim}

\subsection{Description}

The routine \texttt{LALHello()} prints the message ``hello, LSC!'' to the file
specified by \texttt{fileName}, or to \texttt{stdout} if this is NULL.  (The
actual printing is handled by a subroutine \texttt{LALPrintMessage()}, which
is static to the file \texttt{LALHello.c}.)

\subsection{Operating Instructions}

\begin{verbatim}
static Status  status;

/* print to file myfile */
LALHello (&status, "myfile");

/* print to stdout */
LALHello (&status, NULL);
\end{verbatim}


\subsubsection{Arguments}

\begin{itemize}
\item \texttt{status} is a universal status structure.  Its contents are
assigned by the functions.
\item \texttt{fileName} is a character string representing the file name to
which the message will be printed.
\end{itemize}


\subsubsection{Options}

\subsubsection{Error conditions}

These functions all set the universal status structure on return.
Error conditions are described in the following table.

\begin{table}
\begin{tabular}{|r|l|p{2in}|}\hline
status  & status          & Description\\
code    & description     & \\\hline
\verb+LALHELLO_EOPEN  1+ & Could not open file &
File could not be opened---it may be write-protected\\
\verb+LALHELLO_EWRITE 2+ & Error in writing to file &
An error occurred while printing to the file\\
\verb+LALHELLO_ECLOSE 4+ & Error in closing file &
An error occurred while closeing the file\\
\verb+LALHELLO_EFLUSH 8+ & Error in flushing stdout &
An error occurred in trying to flush stdout\\
\hline
\end{tabular}
\caption{Error conditions for the LALHello() function}\label{tbl:CV}
\end{table}

\subsection{Algorithms}

\subsection{Accuracy}

\subsection{Tests}

The program \texttt{LALHelloTest} tests the routine \texttt{LALHello()} by
first printing the message to \texttt{stdout} and then by trying to print the
message to a write-protected file \texttt{protected}.  The second attempt
should fail.

\subsection{Uses}

\begin{itemize}
\item\texttt{LALFopen()}
\item\texttt{LALFclose()}
\end{itemize}

\subsection{Notes}

\subsection{References}

\end{document}
