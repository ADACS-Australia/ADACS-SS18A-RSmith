\documentclass[12pt]{article}
\usepackage{amsmath}

\begin{document}
\huge
\begin{center}
InspiralParameterCalc.c
\end{center}
\normalsize
\vspace{10mm}

\section{Purpose}

The code \texttt{InspiralParameterCalc.c} takes as its input one pair of parameters which may be chosen from the following set of five: $(m_{1}, m_{2}, m, \eta, \mu)$, where $m_{1}$ and $m_{2}$ are the masses of the two compact objects, $m=m_{1}+m_{2}$ is their combined mass, $\eta=m_{1}m_{2}/(m_{1}+m_{2})^{2}$ is the symmetric mass ratio, and $\mu=m_{1}m_{2}/(m_{1}+m_{2})$.

Whichever pair of parameters is given to the function as an input, the function calculates the other three. It also calculates the Newtonian chirp time $\tau_{N}$, the first post--Newtonian chirp time $\tau_{P^{1}N}$, the 1.5 order post--Newtonian chirp time $\tau_{P^{1.5}N}$, the second order post--Newtonian chirptime $\tau_{P^{2}N}$ and the chirp mass $\mathcal{M}$ which is defined as $\mathcal{M}=(\mu^{3} m^{2})^{1/5}$.

The chirp times are related to the masses of the stars and $f_{a}$ in the following way:
\begin{equation}
\tau_{N} = \frac{5}{256} \eta^{-1} m^{-5/3} (\pi f_{a})^{-8/3} \,,
\end{equation}

\begin{equation}
\tau_{P^{1}N} = \frac{3715+4620 \eta}{64512 \eta m (\pi f_{a})^{2}} \,,
\end{equation}

\begin{equation}
\tau_{P^{1.5}N} = \frac{\pi}{8 \eta m^{2/3} (\pi f_{a})^{5/3}}
\end{equation}

and
\begin{equation}
\tau_{P^{2}N} = \frac{5}{128 \eta m^{1/3} (\pi f_{a})^{4/3}} \left[ \frac{3058673}{1016064} + \frac{5429}{1008} \eta + \frac{617}{144} \eta^{2} \right] \,.
\end{equation}

These formulas show that an additional parameter $f_{a}$ is needed. This is the frequency at which the detectors' noise curve rises steeply (the seismic limit).

\section{Algorithms}

The code uses no algorithms


\section{Arguments}

The function header is of the form:

\vspace{5mm}

\begin{tabular}{ll}
void \texttt{InspiralParameterCalc}&(\texttt{Status $\ast$status},     \\
                                   &\texttt{InspiralParamsOutput $\ast$output}, \\
                                   &\texttt{InspiralParamsInput $\ast$params})
\end{tabular}

\vspace{5mm}

The structure which is of type \texttt{Status}, which is pointed to by the pointer \texttt{status} writes information to the screen should the code encounter
a problem. The output structure is of the form \texttt{InspiralParamsOutput} and is pointed to by the pointer \texttt{output}.
The inputs needed come from the input structure which is of type \texttt{InspiralParamsInput}, and which is pointed to by the pointer \texttt{params}.

The output structure has the form

\vspace{5mm}

\begin{tabular}{ll}
\texttt{typedef struct} & \texttt{tagInspiralParamsOutput} \{ \\
                        & \texttt{REAL8 m1;}  \\
                        & \texttt{REAL8 m2;}  \\
                        & \texttt{REAL8 totalMass;}  \\
                        & \texttt{REAL8 mu;}  \\
                        & \texttt{REAL8 eta;}  \\
                        & \texttt{REAL8 chirpMass;}  \\
                        & \texttt{REAL8 tauZero;}  \\
                        & \texttt{REAL8 tauOne;}  \\
                        & \texttt{REAL8 tauOneFive;}  \\
                        & \texttt{REAL8 tauTwo;}  \\
                        & \texttt{REAL8 tauC;}  \\
                        & \} \texttt{InspiralParamsOutput;}
\end{tabular}

\vspace{5mm}

The parameters which are represented by these input are as follows: \texttt{m1} and \texttt{m2} are the masses of the compact objects in solar masses, \texttt{totalMass} is their combined mass $m=m_{1}+m_{2}$, \texttt{mu} is the reduced mass $\mu=m_{1}m_{2}/(m_{1}+m_{2})$, \texttt{eta} is the symmetric mass ratio $\eta=m_{1}m_{2}/(m_{1}+m_{2})^{2}$. \texttt{chirpMass} is the chirpmass $\mathcal{M}=(\mu^{3} m^{2})^{1/5}$, and \texttt{tauZero, tauOne, tauOneFive, tauTwo}  and \texttt{tauC} are the chirptimes $\tau_{N}$, $\tau_{P^{1}N}$, $\tau_{P^{1.5}N}$, $\tau_{P^{2}N}$ and $\tau_{c} = \tau_{N} + \tau_{P^{1}N} - \tau_{P^{1.5}N} + \tau_{P^{2}N}$ respectively.
The range of values which we allow is as follows: The smallest total mass of any system $m_{min}$ allowed is 0.4~$\rm{M_{\odot}}$, and the maximum total mass allowed is $m_{max}=100~\rm{M_{\odot}}$. The symmetric mass ratio $\eta$ should lie in the range $0 \leq \eta \leq 1/4$ and $\mu$ should always be positive. 

The input structure is of the form

\vspace{5mm}

\begin{tabular}{ll}
\texttt{typedef struct} & \texttt{tagInspiralParamsInput} \{ \\
                        & \texttt{REAL8 m1;} \\
                        & \texttt{REAL8 m2;}  \\
                        & \texttt{REAL8 totalMass;} \\
                        & \texttt{REAL8 mu;}  \\
                        & \texttt{REAL8 eta;}  \\
                        & \texttt{REAL8 fLower;}  \\
                        & \texttt{InputMasses massChoice;}  \\
                        & \} \texttt{InspiralParamsInput;}
\end{tabular}

\vspace{5mm}

Here \texttt{fLower} represents $f_{a}$ in the above equations.

The parameter \texttt{MassChoice} is of type \texttt{enum InputMasses}, which determines which pair of input masses the user has defined. This typedef is as follows:

\vspace{5mm}

\begin{tabular}{ll}
\texttt{typedef enum} & \{ \\
                      & \texttt{m1Andm2,} \\
                      & \texttt{totalMassAndEta,}  \\
                      & \texttt{totalMassAndMu} \\
                      & \} \texttt{InputMasses;}
\end{tabular}

\vspace{5mm}



\section{Operating Instructions}

When this function is called, the input structure member \texttt{fLower} must be defined. From the remaining five input members (\texttt{m1,m2,totalMass,mu,eta}), it is not necessary to define all five. Instead, only  one of the following pairs needs to be defined: (\texttt{m1,m2}), (\texttt{totalMass,mu}), or (\texttt{totalMass,eta}). The pair which the user decides to use is as determined by choosing one of the following inputs \\ \texttt{params.massChoice=m1Andm2}, \\ \texttt{params.massChoice=totalMassAndMu} or \\ \texttt{params.massChoice=totalMassAndEta}.

Given these inputs, the function is able to generate all the other parameters it needs in order to compute the waveform.

Here is an example of a code fragment which shows how the members of the input structure (which is pointed to by the pointer \texttt{paramCalc}) are initialized, and how the function is then called. In this example, we have chosen to define the pair \texttt{m1} and \texttt{m2} as inputs.


\vspace{5mm}

\noindent
\begin{verbatim}
/* Declare the structures to be used  */
\end{verbatim}
\texttt{InspiralParamsInput paramCalc;} \\
\texttt{InspiralParamsOutput out1;} \\
\texttt{Status status;} \\
\begin{verbatim}
/* Initialize the inputs  */
\end{verbatim}
\texttt{paramCalc.m1} = 10.0 \\
\texttt{paramCalc.m2} = 10.0 \\
\texttt{paramCalc.fLower} = 40.0 \\
\texttt{paramCalc.massChoice} = m1Andm2 \\

\begin{verbatim}
/* Call the function  */
\end{verbatim}
\texttt{InspiralParameterCalc (\&status, \&out1, \&paramCalc);}
\begin{verbatim}
/* Write a member of the output structure to the screen  */ 

  fprintf(stderr,"the chirpmass is %e\n",out1.chirpMass); 
\end{verbatim}


Inside the function \texttt{InspiralParameterCalc()}, error checks are made upon its arguments, using the ASSERT macro. Because each of the arguments to the function involves a pointer being passed to the function (e.g.\ \texttt{out1, paramCalc}), we first of all check that each of the pointers are not NULL pointers.

Inside the function \texttt{InspiralParameterCalc()}, this looks like:

\vspace{5mm}

\begin{tabular}{ll}
void \texttt{InspiralParameterCalc}&(\texttt{Status $\ast$status},     \\
                                   &\texttt{InspiralParamsOutput $\ast$output}, \\
                                   &\texttt{InspiralParamsInput $\ast$params})
\end{tabular}

\vspace{5mm}

\begin{tabular}{ll}
ASSERT & (output!=NULL,  \\
       &  status,    \\
       &  INSPIRALPARAMETERCALCALC\_ENULL, \\
       &  INSPIRALPARAMETERCALCALC\_MSGENULL1);
\end{tabular}

\vspace{5mm}

This above example checks whether the pointer \texttt{output} is a NULL pointer or not. If it is a NULL pointer, then an error message which is defined by the character string \texttt{INSPIRALPARAMETERCALCALC\_MSGENULL1} is sent to the screen using the \texttt{StatusHandler} function.


\section{Options}

The options available to the user are the choice of input parameters. As explained above, from the following list of five $(m_{1},m_{2},m,\mu,\eta)$, the user need only specify any one of the following pairs $(m_{1},m_{2})$, $(m,\eta)$ or $(m,\mu)$. 

\section{Accuracy}

All variables are decalred to be REAL8, which means that they are double precision.
Each double precision variable has an approximate precision of 15 significant figures.



\section{Error conditions}

We first of all check that each of the pointers passed to the function \\ \texttt{InspiralParameterCalc()} as an argument , i.e.\ \texttt{Status}, \texttt{output} and \texttt{params}, are not NULL pointers. If any of them are NULL, then an error message is sent to the screen.

If the user has not defined one of the required pairs of input parameters $(m_{1},m_{2})$, $(m,\eta)$ or $(m,\mu)$, then an appropriate error message is written to the screen.



\section{Tests}

This function is tested as a part of the test of the function TappRpnTdomFreq(). For a discussion of this test, see the documentation for that function.

\section{Uses}

This function does not call any other functions.


\section{References}
For a fuller description of how this function is used in the generation of an inspiral waveform, see the documentation for the function \texttt{TappRpnTdomFreq()}.
The nomenclature adopted is the same as that used in Sathyaprakash, PRD, 50, R7111, 1994, which may be consulted for further details.









\end{document}
