\chapter{Package \texttt{inspiral}}

This is a module which generates inspiral waveforms. The type of waveform generated
is determined by specifying the value of the {\tt enum}s {\tt Approximant} and {\tt Order.}
All {\tt Approximants} generate the restricted post-Newtonian (PN) waveform given by (we
use units in which $c=G=1$):
\begin{equation}
h(t) = A v^2 \cos(\phi(t)),
\end{equation}
where the coefficient $A$ is set equal to 1 for non-spinning binaries
and to the appropriate modulation function for spinning binaries (see below),
$v$ is the PN expansion parameter, and $\phi(t)$ is the appropriately
defined phasing formula which is either an explicit or an implicit
function of time.  In what follows we summarize the basic formulas used
in generating the waveforms. As we shall see there are are a number of
ways in which these waveforms can be generated.  In the {\tt inspiral} package
these are controlled by two {\tt enums:} The first {\tt enum,} called {\tt Order,}
controls the PN order to which the various quantities are defined and
the second {\tt enum,} called {\tt Approximant,} controls the approximant used
in generating the waveforms.

\section{Taylor and Pade approximants}
Theoretical calculations have given us PN expansions
(i.e.\ a series in terms of ascending powers of $v$) of an energy
function $E(x=v^2)$ and a gravitational wave (GW) luminosity
function $\mathcal{F}(v):$
\begin{equation}
E(x) = E_N \sum_n E_k x^k,\ \ \ \ {\cal F}(v) = {\cal F}_N \sum_j {\cal F}_j v^j.
\end{equation}
One can use kinematical equations $dt = (dt/dE)(dE/dv) dv,$ and
$d\phi/dt = 2\pi F$ and the energy balance equation relating the
luminosity in gravitational waves to the rate of change of
binding energy ${\cal F} = - dE/dt$ , to obtain a phasing formula \cite {dis3}:
\begin{eqnarray}
t(v) & = & t_{\rm ref} + m \int_v^{v_{\rm ref}} \,
\frac{E'(v)}{{\cal F}(v)} \, dv, \nonumber \\
\phi (v) & = & \phi_{\rm ref} + 2 \int_v^{v_{\rm ref}}  v^3 \,
\frac{E'(v)}{{\cal F}(v)} \, dv,
\label{InspiralWavePhasingFormula}
\end{eqnarray}
where $E'(v)=dE/dv,$ $v=(\pi m F)^{1/3}$ is an invariantly defined velocity,
$F$ is the instantaneous GW frequency, and $m$ is the total mass
of the binary.  There are basically three ways of solving the problem:
\begin{enumerate}
\item Leave  $E^{\prime}(v)/\mathcal{F}(v)$ as it is and integrate the
equations numerically. Using standard PN expansions for the
energy and flux functions one generates the {\tt Approximant} called \texttt{TaylorT1.}
If instead one uses the P-approximant for the energy and flux functions
\cite{dis3,dis1} then one generates the approximant called \texttt{PadeT1.}
In reality, it is computationally cheaper to use two ordinary differential
equations instead of the integrals. These are:
\begin{equation}
\frac{dv}{dt} = - \frac{\mathcal{F}(v)}{m E^{\prime}(v)},\ \ \ \
\frac{d \phi(t)}{dt} = \frac{2v^{3}}{m}.
\label{eq:ode2}
\end{equation}
These are implemented in modules {\tt LALInspiralWave1.}

\item Re-expand $E^{\prime}(v)/\mathcal{F}(v)$ in a Taylor expansion
in which case the integrals can be solved analytically to obtain a
{\it parametric} representation of the phasing formula in terms of
polynomial expressions in the auxiliary variable $v$
\begin {eqnarray}
\phi(v)&=& \phi_{\rm ref} +
\phi^v_N (v)\sum_{k=0}^{n} {\phi}^v_k v^k, \nonumber\\
t(v)&=& t_{\rm ref} +t_N(v) \sum_{k=0}^{n} {t}_k v^k,
\label{eq:InspiralWavePhase2}
\end {eqnarray}
This corresponds to \texttt{TaylorT2} in the enum \texttt{Approximant} \cite{dis3}.
These are implemented in modules {\tt LALInspiralWave2.}

\item The second of the polynomials in Eq.~(\ref{eq:InspiralWavePhase2}) can
be inverted and the resulting polynomial for $v$ in terms of
$t$ can be substituted in $\phi(v)$ to arrive at an explicit  time-domain
phasing formula
\begin{equation}
\phi(t)=\phi_{\rm ref}+\phi_N^t \sum_{k=0}^{n}
{\phi}^t_k\theta^k, \ \ \
F(t)= F_N \sum_{k=0}^{n} {F}_k \theta^k,
\label{eq:InspiralWavePhase3}
\end{equation}
where $\theta=[\eta (t_{\rm ref}-t)/(5m)]^{-1/8}$,
$F \equiv d \phi/ 2 \pi dt =v^3/(\pi m)$ is the instantaneous GW frequency
and $\eta=m_1 m_2/m^2$ is the symmetric mass ratio.
This corresponds to the {\tt Approximant} \texttt{TaylorT3}  \cite{BDIWW,BIWW,dis3}.
These are implemented in modules {\tt LALInspiralWave3.}
\end{enumerate}
The expansion coefficients in PN expansions of the various physical
quantities are summarized in Tables \ref{table:energy} and \ref{table:flux}.

\begin {table}[h]
\caption {Taylor coefficients of the energy functions
$E_{T_n}(x) = E_N \sum_{k=0} E_k x^k$ and $e_{T_n}(x) = e_N \sum_{k=0} e_k x^k,$
$e_{P_n}(x) = e_N/(1+c_1x/(1 + c_2x / (1 + c_3x/ \ldots )))$ and the corresponding
 location of the lso and pole.
As there are no terms of order $v^{2k+1}$ we have exceptionally chosen
(for this Table only) the expansion
parameter to be $x\equiv v^2$ rather than $v.$ In all cases the
 $k=0$ coefficient
is equal to 1, the last stable orbit is defined only for $k\ge 1$ in the
case of T-approximants and for $k\ge 2$ in the case of P-approximants
and $N$ denotes the ``Newtonian value''.}
\begin {center}
\begin {tabular}{c c c c}
\hline
$k$    & $N$ & 1  & 2 \\[3pt]
\hline\\[-8pt]
$E_k$
     & $-\frac{\eta v^2}{2}$
     & $-\frac{9+\eta}{12}$
     & $-\frac{81-57\eta+\eta^2}{24}$\\[3pt]
$e_k$
      & $-x(=-v^2)$
      & $-\frac{3+\eta}{3}$
      & $-\frac{36-35\eta}{12}$\\[3pt]
$e_{P_k}$
      & $-x(=-v^2)$
      & $c_1=\frac{3+\eta}{3}$
      & $c_2=-\frac{144-81\eta+4 \eta^2}{36+12\eta}$\\[3pt]
$x^{\rm lso}_{T_k}$
      & ---
      & $\frac{6}{9+\eta}$
      & $\frac {-E_1 + (E_1^2-3E_2)^{1/2}}{3E_2} $\\[3pt]
$x^{\rm lso}_{P_k}$
      & ---
      & ---
      & $\frac{-1 + (-c_1/c_2)^{1/2}}{c_1+c_2}$ \\[3pt]
$x^{\rm pole}_{P_k}$
      & ---
      & ---
      & $\frac{4(3+\eta)}{36-35\eta}$\\[3pt]
\hline
\end {tabular}
\end {center}
\label{table:energy}
\end {table}

\begin {table}[h]
\caption {Taylor coefficients of the flux
${\cal F}_{T_n}(x) \equiv {\cal F}_N(x) \sum_{k=0}^n {{\cal F}}_k(\eta)v^k,$
PN expansion of the GW phase
$\phi_{T_n}(v) =  \phi_{\rm ref} + \phi^v_N (v)\sum_{k=0}^{n} {\phi}^v_k v^k,$
explicit time-domain phasing formula
$\phi_{T_n}(t) = \phi_{\rm ref}+\phi_N^t \sum_{k=0}^{n}
{\phi}^t_k\theta^k,$
PN expansion of time $t_{T_n}(v) = t_{\rm ref} +t^v_N(v) \sum_{k=0}^{n} {t}^v_k v^k,$
explicit time-domain PN expansion of GW frequency,
$F_{T_n}(t) =  F_N^t \sum_{k=0}^{n} {F}^t_k \theta^k,$
and frequency-domain phase function obtained in the stationary phase approximation
to the chirp,
$\psi(f) = 2 \pi f t_{\rm ref} - \phi_{\rm ref} +
\tau_N \sum_{k=0}^5 {\tau}_k (\pi m f)^{(k-5)/3}.$
Here $N$ denotes the ``Newtonian value'' and
$\theta=[\eta (t_{\rm lso}-t)/(5m)]^{-1/8}.$ In all cases the $k=0$ coefficient
is 1 and the  $k=1$ coefficient is zero. In certain cases
the 2.5 PN term involves $v^5 \log v$ or $\theta^5 \log \theta$
 term rather than a  $v^5$ or $\theta^5$ term.
In those cases  we conventionally include the $\log v$ dependence in the
listed coefficient.
Chirp parameters $\tau_k,$ $k\ge 1,$
are related to the expansion parameters $t^v_k$ and $\phi^v_k$ via
$\tau_k = ( 8 \phi^v_k - 5 t^v_k )/3.$
We have given the simplified
expressions for these in all cases, except $k=5$ where no simplification occurs
due to the presence of the log term in $\phi^v_5.$}
\begin {center}
\begin {tabular}{cccccc}
\hline
$k$ &  $N$ & 2  &  3  &  4  &  5  \\
\hline\\[-8pt]
${\cal F}_k$
      & $\frac{32\eta^2 v^{10}}{5}$
      & $- \frac{1247}{336} - \frac{35\eta}{12}$
    & $4\pi$
    & $-\frac{44711}{9072} + \frac{9271\eta}{504} + \frac{65\eta^2}{18}$
    & $-\left(\frac{8191}{672} + \frac{535\eta}{24}\right) \pi$\\[3pt]
$t^v_k$
      & $-\frac{5m}{256 \eta v^8}$
      & $\frac{743}{252} + \frac{11\eta}{3}$
      & $-\frac{32\pi}{5}$
      & $\frac{3058673}{508032} + \frac{5429\eta}{504} + \frac{617\eta^2}{72}$
      & $-\left(\frac{7729}{252}+ \eta\right)\pi$\\[3pt]
$\phi^v_k$
      & $-\frac{1}{16\eta v^5}$
      & $\frac{3715}{1008}+\frac{55\eta}{12}$
      & $-10 \pi$
      & $\frac{15293365}{1016064} + \frac{27145\eta}{1008 } +
\frac{3085\eta^2}{144}$
      & $ \left (\frac{38645}{672} + \frac{15\eta}{8 } \right ) \pi
\ln \left ( \frac{v}{v_{\rm lso}} \right ) $\\[3pt]
$\phi^t_k$
      & $-\frac{2}{\eta \theta^5}$
      & $\frac{3715}{8064}+\frac{55\eta}{96}$
      & $-\frac{3\pi}{4}$
      & $\frac{9275495}{14450688}+\frac{284875\eta}{258048 } +
\frac{1855\eta^2}{2048 }$
      & $\left (\frac {38645}{21504} + \frac{15\eta}{256 } \right ) \pi
\ln \left ( \frac {\theta}{\theta_{\rm lso}} \right ) $\\[3pt]
$F^t_k$
      & $\frac{\theta^3}{8\pi m}$
      & $\frac{743}{2688}+\frac{11\eta}{32}$
      & $-\frac{3\pi}{10}$
       & $\frac {1855099}{14450688} + \frac{56975\eta}{258048 } +
\frac{371\eta^2}{2048 }$
      & $- \left(\frac{7729}{21504} + \frac{3}{256}\eta\right)\pi$\\[3pt]
$\tau_k$
      & $\frac{3}{128\eta}$
      & $ \frac{5}{9}
        \left ( \frac{743}{84} + 11\eta\right )$
      & $-16\pi $
      & $2  \phi^v_4$
      & $ \frac{1}{3} \left ( 8 \phi^v_5 - 5 t^v_5 \right ) $\\
      \hline
\end {tabular}
\end {center}
\label{table:flux}
\end {table}

\section{Stationary Phase Approximation}
Consider a GW signal of the form,
\begin {equation}
h(t)=2a(t)\cos\phi(t)= a(t) \left [ e^{-i \phi(t)} + e^{i \phi(t)} \right ],
\end {equation}
where $\phi(t)$ is the phasing formula, either specified as an explicit
function of time or given implicitly by a set of differential equations
\cite{dis3}.  The quantity $2\pi F(t) = {d\phi(t)}/{dt}$ defines the instantaneous
GW frequency $F(t)$, and is assumed to be
continuously increasing. (We assume $F(t)>0$.)
 Now the Fourier transform $\tilde h(f)$ of $h(t)$ is defined as
\begin {equation}
\tilde{h}(f) \equiv \int_{-\infty}^{\infty} dt e^{2\pi ift} h(t)
            = \int_{-\infty}^{\infty}\,dt\, a(t)
              \left[ e^{2\pi i f t - \phi(t)}  +  e^{2\pi ift +\phi(t)}\right ].
\end {equation}
The above transform can be computed in the stationary
phase approximation (SPA). For positive frequencies only the first term
on the right  contributes and yields the following {\it usual} SPA:
\begin {equation}
\tilde{h}^{\rm uspa}(f)= \frac {a(t_f)} {\sqrt {\dot{F}(t_f)}}
e^{ i\left[ \psi_f(t_f) -\pi/4\right]},\ \
\psi_f(t) \equiv  2 \pi f t -\phi(t),
\label{eq:inspiralspa1}
\end {equation}
and $t_f$ is the saddle point defined by solving for $t$, $ d \psi_f(t)/d t = 0$,
i.e. the time $t_f$ when the GW frequency $F(t)$ becomes equal to the
Fourier variable $f$. In the adiabatic approximation where
the value of $t_f$ is given by the following integral:
\begin{equation}
t_f = t_{\rm ref} + m \int_{v_f}^{v_{\rm ref}} \frac{E'(v)}{{\cal F}(v)} dv,
\phi (v) = \phi_{\rm ref} + 2 \int_v^{v_{\rm ref}} dv v^3 \, \frac{E'(v)}{{\cal F}(v)},
\label{eq:InspiralTimeAndPhaseFuncs}
\end{equation}
where $v_{\rm ref}$ is a fiducial reference point that sets the origin of
time, $v_f \equiv (\pi m f)^{1/3},$ $E'(v)\equiv dE/dv$ is the derivative of
the binding energy of the system and ${\cal F}(v)$ is the gravitational wave
flux.
Using $t_f$ and $\phi(t_f)$ in the above equation and
using it in the expression for $\psi_f(t)$ we find
\begin{equation}
 \psi_f(t_f) = 2 \pi f t_{\rm ref} - \phi_{\rm ref} + 2\int_{v_f}^{v_{\rm ref}}
(v_f^3 - v^3)
\frac{E'(v)}{{\cal {\cal F}}(v)} dv .
\label{eq:InspiralFourierPhase}
\end{equation}
This is the general form of the stationary phase approximation which
can be applied to {\it all} time-domain signals, including the P-approximant
and effective one-body waveforms. In some cases the Fourier domain phasing
can be worked out explicitly, which we now give:

Using PN expansions of energy and flux but
re-expanding the ratio $E'(v)/{\cal F}(v)$ in Eq.~(\ref{eq:InspiralFourierPhase}) one
can solve the integral explicitly. This leads to the following
explicit, Taylor-like, Fourier domain phasing formula:
\begin{equation}
 \psi_f(t_f) = 2 \pi f t_{\rm ref} - \phi_{\rm ref} +
 \psi_N \sum_{k=0}^5 {\psi}_k (\pi m f)^{(k-5)/3}
\label{eq:InspiralFourierPhase:f2}
\end{equation}
where the coefficients ${\psi}_k$ up to 2.5 post-Newtonian approximation are given by:
$$\psi_N =  \frac{3}{128\eta},\ \ \ \psi_0 = 1,\ \ \ \psi_1 = 0,\ \ \
\psi_2 =  \frac{5}{9} \left ( \frac{743}{84} + 11\eta\right ),\ \ \
\psi_3 =  -16\pi,$$
$$\psi_4 = \frac{5}{72}\left(\frac{3058673}{7056} + \frac{5429}{7}\eta +
617\eta^2\right),$$
$$\psi_5 =  \frac{5}{3} \left ( \frac{7729}{252} + \eta \right ) \pi +
   \frac{8}{3} \left ( \frac{38645}{672} + \frac{15}{8} \eta \right )
	\ln \left ( \frac{v}{v_{\rm ref}} \right )\pi.$$
Eq.~(\ref{eq:InspiralFourierPhase:f2}) is (one of) the  standardly used frequency-domain phasing formulas.
This is what is implemented in {\tt LALInspiralStationaryPhaseApproximation2}
and corresponds to \texttt{TaylorF2} in the enum \texttt{Approximant.}

Alternatively, substituting (without doing any re-expansion or re-summation)
for the energy and flux functions their PN expansions
or the P-approximants of energy and flux functions
and solving the integral in Eq.~(\ref{eq:InspiralFourierPhase}) numerically
one obtains the T-approximant SPA or P-approximant SPA, respectively.
However, just as in the time-domain, the frequency-domain phasing is
most efficiently computed by a pair of coupled, non-linear, ODE's:
\begin{equation}
\frac{d\psi}{df} - 2\pi t = 0, \ \ \ \
\frac{dt}{df} + \frac{\pi m^2}{3v^2} \frac{E'(f)}{{\cal F}(f)} = 0,
\label {eq:frequencyDomainODE}
\end{equation}
rather  than by numerically computing the integral in
Eqs.~(\ref{eq:InspiralFourierPhase}).  However, the current implementation
in {\tt LALInspiralStationaryPhaseApproximation1} solves the integral
and corresponds to \texttt{TaylorF1} in the  \texttt{enum Approximant.}

\subsection{Amplitude in the Fourier domain}
The derivative of the frequency that occurs in the
amplitude of the Fourier transform, namely
$1/\sqrt{\dot{F}(t)},$ in Eq.~(\ref{eq:inspiralspa1}),
is computed using
\begin{equation}
\dot{F(t)} = \frac{dF}{dt}
           = \frac{dF}{dv}\frac{dv}{dE}\frac{dE}{dt}
           = \frac{3v^2}{\pi m}\left [\frac{{-\cal F}(v)}{E'(v)} \right ],
\end{equation}
where we have used the fact that the gravitational wave flux
is related to the binding energy $E$ via the energy balance equation
${\cal F} = -dE/dt$ and that $F=v^3/(\pi m).$
At the Newtonian order $E=-\eta m v^2/2,$ and ${\cal F} = 32\eta^2 v^{10}/5,$
giving $\dot{F}(t(v)) = 96\eta v^{11}/(5\pi m^2).$ Taking
$2a(t(v)) = v^2$ (i.e., $h(t) = v^2 \cos (\phi(t)),$ this gives, the
total amplitude of the Fourier transform to be
$$\frac{a(t(v))}{\sqrt{\dot{F}(t(v))}} =  \sqrt{\frac{5\pi m^2}{384\eta}} v_f^{-7/2}.$$
This is the amplitude used in most of literature. However, including the
full PN expansion in $\dot{F}(t),$ gives a better agreement between the
time-domain and Fourier domains signals and this code therefore uses the full
PN expansion \cite{dis2}.

\section{Detection template family}
\label{sec:BCV}

 The Fourier transform of a chirp waveform in the restricted post-Newtonian
 approximation in the stationary phase approximation is given, for
 positive frequencies $f,$ by [cf. Eq.(\ref{eq:InspiralFourierPhase:f2})]
 \begin{equation}
 \tilde h(f) = h_0 f^{-7/6} \exp \left [ \sum_k \psi_k f^{(k-5)/3} \right ],
 \end{equation}
 where $h_0$ is a constant for a given system and $psi_k$ are parameters that
 depend on the two masses of the binary. Since the time-domain waveform
 is terminated at when the instantaneous GW frequency reaches a certain value
 $F_{\rm cut}$ (which is either the last stable
 orbit or the light-ring defined by the model) and since the contribution to
 a Fourier component comes mainly from times when the GW instantaneous frequency
 reaches that value, it is customery to terminate the Fourier transform at the
 same frequency, namely $f_{\rm cut} = F_{\rm cut}.$ In otherwords, the Fourier
 transform is taken to be
 \begin{equation}
 \tilde h(f) = h_0 f^{-7/6} \theta(f-f_{\rm cut}) \exp \left [ \sum_k \psi_k f^{(k-5)/3} \right ],
 \end{equation}
 where $\theta(x<0)=0$ and $\theta(x\ge 0) =1.$
 We have seen that there are different post-Newtonian models such as the
 standard post-Newtonian,
 P-approximants, effective one-body (see Sec.\ref{sec:EOB}),
 and their overlaps with one another is not
 as good as we would like them to be. The main reason for this is that matched
 filtering is sensitive to the phasing of the waves. It is not clear which model
 best describes the true GW signal from a compact binary inspiral although some,
 like the EOB, are more robust in their predictions than others. Thus, Buonanno,
 Chen and Vallisneri proposed \cite{BCV03,BCV03b}
a {\it phenomenological} model as a detection template
 family (DTF) based on the above expression for the Fourier transform. Indeed, they
 proposed to use a DTF that depends on four parameters
$(\psi_0,\, \psi_3,\, f_{\rm cut},\, \alpha)$ for non-spinning sources \cite{BCV03}
and on six parameters
$(\psi_0,\, \psi_3,\, f_{\rm cut},\, \alpha_1,\, \alpha_2,\, \beta)$
for spinning sources \cite{BCV03b}.
We have implemented both the non-spinning and spinning waveforms
in \texttt{LALBCVWaveform} and \texttt{LALBCVSpinWaveform,} respectively.

\subsection{DTF for non-spinning sources}

 The proposed waveform has structure similar to the one above:
 \begin{equation}
 \tilde h(f) = h_0 f^{-7/6} \left (1 - \alpha f^{2/3} \right) \theta(f-f_{\rm cut})
 \exp \left [ \psi_0 f^{-5/3} + \psi_3 f^{-2/3} \right ],
\label{eq:BCV:NonSpinning}
 \end{equation}
 where the motivation to include an amplitude correction term $\alpha$ is based on the
 fact that the first post-Newtonian correction to the amplitude would induce a term like this.
 Note carefully that the phasing does not include the full post-Newtonian expansion
 but only the Newtonian and 1.5 post-Newtonian terms. It turns out this four-parameter
 family of waveforms has good overlap with the two-parameter family of templates
 corresponding to different post-Newtonian models and their improvments.

\subsection{DTF for spinning sources}
In the generic case of spinning black hole binaries there are a total of 17
parameters characterizing the waveform amplitude and shape
(see Sec.\ref{sec:smirches}.  However, the phasing
of the waves is determined, in general, by about 10 parameters. Apostolatos
et al. \cite {ACST94} studied the time-evolution based on which Apostolatos
found \cite{TAA96} that far fewer parameters can be used to capture
the full structure of the spinning black hole binary
waveforms. BCV suggested \cite{BCV03b} that one could use a DTF of waveforms that resembles
the non-spinning case. Their proposal has the following structure:
\begin{equation}
\tilde h(f) = h_0 f^{-7/6}
\left [1 + \alpha_1 \cos ( \beta f^{-2/3} )
         + \alpha_2 \sin ( \beta f^{-2/3} ) \right]
\theta(f-f_{\rm cut})
 \exp \left [ \psi_0 f^{-5/3} + \psi_3 f^{-2/3} \right ],
\label{eq:BCV:Spinning}
\end{equation}
where $\alpha_1,$ $\alpha_2$ and $\beta$ are the parameters designed
to capture the spin-induced modulation of the waveform.

\section{Effective one-body approach}
\label{sec:EOB}
\newcommand{\ww}{\widehat{\omega}}
\newcommand{\wF}{\widehat{\cal F}}
The entry \texttt{EOB} in the {\tt enum Approximant} corresponds to the
effective one-body (EOB) approach of Buonanno and Damour
\cite{BD99,BD00,DJS00,TD02}. The EOB formalism allows one to evolve
the dynamics of two black holes beyond the last stable orbit into
the plunge phase thereby increasing the number of
wave cycles, and the signal-to-noise ratio, that can be extracted.
Here a set of four ordinary differential equations (ODEs) are solved
by making an anzatz for the tangential radiation reaction force.

In practical terms, the time-domain waveform
is obtained as the following function of the reduced time $\hat{t}=t/m$:
\begin{equation}
\label{4.1}
 \quad h(\hat{t}) = {\cal
C} \, v_{\omega}^2 (\hat{t}) \cos (\phi (\hat{t}))\,, \quad v_{\omega}
\equiv \left( \frac{d \varphi}{d \hat{t}} \right)^{\frac{1}{3}}\,, \quad \phi
\equiv 2\varphi \,.
\end{equation}
(The amplitude $\cal C$ is chosen to be equal to 1 in our codes.)
The four ODEs correspond to the evolution of the radial and angular
coordinates and the corresponding momenta:
\begin{eqnarray}
\label{eq:3.28}
&&\frac{dr}{d \hat{t}} = \frac{\partial \widehat{H}}{\partial p_r}
(r,p_r,p_\varphi)\,, \\
\label{3.29}
&& \frac{d \varphi}{d \hat{t}} = \ww \equiv \frac{\partial \widehat{H}}{\partial p_\varphi}
(r,p_r,p_\varphi)\,, \\
\label{3.30}
&& \frac{d p_r}{d \hat{t}} + \frac{\partial \widehat{H}}{\partial r}
(r,p_r,p_\varphi)=0\,, \\
&& \frac{d p_\varphi}{d \hat{t}} = \wF_\varphi(\ww (r,p_r,p_{\varphi}))\,.
\label{3.31}
\end{eqnarray}
The reduced Hamiltonian $\widehat{H}$ (of the one-body problem)
is given, at the 2PN approximation, by
\label{eq:4.4}
\begin{eqnarray}
\label{eq:3.32}
\widehat{H}(r,p_r,p_\varphi) = \frac{1}{\eta}\,\sqrt{1 + 2\eta\,\left [
\sqrt{A(r)\,\left (1 + \frac{p_r^2}{B(r)} + \frac{p_\varphi^2}{r^2} \right )} -1 \right ]}\,,\\
\label{3.34}
{\rm where}\;\;
A(r) \equiv 1 - \frac{2}{r} + \frac{2\eta}{r^3} \,,
\quad \quad B(r) \equiv \frac{1}{A(r)}\,\left (1 - \frac{6\eta}{r^2}
\right )\,.
\end{eqnarray}
The damping force $\cal{F}_{\varphi}$ is approximated by
\begin{equation}
\widehat{{\cal F}}_{\varphi}
=-\frac{1}{\eta v_\omega ^3}{\cal F}_{P_n}(v_\omega)\,,
\label{eq:damp}
\end{equation}
where $
{\cal F}_{P_n} (v_{\omega})
= \frac{32}{5}\,\eta^2\,v_\omega^{10}\,
{\widehat{\cal F}}_{P_n} (v_{\omega})$
is the P-approximant to the flux function.

The initial data $(r_0, p_{r}^0, p_{\varphi}^0)$ are found using
\begin{equation}
r_0^3 \left [ \frac {1 + 2 \eta (\sqrt{z(r_0)} -1 )}{1- 3\eta/r_0^2} \right ]-  \omega_0^{-2} = 0,\ \
p^0_\varphi = \left [\frac {r_0^2 - 3 \eta}{r_0^3 - 3 r_0^2 + 5 \eta} \right ]^{1/2}r_0,\ \
p^0_r = \frac {{\cal F}_\varphi(\omega)}{C(r_0,p^0_\varphi) (dp^0_\varphi/dr_0)}\ \
\end{equation}
where $z(r)$ and $C(r,p_\varphi)$ are given by
\begin{equation}
z(r) = \frac{r^3 A^2(r)}{r^3-3r^2+5 \eta},\ \
C(r,p_\varphi) = \frac{1}{\eta \widehat{H} (r,0,p_\varphi)
 \sqrt{z(r)}} \frac{A^2(r)}{(1-6\eta/r^2)}.
\end{equation}
The plunge waveform is terminated when the radial coordinate attains the value
at the light ring $r_{\rm lr}$ given by the solution to the equation,
\begin{equation}
r_{\rm lr}^3 - 3 r_{\rm lr}^2 + 5 \eta = 0.
\label{eq:LightRing}
\end{equation}

\section{Spinning Modulated Chirps}
\label{sec:smirches}
Waveforms from spinning black hole binaries at 2PN order can be
generated using the choice {\tt SpinTaylorT3} in the {\tt enum
Approximant.} Current implementation closely follows Ref.~\cite{ACST94}.
The orbital plane of a binary consisting of rapidly spinning compact
objects precesses causing the polarization of the wave received at an
antenna to continually change. This change depends on the source location
on the sky and it might therefore be possible to resolve the direction
to a source. It is therefore essential to specify the coordinate system
employed in the description of the waveform. As in Ref.~\cite{ACST94}
the coordinate system is adapted to the detector with the x-y plane
in the plane of the interferometer with the axes along the two arms.

In the restricted post-Newtonian approximation the evolution of a binary
system comprising of two bodies of masses $m_1$ and $m_2,$ spins
${\mathbf S}_1=(s_1,\, \theta_1, \, \varphi_1)$ and
${\mathbf S}_2=(s_2,\, \theta_2, \, \varphi_2),$ orbital angular momentum
${\mathbf L}=(L,\, \theta_0, \, \varphi_0),$ is governed by a
set of differential equations given by:
\begin{eqnarray}
\dot{\mathbf {L}} =
\left [
\left ( \frac{4m_1+3m_2}{2m_1m^3} - \frac{3\,  {\mathbf S}_2 \cdot {\mathbf L}}{2\, L^2m^3} \right ) {\mathbf S}_1 +
\left ( \frac{4m_2+3m_1}{2m_2m^3} - \frac{3}{2}\frac{ {\mathbf S}_1 \cdot {\mathbf L}}{L^2m^3} \right ) {\mathbf S}_2
\right ] \times {\mathbf L} v^6 -\frac{32\eta^2 m}{5L} {\mathbf L} v^7,
\label{eqn:precession1}
\end{eqnarray}
\begin{equation}
\dot{ \mathbf  S}_1  =
\left [ \left ( \frac{4m_1+3m_2}{2m_1m^3} - \frac{3\, {\mathbf S}_2 \cdot {\mathbf L} }{2\, L^2m^3}
\right ) {\mathbf L} \times {\mathbf S}_1 + \frac{ {\mathbf S}_2 \times {\mathbf S}_1 }{2m^3} \right ] v^6,
\label{eqn:precession2}
\end{equation}
\begin{equation}
\dot{ \mathbf  S}_2  =
\left [ \left ( \frac{4m_2+3m_1}{2m_2m^3} - \frac{3\, {\mathbf S}_1 \cdot {\mathbf L} }{2\, L^2m^3}
\right ) {\mathbf L} \times {\mathbf S}_2 + \frac{ {\mathbf S}_1 \times {\mathbf S}_2 }{2m^3} \right ] v^6.
\label{eqn:precession3}
\end{equation}
where as before $v=(\pi m f)^{1/3},$ $f$ is the gravitational wave frequency,
$m = m_1+m_2$ is the total mass, and $\eta = m_1m_2/m^2$ is
the (symmetric) mass ratio.  An overdot denotes the time-derivative.
In the evolution of the orbital angular momentum we have included
the lowest order dissipative term [\,the second term containing
$v^7$ in Equation (\ref{eqn:precession1})\,]
while keeping the non-dissipative modulation effects
caused by spin-orbit and spin-spin couplings
[\,the first term within square brackets containing
$v^6$ in Equation (\ref{eqn:precession1})\,].
The spins evolve non-dissipatively but their orientations change due to
spin-orbit and spin-spin couplings.
Though the non-dissipative terms are not responsible for
gravitational wave emission, and therefore do not shrink the
orbit, they cause to precess the orbit.

In the absence of spins the antenna observes the same polarization at
all times; the amplitude and frequency of the signal both
increase monotonically, giving rise to a chirping signal.
Precession of the orbit and spins cause modulations in the amplitude
and phase of the signal and smear the signal's energy spectrum over a wide band.
For this reason we shall call the spin modulated (sm) chirp,
a {\it smirch} (an anagram of {\it sm} and {\it chir}).

The strain $h(t)$ produced by a smirch at the antenna is given by
\begin{eqnarray}
 h(t) & = & -A(t)\cos[\phi(t)+\varphi(t)],
\label{eqn:waveform}
\end{eqnarray}
where $A(t)$ is the precession-modulated amplitude of the signal,
$\phi(t)$ is its post-Newtonian carrier phase that
increases monotonically and $\varphi(t)$ is the polarization phase caused
by the changing polarization of the wave relative to the antenna. (We have
neglected the Thomas precession of the orbit which induces additional, but
small, corrections in the phase.) For a source with position vector
${\mathbf N} = (D,\, \theta_S,\, \varphi_S)$ the amplitude is given by,
\begin{equation}
 A(t) = \frac{2\eta m v^2}{D}
\left[ \left( 1 + (\hat {\mathbf L}\cdot \hat {\mathbf N})^2\right )^2 F_{+}^2(\theta_S,\varphi_S,\psi)
 + 4 \left ( \hat {\mathbf L}\cdot \hat {\mathbf N} \right )^2 F_\times^2(\theta_S,\varphi_S,\psi) \right]^{1/2}.
\label{eqn:amplitude}
\end{equation}
Here $\hat{\mathbf L} = {\mathbf L}/L,$ $\hat{\mathbf N} = {\mathbf N}/D$ ($D$ is the distance to
the source) and $\psi(t)$ is the precession-modulated polarization angle given by
\begin{equation}
\tan \psi(t) =
\frac{\hat{\mathbf L}(t)\cdot\hat{\mathbf z} - (\hat{\mathbf L}(t)\cdot\hat{\mathbf N})
(\hat{\mathbf z}\cdot\hat{\mathbf N})}
{\hat{\mathbf N}\cdot(\hat{\mathbf L}(t)\times\hat{\mathbf z})}.
\label{eqn:psi(t)}
\end{equation}
Also, $F_+$ and $F_\times$ are the antenna beam pattern functions are given by
\begin{equation}
F_+(\theta_S,\varphi_S,\psi) = \frac{1}{2}\left(1+\cos^2\theta_S\right)\cos{2\phi_S}\cos{2\psi}
- \cos\theta_S\sin{2\phi_S}\sin{2\psi},
\end{equation}
\begin{equation}
F_{\times}(\theta_S,\phi_S,\psi) =
\frac{1}{2}\left(1+\cos^2\theta_S\right)\cos{2\phi_S}\sin{2\psi}
+ \cos\theta_S\sin{2\phi_S}\sin{2\psi}.
\end{equation}
Next, the polarization phase $\varphi(t)$ is
\begin{equation}
\tan \varphi(t) =
\frac{ 2 \hat {\mathbf L}(t) \cdot \hat {\mathbf N}\, F_\times(\theta_S,\varphi_S,\psi)}
{\left [ 1 + \left ({\mathbf L}(t)\cdot {\mathbf N} \right )^2 \right ] F_{+}(\theta_S,\varphi_S,\psi)}.
\label{eqn:varphi}
\end{equation}
And finally, for the carrier phase we use the post-Newtonian
expression, but without the spin-orbit and
spin-spin couplings. These spin couplings modify the carrier phase
by amounts much smaller than the post-Newtonian effects.
To second post-Newtonian order the carrier phase is given by
\begin{equation}
\phi(t) =
\frac{-2}{\eta\theta^5}\left[1+\left(\frac{3715}{8064}+\frac
{55}{96}\eta\right)\theta^2-\frac{3\pi}{4}\theta^3
+\left(\frac{9275495}{14450688}+\frac{284875}{258048}\eta+\frac{1855}{2048}\eta^2\right)\theta^4\right].
\label{eqn:phi}
\end{equation}
where $\theta=[\eta (t_C-t)/(5m)]^{-1/8},$ $t_C$ being the time at which the two stars
merge together and the gravitational wave frequency formally diverges.


\newpage\input{LALInspiralH}
\begin{thebibliography}{}
\bibitem {dis3} Post-Newtonian approximations implemented in this code are
summarized and compared to one another in:
T. Damour, B.R. Iyer and B.S. Sathyaprakash,
Phys. Rev. D {\bf 63,} 044023 (2001); {\em ibid,} 027502 (2002).

\bibitem{BDIWW}
L. Blanchett, B.R. Iyer, C.M. Will and A.G. Wiseman,
Phys.\ Rev.\ Lett.\ {\bf 74}, 3515 (1995).

\bibitem{BIWW}
L. Blanchett, B.R. Iyer, C.M. Will and A.G. Wiseman,
Class. Quantum Grav., {\bf 13}, 575, (1996).

\bibitem {dis1} T. Damour, B.R. Iyer and B.S. Sathyaprakash,
Phys. Rev. D {\bf 57,} 885 (1998).

\bibitem {dis2} T. Damour, B.R. Iyer and B.S. Sathyaprakash,
Phys. Rev. D {\bf 62,} 084036 (2000).

\bibitem{BD99}
A.\ Buonanno and T.\ Damour,
Phys.\ Rev.\ D {\bf 59}, 084006 (1999).

\bibitem{BD00}
A.\ Buonanno and T.\ Damour,
Phys.\ Rev.\ D {\bf 62}, 064015 (2000).

\bibitem{DJS00}
T.\ Damour, P.\ Jaranowski, and G.\ Sch\"afer,
Phys.\ Rev.\ D {\bf 62}, 084011 (2000).

\bibitem{TD02}
T.\ Damour,
Phys.\ Rev.\ D {\bf 64}, 124013 (2002).

\bibitem{BCV03}
A.\ Buonanno, Y.\ Chen, and M.\ Vallisneri,
Phys.\ Rev. \ D {\bf 67,} 024016 (2003)
gr-qc/0205122.

\bibitem{BCV03b}
A.\ Buonanno, Y.\ Chen, and M.\ Vallisneri,
Phys.\ Rev. \ D {\bf 67,} 104025 (2003)
gr-qc/0211087.

\bibitem{ACST94}
T.A. Apostolatos, C. Cutler, G.J. Sussman
and K.S. Thorne Phys. Rev. D {\bf 49,} 6274 (1994);

\bibitem{TAA96}
T.A. Apostolatos Phys. Rev. D {\bf 50,} 605 (1996).

\end{thebibliography}
