\documentclass[12pt]{article}
\usepackage{amsmath}

\begin{document}
\huge
\begin{center}
InspiralPhase.c
\end{center}
\normalsize
\vspace{10mm}

\section{Purpose}

The function \texttt{InspiralPhase} calculates the phase $\phi(v)$ of a gravitational wave from an inspiralling binary system.

It does this using the following equation, which is one of the pair which constitute the gravitational wave phasing formula,

\begin{equation}
\phi(v) =  \phi_{0} - 2 \int_{v_{0}}^{v} v^{3} \frac{E'(v)}{{\cal F}(v)} \, dv \,\,.
\label{phiofv}
\end{equation}

\texttt{InspiralPhase} calculates $\phi(v)$, given $\phi_{0}$, $v_{0}$,  $v$, $E^{\prime}(v)$ and $\mathcal{F}(v)$.




\section{Algorithms}

\texttt{InspiralPhase} uses the function \texttt{DRombergIntegrate}, to calculate the integral.


\section{Arguments}

The function header is of the form:

\vspace{5mm}

\begin{tabular}{ll}
void \texttt{InspiralPhase}&(\texttt{Status $\ast$status},     \\
                                   &\texttt{REAL8 $\ast$phiofv}, \\
                                   &\texttt{REAL8 v}, \\
                                   &\texttt{void $\ast$params})
\end{tabular}

\vspace{5mm}

The structure which is of type \texttt{Status}, which is pointed to by the pointer \texttt{status} writes information to the screen should the code encounter a problem. The output is of type \texttt{REAL8} and is pointed to by the pointer \texttt{phiofv}. This is the phase $\phi(v)$.
The inputs needed come from the input structure which is of type \texttt{void $\ast$}, and which is pointed to by the pointer \texttt{params}. This structure was originally of type \texttt{InspiralPhaseIn}, which was cast into type \texttt{void $\ast$}.

The input structure has the form

\vspace{5mm}

\begin{tabular}{ll}
\texttt{typedef struct} & \texttt{tagInspiralPhaseIn} \{ \\
                        & \texttt{REAL8 v0;} \\
                        & \texttt{REAL8 phi0;} \\
                        & \texttt{EnergyFunction $\ast$dEnergy;}  \\
                        & \texttt{FluxFunction $\ast$flux;}  \\
                        & \texttt{expnCoeffs $\ast$coeffs;}  \\
                        & \} \texttt{InspiralPhaseIn;}
\end{tabular}

\vspace{5mm}


Now \texttt{dEnergy} is a pointer to the function which represents $E^{\prime}(v)$. This function is of the form

\vspace{5mm}

\texttt{typedef REAL8 EnergyFunction(REAL8, expnCoeffs *)};

\vspace{5mm}

Similarly, \texttt{flux} is a pointer to the function which represents $\mathcal{F}(v)$. This function has the form

\vspace{5mm}

\texttt{typedef REAL8 FluxFunction(REAL8, expnCoeffs *)};

\vspace{5mm}

The pointers to these functions are set by calling the function \texttt{ChooseModel}.

\texttt{expnCoeffs} is a structure which holds all of the coefficients needed to calculate the post--Newtonain expansions of $E^{\prime}(v)$ and $\mathcal{F}(v)$. The members of this structure are filled up by calls to the functions \texttt{InspiralSetup} and \texttt{ChooseModel}.









\section{Operating Instructions}

Here is an example of a code fragment which shows how the members of the input structure are initialized, and how the function is then called.

\vspace{5mm}

\noindent
\begin{verbatim}
/* Declare the structures to be used  */
\end{verbatim}
\texttt{InspiralPhaseIn input;} \\
\texttt{REAL8 phi;} \\
\texttt{Status status;} \\
\begin{verbatim}
/* Initialize the inputs  */
\end{verbatim}
\texttt{input.v0} = $(\pi m f_{0})^{1/3}$; \\
\texttt{input.phi0} = 0.0; \\
\texttt{input.dEnergy} = pointer to the derivative of the energy function; \\
\texttt{input.flux} = pointer to the flux function; \\
\texttt{input.coeffs} = structure for the coefficients; \\
\begin{verbatim}
/* Call the function */
\end{verbatim}
\texttt{InspiralPhase (\&status, \&phi, v,  \&input);}
\begin{verbatim}
/* Write the data to the screen  */
  fprintf(stderr,"%e\n",phi); 
\end{verbatim}

Inside the function \texttt{InspiralPhase}, error checks are made upon its arguments, using the ASSERT macro. Because each of the arguments to the function involves a pointer being passed to the function (e.g.\ \texttt{phiofv, params}), we check that each of the pointers are not NULL pointers.
Inside the function \texttt{InspiralPhase}, this looks like:

\vspace{5mm}

\begin{tabular}{ll}
void \texttt{InspiralPhase}&(\texttt{Status $\ast$status},     \\
                                   &\texttt{REAL8 $\ast$phiofv}, \\
                                   &\texttt{REAL8 v}, \\
                                   &\texttt{void $\ast$params})
\end{tabular}

\vspace{5mm}

\begin{tabular}{ll}
ASSERT & (phiofv,  \\
       &  status,    \\
       &  INSPIRALPHASE\_ENULL, \\
       &  INSPIRALPHASE\_MSGENULL);
\end{tabular}

\vspace{5mm}

This above example checks whether the pointer \texttt{phiofv} is a NULL pointer or not. If it is a NULL pointer, then an error message which is defined by the character string \texttt{INSPIRALPHASE\_MSGENULL} is sent to the screen.


\section{Options}

There are no options for this function.


\section{Accuracy}

All variables are decalred to be REAL8, which means that they are double precision.
Each double precision variable has an approximate precision of 15 significant figures.


\section{Error conditions}

We check that each of the pointers passed to the function \\ \texttt{InspiralPhase} as an argument , i.e.\ \texttt{Status}, \texttt{phiofv} and \texttt{params}, are not NULL pointers. If any of them are NULL, then an error message is sent to the screen.


\section{Tests}

This code has been extensively tested by B. Sathyaprakash. This test included a comparison to the routines in the GRASP library.

\section{Uses}

This function calls the function \texttt{DRombergIntegrate},  which itself calls the function \texttt{PhiofVIntegrand}.


\end{document}
