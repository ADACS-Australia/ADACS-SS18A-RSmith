\documentclass[12pt]{article}
\usepackage{amsmath}

\begin{document}
\huge
\begin{center}
RungeKutta4.c
\end{center}
\normalsize
\vspace{10mm}

\section{Purpose}

The code \texttt{RungeKutta4.c} solves a system of $n$ coupled first--order differential equations.



\section{Algorithms}

This code uses a fourth--order Runge--Kutta algorithm.


\section{Arguments}

The function header is of the form:

\vspace{5mm}

\begin{tabular}{ll}
void \texttt{RungeKutta4}&(\texttt{Status $\ast$status},     \\
                                   &\texttt{REAL8Vector $\ast$yout}, \\
                                   &\texttt{rk4In $\ast$input}, \\
                                   &\texttt{void $\ast$params})
\end{tabular}

\vspace{5mm}

The structure which is of type \texttt{Status}, which is pointed to by the pointer \texttt{status} writes information to the screen should the code encounter a problem. The output structure is of the form \texttt{REAL8Vector} and is pointed to by the pointer \texttt{yout}. 
The inputs needed come from the input structures which are of type \texttt{rk4In}, and \texttt{void $\ast$}, which are pointed to by the pointers \texttt{input} and \texttt{params} respectively.

The output structure has the form

\vspace{5mm}

\begin{tabular}{ll}
\texttt{typedef struct} & \texttt{tagREAL8Vector} \{ \\
                        & \texttt{UINT4 length;} \\
                        & \texttt{REAL8 $\ast$data;}  \\
                        & \} \texttt{REAL8Vector;}
\end{tabular}

\vspace{5mm}

The input structure of type \texttt{rk4In} is defined as follows:

\begin{tabular}{ll}
\texttt{typedef struct} & \texttt{tagrk4In} \{ \\
                        & \texttt{TestFunction $\ast$function;} \\
                        & \texttt{REAL8 x;}  \\
                        & \texttt{REAL8 $\ast$y;}  \\
                        & \texttt{REAL8 $\ast$dydx;}  \\
                        & \texttt{REAL8 $\ast$yt;}  \\
                        & \texttt{REAL8 $\ast$dym;}  \\
                        & \texttt{REAL8 $\ast$dyt;}  \\
                        & \texttt{REAL8 h;}  \\
                        & \texttt{REAL8 n;}  \\
                        & \} \texttt{rk4In;}
\end{tabular}


The input structure which is of the form \texttt{void $\ast$} was originally in the form \texttt{InspiralDerivativesIn} but it has been cast into type \texttt{void $\ast$} so that it can be fed into the function \texttt{RungeKutta4}. This structure is only used when it is fed into the function \texttt{InspiralDerivatives}.









\section{Operating Instructions}

Here is an example of a code fragment which shows how the members of the input structure are initialized, and how the function is then called.

\vspace{5mm}

\noindent
\begin{verbatim}
/* Declare the structures to be used  */
\end{verbatim}
\texttt{rk4In $\ast$input;} \\
\texttt{REAL8Vector $\ast$dvandp;} \\
\texttt{REAL8Vector $\ast$vandpnew;} \\
\texttt{RK4In input;} \\
\texttt{InspiralDerivativesIn input2}; \\
\texttt{Status status;} \\
\begin{verbatim}
/* Initialize the inputs  */
\end{verbatim}
\texttt{input.function} = InspiralDerivatives; \\
\texttt{input.x} = $t$; \\
\texttt{input.y} = vandp; \\
\texttt{input.h} = $\delta t$ \\
\texttt{input.yt} = yt \\
\texttt{input.dym} = dym \\
\texttt{input.dyt} = dyt \\
\texttt{input.dydx} = dvandp \\
\begin{verbatim}
/* Call the function */
\end{verbatim}
\texttt{RungeKutta4 (\&status, vandpnew, \&input, \&input2);} \\


Here, \texttt{vandp} has the values of $v$ and $\phi$ at time $t$, and \texttt{dvanddp} has the values of $dv/dt$ and $d\phi/dt$ at time $t$. We wish to know the values of $v$ and $\phi$ at time $t + \delta t$, which is stored in \texttt{vandpnew}, i.e.\ \texttt{$\ast$(vandpnew->data) = $v(t + \delta t)$} and \texttt{$\ast$(vandpnew->data+1) = $\phi(t + \delta t)$}.

The additional arrays are just arrays that \texttt{RungeKutta4} uses internally. They are passed as arguments because \texttt{RungeKutta4} is a function which gets called many times, and if it were to allocate and de--allocate memory for these arrays aeach time it were called, it would run prohibatively slowly. Therefore we allocate and de--allocate the memory for these arrays in the calling function, in which case it only needs to be done once.


Inside the function \texttt{RungeKutta4}, error checks are made upon its arguments, using the ASSERT macro. Because each of the arguments to the function involves a pointer being passed to the function (e.g.\ \texttt{yout, input}), we check that each of the pointers are not NULL pointers.
Inside the function \texttt{RungeKutta4}, this looks like:

\vspace{5mm}

\begin{tabular}{ll}
void \texttt{RungeKutta4}&(\texttt{Status $\ast$status},     \\
                                   &\texttt{REAL8Vector $\ast$yout}, \\
                                   &\texttt{rk4In $\ast$input}, \\
                                   &\texttt{void $\ast$params})
\end{tabular}

\vspace{5mm}

\begin{tabular}{ll}
ASSERT & (yout,  \\
       &  status,    \\
       &  RUNGEKUTTA4\_ENULL, \\
       &  RUNGEKUTTA4\_MSGENULL);
\end{tabular}

\vspace{5mm}

This above example checks whether the pointer \texttt{yout} is a NULL pointer or not. If it is a NULL pointer, then an error message which is defined by the character string \texttt{RUNGEKUTTA4\_MSGENULL} is sent to the screen.


\section{Options}

There are no options available to the user.

\section{Accuracy}

All variables are decalred to be REAL8, which means that they are double precision.
Each double precision variable has an approximate precision of 15 significant figures.

\section{Error conditions}

We check that each of the pointers passed to the function \\ \texttt{RungeKutta4} as an argument , i.e.\ \texttt{Status}, \texttt{yout}, \texttt{input} and \texttt{params}, are not NULL pointers. If any of them are NULL, then an error message is sent to the screen.


\section{Tests}

This code has been extensively tested by B. Sathyaprakash. This test included a comparison to the routines in the GRASP library.

\section{Uses}

This function calls the function which evaluates the derivatives. 

\end{document}
