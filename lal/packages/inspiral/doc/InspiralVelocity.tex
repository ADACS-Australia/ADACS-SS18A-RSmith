\documentclass[12pt]{article}
\usepackage{amsmath}

\begin{document}
\huge
\begin{center}
InspiralVelocity.c
\end{center}
\normalsize
\vspace{10mm}

\section{Purpose}

The function \texttt{InspiralVelocity} calculates the velocity $v$ which corresponds to a time $t$ for in inspiralling binary system.

It does this using the following equation, which is one of the pair which constitute the gravitational wave phasing formula,

\begin{equation}
t(v) =  t_{0} - m \int_{v_{0}}^{v} \frac{E'(v)}{{\cal F}(v)} \, dv \,\,.
\label{tofv}
\end{equation}

\texttt{InspiralVelocity} calculates $v$, given $t(v)$, $t_{0}$, $m$, $v_{0}$, $E^{\prime}(v)$ and $\mathcal{F}(v)$.




\section{Algorithms}

\texttt{InspiralVelocity} uses the function \texttt{DBisectionFindRoot}, which finds roots by the method of bisection.

The equation which this function solves is 

\begin{equation}
t(v) - t_{0} + m \int_{v_{0}}^{v} \frac{E'(v)}{{\cal F}(v)} \, dv = 0 \,\,.
\label{tofv1}
\end{equation}

\section{Arguments}

The function header is of the form:

\vspace{5mm}

\begin{tabular}{ll}
void \texttt{InspiralVelocity}&(\texttt{Status $\ast$status},     \\
                                   &\texttt{REAL8 $\ast$v}, \\
                                   &\texttt{TofVIn $\ast$ak})
\end{tabular}

\vspace{5mm}

The structure which is of type \texttt{Status}, which is pointed to by the pointer \texttt{status} writes information to the screen should the code encounter a problem. The output is of type \texttt{REAL8} and is pointed to by the pointer \texttt{v}. This is the velocity $v$.
The inputs needed come from the input structure which is of type \texttt{TofVIn}, and which is pointed to by the pointer \texttt{ak}.

The input structure has the form

\vspace{5mm}

\begin{tabular}{ll}
\texttt{typedef struct} & \texttt{tagTofVIn} \{ \\
                        & \texttt{REAL8 t;} \\
                        & \texttt{REAL8 v0;} \\
                        & \texttt{REAL8 t0;} \\
                        & \texttt{REAL8 vlso;} \\
                        & \texttt{REAL8 totalmass;} \\
                        & \texttt{EnergyFunction $\ast$dEnergy;}  \\
                        & \texttt{FluxFunction $\ast$flux;}  \\
                        & \texttt{expnCoeffs $\ast$coeffs;}  \\
                        & \} \texttt{TofVIn;}
\end{tabular}

\vspace{5mm}


Now \texttt{dEnergy} is a pointer to the function which represents $E^{\prime}(v)$. This function is of the form

\vspace{5mm}

\texttt{typedef REAL8 EnergyFunction(REAL8, expnCoeffs *)};

\vspace{5mm}

Similarly, \texttt{flux} is a pointer to the function which represents $\mathcal{F}(v)$. This function has the form

\vspace{5mm}

\texttt{typedef REAL8 FluxFunction(REAL8, expnCoeffs *)};

\vspace{5mm}

The pointers to these functions are set by calling the function \texttt{ChooseModel}.

\texttt{expnCoeffs} is a structure which holds all of the coefficients needed to calculate the post--Newtonain expansions of $E^{\prime}(v)$ and $\mathcal{F}(v)$. The members of this structure are filled up by calls to the functions \texttt{InspiralSetup} and \texttt{ChooseModel}.









\section{Operating Instructions}

Here is an example of a code fragment which shows how the members of the input structure are initialized, and how the function is then called.

\vspace{5mm}

\noindent
\begin{verbatim}
/* Declare the structures to be used  */
\end{verbatim}
\texttt{TofVIn input;} \\
\texttt{REAL8 v;} \\
\texttt{Status status;} \\
\begin{verbatim}
/* Initialize the inputs  */
\end{verbatim}
\texttt{input.t} = 0.0; \\
\texttt{input.t0} = 0.0; \\
\texttt{input.v0} = $(\pi m f_{0})^{1/3}$; \\
\texttt{input.vlso} = $1/\sqrt{6}$; \\
\texttt{input.totalmass} = 20.0; \\
\texttt{input.dEnergy} = pointer to the derivative of the energy function; \\
\texttt{input.flux} = pointer to the flux function; \\
\texttt{input.coeffs} = structure for the coefficients; \\
\begin{verbatim}
/* Call the function */
\end{verbatim}
\texttt{InspiralVelocity (\&status, \&v, \&input);}
\begin{verbatim}
/* Write the data to the screen  */
  fprintf(stderr,"%e\n",v); 
\end{verbatim}

Inside the function \texttt{InspiralVelocity}, error checks are made upon its arguments, using the ASSERT macro. Because each of the arguments to the function involves a pointer being passed to the function (e.g.\ \texttt{v, ak}), we check that each of the pointers are not NULL pointers.
Inside the function \texttt{InspiralVelocity}, this looks like:

\vspace{5mm}

\begin{tabular}{ll}
void \texttt{InspiralVelocity}&(\texttt{Status $\ast$status},     \\
                                   &\texttt{REAL8 $\ast$v}, \\
                                   &\texttt{TofVIn $\ast$ak})
\end{tabular}

\vspace{5mm}

\begin{tabular}{ll}
ASSERT & (v,  \\
       &  status,    \\
       &  INSPIRALVELOCITY\_ENULL, \\
       &  INSPIRALVELOCITY\_MSGENULL);
\end{tabular}

\vspace{5mm}

This above example checks whether the pointer \texttt{v} is a NULL pointer or not. If it is a NULL pointer, then an error message which is defined by the character string \texttt{INSPIRALVELOCITY\_MSGENULL} is sent to the screen.


\section{Options}

There are no options for this function.


\section{Accuracy}

All variables are decalred to be REAL8, which means that they are double precision.
Each double precision variable has an approximate precision of 15 significant figures.


\section{Error conditions}

We check that each of the pointers passed to the function \\ \texttt{InspiralVelocity} as an argument , i.e.\ \texttt{Status}, \texttt{v} and \texttt{ak}, are not NULL pointers. If any of them are NULL, then an error message is sent to the screen.


\section{Tests}

This code has been extensively tested by B. Sathyaprakash. This test included a comparison to the routines in the GRASP library.

\section{Uses}

This function calls the function \texttt{DBisectionFindRoot},  which itself calls the function \texttt{TofV}.


\end{document}
