\documentclass[12pt]{article}
\usepackage{amsmath}

\begin{document}
\huge
\begin{center}
InspiralDerivatives.c
\end{center}
\normalsize
\vspace{10mm}

\section{Purpose}

The code \texttt{InspiralDerivatives.c} calculates the two coupled first--order differential equations which we use to solve the gravitational wave phasing equation, as described in the documentation for the function \texttt{TimeDomain2}.

The equations are

\begin{equation}
\frac{dv}{dt} = - \frac{\mathcal{F}(v)}{m E^{\prime}(v)} \,\,.
\label{ode1}
\end{equation}

and

\begin{equation}
\frac{d \phi(t)}{dt} = \frac{2v^{3}}{m}
\label{ode2}
\end{equation}




\section{Algorithms}

This code uses no algorithms.


\section{Arguments}

The function header is of the form:

\vspace{5mm}

\begin{tabular}{ll}
void \texttt{InspiralDerivatives}&(\texttt{Status $\ast$status},     \\
                                   &\texttt{REAL8Vector $\ast$values}, \\
                                   &\texttt{REAL8Vector $\ast$dvalues}, \\
                                   &\texttt{void $\ast$params})
\end{tabular}

\vspace{5mm}

The structure which is of type \texttt{Status}, which is pointed to by the pointer \texttt{status} writes information to the screen should the code encounter a problem. The output structure is of the form \texttt{REAL8Vector} and is pointed to by the pointer \texttt{dvalues}. 
The inputs needed come from the input structures which are of type \texttt{REAL8Vector}, and \texttt{void $\ast$}, which are pointed to by the pointers \texttt{values} and \texttt{params} respectively.

The output structure has the form

\vspace{5mm}

\begin{tabular}{ll}
\texttt{typedef struct} & \texttt{tagREAL8Vector} \{ \\
                        & \texttt{UINT4 length;} \\
                        & \texttt{REAL8 $\ast$data;}  \\
                        & \} \texttt{REAL8Vector;}
\end{tabular}

\vspace{5mm}

The two coupled differential equations $dv/dt$ and $d\phi/dt$ are written to the elements \texttt{dvalues->data} and \texttt{dvalues->data+1} respectively.

By inspecting equations \ref{ode1} and \ref{ode2} we see that they both depend only on $v$. This is fed into the function via the structure \texttt{values}, in that \texttt{values->data} = $v$.

The input structure which is of the form \texttt{void $\ast$} was originally in the form \texttt{InspiralDerivativesIn} but it has been cast into type \texttt{void $\ast$} so that it can be fed into the function \texttt{Runge--Kutta4}. This structure has the form

\begin{tabular}{ll}
\texttt{typedef struct} & \texttt{tagInspiralDerivativesIn} \{ \\
                        & \texttt{REAL8 totalmass;} \\
                        & \texttt{EnergyFunction $\ast$dEnergy;}  \\
                        & \texttt{FluxFunction $\ast$flux;}  \\
                        & \texttt{expnCoeffs $\ast$coeffs;}  \\
                        & \} \texttt{InspiralDerivativesIn;}
\end{tabular}

\vspace{5mm}

Now \texttt{dEnergy} is a pointer to the function which represents $E^{\prime}(v)$. This function is of the form

\vspace{5mm}

\texttt{typedef REAL8 EnergyFunction(REAL8, expnCoeffs *)};

\vspace{5mm}

Similarly, \texttt{flux} is a pointer to the function which represents $\mathcal{F}(v)$. This function has the form

\vspace{5mm}

\texttt{typedef REAL8 FluxFunction(REAL8, expnCoeffs *)};

\vspace{5mm}

The pointers to these functions are set by calling the function \texttt{ChooseModel}.

\texttt{expnCoeffs} is a structure which holds all of the coefficients needed to calculate the post--Newtonain expansions of $E^{\prime}(v)$ and $\mathcal{F}(v)$. The members of this structure are filled up by calls to the functions \texttt{InspiralSetup} and \texttt{ChooseModel}.








\section{Operating Instructions}

Here is an example of a code fragment which shows how the members of the input structure are initialized, and how the function is then called.

\vspace{5mm}

\noindent
\begin{verbatim}
/* Declare the structures to be used  */
\end{verbatim}
\texttt{REAL8Vector $\ast$vandp;} \\
\texttt{REAL8Vector $\ast$dvandp;} \\
\texttt{InspiralDerivativesIn input;} \\
\texttt{Status status;} \\
\begin{verbatim}
/* Initialize the inputs  */
\end{verbatim}
\texttt{input.totalmass} = 20.0; \\
\texttt{input.dEnergy} = pointer to the appropriate function; \\
\texttt{input.flux} = pointer to the appropriate function; \\
\texttt{input.coeffs} = pointer to the appropriate structure; \\
\texttt{$\ast$(vandp->data) = $v$}; \\
\begin{verbatim}
/* Call the function */
\end{verbatim}
\texttt{InspiralDerivatives (\&status, vandp, dvandp \&input);} \\

Inside the function \texttt{InspiralDerivatives}, error checks are made upon its arguments, using the ASSERT macro. Because each of the arguments to the function involves a pointer being passed to the function (e.g.\ \texttt{values, dvalues}), we check that each of the pointers are not NULL pointers.
Inside the function \texttt{InspiralDerivatives}, this looks like:

\vspace{5mm}

\begin{tabular}{ll}
void \texttt{InspiralDerivatives}&(\texttt{Status $\ast$status},     \\
                                   &\texttt{REAL8Vector $\ast$values}, \\
                                   &\texttt{REAL8Vector $\ast$dvalues}, \\
                                   &\texttt{void $\ast$params})
\end{tabular}

\vspace{5mm}

\begin{tabular}{ll}
ASSERT & (values,  \\
       &  status,    \\
       &  INSPIRALDERIVATIVES\_ENULL, \\
       &  INSPIRALDERIVATIVES\_MSGENULL);
\end{tabular}

\vspace{5mm}

This above example checks whether the pointer \texttt{values} is a NULL pointer or not. If it is a NULL pointer, then an error message which is defined by the character string \texttt{INSPIRALDERIVATIVES\_MSGENULL} is sent to the screen.


\section{Options}

There are no options available to the user.

\section{Accuracy}

All variables are decalred to be REAL8, which means that they are double precision.
Each double precision variable has an approximate precision of 15 significant figures.


\section{Error conditions}

We check that each of the pointers passed to the function \\ \texttt{InspiralDerivatives} as an argument , i.e.\ \texttt{Status}, \texttt{values} and \texttt{dvalues}, are not NULL pointers. If any of them are NULL, then an error message is sent to the screen.


\section{Tests}

This code has been extensively tested by B. Sathyaprakash. This test included a comparison to the routines in the GRASP library.

\section{Uses}

This function call the appropriate energy and flux function, $E^{\prime}(v)$ and $\mathcal{F}(v)$ respectively.

\end{document}
