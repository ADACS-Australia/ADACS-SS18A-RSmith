\documentclass[12pt]{article}
\usepackage{amsmath}

\begin{document}
\huge
\begin{center}
InspiralSetup.c
\end{center}
\normalsize
\vspace{10mm}

\section{Purpose}

The code \texttt{InspiralSetup.c} defines all of the pade and taylor coefficients. 

\section{Algorithms}

The code uses no algorithms


\section{Arguments}

The function header is of the form:

\vspace{5mm}

\begin{tabular}{ll}
void \texttt{InspiralSetup}&(\texttt{Status $\ast$status},     \\
                                   &\texttt{expnCoeffs $\ast$ak}, \\
                                   &\texttt{InspiralTemplate $\ast$params})
\end{tabular}

\vspace{5mm}

The structure which is of type \texttt{Status}, which is pointed to by the pointer \texttt{status} writes information to the screen should the code encounter
a problem. The output structure is of the form \texttt{expnCoeffs} and is pointed to by the pointer \texttt{ak}.
The inputs needed come from the input structure which is of type \texttt{InspiralTemplate}, and which is pointed to by the pointer \texttt{params}.

The output structure has the form

\vspace{5mm}

\begin{tabular}{ll}
\texttt{typedef struct} & \texttt{expnCoeffs} \{ \\
                        & \texttt{REAL8 eTaN;}  \\
                        & \texttt{REAL8 eTa0;}  \\
                        & \texttt{REAL8 eTa1;}  \\
                        & \texttt{REAL8 eTa2;}  \\
                        & \texttt{REAL8 eTa3;}  \\
                        & \texttt{REAL8 eTa4;}  \\
                        & \texttt{REAL8 eTa5;}  \\
                        & \texttt{REAL8 ePaN;}  \\
                        & \texttt{REAL8 ePa0;}  \\
                        & \texttt{REAL8 ePa1;}  \\
                        & \texttt{REAL8 ePa2;}  \\
                        & \texttt{REAL8 ePa3;}  \\
                        & \texttt{REAL8 ePa4;}  \\
                        & \texttt{REAL8 ePa5;}  \\
                        & \texttt{REAL8 ETaN;}  \\
                        & \texttt{REAL8 ETa0;}  \\
                        & \texttt{REAL8 ETa1;}  \\
                        & \texttt{REAL8 ETa2;}  \\
                        & \texttt{REAL8 ETa3;}  \\
                        & \texttt{REAL8 ETa4;}  \\
                        & \texttt{REAL8 ETa5;}  \\
                        & \texttt{REAL8 dETaN;}  \\
                        & \texttt{REAL8 dETa0;}  \\
                        & \texttt{REAL8 dETa1;}  \\
                        & \texttt{REAL8 dETa2;}  \\
                        & \texttt{REAL8 dETa3;}  \\
                        & \texttt{REAL8 dETa4;}  \\
                        & \texttt{REAL8 dETa5;}  \\
                        & \texttt{REAL8 fTaN;}  \\
                        & \texttt{REAL8 fTa0;}  \\
                        & \texttt{REAL8 fTa1;}  \\
                        & \texttt{REAL8 fTa2;}  \\
                        & \texttt{REAL8 fTa3;}  \\
                        & \texttt{REAL8 fTa4;}  \\
                        & \texttt{REAL8 fTa5;}  \\
                        & \texttt{REAL8 fPaN;}  \\
\end{tabular}
\begin{tabular}{ll}
                        & \texttt{REAL8 fPa0;}  \\
                        & \texttt{REAL8 fPa1;}  \\
                        & \texttt{REAL8 fPa2;}  \\
                        & \texttt{REAL8 fPa3;}  \\
                        & \texttt{REAL8 fPa4;}  \\
                        & \texttt{REAL8 fPa5;}  \\
                        & \texttt{REAL8 FTaN;}  \\
                        & \texttt{REAL8 FTa0;}  \\                       
			& \texttt{REAL8 FTa1;}  \\
                        & \texttt{REAL8 FTa2;}  \\
                        & \texttt{REAL8 FTa3;}  \\
                        & \texttt{REAL8 FTa4;}  \\
                        & \texttt{REAL8 FTa5;}  \\
                        & \texttt{REAL8 samplingrate;}  \\
                        & \texttt{REAL8 samplinginterval;}  \\
                        & \texttt{REAL8 eta;}  \\
                        & \texttt{REAL8 totalmass;}  \\
                        & \texttt{REAL8 m1;}  \\
                        & \texttt{REAL8 m2;}  \\
                        & \texttt{REAL8 f0;}  \\
                        & \texttt{REAL8 fn;}  \\                       
			& \texttt{REAL8 t0;}  \\
                        & \texttt{REAL8 tn;}  \\
                        & \texttt{REAL8 v0;}  \\
                        & \texttt{REAL8 vn;}  \\
                        & \texttt{REAL8 vf;}  \\

                        & \texttt{REAL8 vlso;}  \\
                        & \texttt{REAL8 vlsoT0;}  \\
                        & \texttt{REAL8 vlsoT2;}  \\
                        & \texttt{REAL8 vlsoT4;}  \\
                        & \texttt{REAL8 vlsoTP;}  \\
                        & \texttt{REAL8 vlsoP2;}  \\
                        & \texttt{REAL8 vlsoP4;}  \\
                        & \texttt{REAL8 vpoleP4;}  \\                       
                        & \} \texttt{expnCoeffs;}
\end{tabular}

\vspace{5mm}


\clearpage
The input structure is of the form

\vspace{5mm}

\begin{tabular}{ll}
\texttt{typedef struct} & \texttt{tagInspiralTemplate} \{ \\
                        & \texttt{INT4 number;} \\
                        & \texttt{REAL8 mass1;} \\
                        & \texttt{REAL8 mass2;}  \\
                        & \texttt{REAL8 spin1[3];}  \\
                        & \texttt{REAL8 spin2[3];}  \\
                        & \texttt{REAL8 inclination;} \\
                        & \texttt{REAL8 eccentricity;} \\
                        & \texttt{REAL8 totalMass;} \\
                        & \texttt{REAL8 mu;}  \\
                        & \texttt{REAL8 eta;}  \\
                        & \texttt{REAL8 fLower;}  \\
                        & \texttt{REAL8 fCutoff;} \\
                        & \texttt{REAL8 tSampling;} \\
                        & \texttt{REAL8 startPhase;} \\
                        & \texttt{REAL8 startTime;} \\
                        & \texttt{REAL8 signalAmplitude;} \\
                        & \texttt{REAL8 nStartPad;} \\
                        & \texttt{REAL8 nEndPad;} \\
                        & \texttt{INT4 ieta;} \\
                        & \texttt{InspiralMethod method;}  \\
                        & \texttt{InputMasses massChoice;}  \\
                        & \texttt{Order order;}  \\
                        & \texttt{Domain domain;}  \\
                        & \texttt{Approximant approximant;}  \\
                        & \} \texttt{InspiralTemplate;}
\end{tabular}

\vspace{5mm}


\section{Operating Instructions}

Here is an example of a code fragment which shows how the members of the input structure are initialized, and how the function is then called.


\vspace{5mm}

\noindent
\begin{verbatim}
/* Declare the structures to be used  */
\end{verbatim}
\texttt{expnCoeffs ak;} \\
\texttt{Status status;} \\
\texttt{InspiralTemplate params};
\begin{verbatim}
/* Initialize the inputs  */
\end{verbatim}
\texttt{params.ieta} = 1; \\
\texttt{params.mass1} = 10.0; \\
\texttt{params.mass2} = 10.0; \\
\texttt{params.startTime} = 0.0; \\
\texttt{params.startPhase} = 0.0; \\
\texttt{params.fCutoff} = 1000.0; \\
\texttt{params.tSampling} = 4000.0; \\
\texttt{params.signalAmplitude} = 1.0; \\
\texttt{params.nStartPad} = 0; \\
\texttt{params.nEndPad} = 0; \\
\texttt{params.order} = twoPN; \\
\texttt{params.domain} = time; \\
\texttt{params.approximant} = taylor;

\begin{verbatim}
/* Call the function  */
\end{verbatim}
\texttt{InspiralSetup (\&status, \&ak, \&params);} \\



Inside the function \texttt{InspiralSetup()}, error checks are made upon its arguments, using the ASSERT macro. Because each of the arguments to the function involves a pointer being passed to the function (e.g.\ \texttt{ak, params}), we first of all check that each of the pointers are not NULL pointers.

Inside the function \texttt{InspiralSetup()}, this looks like:

\vspace{5mm}

\begin{tabular}{ll}
void \texttt{InspiralSetup}&(\texttt{Status $\ast$status},     \\
                                   &\texttt{expnCoeffs $\ast$ak}, \\
                                   &\texttt{InspiralTemplate $\ast$params})
\end{tabular}

\vspace{5mm}

\begin{tabular}{ll}
ASSERT & (ak,  \\
       &  status,    \\
       &  INSPIRALSETUP\_ENULL, \\
       &  INSPIRALSETUP\_MSGENULL1);
\end{tabular}

\vspace{5mm}

This above example checks whether the pointer \texttt{ak} is a NULL pointer or not. If it is a NULL pointer, then an error message which is defined by the character string \texttt{INSPIRALSETUP\_MSGENULL1} is sent to the screen using the \texttt{StatusHandler} function.


\section{Options}

 

\section{Accuracy}

All variables are decalred to be REAL8, which means that they are double precision.
Each double precision variable has an approximate precision of 15 significant figures.



\section{Error conditions}

We first of all check that each of the pointers passed to the function \\ \texttt{InspiralSetup} as an argument , i.e.\ \texttt{Status}, \texttt{ak} and \texttt{params} are not NULL pointers. If any of them are NULL, then an error message is sent to the screen.




\section{Tests}

This code has been extensively tested by B. Sathyaprakash. This test included a comparison to the routines in the GRASP library.

\section{Uses}

This function does not call any other functions.


\section{References}









\end{document}
