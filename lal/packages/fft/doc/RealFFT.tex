% This file is meant to be included in another 
\documentclass{article}
\begin{document}
\section{RealFFT}

\subsection{Purpose}

Perform real-to-complex and complex-to-real fast Fourier transforms of
vectors, sequences of vectors, and (eventually) arrays and sequences of
arrays using the package FFTW~\cite{fj:1998}.

\subsection{Synopsis}

% Syntax: argument definitions, calling signature

\begin{verbatim}
#include "RealFFT.h"

void EstimateFwdRealFFTPlan (Status *status, RealFFTPlan **plan, INT4 size);
void EstimateInvRealFFTPlan (Status *status, RealFFTPlan **plan, INT4 size);
void MeasureFwdRealFFTPlan  (Status *status, RealFFTPlan **plan, INT4 size);
void MeasureInvRealFFTPlan  (Status *status, RealFFTPlan **plan, INT4 size);
void DestroyRealFFTPlan     (Status *status, RealFFTPlan **plan);

void REAL4VectorFFT (
    Status      *status,
    REAL4Vector *out,
    REAL4Vector *in,
    RealFFTPlan *plan
    );

void REAL4VectorSequenceFFT (
    Status              *status,
    REAL4VectorSequence *out,
    REAL4VectorSequence *in,
    RealFFTPlan         *plan
    );

void FwdRealFFT (
    Status         *status,
    COMPLEX8Vector *out,
    REAL4Vector    *in,
    RealFFTPlan    *plan
    );

void InvRealFFT (
    Status         *status,
    REAL4Vector    *out,
    COMPLEX8Vector *in,
    RealFFTPlan    *plan
    );

void RealPowerSpectrum (
    Status      *status,
    REAL4Vector *out,
    REAL4Vector *in,
    RealFFTPlan *plan
    );

void FwdRealSequenceFFT (
    Status                 *status,
    COMPLEX8VectorSequence *out,
    REAL4VectorSequence    *in,
    RealFFTPlan            *plan
    );

void InvRealSequenceFFT (
    Status                 *status,
    REAL4VectorSequence    *out,
    COMPLEX8VectorSequence *in,
    RealFFTPlan            *plan
    );

void RealSequencePowerSpectrum (
    Status              *status,
    REAL4VectorSequence *out,
    REAL4VectorSequence *in,
    RealFFTPlan         *plan
    );
\end{verbatim}

\subsection{Description}

This package provides a LAL-style interface with the FFTW fast Fourier
transform package~\cite{fj:1998}.

The routines \texttt{EstimateFwdRealFFTPlan()},
\texttt{EstimateInvRealFFTPlan()}, \texttt{MeasureFwdRealFFTPlan()}, and
\texttt{MeasureInvRealFFTPlan()} create plans for computing the forward
(real-to-complex) and inverse (complex-to-real) FFTs.  The optimum plan is
either estimated (reasonably fast) or measured (can be time-consuming, but
gives better performance).  The routine \texttt{DestroyRealFFTPlan()} destroys
any of these flavours of plans.

The routines \texttt{FwdRealFFT()} and \texttt{InvRealFFT()} perform the
forward (real-to-complex) and inverse (complex-to-real) FFTs using the plans.
The discrete Fourier transform $H_k$, $k=0\ldots [n/2]$ ($n/2$ rounded down),
of a vector $h_j$, $j=0\ldots n-1$, of length $n$ is defined by
\[
  H_k = \sum_{j=0}^{n-1} h_j e^{2\pi ijk/n}
\]
and, similarly, the inverse Fourier transform is defined by
\[
  h_j = \frac{1}{n}\sum_{k=0}^{[n/2]} H_k e^{-2\pi ijk/n}.
\]
However, the present implementation of the inverse FFT omits the factor of
$1/n$.

The routines in this package require that the vector $h_j$, $j=0\ldots n-1$ be
real; consequently, $H_k=H_{n-k}^\ast$ ($0\le k\le[n/2]$), i.e., the negative
frequency Fourier components are the complex conjugate of the positive
frequency Fourier components when the data is real.  Therefore, one need
compute and store only the first $[n/2]+1$ components of $H_k$; only
the values of $H_k$ for $k=0\ldots [n/2]$ are returned (integer division is
rounded down, e.g., $[7/2]=3$).

The routine \texttt{RealPowerSpectrum()} computes the power spectrum
$P_k=|H_k|^2$, $k=0\ldots [n/2]$, of the data $h_j$, $j=0\ldots n-1$.

The routines \texttt{FwdRealSequenceFFT()} and \texttt{InvRealSequenceFFT()}
operate on sequences of vectors
$\{h^{(0)}_{j_0},\ldots,h^{(m-1)}_{j_{m-1}}\}$, $j_0,\ldots,j_{m-1}=0\ldots
n-1$, to produce the Fourier components
$\{H^{(0)}_{k_0},\ldots,H^{(m-1)}_{k_{m-1}}\}$, $k_0,\ldots,k_{m-1}=0\ldots
[n/2]$, and vice versa respectively.  Here, $m$ is the length of the sequence
of vectors and $n$ and $[n/2]+1$ are the lengths of the vectors themselves.
Similarly, the routine \texttt{RealSequencePowerSpectrum()} computes the power
spectra $\{P^{(0)}_{k_0}=|H^(0)_{k_0}|^2,\ldots,P^{(m-1)}_{k_{m-1}}=
|H^{(m-1)}_{k_{m-1}}|^2\}$, $k_0,\ldots,k_{m-1}=0\ldots n/2$, of the sequence
of vectors $\{h^{(0)}_{j_0},\ldots,h^{(m-1)}_{j_{m-1}}\}$,
$j_0,\ldots,j_{m-1}=0\ldots n-1$.

Notes:
\begin{enumerate}
\item The sign convention used here agrees with the definition in
\textit{Numerical Recipes}~\cite{ptvf:1992}, but is opposite from the one used
by FFTW~\cite{fj:1998}.
\item The result of the inverse FFT must be multiplied by $1/n$ to recover the
original vector.  This is unlike the \textit{Numerical
Recipes}~\cite{ptvf:1992} convension where the factor is $2/n$ for real FFTs.
\item The size $n$ of the transform can be any positive integer; the
performance is $O(n\log n)$.  However, better performance is obtained if $n$
is the product of powers of 2, 3, 5, 7, and zero or one power of either 11 or
13.  Transforms when $n$ is a power of 2 are especially fast.  See
Ref.~\cite{fj:1998}.
\item All of these routines leave the input array undamaged.
\item Thread safety: The creation of plans is \emph{not} thread-safe!  That
is, all plans should be created by a single thread---if two threads are
simultaneously creating plans, possible errors can happen.  However, the use
of plans in the actual FFTing is thread-safe.
\item LALMalloc() is used by all the fftw routines.
\end{enumerate}

\subsection{Operating Instructions}

% Detailed usage 

\begin{verbatim}
const INT4 m = 4;   /* example length of sequence of vectors */
const INT4 n = 32;  /* example vector length */

static Status status; 

RealFFTPlan            *pfwd = NULL;
RealFFTPlan            *pinv = NULL;
REAL4Vector            *hvec = NULL;
COMPLEX8Vector         *Hvec = NULL;
REAL4Vector            *Pvec = NULL;
REAL4VectorSequence    *hseq = NULL;
COMPLEX8VectorSequence *Hseq = NULL;
REAL4VectorSequence    *Pseq = NULL;
CreateVectorSequenceIn  seqinp;
CreateVectorSequenceIn  seqout;

/* create data vector and sequence */
seqinp.length       = m;
seqinp.vectorLength = n;
seqout.length       = m;
seqout.vectorLength = n/2 + 1;

EstimateFwdRealFFTPlan (&status, &pfwd, n);
EstimateInvRealFFTPlan (&status, &pinv, n);
SCreateVector          (&status, &hvec, n);
CCreateVector          (&status, &Hvec, n/2 + 1);
SCreateVector          (&status, &Pvec, n/2 + 1);
SCreateVectorSequence  (&status, &hseq, &seqinp);
CCreateVectorSequence  (&status, &Hseq, &seqout);
SCreateVectorSequence  (&status, &Pseq, &seqout);

/* assign data ... */

/* perform FFTs */
FwdRealFFT                (&status, Hvec, hvec, pfwd);
InvRealFFT                (&status, hvec, Hvec, pinv);
RealPowerSpectrum         (&status, Pvec, hvec, pfwd);
FwdRealSequenceFFT        (&status, Hseq, hseq, pfwd);
InvRealSequenceFFT        (&status, hseq, Hseq, pinv);
RealSequencePowerSpectrum (&status, Pseq, hseq, pfwd);

/* destroy plans, vectors, and sequences */
DestroyRealFFTPlan     (&status, &pfwd);
DestroyRealFFTPlan     (&status, &pinv);
SDestroyVector         (&status, &hvec);
CDestroyVector         (&status, &Hvec);
SDestroyVector         (&status, &Pvec);
SDestroyVectorSequence (&status, &hseq);
CDestroyVectorSequence (&status, &Hseq);
SDestroyVectorSequence (&status, &Pseq);
\end{verbatim}

\subsubsection{Arguments}

% Describe meaning of each argument

\begin{itemize}
\item \texttt{status} is a universal status structure.  Its contents are
assigned by the functions.
\item \texttt{n} is the size of the transform.  It
must be greater than zero, and it is best if it is a power of two.
\item \texttt{pfwd} and \texttt{pinv} are the foward and inverse FFT plans
respectively.
\item \texttt{hvec} is a real vector of length \texttt{n} containing the real
data.
\item \texttt{Hvec} is a complex vector of length \texttt{n/2 + 1} containing
the positive frequencies of the FFT of the data.
\item \texttt{Pvec} is a real vector of length \texttt{n/2 + 1} containing the
positive frequencies of power spectrum of the data.
\item \texttt{m} is the number of vectors in the sequences of vectors.  It
must be greater than zero.
\item \texttt{hseq} is a sequence of length \texttt{m} of real vectors of
length \texttt{n} containing the real data.
\item \texttt{Hvec} is a sequence of length \texttt{m} of complex vectors of
length \texttt{n/2 + 1} containing the positive frequencies of the FFTs of the
data vectors.
\item \texttt{Pvec} is a sequence of length \texttt{m} of real vectors of
length \texttt{n/2 + 1} containing the positive frequencies of the power
spectra of the data vectors.
\end{itemize}

\subsubsection{Options}

None. 

\subsubsection{Error conditions}

% What constitutes an error condition? What do the error codes mean?

These functions all set the universal status structure on return.
Error conditions are described in the following table.

\begin{table}
\begin{tabular}{|r|l|p{2in}|}\hline
status  & status          & Description\\
code    & description     & \\\hline
REALFFT\_ENULL 1   & Null pointer
  & an argument is NULL or contains a NULL pointer\\
REALFFT\_ENNUL 2   & Non-null pointer
  & trying to create a plan that already exists \\
REALFFT\_ESIZE 4   & Invalid input size
  & plan size must be greater than zero\\
REALFFT\_ESZMM 8   & Size mismatch
  & sizes of vector(s) do not agree with plan\\
REALFFT\_ESLEN 16  & Invalid/mismatched sequence length
  & sequence lengths are not the same\\
REALFFT\_ESAME 32  & Input/Output data vectors are the same
  & input and output vectors need to be distinct\\
REALFFT\_ESIGN 64  & Incorrect plan sign
  & incorrect plan is used (e.g., a forward in an inverse transform)\\
REALFFT\_EDATA 128 & Bad input data: DC/Nyquist should be real
  & DC/Nyquist should be real \\
\hline
\end{tabular}
\caption{Error conditions for all RealFFT functions}\label{tbl:CV}
\end{table}
                                
\subsection{Algorithms}

The FFTW~\cite{fj:1998} is used.

% Describe algorithm by which work is done

\subsection{Accuracy}

% For numerical routines address issues related to accuracy:
% approximations, argument ranges, etc.


\subsection{Tests}

% Describe the tests that are part of the test suite

\subsection{Uses}

% What LAL, other routines does this one call?

\begin{itemize}
\item\texttt{SCreateVector()}
\item\texttt{SDestroyVector()}
\item\texttt{SCreateVectorSequence()}
\item\texttt{SDestroyVectorSequence()}
\end{itemize}

\subsection{Notes}

\subsection{References}

% Any references for algorithms, tests, etc.
\begin{thebibliography}{0}
\bibitem{fj:1998}
  M. Frigo and S. G. Johnson,
  \textit{FFTW User's Manual},
  (Massachusetts Institute of Technology, Cambridge, USA, 1998).
  URL: \texttt{http://www.fftw.org/doc}
\bibitem{ptvf:1992}
  W. H. Press, S. A. Teukolsky, W. T. Vetterling, and B. P. Flannery,
  \textit{Numerical Recipes in C: The Art of Scientific Computing}, 2nd ed.
  (Cambridge University Press, Cambridge, England, 1992).
\end{thebibliography}

\end{document}

%%% Local Variables: 
%%% mode: latex
%%% TeX-master: t
%%% End: 

