% This file is meant to be included in another 
\documentclass{article}
\begin{document}
\section{ComplexFFT}

\subsection{Purpose}

Perform complex-to-complex fast Fourier transforms of vectors using the
package FFTW~\cite{fj:1998}.

\subsection{Synopsis}

% Syntax: argument definitions, calling signature

\begin{verbatim}
#include "ComplexFFT.h"

void EstimateFwdComplexFFTPlan( Status *status, ComplexFFTPlan **plan, INT4 size );
void EstimateInvComplexFFTPlan( Status *status, ComplexFFTPlan **plan, INT4 size );
void MeasureFwdComplexFFTPlan( Status *status, ComplexFFTPlan **plan, INT4 size );
void MeasureInvComplexFFTPlan( Status *status, ComplexFFTPlan **plan, INT4 size );
void DestroyComplexFFTPlan( Status *status, ComplexFFTPlan **plan );

void COMPLEX8VectorFFT(
    Status         *status,
    COMPLEX8Vector *out,
    COMPLEX8Vector *in,
    ComplexFFTPlan *plan
    );
\end{verbatim}

\subsection{Description}

This package provides a LAL-style interface with the FFTW fast Fourier
transform package~\cite{fj:1998}.

The routines \texttt{EstimateFwdComplexFFTPlan()},
\texttt{EstimateInvComplexFFTPlan()}, \texttt{MeasureFwdComplexFFTPlan()}, and
\texttt{MeasureInvComplexFFTPlan()} create plans for computing the forward and
inverse FFTs.  The optimum plan is either estimated (reasonably fast) or
measured (can be time-consuming, but gives better performance).  The routine
\texttt{DestroyComplexFFTPlan()} destroys any of these flavours of plans.

The routine \texttt{COMPLEX8VectorFFT()} performs either the forward or
inverse FFT depending on the plan.  The discrete Fourier transform $H_k$,
$k=0\ldots n-1$ of a vector $h_j$, $j=0\ldots n-1$, of length $n$ is defined
by
\[
  H_k = \sum_{j=0}^{n-1} h_j e^{2\pi ijk/n}
\]
and, similarly, the inverse Fourier transform is defined by
\[
  h_j = \frac{1}{n}\sum_{k=0}^{n-1} H_k e^{-2\pi ijk/n}.
\]
However, the present implementation of the inverse FFT omits the factor of
$1/n$.

Notes:
\begin{enumerate}
\item The sign convention used here agrees with the definition in
\textit{Numerical Recipes}~\cite{ptvf:1992}, but is opposite from the one used
by FFTW~\cite{fj:1998}.
\item The result of the inverse FFT must be multiplied by $1/n$ to recover the
original vector.
\item The size $n$ of the transform can be any positive integer; the
performance is $O(n\log n)$.  However, better performance is obtained if $n$
is the product of powers of 2, 3, 5, 7, and zero or one power of either 11 or
13.  Transforms when $n$ is a power of 2 are especially fast.  See
Ref.~\cite{fj:1998}.
\item Thread safety: The creation of plans is \emph{not} thread-safe!  That
is, all plans should be created by a single thread---if two threads are
simultaneously creating plans, possible errors can happen.  However, the use
of plans in the actual FFTing is thread-safe.
\item LALMalloc() is used by all the fftw routines.
\end{enumerate}

\subsection{Operating Instructions}

% Detailed usage 

\begin{verbatim}
const UINT4 n = 17;   /* example length of sequence of vectors */

static Status status; 

ComplexFFTPlan *pfwd = NULL;
ComplexFFTPlan *pinv = NULL;
COMPLEX8Vector *avec = NULL;
COMPLEX8Vector *bvec = NULL;
COMPLEX8Vector *cvec = NULL;

/* create data vector and sequence */
EstimateFwdComplexFFTPlan( &status, &pfwd, n );
EstimateInvComplexFFTPlan( &status, &pinv, n );
CCreateVector( &status, &avec, n );
CCreateVector( &status, &bvec, n );
CCreateVector( &status, &cvec, n );

/* assign data ... */

/* perform FFTs */
COMPLEX8VectorFFT( &status, bvec, avec, pfwd );
COMPLEX8VectorFFT( &status, cvec, bvec, pinv );

/* destroy plans, vectors, and sequences */
DestroyComplexFFTPlan( &status, &pfwd );
DestroyComplexFFTPlan( &status, &pinv );
CDestroyVector( &status, &avec );
CDestroyVector( &status, &bvec );
CDestroyVector( &status, &cvec );
\end{verbatim}

\subsubsection{Arguments}

% Describe meaning of each argument

\begin{itemize}
\item \texttt{status} is a universal status structure.  Its contents are
assigned by the functions.
\item \texttt{n} is the size of the transform.  It
must be greater than zero, and it is best if it is a power of two.
\item \texttt{pfwd} and \texttt{pinv} are the foward and inverse FFT plans
respectively.
\item \texttt{avec} is a complex vector of length \texttt{n} containing the
complex data.
\item \texttt{bvec} is a complex vector of length \texttt{n} containing the
the FFT of the data.
\item \texttt{cvec} is a complex vector of length \texttt{n} containing the
the inverse FFT of the FFT of the data.
\end{itemize}

\subsubsection{Options}

None. 

\subsubsection{Error conditions}

% What constitutes an error condition? What do the error codes mean?

These functions all set the universal status structure on return.
Error conditions are described in the following table.

\begin{table}
\begin{tabular}{|r|l|p{2in}|}\hline
status  & status          & Description\\
code    & description     & \\\hline
COMPLEXFFT\_ENULL 1   & Null pointer
  & an argument is NULL or contains a NULL pointer\\
COMPLEXFFT\_ENNUL 2   & Non-null pointer
  & trying to create a plan that already exists \\
COMPLEXFFT\_ESIZE 4   & Invalid input size
  & plan size must be greater than zero\\
COMPLEXFFT\_ESZMM 8   & Size mismatch
  & sizes of vector(s) do not agree with plan\\
COMPLEXFFT\_ESLEN 16  & Invalid/mismatched sequence length
  & sequence lengths are not the same\\
COMPLEXFFT\_ESAME 32  & Input/Output data vectors are the same
  & input and output vectors need to be distinct\\
\hline
\end{tabular}
\caption{Error conditions for all ComplexFFT functions}\label{tbl:CV}
\end{table}
                                
\subsection{Algorithms}

The FFTW~\cite{fj:1998} is used.

% Describe algorithm by which work is done

\subsection{Accuracy}

% For numerical routines address issues related to accuracy:
% approximations, argument ranges, etc.


\subsection{Tests}

% Describe the tests that are part of the test suite

\subsection{Uses}

% What LAL, other routines does this one call?

\subsection{Notes}

\subsection{References}

% Any references for algorithms, tests, etc.
\begin{thebibliography}{0}
\bibitem{fj:1998}
  M. Frigo and S. G. Johnson,
  \textit{FFTW User's Manual},
  (Massachusetts Institute of Technology, Cambridge, USA, 1998).
  URL: \texttt{http://www.fftw.org/doc}
\bibitem{ptvf:1992}
  W. H. Press, S. A. Teukolsky, W. T. Vetterling, and B. P. Flannery,
  \textit{Numerical Recipes in C: The Art of Scientific Computing}, 2nd ed.
  (Cambridge University Press, Cambridge, England, 1992).
\end{thebibliography}

\end{document}

%%% Local Variables: 
%%% mode: latex
%%% TeX-master: t
%%% End: 

