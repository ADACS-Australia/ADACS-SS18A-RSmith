\chapter{Package \texttt{pulsar}: common routines}

\chapter{Package \texttt{pulsar}: stack-slide routines}
Teviet Creighton
\bigskip

This package provides routines for searching for slowly-modulated
continuous periodic gravitational waves, using coherent or
semi-coherent demodulation techniques.  It is currently under
construction.

There are three parts to this package: common routines, stack-slide algorithm
routines, and hough transform routines.

\newpage\input{PulsarTimesH}
\newpage\input{StackMetricH}
\newpage\input{FlatMeshH}
\newpage\input{StreamInputH}
\newpage\input{ResampleH}

\newpage\begin{thebibliography}{0}
\bibitem{Brady_P:2000}
  P.~R. Brady and T. Creighton, Phys. Rev. D\textbf{61}, 082001
  (2000).
\end{thebibliography}

\chapter{Package \texttt{pulsar}: Hough transform routines}
Steven Berukoff, M. Alessandra Papa
\bigskip


This package provides a routine to perform a demodulation on a set of data.
In particular, this routine works with frequency domain data by combining
short timescale Fourier Transforms (SFTs) into longer time baseline
demodulated Fourier Transforms (DeFTs). If the assumptions under which the
method was developed are met (\cite{Williams:1999}), then the demodulation
procedure concentrates the total power (within $5\%-10\%$) in a single
frequency bin. In practice, due to the discretization of frequency space, this
power may be shared between two neighbouring bins.

The procedure follows that outlined in \cite{Williams:1999} and is part of the
continuous-wave search algorithm outlined in \cite{Schutz:1999}.  Briefly, the
routine takes input SFTs, corrects for modulation effects due to intrinsic
frequency spindown and Earth's motion, and outputs a DeFT of long time
baseline. In general the routine can be easily adapted to correct for an
arbitrary modulation effect simply by the use of a suitable timing routine,
here \verb+tdb()+ .

The package is organized under the headers \verb+LALDemod.h+ and
\verb+ComputeSky.h+ and the modules \verb+LALDemod.c+ and \verb+ComputeSky.c+.


\newpage\input{LALDemodH}
\newpage\input{ComputeSkyH}

\newpage\begin{thebibliography}{0}
\bibitem{Williams:1999}
	Peter R. Williams, Bernard F. Schutz.  gr-qc/9912029.
\bibitem{Schutz:1999}
	Bernard F. Schutz, M. A. Papa.  gr-qc/9905018. 
\end{thebibliography}
