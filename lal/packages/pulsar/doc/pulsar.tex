\chapter{Package \texttt{pulsar}: common routines}
Teviet Creighton
\bigskip

This package provides routines for timing, metric calculation and
mesh-generation relevant for pulsar searches.

\newpage\input{PulsarTimesH}
\newpage\input{FlatMeshH}
\newpage\input{TwoDMeshH}
\newpage\input{TwoDMeshPlotH}
\newpage\input{ResampleH}

\newpage\begin{thebibliography}{0}
\bibitem{Brady_P:2000}
  P.~R. Brady and T. Creighton, Phys. Rev. D\textbf{61}, 082001
  (2000).
\end{thebibliography}

\chapter{Package \texttt{pulsar}: amplitude folding routines}
Greg Mendell
\bigskip

Contains function LALFoldAmplitudes: folds amplitudes into phase bins.

\begin{verbatim}
Files:

FoldAmplitudes.h        header file
FoldAmplitudes.c        source code
FoldAmplitudesTest.c    test code
foldamplitudes.tex      overview
\end{verbatim}

Periodic sources of gravitational radiation will produce measured strains of the following form:
$$
c[i] = A(t_i,\vec{\lambda}) \sin[\Phi(t_i,\vec{\lambda})] + n(t_i)
$$
In this equation $c[i]$ is the discrete time series output of the detector (perhaps after some data conditioning, such as
being resampled, narrow banded, or with instrument line noise removed).
The amplitude, $A(t_i,\vec{\lambda})$, is assumed roughly constant at the gravity wave source,
but is modulated by variation in the detector's response due to the Earth's motion.  The phase, $\Phi(t_i,\vec{\lambda})$,
is modulated by both the intrinsic spin down of the source, and the changes in relative motion between the source
and the detector.  This can be calculated for known pulsars.  The vector $\vec{\lambda}$ is a vector of parameters
that describe the sky position, etc., of the source and location, etc., of the detector.
Finally, $n(t_i)$ is the noise, which also includes any other signals that are not coherent with
the phase $\Phi(t_i,\vec{\lambda})$.

The folded amplitude is given by
$$
c_{\rm F} [j] = \sum_{i'}
\left \{ A(t_i,\vec{\lambda})\sin[\Phi(t_i,\vec{\lambda})] + n(t_i) \right \} ,
$$
where the sum over $i'$ means sum over all $i$'s with $\Phi$ in phase bin $j$.
If the bin sizes are sufficiently small, then $c_{\rm F} [j]$ can be approximated as
$$
c_{\rm F} [j] = \sin\Phi_j\sum_{i'} A(t_i,\vec{\lambda}) + \sum_{i'} n(t_i) ,
$$
where $\Phi_j$ is representive of the phase for bin $j$ (e.g., the phase corresponding to the midpoint of the bin).
However, because of amplitude modulation, the amplitudes that are added to a phase bin are not guaranteed to enter
with the same sign.  Thus, some sort of amplitude demodulation should be done.

If we demodulate $A(t_i,\vec{\lambda})$ (for example, in a minimum way such as multiplying by the sign
of the response function) we multiply each element of the vector $c[i]$ by an amplitude demodulation factor $D(t_i)$
$$
c_{\rm D\, , F} [j] = \sin\Phi_j \sum_{i'} D(t_i) A(t_i,\vec{\lambda}) + \sum_{i'} D(t_i) n(t_i) ,
$$
If the average value of $D(t_i)$ is zero, and is not correlated with the noise, then
$$
\sum_{i'} D(t_i) n(t_i) \approx 0
$$
However, the average value of $D(t_i)$ is probably not zero.
The following is a very preliminary suggestion of how to further reduce the noise.
Consider folding the measured strains, $c[i]$, again,
but this time shifting the phase bins by $\pi$.  Define this phase shifted folded amplitude as:
$$
c_{\pi, \, \rm D\, , F} [j] = \sin(\Phi_j + \pi) \sum_{i''} D(t_i) A(t_i,\vec{\lambda}) + \sum_{i''} D(t_i) n(t_i) ,
$$
where the sum over $i''$ means sum over all $i$'s with $\Phi + \pi$ in phase bin $j$.
This will reverse the sign of the sum of the amplitudes that enter into each phase bin, but
the sum of the noise contributions into each bin should be roughly the same.
If the signal we are searching for is present, then amplitudes, $A(t_i,\vec{\lambda})$ are correlated with $D(t_i)$ such that
$$
\sum_{i'} D(t_i) A(t_i,\vec{\lambda}) \approx \bar{A} = {\rm constant}
$$
Thus,
$$
c_{\rm D\, , F} [j] - c_{\pi, \, \rm D\, , F} [j] \approx 2 \bar{A}\sin\Phi_j ,
$$
plus residual noise.  In practice, one needs to fold the amplitudes only once, and then make the replacement
$$
c_{\rm D\, , F} [j] \rightarrow c_{\rm D\, , F} [j] - c_{\rm D\, , F} [(j + N/2) \, \% \, j] ,
$$
where $N$ is the number of phase bins.  We can then statistically analyze the hypothesis that
the demodulated folded amplitudes correspond to a sinusoid.

\newpage\input{FoldAmplitudesH}

\chapter{Package \texttt{pulsar}: Coherent search routines}
Steven Berukoff, M. Alessandra Papa
\bigskip

This package provides a routine to perform a demodulation on a set of data.
In particular, this routine works with frequency domain data by combining
short timescale Fourier Transforms (SFTs) into longer time baseline
demodulated Fourier Transforms (DeFTs). If the assumptions under which the
method was developed are met (\cite{Williams:1999}), then the demodulation
procedure concentrates the total power (within $5\%-10\%$) in a single
frequency bin. In practice, due to the discretization of frequency space, this
power may be shared between two neighbouring bins.

The procedure follows that outlined in \cite{Williams:1999} and is part of the
continuous-wave search algorithm outlined in \cite{Schutz:1999}.  Briefly, the
routine takes input SFTs, corrects for modulation effects due to intrinsic
frequency spindown and Earth's motion, and outputs a DeFT of long time
baseline. In general the routine can be easily adapted to correct for an
arbitrary modulation effect simply by the use of a suitable timing routine,
here \verb+tdb()+ .

The package is organized under the headers \verb+LALDemod.h+, \verb+LALComputeAM.h+, and
\verb+ComputeSky.h+ and the modules \verb+LALDemod.c+ and \verb+ComputeSky.c+.


\newpage\input{LALDemodH}
\newpage\input{ComputeSkyH}
\newpage\input{ComputeSkyBinaryH}
\newpage\input{LALComputeAMH}

\newpage\begin{thebibliography}{0}
\bibitem{Williams:1999}
        Peter R. Williams, Bernard F. Schutz.  gr-qc/9912029.
\bibitem{Schutz:1999}
        Bernard F. Schutz, M. A. Papa.  gr-qc/9905018.
\end{thebibliography}

\chapter{Package \texttt{pulsar}: known pulsar time-domain search routines}

This package provides routines for a time domain search of gravitational wave
signals from known pulsars.  The documentation and functionality of this
package is \textbf{incomplete}.
\newpage\input{BinaryPulsarTimingH}
\newpage\begin{thebibliography}{0}
\bibitem{TaylorWeisberg:1989}
        J. Taylor and J. Weisberg, \it{Ap. J.}, \bf{345}, 1989.
\bibitem{BlandfordTeukolsky:1976}
        R. Blandford and S. Teukolsky, \it{Ap. J.}, \bf{205}, 1976.
\bibitem{ChLangeetal:2001}
        Ch. Lange {\it et. al.}, \it{Mon. Not. R. Astron. Soc.}, \bf{326}, 2001.
\bibitem{DamourDeruelle:1985}
        T. Damour and N. Deruelle, \it{Ann. Inst. H. Poincar\'e (Phys. Th\'eorique)}, \bf{43}, 1985.
\bibitem{Wex:1998}
        N. Wex, \it{Mon. Not. R. Astron. Soc.}, \bf{298}, 1998.
\end{thebibliography}
\newpage\input{FitToPulsarH}
\newpage\input{PulsarCatH}

