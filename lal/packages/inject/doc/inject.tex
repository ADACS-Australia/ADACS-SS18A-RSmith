\chapter{Package \texttt{inject}}

This package provides routines to simulate gravitational waves and
their effect on a detector.  Conceptually, this can be divided into
three stages:
\begin{enumerate}
\item Generating the gravitational waveform as produced by the source.
The routines currently available are:
\begin{description}
\item[\texttt{GeneratePPNInspiral.h}] Provides routines to generate
parametrized post-Newtonian inspiral waveforms up to 5/2 order.
\item[\texttt{GenerateTaylorCW.h}] Provides a routine to generate
continuous quasiperiodic waveforms with Taylor-parametrized frequency
evolution.
\item[\texttt{GenerateSpinOrbitCW.h}] Provides a routine to generate
Taylor-parameterized waveforms, as above, with additional binary orbit
Doppler modulations.
\end{description}

\item Simulating a detector's theoretical response to an incoming
gravitational wave.  The routines currently available are:
\begin{description}
\item[\texttt{SimulateCoherentGW.h}] Provides routines to simulate the
detector response to a coherent wave with slowly-varying frequency and
amplitude.
\end{description}

\item Injecting the detector's theoretical response with time into a
(noisy) datastream.  This is done by a single set of generic routines
in \verb@Inject.h@.
\end{enumerate}

As the package evolves, new source types may be added under item~1,
and other (perhaps more generic) ways of simulating the detector
response may be added under item~2.  Item~3, however, is unlikely to
need much updating.

In addition to these basic divisions, the package may include routines
that perform other useful tasks in signal injection or source
simulation, such as combining signal generation, detector simulation,
and injection into a single function call, or modelling astrophysical
distributions of sources.

\newpage\input{InjectH}
\newpage\input{SimulateCoherentGWH}
\newpage\input{SkyCoordinatesLaldocH}

\newpage\begin{thebibliography}{1}
\expandafter\ifx\csname bibnamefont\endcsname\relax
  \def\bibnamefont#1{#1}\fi
\expandafter\ifx\csname bibfnamefont\endcsname\relax
  \def\bibfnamefont#1{#1}\fi
\expandafter\ifx\csname url\endcsname\relax
  \def\url#1{\texttt{#1}}\fi
\expandafter\ifx\csname urlprefix\endcsname\relax\def\urlprefix{URL }\fi
\providecommand{\bibinfo}[2]{#2}
\providecommand{\eprint}[2][]{\url{#2}}

\bibitem{GRASP_1.9.8:2000}
\bibinfo{author}{\bibfnamefont{B.}~\bibnamefont{Allen}},
  \emph{\bibinfo{title}{GRASP: a data analysis package for gravitational wave
  detection}}, \bibinfo{address}{Department of Physics, P.O.\ Box 413,
  University of Wisconsin, Milwaukee WI 53201, USA}, \bibinfo{edition}{1st} ed.
  (\bibinfo{year}{2000}), \bibinfo{note}{\texttt{http:\slash\slash
  www.lsc-group.phys.uwm.edu\slash$\sim$ballen\slash grasp-distribution}}.

\bibitem{Will_C:1996}
\bibinfo{author}{\bibfnamefont{C.~M.} \bibnamefont{Will}} \bibnamefont{and}
  \bibinfo{author}{\bibfnamefont{A.~G.} \bibnamefont{Wiseman}},
  \bibinfo{journal}{Phys. Rev. D.}
  \textbf{\bibinfo{volume}{54}}(\bibinfo{number}{8}), \bibinfo{pages}{4813}
  (\bibinfo{year}{1996}).

\bibitem{Anderson_W:2000}
\bibinfo{author}{\bibfnamefont{W.~G.} \bibnamefont{Anderson}},
  \bibinfo{author}{\bibfnamefont{P.~R.} \bibnamefont{Brady}},
  \bibinfo{author}{\bibfnamefont{J.~D.~E.} \bibnamefont{Creighton}},
  \bibnamefont{and} \bibinfo{author}{\bibfnamefont{\'E.~\'E.}
  \bibnamefont{Flanagan}}, \emph{\bibinfo{title}{An excess power statistic for
  detection of burst sources of gravitational radiation}}
  (\bibinfo{year}{2000}), \bibinfo{note}{gr-qc/0008066}.

\bibitem{Lang_K:1999}
\bibinfo{author}{\bibfnamefont{K.~R.} \bibnamefont{Lang}},
  \emph{\bibinfo{title}{Astrophysical Formulae, Volume II: Space, Time, Matter
  and Cosmology}}, Astronomy and Astrophysics Library
  (\bibinfo{publisher}{Springer-Verlag}, \bibinfo{address}{Berlin},
  \bibinfo{year}{1999}), \bibinfo{edition}{3rd} ed.

\bibitem{JKS98}
\bibinfo{author}{\bibfnamefont{P.} \bibnamefont{Jaranowski}},
\bibinfo{author}{\bibfnamefont{A.} \bibnamefont{Kr\'olak}},
\bibinfo{author}{\bibfnamefont{B.~F.} \bibnamefont{Schutz}},
  \emph{\bibinfo{title}{Data analysis of gravitational-wave signals
      from spinning neutron stars: The signal and its detection}},
  \bibinfo{journal}{Phys. Rev. D.}
  \textbf{\bibinfo{volume}{58}}, \bibinfo{pages}{063001}
  \bibinfo{year}{1998})

\end{thebibliography}
