The detection process for binary inspiral chirps is simply the construction of
some statistic of the data followed by some test of the hypotheses ``there is
a signal present in the data'' and ``there is no signal present in the data.''
When these hypotheses are considered to be exclusive, the reception process
reduces to the comparison of the statistic with some pre-assigned threshold.
Thus, there are two issues: first, what is the \emph{optimal} statistic to
construct, and second, how is the threshold to be determined.

Here we show the construction of statistic for the maximum likelihood reciever
used in the \texttt{findchirp} package. It can be shown \cite{wz} that this is
the optimal statistic for detection of inspiral chirps.

Suppose that the detector output is a dimensionless strain $h(t)$. We denote
by $s(t)$ the waveform of the signal (i.e. chirp) that we wish to find in the
data stream hidden in detector noise, $n(t)$. We assume that the noise samples
are drawn from a stationary Normal distribution, though there may be
correlations amongst the noise events (colored noise). The noise correlations
may be expressed in terms of the one-sided noise power spectral density,
\ospsd, defined in equation [\ref{findchirp:eqn:ospsd}]. The detector output
is then
\begin{equation}
h(t) = s(t) + n(t).
\end{equation}
These functions are members of the Banach space $L_2$, so we may construct an
inner product on this space $(\cdot\mid\cdot) : L_2 \times L_2 \rightarrow
\mathcal{C}$ by
\begin{equation}
  (a\mid b) = \int_{-\infty}^\infty df\,
  \frac{\tilde{a}^\ast(f)\tilde{b}(f)+\tilde{a}(f)\tilde{b}^\ast(f)}
       {S_h(|f|)}.
\end{equation}

The signal we wish to recieve is, in general, arbitrary up to an unknown phase
and amplitude. Thus we may write the signal as
\begin{equation}
s(t)=A\cos\theta\times h_c(t)+A\sin\theta\times h_s(t),
\end{equation}
where $A$ is the amplitude of the signal, $\theta$ is the unknown phase and
$h_c(t)$ and $h_s(t)$ are the known waveforms of the signal with some fiducial
normalisation. We suppose further that the waveforms are orthogonal, so
\begin{equation}
\left(h_c\mid h_s\right) = 0
\end{equation}
and
\begin{equation}
\left(h_c\mid h_c\right) = \left(h_s\mid h_s\right) = \sigma^2
\end{equation}
which defines the normalisation constant $\sigma$.

The inspiral signals that we wish to recieve are much shorter than the
observation time, and we do not know when the signal occour. We wish not only
to detect these signals, but also to maesure their arrival time. To do so, we
constuct the time series
\begin{equation}
x(t) = 2 \int_{\infty}^{\infty}df\,e^{2\pi i f t} 
\frac{\tilde{h}(f) \tilde{h_c}^\ast(f)}{S_h\left(\left|f\right|\right)}
\end{equation}
and
\begin{equation}
y(t) = 2 \int_{\infty}^{\infty}df\,e^{2\pi i f t} 
\frac{\tilde{h}(f) \tilde{h_s}^\ast(f)}{S_h\left(\left|f\right|\right)}.
\end{equation}
For each observation period, we construct the statistic
\begin{equation}
\rho(t) = \frac{1}{\sigma}\sqrt{x^2(t) + y^2(t)}
\end{equation}
and threshold on this statistic, which we call the signal-to-noise ratio
(SNR).

Since the time input data are real time series, we have $h(f) = h^\ast(-f)$
and the inner product becomes
\begin{equation}
\left(a\mid b\right) = 2 \int_{-\infty}^{\infty}df\,
\frac{\tilde{a}^\ast(f)\tilde{b}(f)}{S_h\left(\left|f\right|\right)}
\end{equation}
so the normalisation constant $\sigma$ is
\begin{eqnarray}
\sigma^2 &=& 2 \int_{-\infty}^{\infty}df\,
\frac{\tilde{h_c}^\ast(f)\tilde{h_c}(f)}{S_h\left(\left|f\right|\right)} \\
&=& \frac{\tilde{h_s}^\ast(f)\tilde{h_s}(f)}{S_h\left(\left|f\right|\right)}.
\end{eqnarray}

The filter that we have constructed is normalised according to the convention
of Cutler and Flanagan \cite{cutflan}, so that in the case when the detector
output is Gaussian noise, the filter output averaged over an ensemble of
detectors with different realisatons of the noise is
\begin{equation}
\left\langle \rho^2 \right\rangle = \left\langle x^2 + y^2 \right\rangle = 2.
\end{equation}

Since the two chirp waveforms $\tilde{h_c}$ and $\tilde{h_s}$ are assumed to
be orthogonal, it is possible to increase the efficency of the algorithm by
contructing the single time $\rho(t)$ directly using a single complex inverse
FFT rather than computing it from $x(t)$ and $y(t)$ which requires two real
inverse FFTs. Let us write $x(t)$ in the following way
\begin{eqnarray}
x(t) 
&=& 2 \left[ 
\int_{-\infty}^{0} df\, e^{2\pi i f t} \frac{\tilde{h}(f)
\tilde{h_c}^\ast(f)}{S_h\left(\left|f\right|\right)}
+ \int_{0}^{\infty} df\, e^{2\pi i f t} \frac{\tilde{h}(f)
\tilde{h_c}^\ast(f)}{S_h\left(\left|f\right|\right)} 
\right] \\
&=& 2 \left[ 
\int_{0}^{\infty} df\, e^{-2\pi i f t} \frac{\tilde{h}^\ast(f)
\tilde{h_c}(f)}{S_h\left(\left|f\right|\right)}
+ \int_{0}^{\infty} df\, e^{2\pi i f t} \frac{\tilde{h}(f)
\tilde{h_c}^\ast(f)}{S_h\left(\left|f\right|\right)} 
\right] \\
&=& 2 \left[ I^\ast + I \right],
\end{eqnarray}
similarly for $y(t)$
\begin{eqnarray}
y(t) 
&=& 2 \left[ 
\int_{-\infty}^{0} df\, e^{2\pi i f t} \frac{\tilde{h}(f)
\tilde{h_s}^\ast(f)}{S_h\left(\left|f\right|\right)}
+ \int_{0}^{\infty} df\, e^{2\pi i f t} \frac{\tilde{h}(f)
\tilde{h_s}^\ast(f)}{S_h\left(\left|f\right|\right)} 
\right] \\
&=& 2 \left[ 
\int_{0}^{\infty} df\, e^{-2\pi i f t} \frac{\tilde{h}^\ast(f)
\tilde{h_s}(f)}{S_h\left(\left|f\right|\right)}
+ \int_{0}^{\infty} df\, e^{2\pi i f t} \frac{\tilde{h}(f)
\tilde{h_c}^\ast(f)}{S_h\left(\left|f\right|\right)} 
\right] \\
&=& 2 \left[ 
\int_{0}^{\infty} df\, e^{-2\pi i f t} \frac{\tilde{h}^\ast(f)
i\tilde{h_c}(f)}{S_h\left(\left|f\right|\right)}
+ \int_{0}^{\infty} df\, e^{2\pi i f t} \frac{\tilde{h}(f)
(-i)\tilde{h_s}^\ast(f)}{S_h\left(\left|f\right|\right)} 
\right] \\
&=& -2i \left[ 
- \int_{0}^{\infty} df\, e^{-2\pi i f t} \frac{\tilde{h}^\ast(f)
\tilde{h_c}(f)}{S_h\left(\left|f\right|\right)}
+ \int_{0}^{\infty} df\, e^{2\pi i f t} \frac{\tilde{h}(f)
\tilde{h_s}^\ast(f)}{S_h\left(\left|f\right|\right)} 
\right] \\
&=& -2i \left[ - I^\ast + I \right],
\end{eqnarray}
where $I$ is defined to be
\begin{equation}
I = \int_{0}^{\infty} df\, e^{2\pi i f t}\frac{\tilde{h}(f)
\tilde{h_s}^\ast(f)}{S_h\left(\left|f\right|\right)}.
\end{equation}
Now we define the quantity $z(t)$ to be
\begin{equation}
z(t) = 4 I = 4 \int_{0}^{\infty} df\, e^{2\pi i f t}\frac{\tilde{h}(f)
\tilde{h_s}^\ast(f)}{S_h\left(\left|f\right|\right)}
\end{equation}
and then
\begin{eqnarray}
x(t) &=& \Re z(t) \\
y(t) &=& \Im z(t)
\end{eqnarray}
so the filter output becomes
\begin{equation}
\rho(t) = \frac{1}{\sigma} \sqrt{x^2(t) + y^2(t)} 
= \frac{1}{\sigma} \sqrt{|z^2(t)|}
\end{equation}
