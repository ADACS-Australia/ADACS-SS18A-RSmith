From root@star.phys.uwm.edu  Sat Sep  1 10:47:50 2001
Return-Path: <root@star.phys.uwm.edu>
Received: from star.phys.uwm.edu (IDENT:root@star.phys.uwm.edu [129.89.57.99])
	by gravity.phys.uwm.edu (8.9.3/8.9.3) with ESMTP id KAA04174
	for <jolien@gravity.phys.uwm.edu>; Sat, 1 Sep 2001 10:47:49 -0500
Received: (from root@localhost)
	by star.phys.uwm.edu (8.9.3/8.9.3) id KAA06801
	for jolien@gravity.phys.uwm.edu; Sat, 1 Sep 2001 10:49:18 -0500
Date: Sat, 1 Sep 2001 10:49:18 -0500
From: root <root@star.phys.uwm.edu>
Message-Id: <200109011549.KAA06801@star.phys.uwm.edu>
To: jolien@gravity.phys.uwm.edu
Status: RO
Content-Length: 3864
Lines: 92

\subsubsection*{Description}
The \verb|SPStorageAlloc| and \verb|SPStorageFree| functions create and 
destroy an object \verb|**spSegment|, which is a handle to the first 
member of an array of length \verb|numSegments| of type \verb|SPDataSegment|.
The length of the \verb|COMPLEX8Vector| in each \verb|SPDataSegment| is
\verb|numPoints|$/2 +1$, where the parameter \verb|numPoints| must be set 
to the length of the \verb|INT2TimeSeries| IFODMRO time series in the 
\verb|DataSegment| to be processed.
The length of the \verb|INT4Vector| in each \verb|SPDataSegment| is 
determined by the parameter \verb|numChisqBins|, which must be greater
than zero.
The handle passed into \verb|SPStorageAlloc| must be a valid handle (i.e. 
\verb|spSegment|$\neq$\verb|NULL|), but must not point to an existing
object (i.e. \verb|*spSegment|$=$\verb|NULL|).
The handle passed into \verb|SPStorageFree| must point to the first
member of an array of length \verb|numSegments| of type \verb|SPDataSegment|.

The function \verb|SPConditionData| takes an object \verb|*segment|, of type
\verb|DataSegment|, and computes the elements \verb|*a| and \verb|alphasqk| of 
the \verb|*spSegment| structure. It also sets the stucture member \verb|*segment| 
to point to the \verb|DataSegment| that has been conditioned.

\subsubsection*{Algorithm}

Recall that we aim to compute the signal to noise ratio
\begin{equation}
\rho^2(t_j; M, \eta) = 
\frac{16}{\alpha^2} \left(\frac{\Delta t}{N} \right)^2 
  \left| \sum_{k=0}^{N-1} e^{2\pi ijk/N} \tilde{z}_k \right|^2
\label{e:rhosq}
\end{equation}
which requires us to compute the quantities $\alpha^2$ and $\tilde{z}_k$.
Consider the normalisation constant $\alpha^2$. We may write this as 
\begin{eqnarray}
\nonumber
\frac{\alpha^2}{2} 
&=& \frac{\Delta t}{N} 
%  \sum_{k=0}^{N-1} \frac{\tilde{h}_{ck}\tilde{h}^\ast_{ck}}{S_h(|f_k|)}
%  = 2 \frac{\Delta t}{N} 
  \sum_{k=0}^{N/2} \frac{\tilde{h}_{ck}\tilde{h}^\ast_{ck}}{S_h(|f_k|)}
= 2 \left(\frac{\Delta t}{N}\right)
   A_0^2(M,\eta) \sum_{k=0}^{N/2} \frac{k^{-7/3}}{S_h(f_k)} \\
&=& 2 \left(\frac{\Delta t}{N}\right) A_0^2(M,\eta)\, \alpha_{[k]}^2.
\end{eqnarray}
This defines the quantity $\alpha_{[k]}^2$, which the template independent
part of the normalisation.
The quantity $\tilde{z}_k$ may be written for $k\ge 0$ as
\begin{eqnarray}
\tilde{z}_k 
&=& \frac{\tilde{h}_k \tilde{h}^\ast_{ck}}{S_h(f_k)}
= \frac{\tilde{h}_k k^{-7/6} A_0(M,\eta) \exp\left[-i\Psi(f_k;M,\eta)\right]}
         {S_h(f_k)} \\
&=& a_k  A_0(M,\eta) \exp\left[-i\Psi(f_k;M,\eta)\right] 
\end{eqnarray}
where $a_k$ is the template independent part of $\tilde{z}_k$.
\verb|SPConditionData| computes $\alpha^2_{[k]}$ and $a_k$ and stores 
them in \verb|spSegment->alphasqk| and \verb|spSegment->a|.

During construction of the quantities $\alpha^2_{[k]}$ and $a_k$ 
it is necessary to compute the inverse one sided noise power spectrum
$S_h^{-1}(|f_k|)$. \verb|SPConditionData| allows the user to apply a
low frequency cutoff to $S_h^{-1}(|f_k|)$. This specified in units of
Hz by the parameter \verb|fLow|. Below this frequency the inverse
noise power spectrum will be set to zero.

It is also possible to truncate the inverse one sided noise power 
spectrum in the time domain, which is sometimes necessary to smooth
out line features in $S_h^{-1}(|f_k|)$. This is done by setting a
non-zero value of the parameter \verb|invSpecTrunc|. This is an 
integer that specifies the number of \emph{non-zero} points in the
inverse one sided noise power spectum \emph{in the time domain.}

\subsubsection*{Uses}
\begin{verbatim}
LALMalloc()
LALFree()
CreateVector()
CCreateVector()
I4CreateVector()
DestroyVector()
CDestroyVector()
I4DestroyVector()
EstimateFwdRealFFTPlan()
EstimateInvRealFFTPlan()
FwdRealFFT()
InvRealFFT()
DestroyRealFFTPlan()
DestroyRealFFTPlan()
memset()
\end{verbatim}

\subsubsection*{Notes}

