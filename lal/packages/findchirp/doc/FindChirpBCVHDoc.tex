\subsection*{Binary Black Holes}

For the binary black hole inspiral we use the BCV templates.

\subsubsection*{BCV Templates}
\label{BCV}

The signal-to-noise ratio (SNR) for a signal $s$ and a template $h$ is given by
\begin{equation}
\rho(h) = \frac{<s,h>}{\sqrt{<h,h>}},
\end{equation}
with the inner product $<s,h>$ being defined as
\begin{equation}
<s,h> = 2 \int_{-\infty}^{\infty} \frac{\tilde{s}^{\ast}(f) \tilde{h}(f)}
{S_h(|f|)} df =
4 \Re \int_0^{\infty} \frac{\tilde{s}^{\ast}(f) \tilde{h}(f)}{S_h(f)} df
\label{SNRdef}
\end{equation}
and $S_h(f)$ being the one-sided noise power spectral density.
The last equality in Eq.~(\ref{SNRdef}) holds only if $\tilde{s}(f)$ and
$\tilde{h}(f)$ are the Fourier-transforms of real time-functions.

The effective frequency-domain template given by Buonanno, Chen and Vallisneri
is
\begin{equation}
\tilde{h}(f) = A(f) e^{i \psi(f)}
\label{tmplt}
\end{equation}
where
\begin{equation}
A(f) = f^{-7/6} (1-\alpha f^{2/3}) \theta(f_{cut}-f),
\end{equation}
\begin{equation}
\psi(f) = \phi_0 + 2 \pi f t_0 + f^{-5/3} (\psi_0 + \psi_{1/2} f^{1/3} +
\psi_1 f^{2/3} + \psi_{3/2} f + \ldots).
\end{equation}
In these expressions, $t_0$ and $\phi_0$ are the time of arrival and the
frequency-domain phase offset respectively,
and $\theta$ is the Heaviside step function.
For most inspiral templates approximated by the template (\ref{tmplt}),
it is sufficient to use the parameters $\psi_0$ and $\psi_{3/2}$ and set all
other $\psi$ coefficients equal to 0.
So in the following:
\begin{eqnarray}
\psi(f) &=& \phi_0 + 2 \pi f t_0 + f^{-5/3} (\psi_0 + \psi_{3/2} f)\\
        &=& \phi_0 + \psi'(f) . \\
\end{eqnarray}

To simplify the equations, the abbreviation
\begin{equation}
I_k \equiv 4 \int_0^{f_{cut}} \frac{df}{f^k S_h(f)}
\end{equation}
is used in the following.

Notice that in the code, $\psi_0$ is \texttt{psi0} and $\psi_{3/2}$ is
\texttt{psi3}.

\subsubsection*{Normalized Template}
\label{NormOfTemplate}

We begin by normalizing the template $\tilde{h}(f)$.
Specifically, it is assumed that the normalized template is
\begin{equation}
\hat{h}(f) = N \tilde{h}(f)
\end{equation}
where $N$ is a real number. Then:
\begin{eqnarray}
&& <\hat{h}, \hat{h}> = 1 \: \Rightarrow  \:
4 \Re \int_0^{\infty} \frac{\hat{h}^{\ast} \hat{h}}{S_h(f)} df = 1 \:
	\Rightarrow  \\
&& 4 N^2 \int_0^{\infty} \frac{ [ f^{-7/6} (1-\alpha f^{2/3})]^2 \theta
	(f_{cut}-f) }{S_h} df = 1 \: \Rightarrow \\
&& N = \sqrt{ I_{7/3} - 2 \: \alpha \: I_{5/3} + \alpha^2 \: I_1 }
\label{Normalize}
\end{eqnarray}
So the normalized template is
\begin{equation}
\hat{h}(f) = \frac{1}{\sqrt{ I_{7/3} - 2 \: \alpha \: I_{5/3} +
	\alpha^2 \: I_1 }} f^{-7/6} (1-\alpha f^{2/3}) e^{i \phi_0} e^{i \psi'}
	\theta(f_{cut}-f), \: f>0
\end{equation}
and $\hat{h}(f) = \hat{h}^{\ast}(-f), \: f<0$.

Next we construct an orthonormal basis $\{ \hat{h}_j \}$ for the 4-dimensional
linear subspace of templates, with $\phi_0 \in [0,2\pi)$ and
$\alpha \in (-\infty, \infty)$ and all other parameters fixed.
Specifically, we want the basis vectors to satisfy
\begin{equation}
< \hat{h}_i , \hat{h}_j > = \delta_{ij}.
\label{OrthonormBasis}
\end{equation}
For that we construct two real functions $A_1(f)$ and $A_2(f)$, linear
combinations of $f^{-7/6}$ and $f^{-1/2}$, which are related to the 4 basis
vectors via:
\begin{eqnarray}
 \hat{h}_{1,2}(f) &=& A_{1,2}(f) \: e^{i \psi'(f)} \\
 \hat{h}_{3,4}(f) &=& A_{1,2}(f) \: i \: e^{i \psi'(f)}.
\end{eqnarray}
Then, Eq. (\ref{OrthonormBasis}) becomes:
\begin{equation}
4 \Re \int_0^{\infty} \frac{A_i(f) A_j(f)}{S_h} df = \delta_{ij}.
\label{OrthonormA}
\end{equation}
So we choose:
%\begin{equation}
%\begin{pmatrix}A_1(f) \cr A_2(f) \cr
%\end{pmatrix}
%=
%\begin{pmatrix}
%a_1 &0 \cr b_1 &b_2 \cr
%\end{pmatrix}
%\begin{pmatrix}
%f^{-7/6} \cr
%f^{-1/2} \cr
%\end{pmatrix}
%\Rightarrow
%\end{equation}

\begin{eqnarray}
A_1(f) &=& a_1 f^{-7/6} \\
A_2(f) &=& b_1 f^{-7/6} + b_2 f^{-1/2}.
\end{eqnarray}
Imposing condition (\ref{OrthonormA}) gives:
\begin{eqnarray}
\label{A1}
&& 4 \int_0^{\infty} \frac{A_1(f) A_1(f)}{S_h} df = 1 \: \Rightarrow \:
	a_1 = I_{7/3}^{-1/2} \\
\label{A2}
&& 4 \int_0^{\infty} \frac{A_2(f) A_2(f)}{S_h} df = 1 \: \Rightarrow \:
	b_1^2\: I_{7/3} + 2\: b_1\: b_2 \: I_{5/3} +
	b_2^2 \: I_1 = 1 \\
\label{A1A2}
&& 4 \int_0^{\infty} \frac{A_1(f) A_2(f)}{S_h} df = 0 \: \Rightarrow \:
	b_1 = - b_2 \frac{I_{5/3}}{I_{7/3}}
\end{eqnarray}
Solving Eqs. (\ref{A2}) and (\ref{A1A2}) we get
\begin{eqnarray}
b_1 &=& - \frac{I_{5/3}}{I_{7/3}} \Big ( I_1 - \frac{I_{5/3}^2}{I_{7/3}}
	\Big )^{-1/2} \\
b_2 &=& \Big ( I_1 - \frac{I_{5/3}^2}{I_{7/3}} \Big )^{-1/2}.
\end{eqnarray}

The next step is to write the normalized template in terms of the 4 basis
vectors
\begin{eqnarray}
\hat{h}(f) &=& c_1 \hat{h}_1(f) + c_2 \hat{h}_2(f) + c_3 \hat{h}_3(f) +
	c_4 \hat{h}_4(f) \: \Rightarrow \\
\frac{(f^{-7/6} - \alpha f^{-1/2}) e^{i \phi_0}}{\sqrt{I_{7/3} - 2 \alpha
	I_{5/3} + \alpha^2 I_1}} &=& (c_1 +i c_3) a_1 f^{-7/6} +
	(c_2 + i c_4) (b_1 f^{-7/6} + b_2 f^{-1/2} )
\end{eqnarray}
and matching the terms gives
\begin{eqnarray}
c_1 &=& \cos\phi_0 \cos\omega \\
c_2 &=& \cos\phi_0 \sin\omega \\
c_3 &=& \sin\phi_0 \cos\omega \\
c_4 &=& \sin\phi_0 \sin\omega
\end{eqnarray}
if the angle $\omega$ is defined by
\begin{equation}
\tan\omega \equiv - \frac{a_1 \alpha}{b_2 + b_1 \alpha}.
\end{equation}
So
\begin{equation}
\hat{h}(f) = \cos\phi_0 \cos\omega \:\hat{h}_1(f) +
	     \cos\phi_0 \sin\omega \:\hat{h}_2(f) +
	     \sin\phi_0 \cos\omega \:\hat{h}_3(f) +
	     \sin\phi_0 \sin\omega \:\hat{h}_4(f).
\end{equation}

\subsubsection*{Maximization of the SNR}
\label{SNRMaximization}

The Fourier-transformed data is $s(f)$. The SNR is
\begin{eqnarray}
\rho &=& <s,\hat{h}(f)> \\
     &=& \cos\phi_0 \cos\omega \: K_1 +
             \cos\phi_0 \sin\omega \: K_2+
             \sin\phi_0 \cos\omega \: K_3 +
             \sin\phi_0 \sin\omega \: K_4
\end{eqnarray}
where the 4 integrals are defined by
\begin{eqnarray}
\label{K1}
K_1 &=& <s,\hat{h}_1>  = \Re \int_0^{f_{cut}} \frac{4 s^{\ast} a_1 f^{-7/6}
	e^{i \psi'}}{S_h} df \\ \label{K2}
K_2 &=& <s,\hat{h}_2>  = \Re \int_0^{f_{cut}} \frac{4 s^{\ast} (b_1 f^{-7/6}
         + b_2 f^{-1/2} ) e^{i \psi'}}{S_h} df \\ \label{K3}
K_3 &=& <s,\hat{h}_3>  = \Re \int_0^{f_{cut}} \frac{4 s^{\ast} a_1 f^{-7/6} i
        e^{i \psi'}}{S_h} df
	= -\Im \int_0^{f_{cut}} \frac{4 s^{\ast} a_1 f^{-7/6} e^{i \psi'}}{S_h}
	df  \\ \label{K4}
K_4 &=& <s,\hat{h}_4>  = \Re \int_0^{f_{cut}} \frac{4 s^{\ast} (b_1 f^{-7/6}
	+ b_2 f^{-1/2} ) i e^{i \psi'}}{S_h} df
	= -\Im \int_0^{f_{cut}} \frac{4 s^{\ast} (b_1 f^{-7/6} + b_2 f^{-1/2})
	e^{i \psi'}}{S_h} df
\end{eqnarray}
Now set
\begin{eqnarray}
\label{OmegaMinusPhi}
A &=& \omega - \phi_0 \\
B &=& \omega + \phi_0 \label{OmegaPlusPhi}
\end{eqnarray}
so that the expression for the SNR becomes
\begin{eqnarray}
\rho &=& \frac{1}{2} K_1 [\cos(\omega+\phi_0)+\cos(\omega-\phi_0)] +
	\frac{1}{2} K_2 [\sin(\omega+\phi_0)+\sin(\omega-\phi_0)] +\\
	&& \frac{1}{2} K_3[\sin(\omega+\phi_0)-\sin(\omega-\phi_0)]
	+\frac{1}{2} K_4 [\cos(\omega-\phi_0)-\cos(\omega+\phi_0)]\Rightarrow\\
2 \rho &=& (K_1+K_4) \cos A + (K_2-K_3) \sin A + (K_1-K_4) \cos B + (K_2+K_3)
\sin B.
\label{MaxSNR}
\end{eqnarray}

To maximize with respect to $A$ we take the first derivative
\begin{equation}
\frac{\partial(2 \rho)}{\partial A} = - (K_1+K_4) \sin A
+(K_2-K_3) \cos A
\end{equation}
and set that equal to 0, which gives
\begin{equation}
\frac{\partial(2 \rho)}{\partial A} \Big |_{A_0} = 0 \: \Rightarrow
\: \tan A_0 =\frac{K_2-K_3}{K_1+K_4}. \\
\label{TANA}
\end{equation}
Then the sine and cosine of $A_0$ can be found:
\begin{eqnarray}
\sin A_0 &=& \pm \frac{\tan A_0}{\sqrt{1 + \tan^2 A_0}} = \pm
\frac{K_2-K_3}{\sqrt{(K_1+K_4)^2 +(K_2-K_3)^2}},  \label{SINA} \\
\cos A_0 &=& \pm \frac{1}{\sqrt{1 + \tan^2 A_0}} = \pm
\frac{K_1+K_4}{\sqrt{(K_1+K_4)^2 +(K_2-K_3)^2}}.
\label{COSA}
\end{eqnarray}
Notice that for Eq.~(\ref{TANA}) to be satisfied, the same sign must be kept
in Eqs (\ref{SINA}) and (\ref{COSA}).
To find the values that correspond to the maximum, we take the second
derivative of $\rho$ with respect to $A$:
\begin{equation}
\frac{\partial^2(2 \rho)}{\partial A^2}\Big |_{A_0} < 0 \: \Rightarrow
\: \Big [ - (K_1+K_4) \cos A - (K_2-K_3) \sin A \Big ]_{A_0} < 0
\end{equation}
which is satisfied if the $+$ sign is considered in Eqs (\ref{SINA}) and
(\ref{COSA}).

To maximize with respect to $B$ we take the first derivative
\begin{equation}
\frac{\partial(2 \rho)}{\partial B}  = - (K_1-K_4) \sin B
+(K_2+K_3) \cos B
\end{equation}
and set that equal to 0, which gives
\begin{equation}
\frac{\partial(2 \rho)}{\partial B} \Big |_{B_0} = 0 \: \Rightarrow
\: \tan B_0 =\frac{K_2+K_3}{K_1-K_4}. \\
\label{TANB}
\end{equation}
Then the sine and cosine of $B_0$ can be found:
\begin{eqnarray}
\sin B_0 &=& \pm \frac{\tan B_0}{\sqrt{1 + \tan^2 B_0}} = \pm
\frac{K_2+K_3}{\sqrt{(K_1-K_4)^2 +(K_2+K_3)^2}},  \label{SINB} \\
\cos B_0 &=& \pm \frac{1}{\sqrt{1 + \tan^2 B_0}} = \pm
\frac{K_1-K_4}{\sqrt{(K_1-K_4)^2 +(K_2+K_3)^2}}.
\label{COSB}
\end{eqnarray}
Again, the same sign must be kept in Eqs (\ref{SINB}) and (\ref{COSB}).
To find the values that correspond to the maximum, we take the second
derivative of $\rho$ with respect to $B$:
\begin{equation}
\frac{\partial^2(2 \rho)}{\partial B^2}\Big |_{B_0} < 0 \: \Rightarrow
\: \Big [ - (K_1-K_4) \cos B - (K_2+K_3) \sin B \Big ]_{B_0} < 0
\end{equation}
which is satisfied if the $+$ sign is considered in Eqs (\ref{SINB}) and
(\ref{COSB}).

Substituting the expressions for the sines and cosines of $A_0$ and
$B_0$
into Eq.~(\ref{MaxSNR}), the maximum SNR is:
\begin{eqnarray}
 \rho_{max} &=& \frac{1}{2} \sqrt{(K_1+K_4)^2 + (K_2-K_3)^2} +
\frac{1}{2} \sqrt{(K_1-K_4)^2 +(K_2+K_3)^2}\\
2\rho_{max}&=& \sqrt{ K_1^2 + K_2^2 + K_3^2 +K_4^2 + 2(K_1 K_4
	- K_2 K_3)} + \sqrt{  K_1^2 + K_2^2 + K_3^2 +K_4^2 -2(K_1
	K_4 -  K_2 K_3)}
\end{eqnarray}
To achieve a simpler form for the SNR, we can use Eqs (\ref{K1})-(\ref{K4}) to
combine the integrals $K_1$, $K_2$, $K_3$ and $K_4$.
Specifically:
\begin{equation}
K_1^2 + K_2^2 + K_3^2 + K_4^2 = \Big | \int_0^{f_{cut}} \frac{4 s^{\ast} a_1
	f^{-7/6} e^{i \psi'}}{S_h} df \Big |^2 + \Big | \int_0^{f_{cut}}
	\frac{4 s^{\ast} (b_1 f^{-7/6} + b_2 f^{-1/2}) e^{i \psi'}}{S_h} df
	\Big |^2
\end{equation}
and
\begin{equation}
2(K_1 K_4- K_2 K_3) =  2 \Im \Big \{ \int_0^{f_{cut}} \frac{4 s^{\ast} a_1
	f^{-7/6} e^{i \psi'}}{S_h} df \Big (\int_0^{f_{cut}} \frac{4 s^{\ast}
	(b_1 f^{-7/6} + b_2 f^{-1/2}) e^{i \psi'}}{S_h}df\Big )^{\ast} \Big \}.
\end{equation}

\subsubsection*{The $\chi^2$-veto}
\label{ChisquaredVeto}

If we are working with $p$ bins the maximum SNR for the template must be
divided into $p$ equal parts. In this case, since we have two different
amplitude-parts of the template, we have to calculate two sets of bin
boundaries.
For the first set, the quantity
\begin{displaymath}
\int_0^{f_{cut}} \frac{4 a_1^2 f^{-7/3}}{S_h(f)} df
\end{displaymath}
must be divided into $p$ equal pieces.
For the second set, the quantity
\begin{displaymath}
\int_0^{f_{cut}} \frac{4 [b_1 f^{-7/6} + b_2 f^{-1/2}]^2}{S_h(f)} df
\end{displaymath}
must be divided into $p$ equal pieces.

To check if the total SNR is smoothly distributed over the bins,
take:
\begin{eqnarray}
\chi^2 &=& \: p \sum_{l=1}^p  \Big | \int_{f_l}^{f_{l+1}} \frac{4 a_1
     f^{-7/6} \tilde{s}^{\ast} e^{i \psi'}}{S_h(f)} df
     - \frac{1}{p} \int_{0}^{f_{cut}} \frac{4 a_1 f^{-7/6}
 \tilde{s}^{\ast} e^{i \psi'}}{S_h(f)} df \Big |^2 \\
      &+&\: p\sum_{l=1}^p\Big | \int_{f_l}^{f_{l+1}} \frac{4[b_1 f^{-7/6}
	+ b_2 f^{-1/2} ]
     \tilde{s}^{\ast} e^{i \psi'}}{S_h(f)} df-\frac{1}{p}
     \int_{0}^{\infty} \frac{4[b_1 f^{-7/6} + b_2 f^{-1/2}]
\tilde{s}^{\ast} e^{i \psi'}}{S_h(f)} df\Big |^2. \\
\end{eqnarray}


\subsubsection*{Quantities calculated in the code for the case of BCV templates}

\begin{enumerate}
\item The template normalization squared (in \texttt{LALFindChirpBCVTemplate}):
\begin{equation}
\mathtt{tmpltNorm} = d^2 \Big ( \frac{5\mu}{96 M_{\odot}}\Big ) \Big ( \frac{M}        {\pi^2 M_{\odot}} \Big )^{2/3} T_{\odot}^{-1/3}
        \Big (\frac{2 T_{\odot} c}{1 Mpc} \Big )^2.
\end{equation}
\item The exponential $e^{i \psi'}$ (in \texttt{LALFindChirpBCVTemplate}).
\item The BCV Moments (in \texttt{LALFindChirpBCVData}): \\
\begin{eqnarray}
\mathtt{I73} &=& \sum_{k=0}^{N/2} \frac{4 k^{-7/3}}{|dR|^2 S_v(|f_k|)} \\
\mathtt{I53} &=& \sum_{k=0}^{N/2} \frac{4 k^{-5/3}}{|dR|^2 S_v(|f_k|)} \\
\mathtt{I1} &=& \sum_{k=0}^{N/2} \frac{4 k^{-1}}{|dR|^2 S_v(|f_k|)}. \\
\end{eqnarray}
These quantities should be multiplied by $(\Delta t /N)$ (b/c of the FT) and
by ($\mathtt{tmpltNorm}$) (b/c of the template), but that is taken
care of in the calculation of the SNR.
\item The BCV normalization factors (in \texttt{LALFindChirpBCVData}):
\begin{eqnarray}
\mathtt{a1} &=& (\mathtt{I73})^{-1/2} \\
\mathtt{b2} &=& \Big( \mathtt{I1} - \frac{\mathtt{I53}^2}{\mathtt{I73}}
	\Big )^{-1/2} \\
\mathtt{b1} &=& -\frac{\mathtt{I53}}{\mathtt{I73}}  \mathtt{b2}. \\
\end{eqnarray}
Again, these should be multiplied by $[(\Delta t/N)^{-1/2} \mathtt{tmpltNorm}
)^{-1/2}]$
but that is taken care of later.
\item The two FTs required for the calculation of the SNR (in
	\texttt{LALFindChirpBCVFilter}):
\begin{eqnarray}
qj &=&\sum_{k=0}^{N/2}e^{2 \pi i j k/N} \frac{(dR \tilde{v}_k^{\ast})
	\: \mathtt{a1} \: k^{-7/6} e^{i \psi'} }{ |dR|^2 S_v(|f_k|) } \\
q^{{}_{BCV}}_j &=& \sum_{k=0}^{N/2}e^{ 2 \pi i j k/N} \frac{(dR \tilde{v}_k
^{\ast})( \mathtt{b1} \:k^{-7/6} + \mathtt{b2}\: k^{-1/2} ) e^{i \psi'}}
	{|dR|^2 S_v(|f_k|)} \\
\end{eqnarray}
up to the appropriate normalization factors, namely $(\mathtt{tmpltNorm})
^{1/2}( \Delta t/N)$
\item The SNR (in \texttt{LALFindChirpBCVFilter}):
\begin{equation}
\rho^2(t_j) = \Big ( \frac{\Delta t}{N} \Big ) \bigg \{ \frac{1}{2}
	\sqrt{ |q_j|^2 + |q^{{}_{BCV}}_j|^2 + 2 \Im (q_j q^{{}_{BCV}\ast}_j) } +
	\frac{1}{2}\sqrt{|q_j|^2+|q^{{}_{BCV}}_j|^2 -
	2 \Im (q_j q^{{}_{BCV} \ast}_j) }
	\bigg \}.
\end{equation}
\end{enumerate}

