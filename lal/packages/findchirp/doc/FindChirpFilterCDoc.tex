\subsection{Module \texttt{FindChirpFilter.c}}
\label{ss:FindChirpFilter.c}

This module provides the core of the matched filter for binary inspiral
chirps.

\subsubsection*{Prototypes}
\vspace{0.1in}
\input{FindChirpFilterCP}
\idx{LALCreateFindChirpInput()}
\idx{LALDestroyFindChirpInput()}
\idx{LALFindChirpFilterInit()}
\idx{LALFindChirpFilterFinalize()}
\idx{LALFindChirpFilterSegment()}

\subsubsection*{Description}

\begin{description}
\item[\texttt{LALCreateFindChirpFilterInput()}] Memory allocation function to
create the \texttt{FindChirpFilterInput} structure used by the filtering code.
The function uses the parameters in the \texttt{FindChirpFilterParams}
structure to create workspace vectors of the appropriate length.

\item[\texttt{LALDestroyFindChirpInput()}] Memory deallocation function to
destroy the memory allocated by the function
\texttt{LALCreateFindChirpFilterInput()}.

\item[\texttt{LALFindChirpFilterInit()}] This function is used to initialize
the filtering code. It performs memory allocation for the filtering parameter
structures, workspace vectors and creates the FFT plans used by the filtering
code.

\item[\texttt{LALFindChirpFilterFinalize()}] Memory deallocation function
which frees all the memory allocated by \texttt{LALFindChirpFilterInit()}.

\item[\texttt{LALFindChirpFilterSegment()}] This function performs the core of
the match filtering algorithm. It takes a preconditioned data segment and an
inspiral template and searches the data segment for inspiral chirps according
to the algorithm described below.

\end{description}

\subsubsection*{Algorithm}

\paragraph*{Calculation of Wrap Around Due to Matched Filter}

\paragraph*{Algorithm Used to Search for Events}

\subsubsection*{Uses}
\begin{verbatim}
LALCalloc()
LALFree()
\end{verbatim}

\subsubsection*{Notes}

