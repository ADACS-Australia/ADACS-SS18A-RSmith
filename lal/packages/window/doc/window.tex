\chapter{Package \texttt{window}}

This package contains a function to create a vector
containing a window (also called
a taper, lag window, or apodization function).  The choices
currently available are:
\begin{itemize}
\item Rectangular
\item Hann
\item Welch
\item Bartlett
\item Parzen
\item Papoulis
\item Hamming
\item Kaiser.
\end{itemize}
Using window functions is well documented in many places.  Their
principal purpose is to reduce the {\it bias} in power spectrum
estimation.  For example, if a sinusoidal signal is present, it will
give rise to a spike in a power spectrum.  If such a  signal is present
exactly at the frequency of a particular bin, the spike will have some
height.  If a signal at the same amplitude but slighly different
frequency is present, and the frequency is not exactly the same
as one of the bins, then
the spike will be broader and lower. Without widowing, this effect introduces
bias into spectral estimates.

If the signal is first multiplied by the window function, the
relative difference between the two resulting power spectra will be
less evident.  The signal which would have given a single-bin spike
will now be spread over several bins.  The signal that would have
been spread over several bins will now be somewhat more peaked.
In short, the two power spectra will appear more similar (apart from
the difference in frequency of the two signals).  Hence the spectra
is less biased.

Definitions of most of the window functions above may be found in {\it
Numerical Recipes} \cite{numrec} equations 13.4.13-13.4.15.  Definitions of
the remaining windows can be found in {\it Spectral analysis for physical
applications} \cite{pw} Section 6.11. Definition of the Kaiser window can be
found in ``Discrete-time Signal Processing'' by Oppenheim and Schafer, p.474.



\newpage\input{WindowH}
\newpage\begin{thebibliography}{0}
\bibitem{numrec}
W. H. Press, S. A. Teukolsky, W. T. Vetterling, and B. P. Flannery,
  \textit{Numerical Recipes in C: The Art of Scientific Computing}, 2nd ed.
  (Cambridge University Press, Cambridge, England, 1992).
\bibitem{pw}
D.B. Percival and A.T. Walden, {\it Spectral analysis for physical applications}, first edition, Cambridge University Press, (1993).
\bibitem{pm}
J.G. Proakis and D.G. Manolakis, {\it Digital
Signal Processing: principles, algorithms and applications},
3rd Edition, 1995.
Prentice Hall
\end{thebibliography}
