\clearpage
\section{Frequency Series Manipulation}
\label{s:FrequencySeriesManipulation}

This is a suite of functions for creating, destroying, and manipulating LAL
frequency series.  One pair of functions (the XLAL version and its LAL
counterpart) is available for each method and frequency series type.  For
example \texttt{XLALCreateREAL4FrequencySeries()} is available for creating
frequency series of \texttt{REAL4} data, and the LAL-stype wrapper
\texttt{LALCreateREAL4FrequencySeries()} is provided which is equivalent to
the XLAL version in all respects except that it adheres to the LAL calling
conventions (eg.\ it takes a \texttt{LALStatus} pointer as its first
argument, its return type is \texttt{void}, etc.).

\subsection{Creation Functions}

\subsubsection{Name}

\texttt{XLALCreate}\textit{frequencyseriestype}\texttt{()},
\texttt{LALCreate}\textit{frequencyseriestype}\texttt{()}

\subsubsection{Synopsis}

\begin{verbatim}
#include <lal/FrequencySeries.h>
\end{verbatim}
\input{FrequencySeriesCreateP}

\subsubsection{Description}

These functions create LAL frequency series.  An XLAL function returns a
pointer to the newly created series or \texttt{NULL} on failure.  The LAL
counterpart accepts the address of a pointer which it fills with the
address of the newly created series or \texttt{NULL} on failure.
Additionally, the LAL wrapper provides standard LAL-style error checking
via a \texttt{LALStatus} pointer.

\subsubsection{Author}

\verb|Kipp Cannon <kipp@gravity.phys.uwm.edu>|


\subsection{Destruction Functions}

\subsubsection{Name}

\texttt{XLALDestroy}\textit{frequencyseriestype}\texttt{()},
\texttt{LALDestroy}\textit{frequencyseriestype}\texttt{()}

\subsubsection{Synopsis}

\begin{verbatim}
#include <lal/FrequencySeries.h>
\end{verbatim}
\input{FrequencySeriesDestroyP}

\subsubsection{Description}

These functions free all memory associated with a LAL frequency series.  It
is safe to pass \texttt{NULL} to these functions.

\subsubsection{Author}

\verb|Kipp Cannon <kipp@gravity.phys.uwm.edu>|


\subsection{Cutting Functions}

\subsubsection{Name}

\texttt{XLALCut}\textit{frequencyseriestype}\texttt{()},
\texttt{LALCut}\textit{frequencyseriestype}\texttt{()}

\subsubsection{Synopsis}

\begin{verbatim}
#include <lal/FrequencySeries.h>
\end{verbatim}
\input{FrequencySeriesCutP}

\subsubsection{Description}

These functions create a new frequency series by extracting a section of an
existing frequency series.

\subsubsection{Author}

\verb|Kipp Cannon <kipp@gravity.phys.uwm.edu>|


\subsection{Shrinking Functions}

\subsubsection{Name}

\texttt{XLALShrink}\textit{frequencyseriestype}\texttt{()},
\texttt{LALShrink}\textit{frequencyseriestype}\texttt{()}

\subsubsection{Synopsis}

\begin{verbatim}
#include <lal/FrequencySeries.h>
\end{verbatim}
\input{FrequencySeriesShrinkP}

\subsubsection{Description}

These functions reduce an existing frequency series to a section of itself.

\subsubsection{Author}

\verb|Kipp Cannon <kipp@gravity.phys.uwm.edu>|


\subsection{Adding Functions}

\subsubsection{Name}

\texttt{XLALAdd}\textit{frequencyseriestype}\texttt{()}

\subsubsection{Synopsis}

\begin{verbatim}
#include <lal/FrequencySeries.h>
\end{verbatim}
\input{FrequencySeriesAddP}

\subsubsection{Description}

These functions add the second argument to the first argument, returning a
pointer to the first argument on success or NULL on failure.  The two
series must have the same epoch and frequency resolution, and have units
that differ only by a dimensionless factor.

\subsubsection{Author}

\verb|Kipp Cannon <kipp@gravity.phys.uwm.edu>|
