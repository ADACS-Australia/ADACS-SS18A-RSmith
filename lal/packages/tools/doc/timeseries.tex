\clearpage
\section{Time Series Manipulation}
\label{s:TimeSeriesManipulation}

This is a suite of functions for creating, destroying, and manipulating LAL
time series.  One pair of functions (the XLAL version and its LAL
counterpart) is available for each method and series type.  For example
\texttt{XLALCreateREAL4TimeSeries()} is available for creating time series
of \texttt{REAL4} data, and the LAL-stype wrapper
\texttt{LALCreateREAL4TimeSeries()} is provided which is equivalent to the
XLAL version in all respects except that it adheres to the LAL calling
conventions (eg.\ it takes a \texttt{LALStatus} pointer as its first
argument, its return type is \texttt{void}, etc.).

\subsection{Creation Functions}

\subsubsection{Name}

\texttt{XLALCreate}\textit{timeseriestype}\texttt{()},
\texttt{LALCreate}\textit{timeseriestype}\texttt{()}

\subsubsection{Synopsis}

\begin{verbatim}
#include <lal/TimeSeries.h>
\end{verbatim}
\input{TimeSeriesCreateP}

\subsubsection{Description}

These functions create LAL frequency series.  An XLAL function returns a
pointer to the newly created series or \texttt{NULL} on failure.  The LAL
counterpart accepts the address of a pointer which it fills with the
address of the newly created series or \texttt{NULL} on failure.
Additionally, the LAL wrapper provides standard LAL-style error checking
via a \texttt{LALStatus} pointer.

\subsubsection{Author}

\verb|Kipp Cannon <kipp@gravity.phys.uwm.edu>|


\subsection{Destruction Functions}

\subsubsection{Name}

\texttt{XLALDestroy}\textit{timeseriestype}\texttt{()},
\texttt{LALDestroy}\textit{timeseriestype}\texttt{()}

\subsubsection{Synopsis}

\begin{verbatim}
#include <lal/TimeSeries.h>
\end{verbatim}
\input{TimeSeriesDestroyP}

\subsubsection{Description}

These functions free all memory associated with a LAL time series.  It is
safe to pass \texttt{NULL} to these functions.

\subsubsection{Author}

\verb|Kipp Cannon <kipp@gravity.phys.uwm.edu>|


\subsection{Cutting Functions}

\subsubsection{Name}

\texttt{XLALCut}\textit{timeseriestype}\texttt{()},
\texttt{LALCut}\textit{timeseriestype}\texttt{()}

\subsubsection{Synopsis}

\begin{verbatim}
#include <lal/TimeSeries.h>
\end{verbatim}
\input{TimeSeriesCutP}

\subsubsection{Description}

These functions create a new time series by extracting a section of an
existing time series.

\subsubsection{Author}

\verb|Kipp Cannon <kipp@gravity.phys.uwm.edu>|


\subsection{Resizing Functions}

\subsubsection{Name}

\texttt{XLALResize}\textit{timeseriestype}\texttt{()},
\texttt{LALResize}\textit{timeseriestype}\texttt{()},
\texttt{XLALShrink}\textit{timeseriestype}\texttt{()},
\texttt{LALShrink}\textit{timeseriestype}\texttt{()}

\subsubsection{Synopsis}

\begin{verbatim}
#include <lal/TimeSeries.h>
\end{verbatim}
\input{TimeSeriesResizeP}

\subsubsection{Description}

These functions resize an existing time series.  The new time series will
have the given length, and its contents will consist of that part of the
original time series that started at sample first.  If first is negative,
then the new time series is padded at the start by that many samples.  The
time series' epoch is adjusted appropriately.

\subsubsection{Author}

\verb|Kipp Cannon <kipp@gravity.phys.uwm.edu>|


\subsection{Addition Functions}

\subsubsection{Name}

\texttt{XLALAdd}\textit{timeseriestype}\texttt{()}

\subsubsection{Synopsis}

\begin{verbatim}
#include <lal/TimeSeries.h>
\end{verbatim}
\input{TimeSeriesAddP}

\subsubsection{Description}

These functions add the second argument to the first argument, returning a
pointer to the first argument on success or NULL on failure.  The two
series must have the same heterodyne frequency and time resolution, and
have units that differ only by a dimensionless factor.

\subsubsection{Author}

\verb|Kipp Cannon <kipp@gravity.phys.uwm.edu>|


\subsection{Subtraction Functions}

\subsubsection{Name}

\texttt{XLALSubtract}\textit{timeseriestype}\texttt{()}

\subsubsection{Synopsis}

\begin{verbatim}
#include <lal/TimeSeries.h>
\end{verbatim}
\input{TimeSeriesSubtractP}

\subsubsection{Description}

These functions subtract the second argument from the first argument,
returning a pointer to the first argument on success or NULL on failure.
The two series must have the same heterodyne frequency and time resolution,
and have units that differ only by a dimensionless factor.

\subsubsection{Author}

\verb|Kipp Cannon <kipp@gravity.phys.uwm.edu>|
