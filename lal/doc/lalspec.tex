\documentclass[10pt]{ligodcc}
\usepackage{hyperref}
\usepackage{color}
\usepackage{calc}

% make verbatims and teletype font blue
\let\keepverbatim\verbatim
\def\verbatim{\color{blue}\small\keepverbatim}
\makeatletter
\def\verb{\relax\ifmmode\hbox\else\leavevmode\null\fi
  \bgroup
    \color{blue}\small
    \verb@eol@error \let\do\@makeother \dospecials
    \verbatim@font\@noligs
    \@ifstar\@sverb\@verb}
\makeatother
\renewcommand{\texttt}[1]{{\ttfamily\color{blue}#1}}
\DeclareOldFontCommand{\tt}{\normalfont\ttfamily\color{blue}}{\mathtt\color{blue}}

\newcommand{\opt}[1]{\ensuremath{\color{blue}\langle\textit{#1}\rangle}}

\newsavebox{\fminibox}
\newlength{\fminilength}
\newenvironment{fminipage}[1][\linewidth]
  {\setlength{\fminilength}{#1-2\fboxsep-2\fboxrule}%
   \begin{lrbox}{\fminibox}\begin{minipage}{\fminilength}}
  {\end{minipage}\end{lrbox}\noindent\fbox{\usebox{\fminibox}}}
\newenvironment{lalrule}{\begin{quote}\color{red}\begin{fminipage}}
  {\end{fminipage}\end{quote}}

\ligodoc{T04XXXX}{A}
\ligodocdist{%
LIGO Scientific Collaboration\\
\textbf{DRAFT}\\
\bigskip
\begin{minipage}{6in}\color{red}\bfseries\Large
\begin{fminipage}\begin{center}
This document supersedes document LIGO-T990030
``Numerical Algorithms Library Specification and Style Guide''
\end{center}\end{fminipage}\end{minipage}
}
\title{LIGO Scientific Collaboration Algorithm Library Specification and Style
Guide}
\author{Jolien Creighton, Teviet Creighton, \emph{others}}
\rcsid$Id$

\begin{document}
\maketitle


\tableofcontents


\section{Introduction}
\label{s:intro}


\subsection{The LSC Algorithm Library}

The LSC Algorithm Library, hereafter LAL, is a library of routines for use in
gravitational wave data analysis.


\subsection{The goal of the LAL software specification}

From the first edition of this guide: \textbf{The defining purpose of this
document is to establish a software specification that fosters widespread-use
and collaborative-development of a well-tested analysis library.}  The goal is
to develop a \emph{portable} and \emph{convenient} library, both for users and
developers.

To achieve portability, the library is a library of routines written in a
subset of C99, which is well supported on nearly every modern computing
environment, and the routines can easily be used by programs written in
other languages (C++, Fortran, Python, etc.).

The first edition of this specification contained several highly restrictive
rules that were conceived to give LAL routines a standard ``look and feel''
to promote their ease of use and also to promote good programming practice.
Unfortunately, some of these rules turned out to be counterproductive, making
some routines have an awkward interface and encouraging large, monolithic
functions, i.e., promoting poor programming practices.

This second edition makes several significant changes to the original
specification in an attempt to alleviate the most egregious over-restrictive
rules.  However, since there is already a large existing code-base, it is
not possible to completely re-write the specification.  The new specification
must be consistent with the old conventions.  Therefore this new specification
still retains many of the elements of the old specification.

And sometimes for good reason!  Many of the rules in the original
specification, while perhaps being somewhat restrictive and burdensome on the
developer, were nevertheless very important for the concept of a portable data
analysis algorithm library (and did represent good programming practices).

The more rigid rules in this document are set aside in boxes with red type.
Here is the first rigid rule:
\begin{lalrule}
The rules in this specification may need to be periodically reexamined.
Therefore, this is a living document.  The librarian will amend this
document as needed.  Significant changes to the document will be made in
consultation with the LSC Software Change Control Board.
\end{lalrule}


\subsection{The elements of the library specification}

The library specification has the following elements:
\begin{enumerate}
\item A coding style, which is needed in order to establish a library namespace
(so that routines in the LAL library can be used in conjunction with other
libraries), and to maintain a (somewhat) uniform look-and-feel.
\item Function requirements, which are needed in order to maintain portability
and to establish a (relatively) standard API.
\item Standard data structures, macros, and functions, which assist in
providing a common set of tools for developers to promote uniformity (where
desired) and collaborative development.
\end{enumerate}


\subsection{The LAL and XLAL interfaces}

Because the original LAL specification was relatively ``heavy-weight,''
(in that there was significant overhead required for new functions), LAL
functions tended to be large and monolithic, and often a particular ``routine''
was re-written many times in-line in many different functions.  It is not
possible to simultaneously maintain the semblance of the original LAL
requirements on functions and completely remedy this deficiency.  Instead,
this specification introduces a second, parallel interface, called the XLAL
interface, specifically for writing small, light-weight, helper routines.  The
intent is for the LAL interface to be the primary interface for users of the
LAL library, but for the XLAL interface to be a convenient interface for use
within the LAL library.  (Think of the ``X'' being an underscore.)

There are some rules that apply to both the LAL and the XLAL interfaces, and
these will be described first.  Then the rules that are specific to the two
interfaces will be described in separate sections.
\begin{lalrule}
All functions in the LAL library that have external linkage shall be either
LAL functions or XLAL functions.  The LAL functions shall conform to the
general rules for functions and to the rules specific to LAL functions.
The XLAL functions shall conform to the general rules for functions and to
the rules specific to XLAL functions.
\end{lalrule}


\section{Coding style guidelines}

\subsection{Atomic data types}

For historical reasons more than anything I can think of, LAL routines should
use the LAL-specific atomic data types (\verb+REAL4+ rather than
\verb+float+, \verb+REAL8+ rather than \verb+double+, \verb+CHAR+ rather than
\verb+char+, \verb+UCHAR+ rather than \verb+unsigned char+) and should use
\verb+INT2+, \verb+UINT2+, \verb+INT4+, \verb+UINT4+, \verb+INT8+, and
\verb+UINT8+, which have a platform-independent size, rather than
\verb+short int+, \verb+unsigned short int+, \verb+int+, \verb+unsigned int+,
\verb+long int+, and \verb+unsigned long int+ (and especially not
\verb+long long+ or \verb+long double+) which do not.  LAL makes certain
requirements on these types.  For example, \verb+REAL4+ and \verb+REAL8+ must
be single and double precision IEEE-754 floating-point variables.  Invariably
they are equivalent to \verb+float+ or \verb+double+ on a given system (or
else LAL won't work at all on that system).  Similarly, a \verb+INT4+ is a
four-byte integer (and it is assumed that each byte is eight bits on any
system that LAL is installed), so it will be valid over the expected range.

Sometimes when the size of an integer variable is not crucial (e.g., for
return codes from \verb+XLAL+ functions), \verb+int+ is used.  It is also
necessary to allow the \verb+int+ type in LAL for functions such as
\verb+frexp+.  Standard C functions that have arguments that are pointers
to type \verb+double+, e.g., \verb+modf+, can receive a pointer to type
\verb+REAL8+ instead.  That is, the type \verb+REAL8+ can be assumed to be
equivalent to the type \verb+double+.


\subsection{Names of functions, variables, etc.}

These rules are to define a standard namespace scheme.  They apply to all
functions with external linkage (i.e., those functions not preceded by the
\verb+static+ keyword), as well as types, macros, etc., in header files.

\begin{enumerate}
\item All function names use StudlyCaps and begin with either \verb+LAL+
or \verb+XLAL+, e.g., \verb+LALExampleFunction+, \verb+LALDoICare+, etc.
Underscores are not used.
\item All types also use StudlyCaps and begin with a capital letter,
e.g., \verb+LALMyType+.  Custom data structures must be given names that try
to avoid namespace conflicts; we suggest simply prefixing the name with
\verb+LAL+ or \verb+XLAL+ or with the name of one of the LAL atomic data
types, e.g., \verb+REAL4+.
\item Global variables \textbf{of which there are NONE} (except those
specifically allowed by the Librarian), and fields within a structure or
a union, are in studlyCaps beginning with a lower case letter.  Global
variables will begin with either \verb+lal+ or \verb+xlal+, e.g.,
\verb+lalDebugLevel+.
\item Macros are generally all in UPPERCASE and compound macro names may
use underscores.  As with the types, to avoid namespace collisions, it is
recommended that the macro begin with \verb+LAL_+ or  \verb+XLAL_+.
\item Local variables can have any name that does not shadow a standard
global symbol name (whether in LAL or in a standard C library or other likely
names).  Thus, do not call a variable \verb+exit+ or \verb+LALMalloc+ or
even \verb+pow+.  And don't declare the variable \verb+i+ at the top level
of a function and then shadow it in a block within that function.  This is
just good programming practice.
\end{enumerate}

New data types will be declared as shown in this example for the data type
\verb+LALMyType+:
\begin{verbatim}
typedef struct
tagLALMyType
{
  INT4  firstField;
  REAL4 secondField;
}
LALMyType;
\end{verbatim}
Note that the structure name is \verb+tagLALMyType+.


\subsection{Header and source file conventions}

The LAL API is defined by the \emph{installed} header files (there may be
additional header files that are used when compiling LAL that are not
installed, but these then do not form part of the API as they are not made
available to the user).  All functions and variables with external linkage
as well as any datatypes, enumeration constants, macros, etc., that form
part of the API must be defined in these installed header files.
These installed header files will be installed in the location where header
files normally reside on a system in a subdirectory called \verb+lal+.  All
header files should include other LAL header files as follows: suppose
that \verb+LALThisHeader.h+ needs to include \verb+LALAnotherHeader.h+, then it
should do so as follows:
\begin{verbatim}
#include <lal/LALAnotherHeader.h>
\end{verbatim}

All header files should be idempotent.  This means they need to have include
guards.  To do so, the first two lines of \verb+LALThisHeader.h+ should be
something like:
\begin{verbatim}
#ifndef LALTHISHEADER_H
#define LALTHISHEADER_H
\end{verbatim}
and the last line of the file should be
\begin{verbatim}
#endif /* LALTHISHEADER_H */
\end{verbatim}

To be compatible with C++, all declarations should be wrapped as follows:
\begin{verbatim}
#ifdef __cplusplus
extern "C" {
#endif
\end{verbatim}
\opt{declarations}
\begin{verbatim}
#ifdef __cplusplus
}
#endif
\end{verbatim}

It is important that all source files (header files, whether installed or not,
and the C source files) contain the RCS ID information, which is then put into
the LAL library so that it can be examined later.  The convention for this
is to have all \verb+.h+ have lines similar to these (for
\verb+LALThisHeader.h+):
\begin{verbatim}
#include <lal/LALRCSID.h> /* if no other LAL header has been included */
NRCSID( LALTHISHEADERH, "$Id$" );
\end{verbatim}
near the top.  Since all LAL header files will ultimately include
\verb+lal/LALRCSID.h+, it only explicitly needs to be included if no other
LAL header has yet been included.  Note that the string \verb+"$Id$"+ will
be expanded by CVS into some string describing the current version of the file.
Similarly, a \verb+.c+ file such as \verb+LALThisSourceFile.c+ would have
the following
\begin{verbatim}
#include <lal/LALRCSID.h> /* if no other LAL header has been included */
NRCSID( LALTHISSOURCEFILEC, "$Id$" );
\end{verbatim}


\subsection{Language requirements}
LAL code should all be in ``clean C,'' i.e., that language that is a
subset of both C and C++.  This is not quite the same as just the C
programming language.  Only C-style comments should be used and avoid
any constructs that would behave differently with C++-style comments.
Names of variables, functions, etc., should not be any of the reserved
keywords or names for the library.  Some of the keywords and reserved
names are listed here.  The LAL namespace will assist in making sure
that no conflicts arise.  But even local variables names must be chosen
carefully (e.g., so that they aren't the same as a C++ keyword).
A list of keywords and reserved names, along with those standard C library
functions that can be used, is found in appendix~\ref{a:language}


\subsection{Filename conventions}

Purely for the sanity of the Librarian, LAL has a rigid directory structure.
LAL is composed of directories called packages whose names consist of entirely
lower-case letters with no underscores.  Within each package are four
sub-directories called \verb+doc+, \verb+include+ (which contains all the
header files that are installed), \verb+src+ (which contains all the source
files other than the installed header files but including those header files
that are not installed), and \verb+test+.  Source files within these
directories will be named with StudlyCaps (starting with a capital letter, no
underscores).  Refer to the LAL Software Documentation~\cite{lsd}.

For documentation purposes, a \emph{package} contains a set of \emph{headers}
(the installed headers), which contain prototypes for functions that are
organized in \emph{modules} of one or more functions, each module being a
single \verb+.c+ file.  Thus, all the functions with external linkage in a
\verb+.c+ file must be prototyped in the same header file in the same package.


\section{Common rules for both the LAL and the XLAL functions}


\subsection{Function arguments}

Function arguments must be one of the following (atomic) types:
\verb+CHAR+, \verb+UCHAR+, \verb+INT2+, \verb+UINT2+, \verb+INT4+,
\verb+UINT4+, \verb+INT8+, \verb+UINT8+, \verb+int+, \verb+REAL4+,
\verb+REAL8+.  In addition, functions may take a pointer as an argument.
Structures or unions (including \verb+COMPLEX8+ and \verb+COMPLEX16+) must not
be passed directly to a function as an argument; pass a pointer instead.
Arguments may be qualified with \verb+const+ if desired.
\begin{lalrule}
All arguments to functions must be of one of the following types:
\verb+CHAR+, \verb+UCHAR+, \verb+INT2+, \verb+UINT2+, \verb+INT4+,
\verb+UINT4+, \verb+INT8+, \verb+UINT8+, \verb+int+, \verb+REAL4+,
\verb+REAL8+, or a pointer to any object.  XLAL functions may also
have no arguments (\verb+void+), or a variable number of arguments
of the above types (\verb+...+).
\end{lalrule}


\subsection{Functions should not have any dependence on system environment}

The first part of this rule is that functions should not do any file I/O
since there should be no assumptions about the nature of the filesystem.
LAL is not supposed to assume POSIX.  Furthermore, there should be no
assumptions about (or dependence on) the environment under which a LAL
function is called.  This will allow LAL routines to be integrated into
a wide variety of programming environments: they may be used in stand-alone
programs or in loadable modules integrated into other run environments.
Specifically this means that routines in \verb+stdio.h+ are not allowed (except
for \verb+sprintf+, but there is a LAL replacement for this,
\verb+LALSnprintf+, which should be used instead), several routines in
\verb+stdlib.h+ including \verb+rand+ and \verb+srand+ (there are LAL
replacements for these), \verb+system+, and \verb+getenv+.

\begin{lalrule}
Functions will not perform any file I/O or have any dependence on the
system environment.  Specifically the latter means that functions such
as \verb+system+, \verb+getenv+, \verb+rand+, \verb+srand+ will not be
used.
\end{lalrule}


\subsection{Memory management}

Memory should always be allocated or freed with one of the LAL custom
memory managers: \verb+LALMalloc+, \verb+LALCalloc+, \verb+LALRealloc+,
and \verb+LALFree+, and \emph{not} with \verb+malloc+, \verb+calloc+,
\verb+realloc+, and \verb+free+.  The LAL memory managers have additional
memory leak checking ability that will assist in debugging if the debug
level is set appropriately.

\begin{lalrule}
All memory allocation shall be done with the functions \verb+LALMalloc+,
\verb+LALCalloc+, or \verb+LALRealloc+, and shall be freed with the functions
\verb+LALFree+ or \verb+LALRealloc+; the functions \verb+malloc+,
\verb+calloc+, \verb+realloc+, and \verb+free+ shall not be used.
\end{lalrule}

Also, routines should free all memory allocated in that routine \emph{except}
for the memory that is explicitly created by that routine, \emph{even if the
routine exits with a failure code}.  This will prevent memory leaks.  LAL
provides a routine \verb+LALCheckMemoryLeaks+ (which should not be called from
any LAL function---instead it is for users of LAL to put at the end of
\verb+main+) which will make sure that all memory allocated by \verb+LALMalloc+
(etc.) has been freed with \verb+LALFree+.

In fact, it is a good idea not to allocate any temporary memory within a
routine.  All temporary memory needed for a routine should be allocated by LAL
functions that are designed for that purpose.  Hence there are usually
three classes of LAL functions:
\begin{itemize}
\item \verb+LALCreateFoo+ or \verb+LALInitFoo+ functions which create and
initialize storage \verb+foo+ that the use will pass to...
\item \verb+LALBar+ functions which uses the storage \verb+foo+, and then
the user calls...
\item \verb+LALDestroyFoo+ or \verb+LALFinalizeFoo+ functions which destroys
the storage in \verb+foo+.
\end{itemize}


\subsection{Functions must be reentrant and thread-safe}

This rule essentially requires a functions behavior to depend only on its
arguments.  There should be no state saved in static storage within the
function.  That is, never use the \verb+static+ keyword within a function.
In addition, all variables used by a function must be local to the function.
That is, no global variables are allowed.  (There are a few exceptions to this,
e.g., the \emph{reading} of the global \verb+lalDebugLevel+ variable, but these
types of exceptions are under the control of the LAL librarian.)

Furthermore, use of routines that would cause a function to fail to be
reentrant and thread-safe are not allowed.  For example, many of the
\verb+time.h+ routines (\verb+asctime+, \verb+ctime+, \verb+gmtime+,
\verb+localtime+), some of the \verb+string.h+ routines (\verb+strerror+ and
\verb+strtok+), and other routines that are prohibited elsewhere for
additional reasons.
\begin{lalrule}
All functions will be reentrant and thread-safe.  All local variables
must be automatic.  No global variables will be used.  Routines such as
\verb+asctime+, \verb+ctime+, \verb+gmtime+, \verb+localtime+,
\verb+strerror+, and \verb+strtok+ shall not be used as they are not
reentrant and threadsafe.
\end{lalrule}


\subsection{Functions should always return control to the calling program}

That is, routines should never explicitly raise signals, abort, or call the
\verb+exit+ function, nor should they call functions that might do so.  Also,
since LAL is a library, don't change the behavior of \verb+exit+ or signals.
Thus the routines \verb+exit+, \verb+atexit+, and routines in \verb+signal.h+
and \verb+assert.h+ should not be used.  (But note the exception that XLAL
functions do call an error handler, which can be set outside of the library to
abort or exit.)  Also, long-jumps are not allowed, so any routine in
\verb+setjmp.h+ is not allowed.
\begin{lalrule}
All functions will return control to the calling function.  Functions such
as \verb+exit+, \verb+atexit+, \verb+raise+, \verb+assert+, \verb+abort+
shall not be used.  Long-jumps shall not be used.
\end{lalrule}




\section{Rules for LAL functions}

All LAL functions must return \verb+void+ and have as their first argument
a pointer to a \verb+LALStatus+ structure type.  Any number of arguments may
follow the status structure, though it is good style to be economical and to
group miscellaneous data into structures where useful.  The general convention
is to have the first argument following the status structure to be the primary
output from the function (i.e., a pointer to the result that is not used as
input to the function).  This general convention is for the convenience of the
user who will come to appreciate that arguments are typically ordered in LAL as
\begin{verbatim}
void LALREAL4Divide( LALStatus *status, REAL4 *result, REAL4 numerator,
    REAL4 denominator )
\end{verbatim}
rather than the following
\begin{verbatim}
void LALREAL4Divide( LALStatus *status, REAL4 numerator, REAL4 denominator,
    REAL4 *result )
REAL4)
\end{verbatim}
\begin{lalrule}
All LAL functions shall have names that begin with \verb+LAL+ followed by
an uppercase letter.
\end{lalrule}
\begin{lalrule}
All LAL functions shall have no return value (type \verb+void+ return).
\end{lalrule}
\begin{lalrule}
All LAL functions shall have a pointer to a \verb+LALStatus+ structure
as their first argument.  The contents of the \verb+LALStatus+ structure
will be populated appropriately to indicate success or failure of the
function call.  The \verb+LALStatus+ structure is a linked list.  If
a LAL function (the \emph{sub-function}) that is called from within a LAL
function fails (the \emph{top-function}), the status structure returned
by the sub-function shall be the next element in the linked list of status
structures returned by the top-function.
\end{lalrule}

The status structure is maintained from the calling program and keeps a trace
of all levels of LAL functions being called (it is a linked list of status
structures).  If a failure occurs, the status structure can be used to identify
where and which sequence of functions have been called.  The status structure
is central to the ``LAL interface.''  The status structure is not typically
manipulated by hand\ldots LAL provides several status handling macros for
manipulating the status structure and reporting errors.  The use of these
macros imposes additional conventions on writing LAL functions.  See the LAL
Software Documentation~\cite{lsd} for a complete description of these
conventions.  As a brief synopsis, this is what a the source code for a simple
LAL function such as \verb+LALREAL4Divide+ (in file \verb+LALDivide.c+) might
be:
\begin{verbatim}
#include <lal/LALDivide.h>
NRCSID( LALDIVIDEC, "$Id$" );

void
LALREAL4Divide(
    LALStatus *status,
    REAL4     *result,
    REAL4      numerator,
    REAL4      denominator
    )
{
   INITSTATUS( status, "LALREAL4Divide", LALDIVIDEC );
   ASSERT( result != NULL, status, LALDIVIDEH_ENULL, LALDIVIDEH_MSGENULL );
   if ( denominator == 0.0 )
     ABORT( status, LALDIVIDEH_EDIV0, LALDIVIDEH_MSGEDIV0 );
   *result = numerator / denominator;
   RETURN( status );
}
\end{verbatim}
Here the error codes \verb+LALDIVIDEH_ENULL+ and \verb+LALDIVIDEH_EDIV0+
and the corresponding error messages \verb+LALDIVIDEH_MSGENULL+ and
\verb+LALDIVIDEH_MSGEDIV0+ would be defined in \verb+LALDivide.h+.  The
\verb+INITSTATUS+ structure is the first line of a LAL function.  It populates
the status structure with useful information such as the function name and
RCS ID (which are the macro arguments).  The \verb+RETURN+ macro prepares the
status structure to indicate a nominal completion of the function; it should
be used with any successful return.  Error handling in this example is
accomplished using either the \verb+ASSERT+ or \verb+ABORT+ macros.  The
\verb+ASSERT+ macros are usually used to check the sanity of arguments;
the first macro argument is the result of a test that should be true otherwise
the \verb+ASSERT+ macro will populate the status structure with an error
code and message (specified by the third and fourth macro arguments) and
will return from the function.  The \verb+ASSERT+ functions are useful
during debugging and development of code, and they are removed when LAL
is compiled in production mode, so they can be used liberally.  Thus true
failures are captured instead with the \verb+ABORT+ macro.  Like \verb+ASSERT+,
\verb+ABORT+ populates the status structure with an error code and message and
returns, but it does not get removed when LAL is compiled in production mode.
The \verb+ABORT+ macro is the normal way of dealing with error conditions.

There are several other status structure macros that are needed when
preparing a status structure within a LAL function for calling another
LAL function, for checking the result of that function call, and for handling
situations when memory needs to be cleaned up before the function exits.
The conventions for these situations are all described in the LAL Software
Documentation.  Here we will just note that these macros should always be
used, and that a LAL function should never declare its own \verb+LALStatus+
structure for use when calling other LAL functions... the status structure
used must always be one atta(t)ched to the provided status structure (so that
the function call trace is maintained).


\section{Rules for XLAL functions}

The goal is to have XLAL functions be as flexible as possible in their
interface while still requiring strict rules on error reporting.  The
XLAL functions are intended to be ``lightweight'' functions that can
be used internally within the LAL library.  They don't have some of the
burdens of LAL functions.  In particular, they do not have a status structure.
This immediately implies to the following:
\begin{enumerate}
\item \textbf{XLAL functions cannot call LAL functions.}  Since LAL functions
require a status structure, and since this status structure must initiate in
the top-level program that interfaces with the LAL library (so that a trace of
function calls is returned), XLAL functions cannot call LAL functions, even
by having a local status structure.  If you need to call a LAL function
internally, the function must be a LAL function.
\item \textbf{XLAL functions can be ``lightweight.''}  Initializing the status
structure, atta(t)ching new structures to the list, setting the various fields
of the status structure to indicate successes or failures, etc., can be
somewhat burdensome for both the programmer and for the computer.  The lack
of a status structure in XLAL functions will relieve some of this burden and
will hopefully allow for some of the tasks that are now done in large,
monolithic code blocks to be divided into smaller and more modular XLAL
functions.  Compilers can then easily optimize code either by inlining the
XLAL function or not depending on issues that the compiler understands
(e.g., costs of a function call vs. cache misses, etc.).
\item \textbf{XLAL functions must report success or failure in other ways.}
This puts some more burden on the developers to (i) make sure that the XLAL
function correctly reports errors and (ii) understand how particular XLAL
functions report their errors and deal with these appropriately.  The goal
is to design some rules for the XLAL functions that try to approach some
degree of uniformity without overly hampering their interface.
\end{enumerate}
For the purpose of providing a relatively uniform error reporting system, it
is necessary to categorize XLAL functions into four likely types, which are
based on their return types.
\begin{lalrule}
All XLAL functions shall have a name beginning with \verb+XLAL+ followed by
an uppercase letter.
\end{lalrule}
\begin{lalrule}
XLAL functions shall not call LAL functions.
\end{lalrule}
\begin{lalrule}
The return type of XLAL functions shall be one of:
\verb+int+, \verb+CHAR+, \verb+INT2+, \verb+INT4+, or \verb+INT8+ (integer-type
return XLAL functions); \verb+REAL4+ or \verb+REAL8+ (floating-point-type
return XLAL functions); a pointer (pointer-type return XLAL functions);
or no return type (type \verb+void+ return XLAL functions).
\end{lalrule}

\subsection{Four kinds of XLAL functions}

XLAL functions will be one of four types based principally on their return
type, though this is also largely determined by their functional nature.  The
way that the XLAL functions report an error through their return value depends
on which type it is.  In addition, all XLAL functions will report errors by
setting an XLAL error number, \verb+xlalErrno+, and invoking the XLAL error
handler (described below).  For each type of function there is a macro that
will perform all of these tasks.

\begin{enumerate}
\item \textbf{XLAL functions that return an integer.}
These are XLAL functions that
return one of \verb+CHAR+, \verb+INT2+, \verb+INT4+, \verb+INT8+, or \verb+int+.

Simple XLAL functions will return type \verb+int+ that will either be \verb+0+
to indicate success or \verb+-1+ to indicate failure.  However, sometimes it is
useful to have an XLAL function that counts things (e.g., nodes in a linked
list).  For these functions it is useful for the count to be the return value.
Therefore the rule for XLAL functions in this category is:
\begin{lalrule}
All XLAL functions that return an integer type shall return a negative result
to indicate a failure.
In addition, the \verb+xlalErrno+ shall be set to the appropriate error number
and the XLAL error handler shall be invoked.
\end{lalrule}
This means that there cannot be an XLAL function that returns an unsigned
integer type (including \verb+size_t+).

To report an error from this type of function, use the macro
\verb+XLAL_ERROR( func, errnum )+ where \verb+func+ is the function name
string and \verb+errnum+ is the XLAL error number (see below).

\item \textbf{XLAL functions that return a floating-point number.}  These are
XLAL functions that return either \verb+REAL4+ or \verb+REAL8+.

Such functions are quite useful for providing extended mathematical functions
to do things such as compute the value of a distribution at a certain point,
etc.  The value returned must still be checked to see if there was an error.
To flag an error, these functions should return a particular value that would
be impossible to obtain.  The value is given by the constants
\verb+XLAL_REAL4_FAIL_NAN+ which has the same bit pattern as the 32 bit
hexadecimal integer constant \verb+0x7fc001a1+, or \verb+XLAL_REAL8_FAIL_NAN+
which has the same bit pattern as the 64 bit hexadecimal integer constant
\verb+0x7ff80000000001a1+ respectively.  These constants are known as ``quiet''
(as opposed to ``signaling'') NaN (not-a-number) values.  However, owing to the
\verb+1a1+ at the end of the hexadecimal representation, they are not likely to
occur as a result of any calculation (e.g., \verb+0.0/0.0+) as it is unlikely
that any C library will use these particular NaN values.  Thus these values are
identifiable as failures arising from XLAL functions and represent impossible
results.

To summarize:
\begin{lalrule}
All XLAL functions that return a \verb+REAL4+ floating-point type shall return
the \verb+REAL4+ floating-point constant \verb+XLAL_REAL4_FAIL_NAN+ to indicate
a failure.  All XLAL functions that return a \verb+REAL8+ floating-point type
shall return the \verb+REAL8+ floating-point constant
\verb+XLAL_REAL8_FAIL_NAN+ to indicate a failure.
In addition, the \verb+xlalErrno+ shall be set to the appropriate error number
and the XLAL error handler shall be invoked.
\end{lalrule}
To report an error from these types of functions, use one of the macros
\verb+XLAL_ERROR_REAL4( func, errnum )+ or
\verb+XLAL_ERROR_REAL8( func, errnum )+ where \verb+func+ is the function name
string and \verb+errnum+ is the XLAL error number (see below).  The result from
a function call must be checked to see if one of these constants has been
returned.  This can be done with the macros \verb+XLAL_IS_REAL4_FAIL_NAN(val)+
and \verb+XLAL_IS_REAL8_FAIL_NAN(val)+.

\item \textbf{XLAL functions that return a pointer.}  These are often XLAL
functions that are used to create structures, but can also be functions that
return a pointer to the output structure.  An example of the latter, imagine
the function:
\begin{verbatim}
COMPLEX8 *XLALCOMPLEX8Add( COMPLEX8 *result, COMPLEX8 *val1, COMPLEX8 *val2 )
{
  if ( ! result || ! val1 || ! val2 ) /* NULL argument */
    XLAL_ERROR_NULL( "XLALCOMPLEX8Add", XLAL_EFAULT );
  result->re = val1->re + val2->re;
  result->im = val1->im + val2->im;
  return result;
}
\end{verbatim}
The \verb+XLAL_ERROR_NULL( func, errnum )+ macro prints out the function name
\verb+func+, sets the XLAL errno to \verb+errnum+ (in this case the
error number is \verb+XLAL_EFAULT+) and invokes the XLAL error handler
(see below).  It then returns \verb+NULL+.  All functions of this type will
indicate an error by returning \verb+NULL+:
\begin{lalrule}
All XLAL functions that return a pointer type shall return the result
\verb+NULL+ to indicate a failure.
In addition, the \verb+xlalErrno+ shall be set to the appropriate error number
and the XLAL error handler shall be invoked.
\end{lalrule}
To report an error from this type of function, use the macro
\verb+XLAL_ERROR_NULL( func, errnum )+ where \verb+func+ is the function name
string and \verb+errnum+ is the XLAL error number (see below).

\item \textbf{XLAL functions that do not have a return value (return
\texttt{void}.}  These are functions that \emph{really shouldn't} fail.  In
practice, they are almost always free-type functions that destroy memory
created by the create-type functions of the previous function type.  There is
no way to return a success/failure flag via the return value so all
success/failure information must be returned through the XLAL error number
\verb+xlalErrno+.
\begin{lalrule}
All XLAL functions that do not return a result (i.e., they return \verb+void+)
shall set the \verb+xlalErrno+ to the appropriate error number and shall
invoke the XLAL error handler.
\end{lalrule}
To report an error from this type of function, use the macro
\verb+XLAL_ERROR_VOID( func, errnum )+ where \verb+func+ is the function name
string and \verb+errnum+ is the XLAL error number (see below).
\end{enumerate}

\subsection{XLAL error numbers and error handlers}

XLAL functions use the modifiable lvalue \verb+xlalErrno+ (think of it as
a global \verb+int+-type variable) to codify the nature of a failure.  It
should not be used for any other purpose.  The values that \verb+xlalErrno+
is allowed to have are controlled.  It is quite analogous to the standard
C \verb+errno+.

To use \verb+xlalErrno+, set it to zero (no error) before calling an XLAL
function; call the function; and then check the value of \verb+xlalErrno+.
If it is non-zero, an error has occurred, and the value can be used to
determine the nature of the error.  The following table contains the XLAL
return codes and error numbers.  Note that \verb+xlalErrno+ should only be
set to one of the error numbers (or zero if there is no error).
\begin{center}
\begin{tabular}{|lrl|}
\hline
Code & Value & Meaning \\
\hline
\multicolumn{3}{|l|}{\bfseries\itshape Return codes (for XLAL functions that return {\tt int})} \\
\tt XLAL\_SUCCESS    &     0 & Success \\
\tt XLAL\_FAILURE    &  $-$1 & Failure \\
\hline
\multicolumn{3}{|l|}{\bfseries\itshape Error numbers} \\
\multicolumn{3}{|l|}{\it Standard error numbers} \\
\tt XLAL\_EIO        &     5 & I/O error \\
\tt XLAL\_ENOMEM     &    12 & Memory allocation error \\
\tt XLAL\_EFAULT     &    14 & Invalid pointer \\
\tt XLAL\_EINVAL     &    22 & Invalid argument \\
\tt XLAL\_EDOM       &    33 & Input domain error \\
\tt XLAL\_ERANGE     &    34 & Output range error \\
\multicolumn{3}{|l|}{\bfseries\itshape Extended error numbers begin at 128} \\
\multicolumn{3}{|l|}{\it Common error numbers for XLAL functions} \\
\tt XLAL\_EFAILED    &   128 & Generic failure \\
\tt XLAL\_EBADLEN    &   129 & Inconsistent or invalid vector length \\
\multicolumn{3}{|l|}{\bfseries\itshape Specific mathematical and numerical error numbers begin at 256} \\
\multicolumn{3}{|l|}{\it IEEE floating point error numbers} \\
\tt XLAL\_EFPINVAL   &   256 & Invalid floating point operation \\
\tt XLAL\_EFPDIV0    &   257 & Division by zero floating point error \\
\tt XLAL\_EFPOVRFLW  &   258 & Floating point overflow error \\
\tt XLAL\_EFPUNDFLW  &   259 & Floating point underflow error \\
\tt XLAL\_EFPINEXCT  &   260 & Floating point inexact error \\
\multicolumn{3}{|l|}{\it Numerical algorithm error numbers} \\
\tt XLAL\_EMAXITER   &   261 & Exceeded maximum number of iterations \\
\tt XLAL\_EDIVERGE   &   262 & Series is diverging \\
\tt XLAL\_ESING      &   263 & Apparent singularity detected \\
\tt XLAL\_ETOL       &   264 & Failed to reach specified tolerance \\
\tt XLAL\_ELOSS      &   265 & Loss of accuracy \\
\multicolumn{3}{|l|}{\bfseries\itshape Failure from within a function call: ``or'' error number with this} \\
\tt XLAL\_EFUNC      &  1024 & Internal function call failed \\
\hline
\end{tabular}
\end{center}
Note that the last error number, \verb+XLAL_EFUNC+, corresponds to a bit that
can be set on the current error number (using a bitwise-or) to indicate that
the this error occurred from within an internal function call, thereby
preserving some information about the error.

In addition to setting \verb+xlalErrno+ a failure condition should also invoke
the XLAL error handler \verb+XLALErrorHandler+.  This is a function pointer
(actually it can be a macro that results in a function pointer) to a function;
its type is
\begin{verbatim}
typedef void XLALErrorHandlerType( const char *func, const char *file, int line, int errnum );
\end{verbatim}
Thus the error handler takes the name of the function from which it is
invoked, \verb+func+, the file name of the source, \verb+file+, the
line number where it is called, \verb+line+, and the XLAL error number
\verb+errnum+.  The default error handler, \verb+XLALDefaultErrorHandler+,
will print an error message when it is invoked.  The user may set the error
handler to a different error handler, e.g., one that aborts when a failure
occurs.  \emph{However, the error handler should not be changed within a
LAL or an XLAL function.}  Replacing the error handler should always be
done in the top-level program.

To assist in setting \verb+xlalErrno+ and invoking the error handler,
the function
\begin{verbatim}
void XLALError( const char *func, const char *file, int line, int errnum );
\end{verbatim}
is provided which will perform both of these tasks.  The arguments are the same
as those of the error handler.  This function is called as part of the
actions of the macros \verb+XLAL_ERROR+, \verb+XLAL_ERROR_NULL+,
\verb+XLAL_ERROR_VOID+, \verb+XLAL_ERROR_REAL4+, and \verb+XLAL_ERROR_REAL8+,
all of which take two argument: a character string containing the name of
the current function and the integer error number.  These macros call
\verb+XLALError+ with the filename given by \verb+__FILE__+, the line number
\verb+__LINE__+ where the macro occurs, and with the function name and error
number.  They also return from the function with the appropriate failure
return code (depending on which macro was used).


\section{Documentation and unit tests}

The conventions for these are not within the scope of this specification;
they are described in the LAL Software Documentation.  It is a good guide
that every function with external linkage should have a unit test that can
be run automatically to make sure it is (and continues to be) sane.


\section{Other libraries required for LAL}

LAL is not a stand-alone library.  Two other libraries are required to build
and use LAL.  These are the ``Fastest Fourier Transform in the West (version
3)'' \textsc{fftw3} library (compiled in both single and double precision
formats) and the ``GNU Scientific Library'' \textsc{gsl} library.  The way
these libraries are integrated into LAL is different.

The \textsc{fftw3} library is integrated by wrapping certain \textsc{fftw3}
routines within LAL functions.  Other LAL functions should then use these
wrapping functions rather than make direct calls to the \textsc{fftw3} API.
This is possible because only a few functions in \textsc{fftw3} are needed
in LAL.

The \textsc{gsl} library provides many more functions than \textsc{fftw3}.
Some of these functions, e.g., those involving file I/O, are not suitable
for use within LAL.  However, the vast majority of the functions in
\textsc{gsl} are useful.  To facilitate their use within LAL, the macros
\verb+CALLGSL( statement, status )+ and \verb+TRYGSL( statement, status )+
are provided.  These macros wrap the statement \verb+statement+ within a
set of code designed to (i) ensure the thread-safety and standard behavior of
the error handler used by the \textsc{gsl} function call, and (ii) report
any error conditions reported by the \textsc{gsl} function in the LAL
status structure \verb+status+.


\section{Notable exceptions}

\begin{lalrule}
There will be exceptions to (nearly) all of these rules.  Exceptions are
under the strict control of the LAL librarian.
\end{lalrule}

Some parts of LAL must necessarily fail to conform to the above specifications.
Clear examples include \verb+LALMalloc+ and their kin routines which do not
use a LAL status structure and do not have \verb+void+ return (apart from
\verb+LALFree+) --- indeed, since they must manage a heap, they are not really
reentrant either.  The LAL Librarian will endeavor to make LAL conform to
the specification outlined in this document as much as is practical, but there
are occasions when violations must be allowed.  In such cases, the violations
are under strict control of the LAL Librarian.


\section{Beyond LAL... LALSupport, LALMetaIo, LALFrame}

Of the LAL requirements, the most \emph{functionally} limiting is the
requirement that no I/O is allowed.  This requirement is in place to
insulate the bulk of the library from requirements about the nature of
the system on which a program is being run.  The philosophy is that since
the library must always be integrated somehow into an executable program,
assumptions about the system should be made by the program rather than
by the library; hence the I/O should be contained within the program.

To assist a program with various I/O tasks and their integration with LAL,
several libraries associated with LAL are provided.  These libraries do not
need to conform to all of the LAL standards; in particular, their purpose
is to provide the I/O functions that are missing from LAL.

\begin{lalrule}
The \verb+LALSupport+, \verb+LALMetaIo+, and \verb+LALFrame+ libraries
contain routines that need not conform to all of the LAL specifications;
in particular, they contain routines that perform file I/O and/or
require additional libraries.  Routines from these libraries are
intended to be used along with LAL routines, but LAL routines shall not
call any routine from (or in any other way be dependent upon) these
libraries.
\end{lalrule}

Note that there are currently two official exchange data formats within
the LSC: the XML-based ``LIGO Lightweight'' \verb+LIGOlw+ format, and the
binary ``Interferometric Gravitational Wave Detector Data Frame Format''
or ``Frame'' format.  Libraries with routines that are specialized for I/O
with these formats are also available.


\subsection{The \texttt{LALSupport} library}

This library provides the basic file I/O routines that are used in conjunction
with the LAL library.  It is always built and installed along with the LAL
library.

\subsection{The \texttt{LALMetaIo} library}

This library provides I/O routines that are used to read/write the LIGO
lightweight data format.  These routines use the \textsc{metaio} library
routines as their engine.  This library is conditionally built and installed
by LAL if the \textsc{metaio} library is available.

\subsection{The \texttt{LALFrame} library}

This library provides I/O routines that are used to read/write the Frame
data format.  These routines use the \textsc{Frame} library
routines as their engine.  This library is conditionally built and installed
by LAL if the \textsc{Frame} library is available.

\appendix

\section{Language issues}
\label{a:language}

The C99 standard specifies certain keywords and standard library functions.
Only a subset of these are suitable for LAL functions.  However, for maximum
portability, one should avoid various extensions to the C99 keywords and
functions.  Here are some guidelines on writing portable LAL code.

\subsection{Namespace requirements}

Here is a list of keywords and reserved names.  These should be avoided
when choosing names of LAL variables, functions, etc.  The LAL namespace
will help avoid namespace collisions.  The LAL namespace conventions are
also given.

\subsubsection*{Keywords}

The code should avoid any of the following
keywords that are present in C++ as symbol names:
\begin{verbatim}
and          and_eq       asm          auto         bitand         bitor
bool         break        case         catch        char           class
compl        const        const_cast   continue     default        delete
do           double       dynamic_cast else         enum           explicit
export       extern       false        float        for            friend
goto         if           inline       int          long           mutable
namespace    new          not          not_eq       operator       or
or_eq        private      protected    public       register       reinterpret_cast
return       short        signed       sizeof       static         static_cast
struct       switch       template     this         throw          true
try          typedef      typeid       typename     union          unsigned
using        virtual      void         volatile     wchar_t        while
xor          xor_eq
\end{verbatim}
It is not a good idea to use \verb+fortran+ or \verb+entry+ either as these
are sometimes reserved.  Of these keywords, the ones that are C99 keywords are
\begin{verbatim}
_Bool           _Complex        _Imaginary      auto
break           case            char            const
continue        default         do              double
else            enum            extern          float
for             goto            if              inline
int             long            register        restrict
return          short           signed          sizeof
static          struct          switch          typedef
union           unsigned        void            volatile
while
\end{verbatim}
Of these there is (almost) no need to use \verb+auto+, \verb+char+,
\verb+double+, \verb+float+, \verb+long+, \verb+register+, \verb+short+,
\verb+signed+, \verb+unsigned+, or \verb+volatile+.

\subsubsection*{Reserved names}

According to the GNU C library, the following names are reserved (or may be
reserved in the future) by the C library:
\begin{itemize}
\item All global functions or variables that begin with an underscore,
e.g., \texttt{\_whatever}, are reserved.
\item All identifiers that begin with two underscores or with an underscore
followed by an uppercase letter
e.g., \texttt{\_\_whatever}, \texttt{\_Whatever}, are reserved.
\item Names beginning with the capital \texttt{E} and followed by a digit
or uppercase letter are reserved (for error codes).
\item Names beginning with \texttt{is} or \texttt{to} and followed by a
lowercase letter are reserved.
\item Names beginning with \texttt{LC\_} are reserved.
\item Names of all existing mathematical functions but suffixed with either
an \texttt{f} or an \texttt{l} are reserved.
\item Names of all existing mathematical functions but suffixed with either
an \texttt{f} or an \texttt{l} are reserved.
\item Names beginning with \texttt{SIG} or \texttt{SIG\_} and followed by
an uppercase letter are reserved.
\item Names beginning with \texttt{str}, \texttt{mem}, or \texttt{wcs} and
followed by a lowercase letter are reserved.
\item Names ending with \texttt{\_t} are reserved.
\end{itemize}

Certain headers reserve names too.  Since LAL is a library to be used by
others, it is important to respect these when coding the interface.
\begin{itemize}
\item Names prefixed with \texttt{d\_} are reserved in \verb+dirent.h+.
\item Names prefixed with \texttt{l\_}, \texttt{F\_}, \texttt{O\_} and
\texttt{S\_} are reserved in \verb+fcntl.h+.
\item Names prefixed with \texttt{gr\_} are reserved in \verb+grp.h+.
\item Names suffixed with \texttt{\_MAX} are reserved in \verb+limits.h+.
\item Names prefixed with \texttt{pw\_} are reserved in \verb+pwd.h+.
\item Names prefixed with \texttt{sa\_} and \texttt{SA\_} are reserved
in \verb+signal.h+.
\item Names prefixed with \texttt{st\_} and \texttt{S\_} are reserved in
\verb+sys/stat.h+.
\item Names prefixed with \texttt{tms\_} are reserved in \verb+sys/times.h+.
\item Names prefixed with \texttt{c\_}, \texttt{V}, \texttt{I}, \texttt{O},
\texttt{TC} and names prefixed with \texttt{B} followed by a digit are reserved
in \verb+termios.h+.
\end{itemize}

\subsubsection*{LAL namespace}

The LAL namespace will assist in avoiding namespace conflicts.  LAL reserves
any name that is prefixed with \texttt{LAL}, \texttt{LAL\_}, \texttt{XLAL},
\texttt{XLAL\_}, \texttt{lal}, or \texttt{xlal} followed by an uppercase
letter.  In addition LAL reserves names that begin with
\texttt{CHAR}, \texttt{UCHAR}, \texttt{INT2}, \texttt{INT4}, \texttt{INT8},
\texttt{UINT2}, \texttt{UINT4}, \texttt{UINT8}, \texttt{REAL4}, \texttt{REAL8},
\texttt{COMPLEX8}, \texttt{COMPLEX16}, and \texttt{LIGO} followed by an
uppercase letter.

\subsection{Allowed functions from standard C}

C libraries often have various extensions from the C99 standard C library,
but for portability purposes only those functions that are specified by the
C99 standard should be used (note: LAL does require a \emph{hosted} rather
than \emph{freestanding} environment).  Also, many of these should \emph{not}
be used in LAL because they would cause the LAL functions to violate one of the
above rules (e.g., would be used for file I/O, would cause a function to not
be reenterant, etc.).  For completeness, these are the allowed standard C
functions.  There are a few others that could be allowed, but are not
recommended (e.g., \verb+sprintf+, but \verb+LALSprintf+ is provided as a
preferred alternative) and are not listed here for that reason.  Also listed
here are macros and types that are defined in these headers.  If a function
or macro or type is not somewhere on this list, you should probably not use
it in a LAL function.
\begin{verbatim}
<stdio.h>
        sscanf
        EOF

<ctype.h>
        isalnum         isalpha         iscntrl         isdigit
        isgraph         islower         isprint         ispunct
        isspace         isupper         isxdigit
        tolower         toupper

<string.h>
        strcpy          strncpy         strcat          strncat
        strcmp          strncmp         strchr          strrchr
        strspn          strcspn         strpbrk         strstr
        strlen
        memcpy          memmove         memcmp          memchr
        memset
        NULL            size_t

<math.h>
        sin             cos             tan
        asin            acos            atan            atan2
        sinh            cosh            tanh
        exp             log             log10           pow
        sqrt            ceil            floor           fabs
        ldexp           frexp            modf           fmod
        HUGE_VAL

<stdlib.h>
        atof            atoi            atol
        strtod          strtol          strtoul
        bsearch         qsort
        abs             labs            div             ldiv
        NULL            size_t          div_t           ldiv_t

<errno.h>
        errno
        EDOM            ERANGE

<assert.h> NONE

<stdarg.h>
        va_start        va_arg          va_end

<setjmp.h> NONE

<signal.h> NONE

<time.h>
        difftime        mktime          strftime
        NULL            size_t          time_t          struct tm

<limits.h> NONE

<float.h> NONE
\end{verbatim}


\section{LAL Datatypes}

\subsection{Primitive Datatypes}

The primitive datatypes are defined in a separate header
\verb@LALAtomicDatatypes.h@, which is included by
\verb@LALDatatypes.h@.  This is done in order to facilitate the
interface between LAL and non-LAL modules.  By including just
\verb@LALAtomicDatatypes.h@, a non-LAL module can ensure that it is
using the same arithmetic standard as LAL, without being burdened by
LAL's more specialized structures.

Primitive datatypes are those that conceptually store a single number
or quantity.  They include both the \emph{atomic} datatypes and the
complex datatypes.

\subsubsection*{Atomic Datatypes}

Atomic LAL datatypes are platform-independent datatypes corresponding
to the basic types in the C/C++ language.  However, since the C/C++
types are not necessarily the same across platforms, the actual
mapping between LAL and C/C++ datatypes may be different on different
platforms.  The following table lists the LAL atomic datatypes, their
size and range, and the C/C++ datatype to which they \emph{usually}
correspond.

\begin{center}
\begin{tabular}{|lccc|}
\hline
  Type  & Bits & Range                        & Usual C/C++ type  \\
\hline
\tt CHAR    & 8 & \verb@'\0'@ to \verb@'\255'@ & \tt char          \\
\tt UCHAR   & 8 & \verb@'\0'@ to \verb@'\255'@ & \tt unsigned char \\
\hline
\tt INT2    & 16 & $-2^{-15}$ to $2^{15}-1$     & \tt short              \\
\tt INT4    & 32 & $-2^{-31}$ to $2^{31}-1$     & {\tt int} or
                                                {\tt long}              \\
\tt INT8    & 64 & $-2^{-63}$ to $2^{63}-1$     & \tt long long          \\
\tt UINT2   & 16 &          0 to $2^{16}-1$     & \tt unsigned short     \\
\tt UINT4   & 32 &          0 to $2^{32}-1$     & {\tt unsigned int} or
                                                {\tt long}              \\
\tt UINT8   & 64 &          0 to $2^{64}-1$     & \tt unsigned long long \\
\hline
\tt REAL4   & 32 &  $-3.4\times10^{38}$ to $3.4\times10^{38}$  & \tt float  \\
\tt REAL8   & 64 & $-1.8\times10^{308}$ to $1.8\times10^{308}$ & \tt double \\
\hline
\end{tabular}
\end{center}


The unsigned character and integer datatypes store their values
according to the usual binary system.  For signed characters and
integers, setting the most-significant bit indicates that the number
formed from the remaining bits should be added to the lower value of
the range.  The \verb@REAL4@ and \verb@REAL8@ datatypes should store
values according to the IEEE Standard 754 for Binary Floating-Point
Arithmetic, which gives them the following precisions and dynamic
ranges:

\begin{center}
\begin{tabular}{|l|cc|}
\hline
                              & \tt     REAL4       & \tt     REAL8        \\
\hline
Minimum positive subnormal    & $1.4\times10^{-45}$ & $4.9\times10^{-324}$ \\
Minimum positive normal       & $1.2\times10^{-38}$ & $2.2\times10^{-308}$ \\
Maximum finite normal         & $3.4\times10^{38}$  & $1.8\times10^{308}$  \\
Minimum fractional difference & $6.0\times10^{-8}$  & $1.1\times10^{-16}$  \\
Significant decimal digits    & 6--9                & 15--17               \\
\hline
\end{tabular}
\end{center}

The minimum positive subnormal is the smallest positive representable
number.  The minimum positive normal is the smallest positive number
that can be represented with full precision; that is, one whose
mantissa lies in the range [0.5,1).  The maximum finite normal is the
largest representable number other than the reserved code for
$+\infty$.  The minimum fractional difference is the smallest
fractional difference between consecutive representable numbers, or
\emph{half} the difference between 1 and the next representable
number.  Significant decimal digits gives the number of decimal digits
used to represent the binary number in decimal notation: the first is
the maximum number of digits that are guaranteed not to change upon
conversion to binary, the second is the number of digits required to
represent a unique binary quantity.

\subsubsection*{Complex datatypes}

LAL represents complex numbers as structures with two floating-point
fields, storing the real and imaginary parts.  These are considered
primitive datatypes (rather than aggregate or structured datatypes)
because they conceptually represent a single number.  Furthermore,
atomic and complex datatypes are treated equivalently by LAL aggregate
and structured datatypes.

\paragraph{\texttt{COMPLEX8}:}
This structure stores a single-precision complex number in 8~bytes of
memory.  The fields are:
\begin{description}
\item[-- \texttt{REAL4 re}] The real part.
\item[-- \texttt{REAL4 im}] The imaginary part.
\end{description}

\paragraph{\texttt{COMPLEX16}:}
This structure stores a double-precision complex number in 16~bytes of
memory.  The fields are:
\begin{description}
\item[-- \texttt{REAL8 re}] The real part.
\item[-- \texttt{REAL8 im}] The imaginary part.
\end{description}

\subsection{Aggregate datatypes}
\label{ss:aggregate-datatypes}

These datatypes store arbitrarily large sets or collections of
primitive datatypes.  At this level there is no physical
interpretation assigned to the objects (such as names or units); the
aggregate datatypes simply collect and arrange the primitive
datatypes.  The following types of aggregate datatypes are defines:
vectors, arrays, sequences, vector sequences, and array sequences.

\paragraph{\opt{datatype}\texttt{Vector}:}
This structure stores an ordered set of $n$ elements of type
\opt{datatype}, which can be any primitive datatype.  The
data are to be interpreted as being a point in an $n$-dimensional vector space.
The fields are:
\begin{description}
\item[-- \texttt{UINT4 length}] The number of data $n$.
\item[-- \opt{datatype} \texttt{*data}] Pointer to the data array.  The data
are stored sequentially as \verb@data[@$0,\ldots,n-1$\verb@]@.
\end{description}

\paragraph{\opt{datatype}\texttt{Array}:}
This structure stores a set of elements of type \opt{datatype},
which can be any primitive datatype, arranged as an $m$-dimensional
array.  That is, each element can be thought of as having $m$
indices, $\mathsf{A}_{i_0\cdots i_{m-1}}$, where each index $i_k$
runs over its own range $0,\ldots,n_k-1$.  The total number of
elements is then $N=n_0\times\cdots\times n_{m-1}$.  In memory the
array is ``flattened'' so that the elements are stored sequentially in
a contiguous block.  The fields are:
\begin{description}
\item[-- \texttt{UINT4Vector *dimLength}] Pointer to a vector of length
$m$, storing the index ranges $(n_0,\ldots,n_{m-1})$.
\item[-- \opt{datatype} \texttt{*data}] Pointer to the data array.  The data
element $\mathsf{A}_{i_0\cdots i_{m-1}}$ is stored as
\verb@data[@$i_{m-1} + n_{m-2}\times(i_{m-2} +
n_{m-3}\times(\cdots(i_1 + n_0\times i_0)\cdots))$\verb@]@; that is,
the index of \verb@data[]@ runs over the entire range of an index
$i_{k+1}$ before incrementing $i_k$.
\end{description}

\paragraph{\opt{datatype}\texttt{Sequence}:}
This structure stores an ordered set of $l$ elements of type
\opt{datatype}, which can be any primitive datatype.  It is
identical to \opt{datatype}\verb@Vector@ and is retained for historical
purposes only.

\paragraph{\opt{datatype}\texttt{VectorSequence}:}
This structure stores an ordered set of $l$ elements of type
\opt{datatype}\verb@Vector@, where \opt{datatype} can be any primitive
datatype.  Mathematically the sequence can be written as
$\{\vec{v}^{(0)},\ldots,\vec{v}^{(l-1)}\}$, where each element
$\vec{v}^{(j)}=(v^{(j)}_0,\ldots,v^{(i)}_{n-1})$ is a vector of length
$n$.  In memory the elements are ``flattened''; that is, they are
stored sequentially in a contiguous block of memory.  The fields are:
\begin{description}
\item[-- \texttt{UINT4 length}] The number of vectors $l$.
\item[-- \texttt{UINT4 vectorLength}] The length $n$ of each vector.
\item[-- \opt{datatype} \texttt{*data}] Pointer to the data array.  The data
element $v^{(j)}_i$ is stored as \verb@data[@$j\times n + i$\verb@]@;
that is, the index of \verb@data[]@ runs over the internal index of
each vector element before incrementing to the next vector element.
\end{description}


\paragraph{\opt{datatype}\texttt{ArraySequence}:}
This structure stores an ordered set of $l$ elements of type
\opt{datatype}\verb@Array@, where \opt{datatype} can be any primitive
datatype.  The indexing of an array sequence can get quite
complicated; it helps to read first the documentation for data arrays,
above.  Mathematically the data can be written as a set
$\{\mathsf{A}^{(j)}_{i_0\cdots i_{m-1}}$, where the sequence number
$j$ runs from 0 to $l-1$, and each array index $i_k$ runs over its own
range $0,\ldots,n_k-1$.  The total number of data in a given array
element is then $N=n_0\times\cdots\times n_{m-1}$, and the total
number of data in the sequence is $N\times l$.  In memory the array is
``flattened'' so that the elements are stored sequentially in a
contiguous block.  The fields are:
\begin{description}
\item[-- \texttt{UINT4 length}] The number $l$ of array elements in the
sequence.
\item[-- \texttt{UINT4 arrayDim}] The number of data $N$ (\emph{not} the
number of indecies $m$) in each array element of the sequence.
\item[-- \texttt{UINT4Vector *dimLength}] Pointer to a vector of length
$m$, storing the index ranges $(n_0,\ldots,n_{m-1})$.
\item[-- \opt{datatype} \texttt{*data}] Pointer to the data.  The element
$\mathsf{A}^{(j)}_{i_0\cdots i_{m-1}}$ is stored as
\verb@data[@$j\times N + i_{m-1} + n_{m-2}\times(i_{m-2} +
n_{m-3}\times(\cdots(i_1 + n_0\times i_0)\cdots))$\verb@]@; that is,
the index of \verb@data[]@ runs over the internal indecies of each
array element before incrementing to the next array element.
\end{description}

\subsection{Structured datatypes}
\label{ss:structured-datatypes}

These datatypes embed primitive and aggregate datatypes inside
structures that define their physical meaning.  Most of these
structures are wrappers for aggregate datatypes that store a physical
quantity as a function of time or frequency.  Other structures store
specific physical information, such as the GPS time, or the factored
response function of a filter.

\paragraph{\texttt{LIGOTimeGPS}:}
This structure stores the time, to nanosecond precision, synchronized
to the Global Positioning System time reference.  The zero time for
the GPS standard is the moment of midnight beginning January~6, 1980,
UTC.  The \verb@LIGOTimeGPS@ structure can represent times up to
68~years on either side of this epoch.  (Note that this is better than
an equivalently-sized \verb@REAL8@ representation of time, which can
maintain nanosecond precision only for times within 104~days of its
reference point.  However, the \verb@REAL8@ representation does allow
one to cover arbitrarily long timescales at correspondingly lower
precision.)  The fields are:
\begin{description}
\item[-- \texttt{INT4 gpsSeconds}] The number of seconds since the GPS
reference time.
\item[-- \texttt{INT4 gpsNanoSeconds}] The number of nanoseconds since
the last GPS second.
\end{description}


\paragraph{\texttt{LALUnit}:}
This structure stores units in the mksA system (plus Kelvin, Strain,
and ADC Count).  It also stores an overall power-of-ten scaling factor.
The fields are:
\begin{description}
\item[-- \texttt{INT2 powerOfTen}] The power $p$ of ten scaling factor.
\item[-- \texttt{INT2 unitNumerator[LALNumUnits]}] Array of unit numerators,
  $N_i$, $i=0\ldots\textrm{LALNumUnits}-1$.
\item[-- \texttt{INT2 unitDenominatorMinusOne[LALNumUnits]}] Array of unit
  denominators-minus-one, $D_i$, $i=0\ldots\textrm{LALNumUnits}-1$.
\end{description}
Thus, the units are given by
\begin{equation}
  10^p\times\textrm{m}^{N_0/(1+D_0)}\times\textrm{kg}^{N_1/(1+D_1)}
  \times\textrm{s}^{N_2/(1+D_2)}\times\textrm{A}^{N_3/(1+D_3)}
  \times\textrm{K}^{N_4/(1+D_4)}\times\textrm{strain}^{N_5/(1+D_5)}
  \times\textrm{count}^{N_6/(1+D_6)}
\end{equation}
The indexes of the units can be specified using the constants
\texttt{LALUnitIndexMeter},
\texttt{LALUnitIndexKiloGram},
\texttt{LALUnitIndexSecond},
\texttt{LALUnitIndexAmpere},
\texttt{LALUnitIndexKelvin},
\texttt{LALUnitIndexStrain},
\texttt{LALUnitIndexADCCount},
while \texttt{LALNumUnits} is the total number of units.

\paragraph{\opt{datatype}\texttt{TimeSeries}:}
This structure represents a sequence of data of type \opt{datatype}
(where \opt{datatype} can be any primitive datatype), sampled over
uniform time intervals $t_0, t_0+\Delta t, \ldots , t_0+l\Delta t$.
Essentially this is a \opt{datatype}\verb@Sequence@ with extra fields
defining the sample times and the type of data being sampled.  The raw
data may also have been \emph{heterodyned}; that is, multiplied by a
sinusoid of some frequency $f_0$, low-pass filtered, and resampled, in
order to extract the behavior in a small bandwidth about $f_0$.  The
fields are:
\begin{description}
\item[-- \texttt{CHAR name[LALNameLength]}] The name of the data series (i.e.\
the type of data being sampled).
\item[-- \texttt{LIGOTimeGPS epoch}] The start time $t_0$ of the data
series.
\item[-- \texttt{REAL8 deltaT}] The sampling interval $\Delta t$, in
seconds.
\item[-- \texttt{REAL8 f0}] The heterodyning frequency $f_0$, in hertz.
\item[-- \texttt{LALUnit sampleUnits}] The physical units of the
quantity being sampled.
\item[-- \opt{datatype}\texttt{Sequence *data}] The sequence of sampled data.
\end{description}


\paragraph{\opt{datatype}\texttt{FrequencySeries}:}
This structure represents a frequency spectrum of data of type
\opt{datatype} (where \opt{datatype} can be any primitive
datatype), sampled over uniform frequency intervals $f_0, f_0+\Delta
f, \ldots , f_0+l\Delta f$.  Essentially this is a
\opt{datatype}\verb@Sequence@ with extra fields defining the sample
frequencies, the timestamp of the spectrum, and the type of data being
sampled.  The fields are:
\begin{description}
\item[-- \texttt{CHAR name[LALNameLength]}] The name of the data series (i.e.\
the type of data being sampled).
\item[-- \texttt{LIGOTimeGPS epoch}] The start time of the \emph{time}
series from which the spectrum was calculated.
\item[-- \texttt{REAL8 f0}] The lowest frequency $f_0$ being sampled, in
hertz.
\item[-- \texttt{REAL8 deltaF}] The frequency sampling interval $\Delta
f$, in hertz.
\item[-- \texttt{LALUnit sampleUnits}] The physical units of the
quantity being sampled.
\item[-- \opt{datatype}\texttt{Sequence *data}] The sequence of sampled data.
\end{description}


\paragraph{\opt{datatype}\texttt{ZPGFilter}:}
This structure stores the complex frequency response of a filter or
transfer function in a factored form, where \opt{datatype} can be
either \verb@COMPLEX8@ or \verb@COMPLEX16@.  One defines a
(dimensionless) complex frequency variable $\zeta(f\Delta t)$, where
$\Delta t$ is the time sampling interval of the data to which the
filter will be applied (in the case of a digital filter), or some
other reference timescale (in the case of an analog filter).  The
complex response function can then be given (or approximated) as
$H(f)=g\times\prod_k(\zeta-z_k)/\prod_l(\zeta-p_l)$, where $z_k$ are
the complex \emph{zeros}, $p_l$ are the complex \emph{poles}, and $g$
is the complex \emph{gain} of the response function.  Some common
complex frequency representations are the $z$-plane representation
$\zeta(f\Delta t)=\exp(2\pi if\Delta t)$, which maps the Nyquist
interval $f\in[0,1/2\Delta t)$ onto the upper-half unit circle in
$\zeta$, and the $w$-plane representation $\zeta(f\Delta t)=\tan(\pi
f\Delta t)$, which maps the Nyquist interval onto the positive real
axis in $\zeta$.  The fields of \opt{datatype}\verb@ZPGFilter@ are:
\begin{description}
\item[-- \texttt{CHAR name[LALNameLength]}] The name of the filter or transfer
function.  This should also mention its complex frequency
representation.
\item[-- \texttt{REAL8 deltaT}] The sampling time or reference timescale
$\Delta t$ for the filter, in seconds.  If zero, it will be treated as
being equal to the sampling interval of the data being filtered.
\item[-- \opt{datatype}\texttt{Vector *zeros}] Pointer to a vector storing
the zeros $z_k$ of the filter.
\item[-- \opt{datatype}\texttt{Vector *poles}] Pointer to a vector storing
the poles $p_k$ of the filter.
\item[-- \opt{datatype} \texttt{gain}] The gain $g$ of the filter.
\end{description}


\subsection{The LAL universal status structure \texttt{LALStatus}}
\label{ss:status-structure}

This structure is the means by which LAL functions report their
success or failure; it provides a useful mechanism for tracking
progress and errors through nested function calls.  The error
reporting structure is a linked list of \verb@LALStatus@ structures, with
each node corresponding to a given function in the current calling
sequence.  When a function terminates successfully, its node is
dropped from the list.  If a function encounters an error, it must
still return control to the calling routine, reporting the error
through its \verb@LALStatus@.  The calling routine must either deal with
the error (pruning the linked list if it succeeds), or else return an
error itself.  A fatal error will thus return a linked list of
\verb@LALStatus@ structures to the top-level routine, where the tail of
the list identifies the source of the error, and the intermediate
nodes identify the sequence of nested function calls that led to the
error.  The fields of the \verb@LALStatus@ are as follows:
\begin{description}
\item[-- \texttt{INT4 statusCode}] A numerical code identifying the type
of error, or 0 for nominal status.
\item[-- \texttt{const CHAR *statusDescription}] A description of the
current status or error.
\item[-- \texttt{volatile const CHAR *Id}] The RCS ID string of the
source file of the current function.
\item[-- \texttt{const CHAR *function}] The name of the current function.
\item[-- \texttt{const CHAR *file}] The name of the source file of the
current function.
\item[-- \texttt{INT4 line}] The line number in the source file where the
current \verb@statusCode@ was set.
\item[-- \texttt{LALStatus *statusPtr}] Pointer to the next node in the
list; \verb@NULL@ if this function is not reporting a subroutine
error.
\item[-- \texttt{INT4 level}] The current level in the nested calling
sequence.
\end{description}


\section{The \texttt{LALStatus} structure}

LAL routines store their current execution status in a linked list of
structures of type \verb@LALStatus@, with each node in the list
representing a subroutine in the current calling sequence.  The
\verb@LALStatus@ structure is described in Sec.~\ref{ss:status-structure}
of the header \verb@LALDatatypes.h@, but for completeness, we explain
its fields below:
\begin{description}
\item[-- \texttt{INT4 statusCode}] A code indicating the exit status of a
function.  0 represents a normal exit.  Negative values are reserved
for certain standard error types.  The authors of individual functions
should assign positive values to the various ways in which their code
can fail.
\item[-- \texttt{const CHAR *statusDescription}] An explanatory string
corresponding to the numerical status code.
\item[-- \texttt{volatile const CHAR *Id}] A character string identifying
the source file and version number of the function being reported on.
\item[-- \texttt{const CHAR *function}] The name of the function.
\item[-- \texttt{const CHAR *file}] The file name of the \verb@.c@ file
containing the function code.
\item[-- \texttt{INT4 line}] The line number in the \verb@.c@ file of the
instruction where any error was reported.
\item[-- \texttt{LALStatus *statusPtr}] A recursive pointer to another
status pointer.  This structure is used to report an error in a
subroutine of the current function.  Thus if an error occurs in a
deeply-nested routine, the status structure returned to the main
program will be the head of a linked list of status structures, one
for each nested level, with the tail structure reporting the actual
error that caused the overlying routines to fail.
\item[-- \texttt{INT4 level}] The nested-function level where any error
was reported.
\end{description}
In almost all circumstances the programmer will \emph{not} have to
access this structure directly, relying instead on the macros defined
in the header \verb+LALStatusMacros.h+.  The exception is the \verb@statusCode@
field, which the programmer may want to query directly.

The \verb@statusCode@ field is set to a nonzero value any time an
error condition arises that would lead to abnormal termination of the
current function.  Programmers can assign positive error codes to the
various types of error that may be encountered in their routines.
Additionally, the following following status codes are reserved to
report certain standard conditions:

\begin{center}
\begin{tabular}{|cp{3.5cm}p{6.5cm}|}
\hline
Code & Message & Explanation \\
\hline

\tt 0 & & Nominal execution; the function returned
successfully. \\

\tt -1 & \vspace{-1.4ex}\tt Recursive error & The function aborted due
to failure of a subroutine. \\

\tt -2 & \vspace{-1.4ex}\tt INITSTATUS: non-null status pointer & The
status structure passed to the function had a non-\verb@NULL@
\verb@statusPtr@ field, which blocks the function from calling
subroutines (it is symptomatic of something screwy going on in the
calling routine). \\

\tt -4 & \vspace{-1.4ex}\tt ATTATCHSTATUSPTR: memory allocation error
& The function was unable to allocate a \verb@statusPtr@ field to pass
down to a subroutine. \\

\tt -8 & \vspace{-1.4ex}\tt DETATCHSTATUSPTR: null status pointer &

The \verb@statusPtr@ field could not be deallocated at the end of all
subroutine calls; one of the subroutines must have lost it or set it
to \verb@NULL@. \\

\hline
\end{tabular}
\end{center}


\section{The \texttt{lalDebugLevel}}

The \verb@lalDebugLevel@ is a global variable, set at runtime, that
determines how much and what kind of debugging information will be
reported.  It is declared as an \verb@extern int@ in the header
\verb@LALStatusMacros.h@, and is therefore accessible in any standard
LAL module that includes this header.  Note, however, that it is
declared to be of the C type \verb@int@, which is usually but not
always a 32-bit integer (on some systems it may only be 16 bits).

The value of \verb@lalDebugLevel@ should be thought of not as a number,
but as a \emph{bit mask}, wherein each bit in the binary
representation turns on or off a specific type of status reporting.
At present, there are five types of status reporting, each associated
with a bit in \verb@lalDebugLevel@.

\paragraph{Error messages} tell the operator that a computation has
terminated abnormally, and has failed to produce an acceptable result.
Normally this is associated with assigning a non-zero
\verb@statusCode@; an error message is printed automatically whenever
a function exits with non-zero \verb@statusCode@.

\paragraph{Warning messages} tell the user that a computation is
working, but with unusual behavior that might indicate an unreliable
or meaningless result.  Warnings do not normally result in a non-zero
\verb@statusCode@.

\paragraph{Information messages} tell the operator that the
computation is proceeding as expected, and simply provide additional
information about its progress.

\paragraph{Tracing messages} are printed automatically a subroutine
is called or returned; they simply track the current sequence of
function calls.

\paragraph{Memory information messages} are a special type of
information message; they tell the operator when and how much memory
is allocated or freed from the memory heap.

\paragraph{}The module \verb@LALError.c@ defines functions for
printing each of these types of status message.  Each type of message
is turned on by setting the corresponding bit in \verb@lalDebugLevel@ to
1, and is suppressed by setting the bit to 0.  This header file
\verb@#define@s flags with numerical values designed to switch on the
appropriate bits.  Combinations of bits can be switched on by
combining these flags using the bitwise-\textit{or} operator,
\verb@|@.  The flags are defined as follows:

\begin{center}
\begin{tabular}{|lccl|}
\hline
Flag & Octal & Decimal & Meaning \\
\hline
\multicolumn{4}{|l|}{\it Primitive flags} \\
\tt LALNDEBUG   & 000000 &     0 & No debugging or status messages \\
\tt LALERROR    & 000001 &     1 & Turn on error messages \\
\tt LALWARNING  & 000002 &     2 & Turn on warning messages \\
\tt LALINFO     & 000004 &     4 & Turn on info messages \\
\tt LALTRACE    & 000010 &     8 & Turn on tracing messages \\
\tt LALMEMINFO  & 000020 &    16 & Turn on memory messages \\
\tt LALNMEMDBG  & 000040 &    32 & Turn off all memory debugging \\
\tt LALNMEMPAD  & 000100 &    64 & Turn off memory padding \\
\tt LALNMEMTRK  & 000200 &   128 & Turn off memory tracking \\
\tt LALMEMDBG   & 040000 & 16384 & Turn on memory debugging without messages \\
\multicolumn{4}{|l|}{\it Combination flags} \\
\tt LALMSGLVL1  & 000001 &     1 & Error messages only \\
\tt LALMSGLVL2  & 000003 &     3 & Error and warning messages \\
\tt LALMSGLVL3  & 000007 &     7 & Error, warning, and info messages \\
\tt LALMEMTRACE & 000030 &    24 & Memory and tracing messages \\
\tt LALALLDBG   & 077437 & 32543 & All messages and debugging \\
\hline
\end{tabular}
\end{center}

The most significant bit
of \verb@lalDebugLevel@ has a special meaning in that it is not
associated with any type of status message.  However, certain pieces
of debugging or error-tracking code --- such as the memory leak
detection code in \verb@LALMalloc.c@ --- do not write status messages
and are not associated with a \verb@lalDebugLevel@ bit; instead, these
pieces of code are turned on for \emph{any} nonzero value of
\verb@lalDebugLevel@, unless the \verb@LALNMEMDBG@ bit is set.
Switching on only the most significant bit with
\verb@LALMEMDBG@ activates this code without turning on any other
error reporting.

\section{LAL Constants}

\subsection{Mathematical Constants}

The following constants define the precision and range of
floating-point arithmetic in LAL.  They are taken from the IEEE
standard~754 for binary arithmetic.  All numbers are dimensionless.

\begin{center}
\begin{tabular}{|lll|}
\hline
Name & Value & Description \\
\hline
\tt LAL\_REAL4\_MANT & 24 &
        Bits in {\tt REAL4} mantissa \\
\tt LAL\_REAL4\_MAX  & $3.40282347\times10^{38}$ &
        Largest {\tt REAL4} \\
\tt LAL\_REAL4\_MIN  & $1.17549435\times10^{-38}$ &
        Smallest positive {\tt REAL4} \\
\tt LAL\_REAL4\_EPS  & $1.19209290\times10^{-7}$ &
        $2^{-(\mathtt{LAL\_REAL4\_MANT}-1)}$ \\
\hline
\tt LAL\_REAL8\_MANT & 53 &
        Bits in {\tt REAL8} mantissa \\
\tt LAL\_REAL8\_MAX  & $1.7976931348623157\times10^{308}$ &
        Largest {\tt REAL8} \\
\tt LAL\_REAL8\_MIN  & $2.2250738585072014\times10^{-308}$ &
        Smallest positive {\tt REAL8} \\
\tt LAL\_REAL8\_EPS  & $2.2204460492503131\times10^{-16}$ &
        $2^{-(\mathtt{LAL\_REAL8\_MANT}-1)}$ \\
\hline
\end{tabular}
\end{center}

\noindent\verb@LAL_REAL4_EPS@ and \verb@LAL_REAL8_EPS@ can be thought
of as the difference between 1 and the next representable \verb@REAL4@
or \verb@REAL8@ number.

\vspace{3ex}

The following are fundamental mathematical constants.  They are mostly
taken from the GNU C \verb@math.h@ header (with the exception of
\verb@LAL_TWOPI@, which was computed using Maple).  All numbers are
dimensionless.

\begin{center}
\begin{tabular}{|llc|}
\hline
Name & Value & Expression \\
\hline
\tt LAL\_E         & 2.7182818284590452353602874713526625 & $e$ \\
\tt LAL\_LOG2E     & 1.4426950408889634073599246810018922 & $\log_2 e$ \\
\tt LAL\_LOG10E    & 0.4342944819032518276511289189166051 & $\log_{10} e$ \\
\tt LAL\_LN2       & 0.6931471805599453094172321214581766 & $\log_e 2$ \\
\tt LAL\_LN10      & 2.3025850929940456840179914546843642 & $\log_e 10$ \\
\tt LAL\_SQRT2     & 1.4142135623730950488016887242096981 & $\sqrt{2}$ \\
\tt LAL\_SQRT1\_2  & 0.7071067811865475244008443621048490 & $1/\sqrt{2}$ \\
\tt LAL\_GAMMA     & 0.5772156649015328606065120900824024 & $\gamma$ \\
\tt LAL\_PI        & 3.1415926535897932384626433832795029 & $\pi$ \\
\tt LAL\_TWOPI     & 6.2831853071795864769252867665590058 & $2\pi$ \\
\tt LAL\_PI\_2     & 1.5707963267948966192313216916397514 & $\pi/2$ \\
\tt LAL\_PI\_4     & 0.7853981633974483096156608458198757 & $\pi/4$ \\
\tt LAL\_1\_PI     & 0.3183098861837906715377675267450287 & $1/\pi$ \\
\tt LAL\_2\_PI     & 0.6366197723675813430755350534900574 & $2/\pi$ \\
\tt LAL\_2\_SQRTPI & 1.1283791670955125738961589031215452 & $2/\sqrt{\pi}$ \\
\tt LAL\_PI\_180   & 1.7453292519943295769236907684886127$\times10^{-2}$ &
  $\pi/180$ \\
\tt LAL\_180\_PI   & 57.295779513082320876798154814105170 & $180/\pi$ \\
\hline
\end{tabular}
\end{center}

\subsection{Physical Constants}

The following physical constants are defined to have exact values.
The values of $c$ and $g$ are taken from~\cite{Barnet:1996},
$p_\mathrm{atm}$ is from~\cite{Lang:1992}, while $\epsilon_0$ and
$\mu_0$ are computed from $c$ using exact formulae.  They are given in
the SI units shown.

\begin{center}
\begin{tabular}{|lll|}
\hline
Name & Value & Description \\
\hline
\tt LAL\_C\_SI        & $299\,792\,458\,\mathrm{m}\,\mathrm{s}^{-1}$ &
        Speed of light $c$ in free space \\
\tt LAL\_EPSILON0\_SI & \multicolumn{2}{l|}{
        $8.8541878176203898505365630317107503\times10^{-12}\,
        \mathrm{C}^2\mathrm{N}^{-1}\mathrm{m}^{-2}$} \\
& & Permittivity $\epsilon_0$ of free space \\
\tt LAL\_MU0\_SI      & \multicolumn{2}{l|}{
        $1.2566370614359172953850573533118012\times10^{-6}\,
        \mathrm{N}\,\mathrm{A}^{-2}$} \\
& & Permeability $\mu_0$ of free space \\
\tt LAL\_GEARTH\_SI   & $9.80665\,\mathrm{m}\,\mathrm{s}^{-2}$ &
        Standard gravity $g$ \\
\tt LAL\_PATM\_SI     & $101\,325\,\mathrm{Pa}$ &
        Standard atmospheric pressure $p_\mathrm{atm}$ \\
\hline
\end{tabular}
\end{center}

The following are measured fundamental physical constants, with values
given in~\cite{Barnet:1996}.  When not dimensionless, they are given
in the SI units shown.

\begin{center}
\begin{tabular}{|lll|}
\hline
Name & Value & Description \\
\hline
\tt LAL\_G\_SI     & $6.67259\times10^{-11}\,\mathrm{N}\,\mathrm{m}^{2}
        \mathrm{kg}^{-2}$ & Gravitational constant $G$ \\
\tt LAL\_H\_SI     & $6.6260755\times10^{-34}\,\mathrm{J}\,\mathrm{s}$ &
        Planck constant $h$ \\
\tt LAL\_HBAR\_SI  & $1.05457266\times10^{-34}\,\mathrm{J}\,\mathrm{s}$ &
        Reduced Planck constant $\hbar$ \\
\tt LAL\_MPL\_SI   & $2.17671\times10^{-8}\,\mathrm{kg}$ & Planck mass \\
\tt LAL\_LPL\_SI   & $1.61605\times10^{-35}\,\mathrm{m}$ & Planck length \\
\tt LAL\_TPL\_SI   & $5.39056\times10^{-44}\,\mathrm{s}$ & Planck time \\
\tt LAL\_K\_SI     & $1.380658\times10^{-23}\,\mathrm{J}\,\mathrm{K}^{-1}$ &
        Boltzmann constant $k$ \\
\tt LAL\_R\_SI     & $8.314511\,\mathrm{J}\,\mathrm{K}^{-1}$ &
        Ideal gas constant $R$ \\
\tt LAL\_MOL       & $6.0221367\times10^{23}$ & Avogadro constant \\
\tt LAL\_BWIEN\_SI & $2.897756\times10^{-3}\,\mathrm{m}\,\mathrm{K}$ &
        Wien displacement law constant $b$ \\
\tt LAL\_SIGMA\_SI & $5.67051\times10^{-8}\,\mathrm{W}\,\mathrm{m}^{-2}
        \mathrm{K}^{-4}$ & Stefan-Boltzmann constant $\sigma$ \\
\tt LAL\_AMU\_SI   & $1.6605402\times10^{-27}\,\mathrm{kg}$ &
        Atomic mass unit \\
\tt LAL\_MP\_SI    & $1.6726231\times10^{-27}\,\mathrm{kg}$ & Proton mass \\
\tt LAL\_ME\_SI    & $9.1093897\times10^{-31}\,\mathrm{kg}$ & Electron mass \\
\tt LAL\_QP\_SI    & $1.60217733\times10^{-19}\,\mathrm{C}$ & Proton charge \\
\tt LAL\_ALPHA     & $7.297354677\times10^{-3}$ & Fine structure constant \\
\tt LAL\_RE\_SI    & $2.81794092\times10^{-15}\,\mathrm{m}$ &
        Classical electron radius $r_e$ \\
\tt LAL\_LAMBDAE\_SI & $3.86159323\times10^{-13}\,\mathrm{m}$ &
        Electron Compton wavelength $\lambda_e$ \\
\tt LAL\_AB\_SI    & $5.29177249\times10^{-11}\,\mathrm{m}$ & Bohr radius $a$\\
\tt LAL\_MUB\_SI   & $9.27401543\times10^{-24}\,\mathrm{J}\,\mathrm{T}^{-1}$ &
        Bohr magneton $\mu_B$ \\
\tt LAL\_MUN\_SI   & $5.05078658\times10^{-27}\,\mathrm{J}\,\mathrm{T}^{-1}$ &
        Nuclear magneton $\mu_N$ \\
\hline
\end{tabular}
\end{center}


\subsection{Astrophysical Parameters}

The following parameters are derived from measured properties of the
Earth and Sun.  The values are taken from~\cite{Barnet:1996}, except
for the obliquity of the ecliptic plane and the eccentricity of
Earth's orbit, which are taken from~\cite{Lang:1992}.  All values are
given in the SI units shown.

\begin{center}
\begin{tabular}{|lll|}
\hline
Name & Value & Description \\
\hline
\tt LAL\_REARTH\_SI & $6.378140\times10^6\,\mathrm{m}$ &
        Earth equatorial radius \\
\tt LAL\_AWGS84\_SI & $6.378137\times10^6\,\mathrm{m}$ &
        Semimajor axis of WGS-84 Reference Ellipsoid \\
\tt LAL\_BWGS84\_SI & $6.356752314\times10^6\,\mathrm{m}$ &
        Semiminor axis of WGS-84 Reference Ellipsoid \\
\tt LAL\_MEARTH\_SI & $5.97370\times10^{24}\,\mathrm{kg}$ & Earth mass \\
\tt LAL\_IEARTH     & $0.409092804\,\mathrm{rad}$ &
        Obliquity of the ecliptic (2000) \\
\tt LAL\_EEARTH     & 0.0167 & Earth orbital eccentricity \\
\tt LAL\_RSUN\_SI   & $6.960\times10^8\,\mathrm{m}$ & Solar equatorial radius\\
\tt LAL\_MSUN\_SI   & $1.98892\times10^{30}\,\mathrm{kg}$ & Solar mass \\
\tt LAL\_MRSUN\_SI  & $1.47662504\times10^3\,\mathrm{m}$ &
        Geometrized solar mass (length) \\
\tt LAL\_MTSUN\_SI  & $4.92549095\times10^{-6}\,\mathrm{s}$ &
        Geometrized solar mass (time) \\
\tt LAL\_LSUN\_SI   & $3.846\times10^{26}\,\mathrm{W}$ & Solar luminosity \\
\tt LAL\_AU\_SI     & $1.4959787066\times10^{11}\,\mathrm{m}$ &
        Astronomical unit \\
\tt LAL\_PC\_SI     & $3.0856775807\times10^{16}\,\mathrm{m}$ & Parsec \\
\tt LAL\_YRTROP\_SI & $31\,556\,925.2\,\mathrm{s}$ & Tropical year (1994) \\
\tt LAL\_YRSID\_SI  & $31\,558\,149.8\,\mathrm{s}$ & Sidereal year (1994) \\
\tt LAL\_DAYSID\_SI & $86\,164.09053\,\mathrm{s}$ & Mean sidereal day \\
\tt LAL\_LYR\_SI    & $9.46052817\times10^{15}\,\mathrm{m}$ &
        $c\times$tropical year (1994) \\
\hline
\end{tabular}
\end{center}

The following cosmological parameters are derived from measurements of
the Hubble expansion rate and of the cosmic background radiation
(CBR).  Data are taken from~\cite{Barnet:1996}.  In what follows, the
normalized Hubble constant $h_0$ is equal to the actual Hubble
constant $H_0$ divided by $\langle H
\rangle=100\,\mathrm{km}\,\mathrm{s}^{-1}\mathrm{Mpc}^{-1}$.  Thus the
Hubble constant can be written as:
$$
H_0 = \langle H \rangle h_0 \; .
$$
Similarly, the critical energy density $\rho_c$ required for spatial
flatness is given by:
$$
\rho_c = \langle\rho\rangle h_0^2 \; .
$$
Current estimates give $h_0$ a value of around 0.65, which is what is
assumed below.  All values are in the SI units shown.

\begin{center}
\begin{tabular}{|lll|}
\hline
Name & Value & Description \\
\hline
\tt LAL\_H0\_SI      & $2\times10^{-18}\,\mathrm{s}^{-1}$ &
        Approx.\ Hubble constant $H_0$ \\
\tt LAL\_H0FAC\_SI   & $3.2407792903\times10^{-18}\,\mathrm{s}^{-1}$ &
        $H_0/h_0$ \\
\tt LAL\_RHOC\_SI    & $7\times10^{-10}\,\mathrm{J}\,\mathrm{m}^{-3}$ &
        Approx.\ critical energy density $\rho_c$ \\
\tt LAL\_RHOCFAC\_SI & $1.68860\times10^{-9}\,\mathrm{J}\,\mathrm{m}^{-3}$ &
        $\rho_c/h_0^2$ \\
\tt LAL\_TCBR\_SI    & $2.726 \mathrm{K}$ &
        CBR temperature \\
\tt LAL\_VCBR\_SI    & $3.695\times10^5\,\mathrm{m}\,\mathrm{s}^{-1}$ &
        Solar velocity with respect to CBR \\
\tt LAL\_RHOCBR\_SI  & $4.177\times10^{-14}\,\mathrm{J}\,\mathrm{m}^{-3}$ &
        Energy density of CBR \\
\tt LAL\_NCBR\_SI    & $4.109\times10^8\,\mathrm{m}^{-3}$ &
        Number density of CBR photons \\
\tt LAL\_SCBR\_SI    & $3.993\times10^{-14}\,\mathrm{J}\,\mathrm{K}^{-1}
        \mathrm{m}^{-3}$ & Entropy density of CBR \\
\hline
\end{tabular}
\end{center}

\newpage\begin{thebibliography}{0}
\bibitem{lsd}
  LAL Software Documentation,
  \url{http://www.lsc-group.phys.uwm.edu/lal/lsd.pdf}
\bibitem{Barnet:1996}
  Particle Data Group, R.~M. Barnett et al., Phys. Rev. D\textbf{54},
  1 (1996)
\bibitem{Lang:1992}
  K.~R. Lang, \textit{Astrophysical Data: Planets and Stars}.
  Springer-Verlag, New York (1992)
\end{thebibliography}


\end{document}
