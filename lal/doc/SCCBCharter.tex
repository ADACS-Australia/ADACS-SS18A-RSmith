\documentclass[]{ligodcc}

%% use the following line for latex with dvips
%\usepackage[dvips]{hyperref,color}
\usepackage{hyperref,color}

%\let\keepverbatim\verbatim
%\def\verbatim{\color{blue}\keepverbatim}
%\renewcommand{\texttt}[1]{{\ttfamily\color{blue}#1}}

\setcounter{secnumdepth}{6}
\setcounter{tocdepth}{6}
\let\keepverbatim\verbatim
\def\verbatim{\color{blue}\keepverbatim}
\def\verb{\relax\ifmmode\hbox\else\leavevmode\null\fi
  \bgroup
    \color{blue}%
    \verb@eol@error \let\do\@makeother \dospecials
    \verbatim@font\@noligs
    \@ifstar\@sverb\@verb}
\renewcommand{\texttt}[1]{{\ttfamily\color{blue}#1}}

%\def\tt{\ttfamily\color{blue}}
\DeclareOldFontCommand{\tt}{\normalfont\ttfamily\color{blue}}{\mathtt\color{blue}}


%% or use the following lines for pdflatex
%\usepackage[pdftex]{hyperref,color}
%\voffset=-1.5in

%\def\tt{\ttfamily\color{blue}}
\DeclareOldFontCommand{\tt}{\normalfont\ttfamily\color{blue}}{\mathtt\color{blue}}

\ligodoc{T010050-00}{Z}
\ligodocdist{LSC and LIGO \vspace{1.5in} \\
This is an internal working note of the \\
LIGO Laboratory and the \\
LIGO Scientific Collaboration.}
\title{ Operating Procedures for the LIGO/LSC  Software Change Control Board}
\author{Alan Wiseman}
\rcsid$Id$
\begin{document}


\maketitle

\paragraph*{Charter of the Software Change Control Board (SCCB)\\}
\begin{enumerate}
\item
\vskip -0.3in
The purpose of the SCCB:

The LIGO Lab and the LSC have adopted software specifications that are
intended to foster widespread use and collaborative development of
well-tested analysis packages.  In order to meet this goal, the
software and software specifications must remain reasonably stable
so the ground won't be shifting under the developers.  On the other
hand, as real problems are identified, changes may need to be made.
The charge to the SCCB is to strike a balance between stability and
flexibility of the software. They should do this by adding order,
common sense and some inertia to the change process.  The purpose of
this charter is to lay out how the SCCB should operate.


\item
The software governed by the SCCB:

The LSC Software Coordinator in conjunction with the LIGO Lab
directorate will decide what is to be governed by the SCCB.  The list
of items that are formally under the charge of the committee will be
appended to this document.

As a general rule, the initial drafting of software specifications
should be done by working groups and not by the SCCB.  Only after the
specifications have had some ``field testing'' should they be placed
under the purview of the SCCB.

\item
Procedure for requesting a change:

An individual or group should submit a written (email) request for
software changes to the LSC Software Coordinator [currently
agw@gravity.phys.uwm.edu].  In most cases, this should be a
light-weight, one-page email. The request should include:

   \begin{enumerate}
   \item
   \vskip -0.1in
   a description of the change.
   \item
   the rationale behind the requested change, including the
   (dis)advantages of (not) making the change.
   \item
   an estimate of the impact of the change on other systems and
   software.
   \item
   a plan for implementing the change that addresses who will
   do the work.
   \end{enumerate}

\item
The criterion for making a change:

The change should be adopted if the benefits outweigh the costs of
making the change.

\item
The SCCB decision process should include:
\begin{enumerate}
   \item
   iteration with those requesting the changes.
   \item
   technical input from the Board itself, or others
   the Board feels should be consulted.
   \item
   an assessment of the impact of (not) making the change.
   \item
   input from the Lab and the LSC hierarchy, as well as
   other projects (GEO, VIRGO, TAMA, etc.) that might be affected
   by the change. In particular, the SCCB will work with the Lab to honor
   agreements with other projects that pertain to jointly maintained
   software specifications.
   \item
   input from the rank and file code developers that might be affected
   by the change.
   \item
   the viability of the plan for implementing the change.
   \item
   common sense.
\end{enumerate}

The committee will try to reach a consensus on whether the change
should be adopted.  In the absence of a clear consensus, the board
will formally vote on the changes. The software coordinator will break
tie votes.

\item
Responsibilities of the LSC Software Coordinator in the SCCB process:
\begin{enumerate}
   \item
   Make sure that requested changes move through the process, i.e.
   pass the information to the committee, convene the meetings, etc..
   \item
   Set the pace and priorities for the decision process. Some changes are
   urgent, others are trivial, and still others things can (should) wait.
   \item
   Inform the Lab Directorate and the LSC  spokesperson what changes are
   before the committee. This is to insure they have an opportunity for
   input regarding scheduling and coordination with other projects.
   \item
   Insure that the rank and file programmers are notified and have
   a chance to comment on significant changes.
   \item
   If the SCCB concludes that a change should be made, the Software
   Coordinator will act on behalf of the LSC spokesperson and
   LSC Executive Committee to see that the change is carried out.
\end{enumerate}

\end{enumerate}

\paragraph*{SCCB Members: \\}

The LSC Software Coordinator (in consultation with the LSC
spokesperson) will appoint the members of the SCCB. Currently, they
are:
\begin{enumerate}
\item
\vskip -0.1in
Stuart Anderson
\item
Jolien Creighton
\item
John Zweizig
\end{enumerate}

\paragraph*{Software specifications formally under the SCCB's purview
(as of April 2001):\\}
\begin{enumerate}
\item
\vskip -0.3in
The LAL Spec. [Formally known as ``Numerical Algorithms Library
Specification and Style Guide'', dcc number T990030. The most up-to-date
version can always be found in the developers LAL CVS archive.]
Significant changes to LAL Spec that are pending before the committee
will be announced on the LAL developers email lists.
\item
The FRAME Spec. [Formally known as ``Specification of a Common Data
Frame Format for Interferometric Gravitational Wave Detectors (IGWD)'',
dcc T970130-D-E].
\end{enumerate}

\paragraph*{Maintaining this document:\\}
The LSC Software Coordinator will maintain the LaTeX source for this
document in the LAL code development archive.  The most up to date
version can be found there.  A pdf version of this document will be
kept in the LIGO document control center.

\end{document}
