%%%%$Id$%%%

% \documentclass[10pt]{ligodcc}
\documentclass[prd,preprintnumbers,superscriptaddress,eqsecnum]{revtex4}

\usepackage{hyperref}
\usepackage{color}
\usepackage{calc}
\usepackage{amsmath,amssymb,graphicx,rcs}
\usepackage{tensor}

% make verbatims and teletype font blue
% \let\keepverbatim\verbatim
% \def\verbatim{\color{blue}\small\keepverbatim}
% \makeatletter
% \def\verb{\relax\ifmmode\hbox\else\leavevmode\null\fi
%   \bgroup
%     \color{blue}\small
%     \verb@eol@error \let\do\@makeother \dospecials
%     \verbatim@font\@noligs
%     \@ifstar\@sverb\@verb}
% \makeatother
% \renewcommand{\texttt}[1]{{\ttfamily\color{blue}#1}}
% \DeclareOldFontCommand{\tt}{\normalfont\ttfamily\color{blue}}{\mathtt\color{blue}}

% \newcommand{\opt}[1]{\ensuremath{\color{blue}\langle\textit{#1}\rangle}}

% \newsavebox{\fminibox}
% \newlength{\fminilength}
% \newenvironment{fminipage}[1][\linewidth]
%   {\setlength{\fminilength}{#1-2\fboxsep-2\fboxrule}%
%    \begin{lrbox}{\fminibox}\begin{minipage}{\fminilength}}
%   {\end{minipage}\end{lrbox}\noindent\fbox{\usebox{\fminibox}}}
% \newenvironment{lalrule}{\begin{quote}\color{red}\begin{fminipage}}
%   {\end{fminipage}\end{quote}}

% \ligodoc{T070072-00}{Z}
% \ligodocdist{%
% LIGO Scientific Collaboration\\ Numerical Relativity Community\\
% \textbf{DRAFT}\\
% \bigskip
% }


\numberwithin{equation}{section}
\newcommand{\Ys}{{{}^{-s}Y}}
\newcommand{\Ytwo}{{{}^{-2}Y}}
%\newcommand{\tens}[1]{\mathbf{#1}}
%\newcommand{\tens}[1]{\overleftrightarrow{#1}}
\newcommand{\tens}[1]{\mytensor{#1}}
\newcommand{\xhat}{\vec{e}_x}
\newcommand{\yhat}{\vec{e}_y}
\newcommand{\zhat}{\vec{e}_z}
\newcommand{\ihat}{\vec{e}_i}
\newcommand{\jhat}{\vec{e}_j}
\newcommand{\rhat}{\vec{e}_{r}}
\newcommand{\iotahat}{\vec{e}_{\iota}}
\newcommand{\phihat}{\vec{e}_{\phi}}
\newcommand{\eplus}{\tens{e}_+}
\newcommand{\ecross}{\tens{e}_\times}


% \rcsid$Id$
% \RCS $Date$
% \date{\RCSDate}l


\begin{document}

\newcommand*{\CIT}{Theoretical Astrophysics, California Institute of
  Technology, Pasadena, California 91125, USA}\affiliation{\CIT}
\newcommand*{\LIGO}{LIGO Laboratory, California Institute of
  Technology, Pasadena, California 91125, USA}\affiliation{\LIGO}
\newcommand*{\CU}{Cardiff University, Cardiff, CF2 3YB, United 
  Kingdom}\affiliation{\CU}
\newcommand*{\AEI}{Max-Planck-Institut f\"ur
  Gravitationsphysik, Albert-Einstein-Institut, Am M\"uhlenberg 1,
  D-14476 Golm, Germany}\affiliation{\AEI}
\newcommand*{\UFL}{Department of Physics, University of Florida,
  Gainesville, FL 32611, USA}\affiliation{\UFL}
\newcommand*{\CCT}{Center for Computation and Technology, Louisiana State
  University, Baton Rouge, LA 70803, USA}\affiliation{\CCT}
\newcommand*{\LSU}{Department of Physics and Astronomy, Louisiana State
  University, Baton Rouge, LA 70803, USA}\affiliation{\LSU}
\newcommand*{\Jena}{Theoretical Physics Institute, University of Jena,
  07743 Jena, Germany}\affiliation{\Jena}


\title{Data formats for numerical relativity waves}

\author{D.~A.~Brown}\email{dbrown@ligo.caltech.edu}\affiliation{\CIT}\affiliation{\LIGO}
\author{S.~Fairhurst}\email{fairhurst\_s@ligo.caltech.edu}\affiliation{\CIT}\affiliation{\LIGO}\affiliation{\CU}
\author{B.~Krishnan}\email{badri.krishnan@aei.mpg.de}\affiliation{\AEI}
\author{R.~A.~Mercer}\email{ram@phys.ufl.edu}\affiliation{\UFL}
\author{R.~K.~Kopparapu}\email{kumar@theory.phys.lsu.edu}\affiliation{\CCT}\affiliation{\LSU}
\author{L.~Santamaria}\email{lucia.santamaria@uni-jena.de}\affiliation{\Jena}
\author{J.~T.~Whelan}\email{john.whelan@aei.mpg.de}\affiliation{\AEI}

%\tableofcontents

\begin{abstract}
  This document suggests possible data formats to further the
  interaction between gravitational wave source modeling groups and
  the gravitational wave data analysis community. The aim is to have a
  simple format which is nevertheless sufficiently general, and is
  applicable to various kinds of sources including binaries of compact
  objects and systems undergoing gravitational collapse.
\end{abstract}

\preprint{LIGO-T070072-00-Z}


\begin{verbatim}

Dear Colleagues,

With recent advances in the fields of numerical simulations of
gravitational-wave sources and gravitational wave detection, we have
reached a time when closer collaborations will benefit both
fields. Close interactions between these fields may enhance the
chances of detecting gravitational waves, and enable us to better
understand the physics and astrophysics involved.

We would like to develop a uniform interface to public waveforms
produced by the source-modeling community that could be used by the
LIGO Scientific Collaboration (LSC), and other detector groups.
In this document, we suggest a simple format for the waveforms.

The software tools designed around this format, by the LSC, will be
released under the GPL. We expect that data analysis groups will
use the public waveforms in their analyses when appropriate.

While this interface document proposes technical standards for
numerical relativity waveforms, we believe that the ability to extract
the best astrophysical information from NR waveforms in gravitational
wave searches will depend not just on adopting standards, but on the
ability of the gravitational wave detector communities and the numerical
relativity communities to interact closely and develop a sufficiently
detailed understanding of each other's technical methods and
limitations. To effectively use NR waveforms, gravitational wave
scientists will need to understand the physical limitations and
subtleties of numerical data, and to effectively produce waveforms,
numerical relativists will need to understand the instrumental
limitations and subtleties of gravitational wave interferometers. We
hope that this opens the way to deeper interactions between the GW and
NR communities and look forward to closer collaborations that may
develop between the two communities as we explore the
gravitational-wave Universe together. 

The LIGO Scientific Collaboration

\end{verbatim}


\newpage


\maketitle


\section{Introduction}
\label{sec:intro}

Numerical relativity has made enormous progress within the last few
years. Many numerical relativity groups now have sufficiently stable
and accurate codes which can simulate the inspiral, merger, and
ringdown phases of binary black hole coalescence. Similarly,
significant progress has been made in the numerical simulation of
stellar gravitational collapse and there now seems to be a much better
understanding of how supernova explosions happen. All these processes
are among the most promising sources of gravitational radiation and
therefore, there is significant interest in using these numerical
relativity results within various data analysis pipelines used within
the gravitational wave data analysis community. A dialog between
numerical relativists and data analysts from the LIGO Scientific
Collaboration (LSC) was recently initiated in November 2006 through a
meeting in Boston. It seems appropriate to continue this dialog at a
more concrete level, and to start incorporating numerical relativity
results within various data analysis software.

The aim of this document is to suggest formats for data exchange
between numerical relativists and data analysts. It is clear that
there are still outstanding conceptual and numerical issues remaining
in these numerical simulations; the goal of this document is not to
resolve them. The goal is primarily to spell out the technical
details of the waveform data so that they can be incorporated
seamlessly within the data analysis software currently being developed
within the LSC. The relevant software development is being carried
out as part of the LSC Algorithms Library \footnote{Available from
 \texttt{http://www.lsc-group.phys.uwm.edu/daswg/projects/lal.html}.}
which contains core routines for gravitational wave data analysis
written in ANSI C89, and is distributed under the GNU General Public
License. The latest version of this document is available within this
library.\footnote{The location of the document is
 \texttt{lal/doc/NRDataFormat.tex}.}
  
The remainder of this document is structured as follows: section
\ref{sec:multipoles} describes our conventions for decomposing the
gravitational wave data in terms of spherical harmonics, section
\ref{sec:format} specifies the data formats for binary black hole
simulations, and finally section \ref{sec:openissues} enumerates some
open issues in binary black hole simulations which could be topics of
further discussion between data analysts and numerical relativists.



\section{Multipole expansion of the wave}
\label{sec:multipoles}


The output of a numerical relativity code is the full spacetime of a
binary black hole system. On the other hand, what is required for
gravitational wave data analysis purposes is the strain $h(t)$, as
measured by a detector located far away from the source. The quantity
of interest is therefore the gravitational wave metric perturbation
$h_{ab}$ in the wave-zone, where $a$ and $b$ are space-time indices.
We always work in the Transverse Traceless (TT) gauge so that all
information about the metric perturbation is contained in the TT
tensor $h_{ij}$, where $i$ and $j$ are spatial indices. The wave
falls off as $1/r$ where $r$ is the distance from the source:
\begin{equation}
  \label{eq:1}
  h_{ij} = A_{ij}\frac{M}{r} + \mathcal{O}\left(r^{-2}\right)\,.
\end{equation}
Here $A_{ij}$ is a transverse traceless tensor and $M$ is the total
mass of the system; this approximation is, naturally, only valid far
away from the source.

There are different methods for extracting $h_{ij}$ from a numerical
evolution. One common method is to use the complex Weyl tensor
component $\Psi_4$ which is related to the second time derivative of
$h_{ij}$. Another method is to use the Zerilli function which
approximates the spacetime in the wave-zone as a perturbation of a
Schwarzschild spacetime. For our purposes, it is not important how
the wave is extracted, and different numerical relativity groups are
free to use methods they find appropriate. The starting point of our
analysis are the multipole moments of $h_{ij}$ and it is important to
describe explicitly our conventions for the multipole decomposition.
In addition to these multipole moments, we also request the
corresponding values of $\Psi_4$ or the Zerilli function in the
formats described later.

Let $(x,y,z,t)$ be a Cartesian coordinate system in the wave zone,
sufficiently far away from the source. Let $\xhat$, $\yhat$ and
$\zhat$ denote the spatial orthonormal coordinate basis vectors.
Given this coordinate system, we define standard spherical coordinates
$(r,\iota,\phi)$ where $\iota$ is the inclination angle from the
$z$-axis and $\phi$ is the phase angle. At this point, we have not
specified anything about the source. In fact, the source could be a
binary system, a star undergoing gravitational collapse or anything
else that could be of interest for gravitational wave source modeling.
In a later section we will specialize to binary black hole systems and
suggest possibilities for some of the various choices that have to be
made. However, as far as possible, these choices are eventually to be
made by the individual source modeling group.

We break up $h_{ij}$ into modes in this coordinate system. In the
wave zone, the wave will be propagating in the direction of the radial
unit vector
\begin{subequations}
  \begin{alignat}{3}
    \rhat &=  &\xhat&\,\sin\iota\cos\phi\, &+&\, \yhat\,\sin\iota\sin\phi
    + \zhat\,\cos\iota\,.
\intertext{A natural set of orthogonal basis vectors from which to build the
transverse traceless basis tensors is}
    \iotahat &=
    &\xhat&\,\cos\iota\cos\phi \,&+&\, \yhat\,\cos\iota\sin\phi
    - \zhat\,\sin\iota \,,\\
    \phihat &=
    -\,&\xhat&\,\sin\phi \,&+&\, \yhat\,\cos\phi\,.
  \end{alignat}
\end{subequations}
In the transverse traceless gauge, $h_{ij}$ has two independent
polarizations 
\begin{equation}
  \label{eq:2}
  \tens{h} = \sum_{i,j}h_{ij}\,\ihat\otimes\jhat
  = h_+ \eplus + h_\times \ecross\,,
\end{equation}
where $\eplus$ and $\ecross$ are the usual basis tensors for
transverse-traceless tensors in the wave frame
\begin{equation}
  \label{eq:9}
  \eplus = \iotahat\otimes\iotahat -
  \phihat\otimes\phihat\,, \qquad \textrm{and} \qquad 
  \ecross = \iotahat\otimes\phihat +
  \phihat\otimes\iotahat\,. 
\end{equation}
It is convenient to use the combination $h_+ - ih_\times$, which is
related to $\Psi_4$ by two time derivatives\footnote{We define
  $\Psi_4$ as $\Psi_4 := C_{abcd}\bar{m}^a n^b \bar{m}^c n^d$ where
  $C_{abcd}$ is the Weyl tensor and $a,b\ldots$ denote abstract
  spacetime indices. If we denote the unit timelike normal to the
  spatial slice as $e_{\hat{t}}^a$ and the promotions of
  $\{\rhat,\iotahat,\phihat\}$ to the full spacetime as
  $\{e_{\hat{r}}^a,e_{\hat{\iota}}^a,e_{\hat{\phi}}^a\}$, then the
  null tetrad adapted to the constant $r$ spheres is
  $\{\ell^a,n^a,m^a,\bar{m}^a\}$ where $\ell^a = (e_{\hat{t}}^a +
  e_{\hat{r}}^a)/\sqrt{2}$, $n^a = (e_{\hat{t}}^a -
  e_{\hat{r}}^a)/\sqrt{2}$, $m^a = (e_{\hat{\iota}}^a +
  ie_{\hat{\phi}}^a)/\sqrt{2}$, and $\bar{m}^a$ is the complex
  conjugate of $m^a$.}
\begin{equation}
  \label{eq:3}
  \Psi_4 = \ddot{h}_+ - i\ddot{h}_\times\,.
\end{equation}
It can be shown that $h_+-ih_\times$ can be decomposed into
modes using spin weighted spherical harmonics $\Ys_{lm}$ of weight
-2:
\begin{equation}
  \label{eq:4}
  h_+ - ih_\times = \frac{M}{r}\sum_{\ell=2}^{\infty}\sum_{m=-\ell}^\ell H_{\ell m}(t)\,
  \Ytwo_{\ell m}(\iota,\phi)\,.
\end{equation}
The expansion parameters $H_{lm}$ are complex functions of the retarded time
$t-r$ and, if we fix $r$ to be the radius of the sphere at which we
extract waves, then $H_{lm}$ are functions of $t$ only. 

The explicit expression for the spin weighted spherical harmonics
in terms of the Wigner $d$-functions is
\begin{equation}
  \label{eq:5}
   \Ys_{lm} = (-1)^s\sqrt{\frac{2\ell+1}{4\pi}} d^\ell_{m,s}(\iota)e^{im\phi},
\end{equation}
where
\begin{equation}
  \label{eq:6}
  d^\ell_{m,s}(\iota) = \sum_{k = k_1}^{k_2}
    \frac{(-1)^k[(\ell+m)!(\ell-m)!(\ell+s)!(\ell-s)!]^{1/2}}{(\ell +m
      -k)!(\ell-s-k)!k!(k+s-m)!}  
    \times \left(\frac{\cos\iota}{2}\right)^{2\ell+m-s-2k}\left(\frac{\sin\iota}{2}\right)^{2k+s-m}  
\end{equation}
with $k_1 = \textrm{max}(0, m-s)$ and $k_2=\textrm{min}(\ell+m,
\ell-s)$. For reference,
\begin{eqnarray}
  \label{eq:7}
  \Ytwo_{22} &=& \sqrt{\frac{5}{64\pi}}(1+\cos\iota)^2e^{2i\phi} \,,\\
  \Ytwo_{21} &=& \sqrt{\frac{5}{16\pi}}  \sin\iota( 1 + \cos\iota )e^{i\phi} \,,\\
  \Ytwo_{20} &=& \sqrt{\frac{15}{32\pi}} \sin^2\iota \,,\\
  \Ytwo_{2-1} &=& \sqrt{\frac{5}{16\pi}}  \sin\iota( 1 - \cos\iota
  )e^{-i\phi} \,,\\
  \Ytwo_{2-2} &=& \sqrt{\frac{5}{64\pi}}(1-\cos\iota)^2e^{-2i\phi}\,.
\end{eqnarray}
The mode expansion coefficients $H_{lm}$ are given by
\begin{equation}
  \label{eq:10}
  MH_{\ell m} = \oint \Ytwo_{lm}^\star(\iota,\phi)(rh_+-irh_\times )\,d\Omega\,.
\end{equation}
If $\Psi_4$ is used for wave extraction, then $H_{lm}$ is given by two
time integrals of the corresponding mode of $\Psi_4$. In this case, it
is important that the information provided contains details about how
the integration constants are chosen. We define $h_+^{(\ell m)}$ and
$h_\times^{(\ell m)}$ as
\begin{equation}
  \label{eq:11}
  rh_+^{(\ell m)}(t) -irh_\times^{(\ell m)}(t) := MH_{\ell m}(t)\,. 
\end{equation}
It is these modes $rh_{+,\times}^{(\ell m)}$ of $rh_+$ and $rh_\times$
that we suggest to be provided as functions of time in units of $M$.




\section{Data formats}
\label{sec:format}


Let us now specialize in simulations of binary black hole coalescence.
A numerical relativity simulation has many parameters that need to be
specified, and several of them may not be directly relevant to the
data analysis problem. We need to specify which parameters of the
numerical simulation will be significantly useful for the astrophysics
of a binary black hole system in a circular orbit. For our purposes, a
single numerical waveform is defined by at least seven parameters: the
mass ratio $q = M_1/M_2$ and the three components of the individual
spins $\vec{S}_1$ and $\vec{S}_2$. These parameters will be referred
to as the ``metadata'' for a waveform; more parameters can be added as
necessary. We use the convention that $M_1$ denotes the larger of the
two masses so that $q\geq 1$. The choice of precisely how $M_1$, $M_2$
and the spins are calculated is left up to the individual numerical
relativity groups. In addition, the start frequency $f_0/M$ of the
waveform, in units of $M$, is an important parameter relevant for data
analysis.  For a given value of the mass, this gives the physical
start frequency of the waveform, and this will need to be lesser than
the lower cut-off frequency relevant for a particular detector.  For
example, $f_0 = 40\,$Hz is an appropriate value appropriate for the
initial LIGO detectors.

For the wavfeorm data itself, we suggest the data for a single mode
$rh_{+,\times}^{(\ell m)}$ to be written as a plain text file in three
columns for the time $t$, $rh_+^{(\ell m)}$ and $rh_\times^{(\ell m)}$
respectively. For a given simulation, numerical groups may wish to
decide the maximum value of $\ell = \ell_{\rm max}$ to which they will
provide the waveform. From the data analysis standpoint, it is most
useful that for every $\ell \leq \ell_{\rm max}$, waveforms are
provided for all values of $m = -\ell, \ldots, \ell$, irrespective of
any symmetries that may be present in the simulation. If there are
certain modes which, due to small amplitude, cannot be accurately
determined, these can be set to zero. Numerical groups often extract
the waveform from the simulation at several different radii and then
use Richardson extrapolation to determine the most accurate waveform.
For data analysis purposes, we do not consider waveforms from
different radii as distinct, and would prefer only the most accurate
determination of the waveform from any given simulation.

It is natural to use the total mass $M$ of the binary as the unit for
the time and strain columns. However, there can be subtleties in the
choice of $M$. It could be the ADM mass of the spacetime, an
approximation to the ADM mass measured at the wave-extraction sphere,
or it could be the sum of the individual masses. Again, the choice is
left up to the numerical relativity group which produced the waveform,
and it depends on whatever best represents the time coordinate and the
scale of $h_{ij}$ in the particular simulation.

For data analysis purposes, we would prefer the sampling in time to be
uniform. If the result of a simulation, or set of simulations, yields a
waveform sampled non-uniformly, we ask that the NR group performs an
interpolation to give a uniformly sampled waveform. A sampling rate
of $1\times M$ is usually sufficient for our purposes, but this is not
a requirement. The strain multiplied by the distance will also be in
units of the total mass $M$ of the binary. There can be any number of
comment lines at the top of the file and it is envisioned that the
details of the simulation and the mode contained in the file will be
held in the comment lines.

This could be an example of a data file:

\begin{verbatim}
# numerical waveform from ....
# equal mass, non spinning, 5 orbits, l=m=2
# time       hplus        hcross
0.000000e+00 1.138725e-02 -8.319811e-04
2.000000e-01 1.138725e-02 -1.247969e-03
4.000000e-01 1.138726e-02 -1.663954e-03
6.000000e-01 1.138727e-02 -2.079936e-03
8.000000e-01 1.138728e-02 -2.495913e-03
1.000000e-00 1.138728e-02 -2.911884e-03
1.200000e+00 1.138729e-02 -3.327850e-03
1.400000e+00 1.138730e-02 -3.743807e-03
1.600000e+00 1.138731e-02 -4.159757e-03
1.800000e+00 1.138733e-02 -4.575696e-03
2.000000e+00 1.138734e-02 -4.991627e-03
2.200000e+00 1.138735e-02 -5.407545e-03
2.400000e+00 1.138737e-02 -5.823452e-03
2.600000e+00 1.138739e-02 -6.239345e-03
2.800000e+00 1.138740e-02 -6.655225e-03
3.000000e+00 1.138752e-02 -7.071059e-03
3.200000e+00 1.138754e-02 -7.486903e-03
3.400000e+00 1.138757e-02 -7.902739e-03
......
\end{verbatim}

The metadata information for the different datafiles will be stored in a
separate file. This metadata can contain (at least) two sections,
one for the simulation metadata, and the other listing the filenames
which correspond to the various $(\ell,m)$ modes of the waveform. There
will be a separate metadata file for each simulation.

This could be an example of a metadata file:

\begin{verbatim}
[metadata]
simulation-details = NRfile.dat
nr-group = friendlyNRgroup
email = myemail@somewhere.edu
mass-ratio = 1.0
spin1x = 0.0
spin1y = 0.0
spin1z = 0.5
spin2x = 0.0
spin2y = 0.8
spin2z = 0.0
f0M = 0.1

[ht-data]
2,2  = example1_22.dat
2,1  = example1_21.dat
2,0  = example1_20.dat
2,-1 = example1_2-1.dat
2,-2 = example1_2-2.dat
\end{verbatim}

It would be desirable if the waveform data are reproducible at a later
date if necessary. For this purpose, the numerical relativity groups
can submit a file with the parameters of the simulation. There is no
requirement on the format of this file. The
\texttt{simulation-details} line will contain the name of the file
describing the parameters used to describe the NR simulation. In
addition, we ask for the numerical relativity groups to provide a
\texttt{nr-group} name and a contact \texttt{email}.

The remainder of the entries in the \texttt{[metadata]} section
describe the physical parameters of the waveform. To begin with, for
non-spinning waveforms, the only required parameter is the mass ratio.
For waveforms with spin, the initial spins of the two black holes (in
the co-ordinates discussed in the previous section) must also be
specified.  The start frequency of the waveform in units of $M$ is
denoted by $\mathtt{f0M}$.  We emphasize that whenever necessary, we
will add more parameters such as, for example, the eccentricity of the
orbit. Lines starting with a $\%$ or $\#$ will be taken to be comment
lines; there can be an arbitrary number of comment lines.

The section \texttt{[ht-data]} contains one line for each $(\ell,m)$
mode. These give the file names containing the corresponding modes.
The filenames can be specified as relative paths to the data files
starting from the location of the metadata file. Thus, if the
datafiles are stored in a sub-directory called \texttt{data}, then the
metadata file would read:
\begin{verbatim}
[ht-data]
2,2  = data/example1_22.dat
2,1  = data/example1_21.dat
2,0  = data/example1_20.dat
2,-1 = data/example1_2-1.dat
2,-2 = data/example1_2-2.dat
\end{verbatim}

If the waveforms have been calculated using $\Psi_4$, then for
cross-checking purposes, we request that datafiles containing the real
and imaginary parts of $\Psi_4$ are also provided in the same format as
for the waveforms, i.e. three columns which are respectively time, real
part of $\Psi_4$ and imaginary part of $\Psi_4$. Again there can be an
arbitrary number of comment lines, but in this case there does not need
to be a metadata file. This data can be referenced in the metadata
file in an additional section:

\begin{verbatim}
[psi4-data]
2,2  = data/example1_psi4_22.dat
2,1  = data/example1_psi4_21.dat
2,0  = data/example1_psi4_20.dat
2,-1 = data/example1_psi4_2-1.dat
2,-2 = data/example1_psi4_2-2.dat
\end{verbatim}

Similarly, if the waveforms were extracted using the Zerilli formalism, a
section \texttt{[zerilli-data]} would be added.

To summarize, the numerical relativity groups are asked to submit a
tarball containing the following information for each simulation:
\begin{description}
\item[1.][Required] The data files for $h_+$ and $h_\times$ -- one for
  each $(\ell,m)$ mode of the simulation.
\item[2.][Required] The meta-data file.
\item[3.][Optional] Data files for the functions (e.g. $\Psi_4$ or the
  Zerilli function) which were used to construct $h_{+}$ and $h_{\times}$.
\item[4.][Required] Parameter file for reproducing the waveform.
\end{description}




\section{Open Issues for binary black hole systems}
\label{sec:openissues}

We now list some open issues for binary black hole simulations which
could be topics for further discussion.

We associate the coordinate system $(x,y,z,t)$ with the binary system
as follows. The orbital plane of the binary at $t=0$ is taken to be
the $x$-$y$ plane with the $z$-axis in the direction of the orbital
angular momentum.
\begin{itemize}
\item Is this the best choice of the $z$-axis?  Would it be better to
  choose, say, the spin of the final black hole as the $z$-axis? The
  decision will be determined by requirements of simplicity and
  having as few modes to work with as possible.
\item The orbital plane is unambiguous when the black holes are non-spinning
  but it can be ambiguous in many situations, especially when the spin
  of the black holes causes the orbital plane to precess significantly.
  In such cases, it is left to the numerical relativity group to decide
  what the best choice of the ``orbital plane'' is. 
\end{itemize}
What is the right choice for parameters such as the individual masses,
the total mass and the spins?  Here are some possibilities:
\begin{itemize}
\item $M_1$ and $M_2$ could be the parameters appearing in the initial
  data construction. Alternatively, for non spinning black holes,
  they could be the irreducible masses of the two horizons:
  \begin{equation}
    \label{eq:12}
    M_{irr} = \sqrt{\frac{A}{16\pi}},
  \end{equation}
  where $A$ is the horizon area. For spinning black holes it could be
  given by the Christodoulou formula:
  \begin{equation}
    \label{eq:13}
    M_{J} = \sqrt{\frac{A}{16\pi} + \frac{4\pi J^2}{A}},
  \end{equation}
  where $J$ is an appropriately defined spin for the individual black
  holes. The calculation of $J$ is again left up to the numerical
  relativity group.
\item For the total mass $M$, is it better to use the sum of individual
  horizon masses (including the effect of angular momentum), or could
  it be the total ADM mass or rather, an approximation to the ADM mass
  calculated at the sphere where the waves are extracted? This could
  be specified in the metadata file, for example through the additional lines
  \begin{verbatim}
     mADM = 1.0
     mChristodoulou = 0.97.
  \end{verbatim}
\end{itemize}
What is the best choice for the radiation extraction sphere?
\begin{itemize}
\item How far away do we need to take the sphere?  Is it sufficient to
  take the sphere to be a coordinate sphere, or do we need some
  further gauge conditions?
\end{itemize}
How important is the choice of initial data?
\begin{itemize}
\item Clearly, the initial data used in almost all current simulations
  do not exactly represent real astrophysical binary black hole
  systems. Is this deviation important for gravitational wave
  detection?
\end{itemize}
How large are the error-bars on the numerical results?
\begin{itemize}
\item Is there a reliable way to estimate the systematic errors due to
  finite resolution effects, different gauge choices, wave extraction
  methods etc?  
\end{itemize}

Numerical relativity groups are welcome to raise any other issues that
might be important.


\section*{Acknowledgments}

We are grateful to various numerical relativists for numerous
discussions and suggestions. In particular, we would like to thank
the following people for valuable inputs to this document: Peter
Diener, Sascha Husa, Luis Lehner, Lee Lindblom, Harald Pfeiffer,
Luciano Rezzolla, and Erik Schnetter. 

\end{document}
