\documentclass[oneside]{book}
\usepackage{../lal,fancyhdr,epsfig,psfig,color,hyperref,makeidx}
\includeonly{std,support,hello,factories,tdfilters}

%define page size
\setlength{\textheight}{9.0in}
\setlength{\textwidth}{6.0in}
\setlength{\topmargin}{-0.00in}
\setlength{\oddsidemargin}{-0.25in}
\setlength{\evensidemargin}{\oddsidemargin}
\sloppy

\pagestyle{fancy}
\fancyhf{}
\lhead{\bf\nouppercase\rightmark}
\rhead{ \bf Pg \thepage}

\makeindex

\def\rcs#1{\def\next##1#1{\mbox{##1}}\next}
\newfont{\lsdfont}{cmbx10 at 72pt}

\begin{document}

% \reversemarginpar
% \let\marginpar\mparorig
% \providecommand{\marginpar}[1]{\mbox{}\mparorig{\raggedleft\hspace{0pt}#1}}

% The title page:
\title{\sffamily\bfseries\Huge
\textcolor{red}{\lsdfont L}AL
\raisebox{-2.5ex}{\textcolor{green}{\lsdfont S}\hspace{-0.1em}oftware}
\hspace{-2em}
\raisebox{-0.5ex}{\textcolor{blue}{\lsdfont D}\hspace{-0.2em}ocumentation}}
\author{\bf Members of the LSC}
\date{RCS \rcs$Revision$\rcs$Date$UTC --- Compiled:
\number\year/\ifnum\month<10 0\fi\number\month/\ifnum\day<10 0\fi\number\day}
\maketitle



% The table of contents.
\tableofcontents

\chapter*{Preface}
\textbf{The LAL specification and style guide has precedence over any
statements about programming practice stated in this document, period.}

\part{Coding and Documentation Instructions}

\chapter{Instructions for code documentation}
\section{How to import code and comments from the source code to 
your \LaTeX\ documentation}

In addition to a knowledge of \LaTeX, there are only four additional
instructions you need to know to extract a code fragment, a comment block, or
a piece of \LaTeX\ source from your source-code files and insert it into your
documentation file.  They are:
\begin{verbatim}
 <lalVerbatim file="FileName"> 
 </lalVerbatim> 

 <lalLaTeX file="FileName"> 
 </lalLaTeX > 
\end{verbatim}
Use these commands (and the standard \LaTeX\ command \verb@\input{}@), changes
in the in-line documentation of the source-code will be captured---exactly as
they appear in the source---in the comprehensive manual.  This allows the
programmers to simultaneously serve several difficult masters:
\begin{description}
\item[$\bullet$ ] 
Produce a comprehensive, coherent manual of the entire software package.
\vspace*{-0.05in}
\item[$\bullet$ ] 
Keep the primary source of the documentation close to the code.
\item[$\bullet$] 
Insure that changes in the code and comments are immediately and automatically
captured in the comprehensive manual.
\item[$\bullet$ ] 
Use \LaTeX\ which is suitable for equation writing.
\vspace*{-0.051in}
\end{description}


Using the first pair of instructions, everything between
\texttt{<lalVerbatim file="FileName">} and \texttt{</lalVerbatim>} will be
wrapped in a \LaTeX\ verbatim environment and written to the file
\texttt{FileName.tex} for later insertion in the documentation.  The insertion
can be done with the \LaTeX\ command \verb@\input{FileName}@.

As an example, look at the code fragment in the source file 
\texttt{packages/hello/src/LALHello.c}.
\begin{verbatim}
#include <stdio.h>
#include "LALStdlib.h"
#include "LALHello.h"

/* <lalVerbatim file="LALHelloNRCSID"> */
NRCSID( LALHELLOC, "$Id$" );
/* </lalVerbatim> */
\end{verbatim}
When you run \texttt{make dvi} (this will be automated) in the package
directory, a parser sifts the code file. It wraps what is in between
\texttt{<lalVerbatim file="LALHelloNRCSID"> } and the \texttt{</lalVerbatim>}
in the \LaTeX\ "verbatim environment", and writes the result to the file
\texttt{LALHelloNRCSID.tex }.  This file can be put into the documentation
with the \LaTeX\ command {\verb@\input{LALHelloNRCSID}@}. The result is: 
\input{LALHelloNRCSID}
The marginpar is the line number and file name of where the fragment came
from.

The second pair of commands (\texttt{<lalLaTeX file="FileName">} and the
\texttt{</lalLaTeX>}) writes the intervening \LaTeX\ source to the file (no
verbatim wrapping this time) for later insertion.

\part{Documentation of the LAL packages}
\chapter{Package \texttt{std}}

This package contains headers providing basic datatypes, constants,
and macros that support the LAL standard.

\newpage\input{LALStdlibH}
\newpage\input{LALRCSIDH}
\newpage\input{LALDatatypesH}
\newpage\input{LALStatusMacrosH}
\newpage\input{LALConstantsH}
\newpage\input{LALStdioH}
\newpage\input{LALVersionH}
\newpage\input{LALMallocH}
\newpage\input{LALErrorH}
\newpage\input{LALGSLH}
\newpage\input{StringInputH}
\newpage\input{GridH}


\newpage\begin{thebibliography}{0}
\bibitem{Barnet:1996}
  Particle Data Group, R.~M. Barnett et al., Phys. Rev. D\textbf{54},
  1 (1996)
\bibitem{Lang:1992}
  K.~R. Lang, \textit{Astrophysical Data: Planets and Stars}.
  Springer-Verlag, New York (1992)
\end{thebibliography}

\chapter{Package \texttt{support}}

%% $Id$

This package covers LAL support routines.

These routines do not conform to LAL requirements, and many of them should be
used only for debugging and in test code, not in production code.  These are
compiled and installed as a separate library \texttt{lalsupport}.

\newpage\input{LALStdioH}
\newpage\input{LALVersionH}
\newpage\input{LALMallocH}
\newpage\input{LALErrorH}
\newpage\input{GridH}
\newpage\input{PrintVectorH}
\newpage\input{PrintFTSeriesH}
\newpage\input{ReadFTSeriesH}
\newpage\input{ReadNoiseSpectrumH}
\newpage\input{StringInputH}
\newpage\input{StreamInputH}
\newpage\input{StreamOutputH}
\newpage\input{LALInitBarycenterH}
\newpage\input{LALXMGRInterfaceH}
\newpage\input{LIGOLwXMLH}
\newpage\input{LIGOLwXMLReadH}

\chapter{Package \texttt{hello}}

This is an example package that illustrates how to write and document LAL
code.  This pacakage contains a routine that prints ``hello, LSC!''

\newpage\input{LALHelloH}

\chapter{Package \texttt{factories}}

This package provides routines for creating and destroying the LAL aggregate
datatypes.

\newpage\input{AVFactoriesH}
\newpage\input{SeqFactoriesH}

%\chapter{Package \texttt{vectorops}}

This package contains routines for manipulating vectors.

\newpage\input{VectorOpsH}
\newpage\input{VectorIndexRangeH}
\newpage\input{MatrixH}
\newpage\begin{thebibliography}{0}
\bibitem{ptvf:1992}
  W. H. Press, S. A. Teukolsky, W. T. Vetterling, and B. P. Flannery,
  \textit{Numerical Recipes in C: The Art of Scientific Computing}, 2nd ed.
  (Cambridge University Press, Cambridge, England, 1992).
\end{thebibliography}

%\chapter{Package \texttt{utilities}}

This package contains various numerical utilities for use in LAL.

\newpage\input{RandomH}
\newpage\input{FindRootH}
\newpage\input{IntegrateH}
\newpage\input{InterpolateH}
\newpage
\section{Header \texttt{Sort.h}}

Provides routines for sorting, indexing, and ranking real vector
elements.

\subsection{Synopsis}
\begin{verbatim}
#include "Sort.h"
\end{verbatim}


\subsection{Error conditions}
\begin{tabular}{|c|l|l|}
\hline
status & status                      & Explanation                      \\
 code  & description                 &                                  \\
\hline
\tt 1  & \tt Null pointer            & Missing a required pointer.      \\
\tt 2  & \tt Length mismatch         & Vectors are of different length. \\
\tt 3  & \tt Memory allocation error & Could not allocate memory.       \\
\hline
\end{tabular}

\subsection{Structures}
\newpage
\subsection{Module \texttt{HeapSort.c}}

Sorts, indexes, or ranks vector elements using the heap sort
algorithm.

\subsubsection{Prototypes}
\vspace{0.1in}
\input{HeapSortD}

\subsubsection{Description}

These routines sort a vector \verb@*data@ (of type \verb@REAL4Vector@
or \verb@REAL8Vector@) into ascending order using the in-place
heapsort algorithm, or construct an index vector \verb@*index@ that
indexes \verb@*data@ in increasing order (leaving \verb@*data@
unchanged), or construct a rank vector \verb@*rank@ that gives the
rank order of the corresponding \verb@*data@ element.

The relationship between sorting, indexing, and ranking can be a bit
confusing.  One way of looking at it is that the original array is
ordered by index, while the sorted array is ordered by rank.  The
index array gives the index as a function of rank; i.e.\ if you're
looking for a given rank (say the 0th, or smallest element), the index
array tells you where to look it up in the unsorted array:
\begin{verbatim}
unsorted_array[index[i]] = sorted_array[i]
\end{verbatim}
The rank array gives the rank as a function of index; i.e.\ it tells
you where a given element in the unsorted array will appear in the
sorted array:
\begin{verbatim}
unsorted_array[j] = sorted_array[rank[j]]
\end{verbatim}
Clearly these imply the following relationships, which can be used to
construct the index array from the rank array or vice-versa:
\begin{verbatim}
index[rank[j]] = j
rank[index[i]] = i
\end{verbatim}

\subsubsection{Algorithm}

These routines use the standard heap sort algorithm described in
Sec.~8.3 of Ref.~\cite{ptvf:1992}.

The \verb@SHeapSort()@ and \verb@DHeapSort()@ routines are entirely
in-place, with no auxiliary storage vector.  The \verb@SHeapIndex()@
and \verb@DHeapIndex()@ routines are also technically in-place, but
they require two input vectors (the data vector and the index vector),
and leave the data vector unchanged.  The \verb@SHeapRank()@ and
\verb@DHeapRank()@ routines require two input vectors (the data and
rank vectors), and also allocate a temporary index vector internally;
these routines are therefore the most memory-intensive.  All of these
algorithms are $N\log_2(N)$ algorithms, regardless of the ordering of
the initial dataset.


\subsubsection{Uses}
\begin{verbatim}
I4CreateVector()
I4DestroyVector()
\end{verbatim}

\subsubsection{Notes}


\newpage
\subsection{Program \texttt{SortTest.c}}

A program to test sorting routines.

\subsubsection{Usage}
\begin{verbatim}
SortTest [-s seed] [-d [debug-level]]
\end{verbatim}

\subsubsection{Description}

This test program creates rank and index arrays for an unordered list
of numbers, and then sorts the list.  The data for the list are
generated randomly, and the output is to \verb@stdout@ unless
redirected.  \verb@SortTest@ returns 0 if it executes successfully,
and 1 if any of the subroutines fail.

The \verb@-s@ option sets the seed for the random number generator; if
\verb@seed@ is set to zero (or if no \verb@-s@ option is given) then
the seed is taken from the processor clock.  The \verb@-d@ option
increases the default debug level from 0 to 1, or sets it to the
specified value \verb@debug-level@.


\subsubsection{Algorithm}

\subsubsection{Uses}
\begin{verbatim}
debuglevel
CreateI4Vector()
CreateSVector()
DestroyI4Vector()
DestroySVector()
CreateRandomParams()
DestroyRandomParams()
UniformDeviate()
LALPrintError()
\end{verbatim}

\subsubsection{Notes}



\newpage\input{ODEH}
\newpage\input{DirichletH}
\newpage\input{CoarseGrainFrequencySeriesH}
\newpage\input{MatrixUtilsH}
\newpage\input{LALRunningMedianH}

\newpage\begin{thebibliography}{0}
\bibitem{ptvf:1992}
  W. H. Press, S. A. Teukolsky, W. T. Vetterling, and B. P. Flannery,
  \textit{Numerical Recipes in C: The Art of Scientific Computing}, 2nd ed.
  (Cambridge University Press, Cambridge, England, 1992).
\input{DirichletCB}
\end{thebibliography}

\chapter{Package \texttt{tdfilters}}

This package covers LAL routines for constructing and applying digital
time-domain filters.  It is organized under three headers.  The
\verb@ZPGFilter.h@ header provides routines for manipulating filters
in the ``zeros, poles, gain'' representation, which is typically the
simplest way of representing a filter response.  These routines create
and destroy ZPG filters, and can transform the complex variable used
to represent them.  The \verb@IIRFilter.h@ header provides routines
for creating actual time-domain filters from the ZPG representation,
and applying these filters to data.  The \verb@BandPassTimeSeries.h@
header provides routines an actual implementation of these utilities
to the specific task of high- or low-pass filtering of a data stream.
These routines also serve as an example for the more general task of
designing time-domain filters with any desired frequency response.

The module \verb@ButterworthTimeSeries.c@ provides specific advice and
guidelines for building a numerically stable time-domain filter, but
the general procedure is as follows.  (1) Decide on the desired filter
response, and express it as a rational function of the frequency
variable $w=\tan(\pi f\Delta t)$ (which maps the Nyquist interval onto
the positive real axis).  (2) Factorize this rational function into
zeros and poles, restricting oneself to the upper half of the $w$
complex plane.  Assign these to one or more objects of type
\verb@<datatype>ZPGFilter@.  (3) Use \verb@WToZ<datatype>ZPGFilter()@
to transform the filter to the more conventional $z=\exp(2\pi if\Delta
t)$ frequency variable.  (4) Use the routines in \verb@IIRFilter.h@ to
create IIR filters from the ZPG filters, and to apply them to data.

\newpage
\section{Header \texttt{ZPGFilter.h}}

Provides routines to manipulate ZPG filters.

\subsection{Synopsis}
\begin{verbatim}
#include "ZPGFilter.h"
\end{verbatim}

\noindent This header covers routines that create, destroy, and
transform objects of type \verb@<datatype>ZPGFilter@, where
\verb@<datatype>@ is either \verb@COMPLEX8@ or \verb@COMPLEX16@.
Generically, these data types can be used to store any rational
complex function in a factored form.  Normally this function is a
filter response, or ``transfer function'' $T(z)$, expressed in terms
of a complex frequency parameter $z=\exp(2\pi if\Delta t)$, where
$\Delta t$ is the sampling interval.  The rational function is
factored as follows:
$$
T(f) = g\times\frac{\prod_k (z-a_k)}{\prod_l (z-b_l)}
$$
where $g$ is the gain, $a_k$ are the (finite) zeros, and $b_l$ are the
(finite) poles.  It should be noted that rational functions always
have the same number of zeros as poles if one includes the point
$z=\infty$; any excess in the number of finite zeros or poles in the
rational expression simply indicates that there is a corresponding
pole or zero of that order at infinity.  It is also worth pointing out
that the ``gain'' is just the overall prefactor of this rational
function, and is not necessarily equal to the actual gain of the
transfer function at any particular frequency.

Another common complex frequency space is the $w$-space, obtained
from the $z$-space by the bilinear transformation:
$$
w = i\left(\frac{1-z}{1+z}\right) = \tan(\pi f\Delta t) , \quad
z = \frac{1+iw}{1-iw} \; .
$$
Other variables can also be used to represent the complex frequency
plane.  The \verb@<datatype>ZPGFilter@ structure can be used to
represent the transfer function in any of these spaces by transforming
the coordinates of the zeros and poles, and incorporating any residual
factors into the gain.  Care must be taken to include any zeros or
poles that are brought in from infinity by the transformation, and to
remove any zeros or poles which were sent to infinity.  Thus the
number of zeros and poles of the \verb@<datatype>ZPGFilter@ is not
necessarily constant under transformations!  Routines invoking the
\verb@<datatype>ZPGFilter@ data types should document which complex
variable is assumed.


\subsection{Error conditions}
\begin{tabular}{|c|l|l|}
\hline
status & status      & Explanation \\
 code  & description &             \\
\hline
\tt 1  & \tt Null pointer            & Missing a required pointer.           \\
\tt 2  & \tt Output already exists   & Can't allocate to a non-null pointer. \\
\tt 3  & \tt Memory allocation error & Could not allocate memory.            \\
\tt 4  & \tt Bad filter parameters   & Filter creation parameters outside of \\
       &                             & acceptable ranges.                    \\
\hline
\end{tabular}

\subsection{Structures}
\newpage
\subsection{Module \texttt{CreateZPGFilter.c}}

Creates ZPG filter objects.

\subsubsection{Prototypes}
\vspace{0.1in}
\input{CreateZPGFilterD}

\subsubsection{Description}

These functions create an object \verb@**output@, of type
\verb@COMPLEX8ZPGFilter@ or \verb@COMPLEX16ZPGFilter@, having
\verb@numZeros@ zeros and \verb@numPoles@ poles.  The values of those
zeros and poles are not set by these routines (in general they will
start out as garbage).  The handle passed into the functions must be a
valid handle (i.e.\ \verb@output@$\neq$\verb@NULL@), but must not
point to an existing object (\i.e.\ \verb@*output@=\verb@NULL@).

\subsubsection{Algorithm}

\subsubsection{Uses}
\begin{verbatim}
LALMalloc()
CCreateVector()
ZCreateVector()
\end{verbatim}

\subsubsection{Notes}


\newpage
\subsection{Module \texttt{DestroyZPGFilter.c}}

Destroys ZPG filter objects.

\subsubsection{Prototypes}
\vspace{0.1in}
\input{DestroyZPGFilterD}

\subsubsection{Description}

These functions destroy an object \verb@**output@ of type
\verb@COMPLEX8ZPGFilter@ or \verb@COMPLEX16ZPGFilter@, and set
\verb@*output@ to \verb@NULL@.

\subsubsection{Algorithm}

\subsubsection{Uses}
\begin{verbatim}
LALFree()
CDestroyVector()
ZDestroyVector()
\end{verbatim}

\subsubsection{Notes}


\newpage
\subsection{Module \texttt{BilinearTransform.c}}

Transforms the complex frequency coordinate of a ZPG filter.

\subsubsection{Prototypes}
\vspace{0.1in}
\input{BilinearTransformD}

\subsubsection{Description}

These functions perform an in-place bilinear transformation on an
object \verb@*filter@ of type \verb@<datatype>ZPGFilter@, transforming
from $w$ to $z=(1+iw)/(1-iw)$.  Care is taken to ensure that zeros and
poles at $w=\infty$ are correctly transformed to $z=-1$, and zeros and
poles at $w=-i$ are correctly transformed to $z=\infty$.  In addition
to simply relocating the zeros and poles, residual factors are also
incorporated into the gain of the filter (i.e.\ the leading
coefficient of the rational function).

\subsubsection{Algorithm}

The vectors \verb@filter->zeros@ and \verb@filter->poles@ only record
those zeros and poles that have finite value.  If one includes the
point $\infty$ on the complex plane, then a rational function always
has the same number of zeros and poles: a number \verb@num@ that is
the larger of \verb@z->zeros->length@ or \verb@z->poles->length@.  If
one or the other vector has a smaller length, then after the
transformation that vector will receive additional elements, with a
complex value of $z=-1$, to bring its length up to \verb@num@.
However, each vector will then \emph{lose} those elements that
previously had values $w=-i$, (which are sent to $z=\infty$,) thus
possibly decreasing the length of the vector.  These routines handle
this by simply allocating a new vector for the transformed data, and
freeing the old vector after the transformation.

When transforming a zero $w_k$ on the complex plane, one makes use of
the identity:
$$
(w - w_k) = -(w_k + i)\times\frac{z-z_k}{z+1} \; ,
$$
and similarly, when transforming a pole at $w_k$,
$$
(w - w_k)^{-1} = -(w_k + i)^{-1}\times\frac{z+1}{z-z_k} \; ,
$$
where $z=(1+iw)/(1-iw)$ and $z_k=(1+iw_k)/(1-iw_k)$.  If there are an
equal number of poles and zeros being transformed, then the factors of
$z+1$ will cancel; otherwise, the remaining factors correspond to the
zeros or poles at $z=-1$ brought in from $w=\infty$.  The factor
$(z-z_k)$ represents the transformation of the zero or pole at $w_k$.
The important factor to note, though, is the factor $-(w_k+i)^{\pm1}$.
This factor represents the change in the gain \verb@filter->gain@.
When $w_k=-i$, the transformation is slightly different:
$$
(w + i) = \frac{2i}{z+1} \; ;
$$
thus the gain correction factor is $2i$ (rather than 0) in this case.

The algorithms in this module computes and stores all the gain
correction factors before applying them to the gain.  The correction
factors are sorted in order of absolute magnitude, and are multiplied
together in small- and large-magnitude pairs.  In this way one reduces
the risk of overrunning the floating-point dynamical range during
intermediate calculations.

As a similar precaution, the routines in this module use the algorithm
discussed in the \verb@VectorOps@ package whenever they perform
complex division, to avoid intermediate results that mey be the
product of two large numbers.  When transforming $z=(1+iw)/(1-iw)$,
these routines also test for special cases (such as $w$ purely
imaginary) that have qualitatively significant results ($z$ purely
real), so that one doesn't end up with, for instance, an imaginary
part of $10^{-12}$ instead of 0.

\subsubsection{Uses}
\begin{verbatim}
I4CreateVector()
SCreateVector()
DCreateVector()
CCreateVector()
ZCreateVector()
I4DestroyVector()
SDestroyVector()
DDestroyVector()
CDestroyVector()
ZDestroyVector()
SHeapIndex()
DHeapIndex()
\end{verbatim}

\subsubsection{Notes}



\newpage
\section{Header \texttt{IIRFilter.h}}

Provides routines to make and apply IIR filters.

\subsection{Synopsis}
\begin{verbatim}
#include "IIRFilter.h"
\end{verbatim}

\noindent This header covers routines that create, destroy, and apply
generic time-domain filters, given by objects of type
\verb@<datatype>IIRFilter@, where \verb@<datatype>@ is either
\verb@REAL4@ or \verb@REAL8@.

An IIR (Infinite Impulse Response) filter is a generalized linear
causal time-domain filter, in which the filter output $y_n=y(t_n)$ at
any sampled time $t_n=t_0+n\Delta t$ is a linear combination of the
input $x$ \emph{and} output $y$ at previous sampled times:
$$
y_n = \sum_{k=0}^M c_k x_{n-k} + \sum_{l=1}^N d_l y_{n-l} \; .
$$
The coefficients $c_k$ are called the direct filter coefficients, and
the coefficients $d_l$ are the recursive filter coefficients.  The
filter order is the larger of $M$ or $N$, and determines how far back
in time the filter must look to determine its next output.  However,
the recursive nature of the filter means that the output can depend on
input arbitrarily far in the past; hence the name ``infinite impulse
response''.  Nonetheless, for a well-designed, stable filter, the
actual filter response to an impulse should diminish rapidly beyond
some characteristic timescale.

Note that nonrecursive FIR (Finite Impulse Response) filters are
considered a subset of IIR filters, having $N=0$.

For practical implementation, it is convenient to express the bilinear
equation above as two linear equations involving an auxiliary sequence
$w$:
$$
w_n = x_n + \sum_{l=1}^N d_l w_{n-l} \; ,
$$
$$
y_n = \sum_{k=0}^M c_k w_{n-k} \; .
$$
The equivalence of this to the first expression is not obvious, but
can be proven by mathematical induction.  The advantage of the
auxiliary variable representation is twofold.  First, when one is
feeding data point by point to the filter, the filter needs only
``remember'' the previous $M$ or $N$ (whichever is larger) values of
$w$, rather than remembering the previous $M$ values of $x$ \emph{and}
the previous $N$ values of $y$.  Second, when filtering a large stored
data vector, the filter response can be computed in place: one first
runs forward through the vector replacing $x$ with $w$, and then
backward replacing $w$ with $y$.

Although the IIR filters in these routines are explicitly real, one
can consider formally their complex response.  A sinusoidal input can
thus be written as $x_n=X\exp(2\pi ifn\Delta t)=z^n$, where $X$ is a
complex amplitude and $z=\exp(2\pi if\Delta t)$ is a complex
parametrization of the frequency.  By linearity, the output must also
be sinusoidal: $y_m=Y\exp(2\pi ifm\Delta t)=z^m$.  Putting these into
the bilinear equation, one can easily compute the filter's complex
transfer function:
$$
T(z) = \frac{Y}{X} = \frac{\sum_{k=0}^M c_k z^{-k}}
                      {1 - \sum_{l=1}^N d_l z^{-l}}
$$
This can be readily converted to and from the ``zeros, poles, gain''
representation of a filter, which expresses $T(z)$ as a factored
rational function of $z$.

It should also be noted that, in the routines covered by this header,
I have adopted the convention of including a redundant recursive
coefficient $d_0$, in order to make the indexing more intuitive.  For
formal correctness $d_0$ should be set to $-1$, although the filtering
routines never actually use this coefficient.


\subsection{Error conditions}
\begin{tabular}{|c|l|l|}
\hline
status & status      & Explanation \\
 code  & description & \\
\hline
\tt 1  & \tt Null pointer            & Missing a required pointer.           \\
\tt 2  & \tt Output already exists   & Can't allocate to a non-null pointer. \\
\tt 3  & \tt Memory allocation error & Could not allocate memory.            \\
\tt 4  & \tt Input has unpaired      & For real filters, complex poles or    \\
       & \tt nonreal poles or zeros  & zeros must come in conjugate pairs.   \\
\hline
\end{tabular}


\subsection{Structures}
\begin{verbatim}
<datatype>IIRFilter
\end{verbatim}

\noindent This structure stores the direct and recursive filter
coefficients, as well as the history of the auxiliary sequence $w$.
The length of the history vector gives the order of the filter.  The
fields are:

\begin{description}
\item[\texttt{CHAR *name}] A user-assigned name.

\item[\texttt{<datatype>Vector *directCoef}] The direct filter
  coefficients.

\item[\texttt{<datatype>Vector *recursCoef}] The recursive filter
  coefficients.

\item[\texttt{<datatype>Vector *history}] The previous values of $w$.
\end{description}

\newpage
\subsection{Module \texttt{CreateIIRFilter.c}}

Creates IIR filter objects.

\subsubsection{Prototypes}
\vspace{0.1in}
\input{CreateIIRFilterD}

\subsubsection{Description}

These functions create an object \verb@**output@ of type
\verb@<datatype>IIRFilter@, where \verb@<datatype>@ is \verb@REAL4@ or
\verb@REAL8@.  The filter coefficients are computed from the zeroes,
poles, and gain of an input object \verb@*input@ of type
\verb@COMPLEX8ZPGFilter@ or \verb@COMPLEX16ZPGFilter@, respectively.
The ZPG filter should express the factored transfer function in the
$z=\exp(2\pi if)$ plane.  Initially the output handle must be a valid
handle (\verb@output@$\neq$\verb@NULL@) but should not point to an
existing object (\verb@*output@=\verb@NULL@)

\subsubsection{Algorithm}

An IIR filter is a real time-domain filter, which imposes certain
constraints on the zeros, poles, and gain of the filter transfer
function.  The function \verb@Create<datatype>IIRFilter()@ deals with
the constraints either by aborting if they are not met, or by
adjusting the filter response so that they are met.  In the latter
case, warning messages will be issued if the external parameter
\verb@debuglevel@ is 1 or more.  The specific constraints, and how
they are dealt with, are as follows:

First, the filter must be \emph{causal}; that is, the output at any
time can be generated entirely from the input at previous times.  In
practice this means that the number of (finite) poles in the $z$ plane
must equal or exceed the number of (finite) zeros.  If this is not the
case, \verb@Create<datatype>IIRFilter()@ will add additional poles at
$z=0$.  Effectively this is just adding a delay to the filter response
in order to make it causal.

Second, the filter should be \emph{stable}, which means that all poles
should be located on or within the circle $|z|=1$.  This is not
enforced by \verb@Create<datatype>IIRFilter()@, which can be used to
make unstable filters; however, warnings will be issued if
\verb@debuglevel@ is 1 or more.  (In some sense the first condition is
a special case of this one, since a transfer function with more zeros
than poles actually has corresponding poles at infinity.)

Third, the filter must be \emph{physically realizable}; that is, the
transfer function should expand to a rational function of $z$ with
real coefficients.  Necessary and sufficient conditions for this are
that the gain be real, and that all zeros and poles either be real or
come in complex conjugate pairs.  The routine
\verb@Create<datatype>IIRFilter()@ enforces this by using only the
real part of the gain, and only the real or positive-imaginary zeros
and poles; it assumes that the latter are paired with
negative-imaginary conjugates.  The routine will abort if this
assumption results in a change in the given number of zeros or poles,
but will otherwise simply modify the filter response.  This allows
\verb@debuglevel@=0 runs to proceed without lengthy and usually
unnecessary error trapping; when \verb@debuglevel@ is 1 or more, the
routine checks to make sure that each nonreal zero or pole does in
fact have a complex-conjugate partner.

\subsubsection{Uses}
\begin{verbatim}
debuglevel
LALMalloc()
SCreateVector()
DCreateVector()
\end{verbatim}

\subsubsection{Notes}


\newpage
\subsection{Module \texttt{DestroyIIRFilter.c}}

Destroys IIR filter objects.

\subsubsection{Prototypes}
\vspace{0.1in}
\input{DestroyIIRFilterD}

\subsubsection{Description}

These functions destroy an object \verb@**input@ of type
\texttt{REAL4IIRFilter} or \texttt{REAL8IIRFilter}, and set
\verb@*input@ to \verb@NULL@.

\subsubsection{Algorithm}

\subsubsection{Uses}
\begin{verbatim}
void LALFree()
void SDestroyVector()
void DDestroyVector()
\end{verbatim}

\subsubsection{Notes}


\newpage
\subsection{Module \texttt{IIRFilter.c}}

Computes an instant-by-instant IIR filter response.

\subsubsection{Prototypes}
\vspace{0.1in}
\input{IIRFilterD}

\subsubsection{Description}

These functions pass a time-domain datum to an object \verb@*filter@
of type \verb@REAL4IIRFilter@ or \verb@REAL8IIRFilter@, and return the
filter response.  This is done using the auxiliary data series
formalism described in the header \verb@IIRFilter.h@.

There are two pairs of routines in this module.  The functions
\verb@IIRFilterReal4()@ and \verb@IIRFilterREAL8()@ conform to the LAL
standard, with status handling and error trapping; the input datum is
passed in as \verb@input@ and the response is returned in
\verb@*output@.  The functions \verb@SIIRFilter()@ and
\verb@DIIRFilter()@ are non-standard lightweight routines, which may
be more suitable for multiple callings from the inner loops of
programs; they have no status handling or error trapping.  The input
datum is passed in by the variable \verb@x@, and the response is
returned through the function's return statement.

\subsubsection{Algorithm}

\subsubsection{Uses}

\subsubsection{Notes}


\newpage
\subsection{Module \texttt{IIRFilterVector.c}}

Applies an IIR filter to a data stream.

\subsubsection{Prototypes}
\vspace{0.1in}
\input{IIRFilterVectorD}

\subsubsection{Description}

These functions apply a generic time-domain filter given by an object
\verb@*filter@ of type \verb@REAL4IIRFilter@ or \verb@REAL8IIRFilter@
to a list \verb@*vector@ of data representing a time series.  This is
done in place using the auxiliary data series formalism described in
\verb@IIRFilter.h@, so as to accomodate potentially large data series.
To filter a piece of a larger dataset, the calling routine may pass a
vector structure whose data pointer and length fields specify a subset
of a larger vector.

\subsubsection{Algorithm}

\subsubsection{Uses}
\begin{verbatim}
LALMalloc()
LALFree()
\end{verbatim}

\subsubsection{Notes}


\newpage
\subsection{Module \texttt{IIRFilterVectorR.c}}

Applies a time-reversed IIR filter to a data stream.

\subsubsection{Prototypes}
\vspace{0.1in}
\input{IIRFilterVectorRD}

\subsubsection{Description}

These functions apply a generic time-domain filter \verb@*filter@ to a
time series \verb@*vector@, as with the routines
\verb@IIRFilterREAL4Vector()@ and \verb@IIRFilterREAL8Vector()@, but
do so in a time-reversed manner.  By successively applying normal and
time-reversed IIR filters to the same data, one squares the magnitude
of the frequency response while canceling the phase shift.  This can
be significant when one wishes to preserve phase correlations across
wide frequency bands.

\subsubsection{Algorithm}

Because these filter routines are inherently acausal, the
\verb@filter->history@ vector is meaningless and unnecessary.  These
routines neither use nor modify this data array.

\subsubsection{Uses}

\subsubsection{Notes}



\newpage
\section{Header \texttt{BandPassTimeSeries.h}}

Provides routines to low- or high-pass filter a time series.

\subsection{Synopsis}
\begin{verbatim}
#include "BandPassTimeSeries.h"
\end{verbatim}

\noindent This header covers routines that apply a time-domain low- or
high-pass filter to a data series of type \verb@<datatype>TimeSeries@.
Further documentation is given in the individual routines' modules.


\subsection{Error conditions}
\begin{tabular}{|c|l|l|}
\hline
status & status                    & Explanation                           \\
 code  & description               &                                       \\
\hline
\tt 1  & \tt Null pointer          & Missing a required pointer.           \\
\tt 2  & \tt Bad filter parameters & Filter creation parameters outside of \\
       &                           & acceptable ranges.                    \\
\hline
\end{tabular}


\subsection{Structures}

\begin{verbatim}
struct PassBandParamStruc
\end{verbatim}

\noindent This structure stores data used for constructing a low- or
high-pass filter: either the order and characteristic frequency of the
filter, or the frequencies and desired attenuations at the ends of
some transition band.  In the latter case, a nonzero filter order
parameter n indicates a maximum allowed order.  The fields are:

\begin{description}
\item[\texttt{CHAR *name}] A user-assigned name.

\item[\texttt{INT4 n}] The maximum desired filter order (actual order
  may be less if specified attenuations do not require a high order).

\item[\texttt{REAL8 f1}, \texttt{f2}] The reference frequencies of the
  transition band.

\item[\texttt{REAL8 a1}, \texttt{a2}] The minimal desired attenuation
  factors at the reference frequencies.
\end{description}

\newpage
\subsection{Module \texttt{ButterworthTimeSeries.c}}

Applies a low- or high-pass Butterworth filter to a time series.

\subsubsection{Prototypes}
\vspace{0.1in}
\input{ButterworthTimeSeriesD}

\subsubsection{Description}

These routines perform an in-place time-domain band-pass filtering of
a data sequence \verb@*series@, using a Butterworth filter generated
from parameters \verb@*params@.  The routines construct a filter with
the square root of the desired amplitude response, which it then
applied to the data once forward and once in reverse.  This gives the
full amplitude response with little or no frequency-dependent phase
shift.

\subsubsection{Algorithm}

The frequency response of a Butterworth low-pass filter is easiest to
express in terms of the transformed frequency variable $w=\tan(\pi
f\Delta t)$, where $\Delta t$ is the sampling interval (i.e.\
\verb@series->deltaT@).  In this parameter, then, the \emph{power}
response (attenuation) of the filter is:
$$
|R|^2 = a = \frac{1}{1+(w/w_c)^{2n}} \; ,
$$
where $n$ is the filter order and $w_c$ is the characteristic
frequency.  Similarly, a Butterworth high-pass filter is given by
$$
|R|^2 = a = \frac{1}{1+(w_c/w)^{2n}} \; .
$$
If one is given a filter order $n$, then the characteristic frequency
can be determined from the attenuation at some any given frequency.
Alternatively, $n$ and $w_c$ can both be computed given attenuations
at two different frequencies.

Frequencies in \verb@*params@ are assumed to be real frequencies $f$
given in the inverse of the units used for the sampling interval
\verb@series->deltaT@.  In order to be used, the pass band parameters
must lie in the ranges given below; if a parameter lies outside of its
range, then it is ignored and the filter is calculated from the
remaining parameters.  If too many parameters are missing, the routine
will fail.  The acceptable parameter ranges are:

\begin{description}
\item[\texttt{params->nMax}]   = 1, 2, $\ldots$
\item[\texttt{params->f1}, \texttt{f2}] $\in
  (0,\{2\times\verb@series->deltaT@\}^{-1}) $
\item[\texttt{params->a1}, \texttt{a2}] $\in (0,1) $
\end{description}

If both pairs of frequencies and amplitudes are given, then \verb@a1@,
\verb@a2@ specify the minimal requirements on the attenuation of the
filter at frequencies \verb@f1@, \verb@f2@.  Whether the filter is a
low- or high-pass filter is determined from the relative sizes of
these parameters.  In this case the \verb@nMax@ parameter is optional;
if given, it specifies an upper limit on the filter order.  If the
desired attenuations would require a higher order, then the routine
will sacrifiece performance in the stop band in order to remain within
the specified \verb@nMax@.

If one of the frequency/attenuation pairs is missing, then the filter
is computed using the remaining pair and \verb@nMax@ (which must be
given).  The filter is taken to be a low-pass filter if \verb@f1@,
\verb@a1@ are given, and high-pass if \verb@f2@, \verb@a2@ are given.
If only one frequency and no corresponding attenuation is specified,
then it is taken to be the characteristic frequency (i.e. the
corresponding attenuation is assumed to be 1/2).  If none of these
conditions are met, the routine will return an error.

Once an order $n$ and characteristic frequency $w_c$ are known, the
zeros and poles of a ZPG filter are readily determined.  A stable,
physically realizable Butterworth filter will have $n$ poles evenly
spaced on the upper half of a circle of radius $w_c$; that is,
$$
R = \frac{(-iw_c)^n}{\prod_{k=0}^{n-1}(w - w_c e^{2\pi i(k+1/2)/n})}
$$
for a low-pass filter, and
$$
R = \frac{w^n}{\prod_{k=0}^{n-1}(w - w_c e^{2\pi i(k+1/2)/n})}
$$
for a high-pass filter.  By choosing only poles on the upper-half
plane, one ensures that after transforming to $z$ the poles will have
$|z|<1$.

Although higher orders $n$ would appear to produce better (i.e.\
sharper) filter responses, one rapidly runs into numerical errors, as
one ends up adding and subtracting $n$ large numbers to obtain small
filter responses.  One way around this is to break the filter up into
several lower-order filters.  The routines in this module do just
that.  Poles are paired up across the imaginary axis, (and combined
with pairs of zeros at $w=0$ for high-pass filters,) to form $[n/2]$
second-order filters.  If $n$ is odd, there will be an additional
first-order filter, with one pole at $w=iw_c$ (and one zero at $w=0$
for a high-pass filter).

Each ZPG filter in the $w$-plane is first transformed to the $z$-plane
by a bilinear transformation, and is then used to construct a
time-domain IIR filter.  Each filter is then applied to the time
series.  As mentioned in the description above, the filters are
designed to give an overall amplitude response that is the square root
of the desired attenuation; however, each time-domain filter is
applied to the data stream twice: once in the normal sense, and once
in the time-reversed sense.  This gives the full attenuation with very
little frequency-dependent phase shift.

\subsubsection{Uses}
\begin{verbatim}
debuglevel
LALPrintError()
CreateREAL4IIRFilter()
CreateREAL8IIRFilter()
DestroyREAL4IIRFilter()
DestroyREAL8IIRFilter()
CreateCOMPLEX8ZPGFilter()
CreateCOMPLEX16ZPGFilter()
DestroyCOMPLEX8ZPGFilter()
DestroyCOMPLEX16ZPGFilter()
IIRFilterREAL4Vector()
IIRFilterREAL8Vector()
IIRFilterREAL4VectorR()
IIRFilterREAL8VectorR()
\end{verbatim}

\subsubsection{Notes}


\newpage
\subsection{Program \texttt{BandPassTest.c}}

Tests time-domain high- and low-pass filters.

\subsubsection{Usage}
\begin{verbatim}
BandPassTest [-o [outfile]] [-d [debug-level]]
\end{verbatim}

\subsubsection{Description}

This program generates a time series with an impulse in it, and passes
it through a time-domain low-pass or high-pass filter.  By default,
running this program with no arguments simply tests the subroutines,
producing no output.  All filter parameters are set from
\verb@#define@d constants.  The program returns a value of 0 upon
successful completion, 1 if any of its function calls failed, or 2 if
output file creation failed.

The \verb@-o@ flag tells the program to print the impulse response to
a data file; if \verb@outfile@ is not specified, it will write to the
file \verb@out.dat@.  The \verb@-d@ option increases the default debug
level from 0 to 1, or sets it to the specified value
\verb@debug-level@.

\subsubsection{Algorithm}

\subsubsection{Uses}
\begin{verbatim}
debuglevel
SCreateVector()
SDestroyVector()
ButterworthREAL4TimeSeries()
LALPrintError()
\end{verbatim}

\subsubsection{Notes}




%\chapter{Package \texttt{window}}

This package contains a function to create a vector
containing a window (also called
a taper, lag window, or apodization function).  The choices
currently available are:
\begin{itemize}
\item Rectangular
\item Hann
\item Welch
\item Bartlett
\item Parzen
\item Papoulis
\item Hamming
\item Kaiser.
\end{itemize}
Using window functions is well documented in many places.  Their
principal purpose is to reduce the {\it bias} in power spectrum
estimation.  For example, if a sinusoidal signal is present, it will
give rise to a spike in a power spectrum.  If such a  signal is present
exactly at the frequency of a particular bin, the spike will have some
height.  If a signal at the same amplitude but slighly different
frequency is present, and the frequency is not exactly the same
as one of the bins, then
the spike will be broader and lower. Without widowing, this effect introduces
bias into spectral estimates.

If the signal is first multiplied by the window function, the
relative difference between the two resulting power spectra will be
less evident.  The signal which would have given a single-bin spike
will now be spread over several bins.  The signal that would have
been spread over several bins will now be somewhat more peaked.
In short, the two power spectra will appear more similar (apart from
the difference in frequency of the two signals).  Hence the spectra
is less biased.

Definitions of most of the window functions above may be found in {\it
Numerical Recipes} \cite{numrec} equations 13.4.13-13.4.15.  Definitions of
the remaining windows can be found in {\it Spectral analysis for physical
applications} \cite{pw} Section 6.11. Definition of the Kaiser window can be
found in ``Discrete-time Signal Processing'' by Oppenheim and Schafer, p.474.



\newpage\input{WindowH}
\newpage\begin{thebibliography}{0}
\bibitem{numrec}
W. H. Press, S. A. Teukolsky, W. T. Vetterling, and B. P. Flannery,
  \textit{Numerical Recipes in C: The Art of Scientific Computing}, 2nd ed.
  (Cambridge University Press, Cambridge, England, 1992).
\bibitem{pw}
D.B. Percival and A.T. Walden, {\it Spectral analysis for physical applications}, first edition, Cambridge University Press, (1993).
\bibitem{pm}
J.G. Proakis and D.G. Manolakis, {\it Digital
Signal Processing: principles, algorithms and applications},
3rd Edition, 1995.
Prentice Hall
\end{thebibliography}

%\chapter{Package \texttt{fft}}

This package contains various routines for performing FFTs.

\newpage\input{RealFFTH}
\newpage\input{ComplexFFTH}
\newpage\input{TimeFreqFFTH}

\newpage\begin{thebibliography}{0}
\bibitem{fj:1998}
  M. Frigo and S. G. Johnson,
  \textit{FFTW User's Manual},
  (Massachusetts Institute of Technology, Cambridge, USA, 1998).
  URL: \texttt{http://www.fftw.org/doc}
\bibitem{ptvf:1992}
  W. H. Press, S. A. Teukolsky, W. T. Vetterling, and B. P. Flannery,
  \textit{Numerical Recipes in C: The Art of Scientific Computing}, 2nd ed.
  (Cambridge University Press, Cambridge, England, 1992).
\end{thebibliography}

%\include{timefreq}
%\section{Stochastic Search Programs}
\label{section:stochastic}

This section of \textsc{LALApps} contains programs that can be used to
search interferometer data for stochastic gravitational wave
backgrounds.

\clearpage
\section{Program \texttt{lalapps\_olapredfcn}}
\label{program:lalapps-olapredfcn}
\idx[Program]{lalapps\_olapredfcn}

\begin{entry}

\item[Name]
%
  \verb$lal_olapredfcn$ --- computes overlap reduction function given
  a pair of known detectors.

\item[Synopsis]
%
  \verb$lal_olapredfcn $[\verb$-h$]\verb$ $[\verb$-q$]\verb$ $[\verb$-v$]
  \verb$ $[\verb$-d debugLevel $]\verb+ \+\newline
  \verb$   $
  \verb$-s siteID1 $[\verb$-a azimuth1$]
  \verb$-t siteID2 $[\verb$-b azimuth2$]\verb+ \+\newline
  \verb$   $
  [\verb$-f fLow$]\verb$ -e deltaF$\verb$ -n numPoints$\verb$ -o outfile$
                         
\item[Description]
%
  \verb$lal_olapredfcn$ computes the overlap reduction function
  $\gamma(f)$ for a pair of known gravitational wave detectors.  It
  uses the LAL function \verb$LALOverlapReductionFunction()$, which is
  documented in the LAL Software Documentation under the
  \texttt{stochastic} package.

\item[Options]\leavevmode
\begin{entry}
\item[\texttt{-h}]
  Print a help message.
\item[\texttt{-q}]
  Run silently (redirect standard input and error to \texttt{/dev/null}).
\item[\texttt{-v}]
  Run in verbose mode.
\item[\texttt{-d} \textit{debugLevel}]
  Set the LAL debug level to \textit{debugLevel}.
\item[\texttt{-s} \textit{siteID1} \texttt{-t} \textit{siteID2}]
  Use detector sites identified by \textit{siteID1} and
  \textit{siteID2}; ID numbers between \texttt{LALNumCachedDetectors}
  (defined in the \texttt{tools} package of LAL) refer to detectors
  cached in the constant array \verb$lalCachedDetectors[]$.  (At this
  point, these are all interferometers.)  Additionally, the five
  resonant bar detectors of the IGEC collaboration can be specified.
  The bar geometry data (summarized in table~\ref{table:cachedBars})
  is used by the fucntion \verb$LALCreateDetector()$ from the
  \texttt{tools} package of LAL to generate the Cartesian position
  vector and response tensor which are used to calculate the overlap
  reduction function.  The ID numbers for the bars depend on the value
  of \texttt{LALNumCachedDetectors}; the correct ID numbers can be
  obtained by with the command
\begin{verbatim}
./lalapps_olapredfcn -h
\end{verbatim}
\item[\texttt{-a} \textit{azimuth1} \texttt{-b} \textit{azimuth2}]
%
  If \textit{siteID1} (\textit{siteID2}) is a bar detector, assume it
  has an azimuth of \textit{azimuth1} (\textit{azimuth2}) degrees East
  of North rather than the default IGEC orientation given in
  table~\ref{table:cachedBars}.  Note that this convention, measuring
  azimuth in degrees clockwise from North is not the same as that used
  in LAL (which comes from the frame spec).  Note also that any
  specified azimuth angle is ignored if the corresponding detector is
  an interferometer.
\item[\texttt{-f} \textit{fLow}]
  Begin the frequency series at a frequency of \textit{fLow}\,Hz; if this
  is omitted, the default value of 0\,Hz is used.
\item[\texttt{-e} \textit{deltaF}]
  Construct the frequency series with a frequency spacing of
  \textit{deltaF}\,Hz
\item[\texttt{-n} \textit{numPoints}]
  Construct a frequency series with \textit{numPoints} points.
\item[\texttt{-o} \textit{outfile}]
  Write the output to file \textit{outfile}.  The format of this file
  is that output by the routine \verb$LALPrintFrequencySeries()$ in
  the \texttt{support} package of LAL, which consists of a header
  describing metadata followed by two-column rows, each containing the
  doublet $\{f,\gamma(f)\}$.
\end{entry}

\begin{table}[tbp]
  \begin{center}
    \begin{tabular}{|c|c|c|c|}
\hline
      Name & Longitude & Latitude & Azimuth
\\ \hline
\verb$AURIGA$ & $11^\circ56'54''$E & $45^\circ21'12''$N & N$44^\circ$E 
\\ \hline
\verb$NAUTILUS$ & $12^\circ40'21''$E & $41^\circ49'26''$N & N$44^\circ$E 
\\ \hline
\verb$EXPLORER$ & $6^\circ12'$E & $46^\circ27'$N & N$39^\circ$E 
\\ \hline
\verb$ALLEGRO$ & $91^\circ10'43.\!\!''766$W & $30^\circ24'45.\!\!''110$N 
& N$40^\circ$W
\\ \hline
\verb$NIOBE$ & $115^\circ49'$E & $31^\circ56'$S & N$0^\circ$E 
\\ \hline
    \end{tabular}
    \caption{Location and orientation data for the five IGEC resonant
      bar detectors, stored in the \texttt{lalCachedBars[]}
      array.  The data are taken from
      \texttt{http://igec.lnl.infn.it/cgi-bin/browser.pl?Level=0,3,1}
      except for the latitude and longitude of ALLEGRO, which were
      taken from Finn \& Lazzarini, gr-qc/0104040.  Note that the
      elevation above the WGS-84 reference ellipsoid and altitude
      angle for each bar is not given, and therefore set to zero.}
    \label{table:cachedBars}
  \end{center}
\end{table}


\item[Example usage]
  To compute the overlap reduction function for LIGO Hanford and
  LIGO Livingston, with a resolution of 1\,Hz from 0\,Hz to 1024\,Hz:
\begin{verbatim}
lalapps_olapredfcn -s 0 -t 1 -e 1 -n 1025 -o LHOLLO.dat
\end{verbatim}
  
  To compute the overlap reduction function for ALLEGRO in its optimal
  orientation of $72.\!\!^\circ08$ West of South (see Finn \& Lazzarini,
  gr-qc/0104040) and LIGO Livingston, with a resolution of 0.5\,Hz from
  782.5\,Hz to 1032\,Hz (assuming \texttt{lalNumCachedBars} is 6):
\begin{verbatim}
lalapps_olapredfcn -s 9 -a 252.08 -t 1 -f 782.5 -e 0.5 -n 500 -o ALLEGROLHO.dat
\end{verbatim}

\item[Author]
John T.~Whelan

\end{entry}


\clearpage
\subsection{Program \texttt{lalapps\_stochastic\_pipe}}
\label{program:stochastic-pipeline}
\idx[Program]{stochastic\_pipeline.py}

\begin{entry}
\item[Name]
\verb$lalapps_stochastic_pipe$ --- python script to generate Condor DAGs
to run the stochastic pipeline.

\item[Synopsis]
\begin{verbatim}
  -h, --help               display this message
  -v, --version            print version information and exit

  -d, --datafind           run LSCdataFind to create frame cache files
  -s, --stochastic         run lalapps_stochastic

  -p, --playground-only    only create chunks that overlap with playground
  -P, --priority PRIO      run jobs with condor priority PRIO

  -f, --config-file FILE   use configuration file FILE
  -l, --log-path PATH      directory to write condor log file
\end{verbatim}

\item[Description] \verb$lalapps_stochastic_pipe$ generates a Condor DAG
to run the stochastic search pipeline. The configuration should specify
the parameters needed to run the jobs and must be specified with the
\verb$--config-file$ option. A file containing science segments to
analysed should be specified in the \verb$[input]$ section of the
configuration file with a line such as
\begin{verbatim}
segments = S2H1L1v03_selectedsegs.txt
\end{verbatim}
This should contain four whitespace separated columns:
\begin{verbatim}
  segment_id    gps_start_time    gps_end_time    duration
\end{verbatim}
that define the science segments to be used. Lines starting with an
octothorpe are ignored.

\item[Example]
Generate a DAG to run a stochastic search on a pair of interferometers
specified in the configuration file. The generated DAG is then submitted
with \texttt{condor\_submit\_dag}
\begin{verbatim}
lalapps_inspiral_pipe --log-path /home/ram/dag_logs \
--datafind --stochastic --config-file stochastic_H1L1.ini

condor_submit_dag stochastic_H1L1.dag
\end{verbatim}

\item[Author]
Adam Mercer
\end{entry}

\clearpage
\subsection{Program \texttt{lalapps\_stochastic}}
\label{program:lalapps-stochastic}
\idx[Program]{lalapps\_stochastic}

\begin{entry}
\item[Name]
\verb$lalapps_stochastic$ --- standalone stochastic analysis code.

\item[Synopsis]
\begin{verbatim}
 --help                        print this message
 --version                     display version
 --verbose                     verbose mode
 --debug-level LEVEL           set lalDebugLevel
 --gps-start-time SEC          GPS start time
 --gps-end-time SEC            GPS end time
 --segment-duration SEC        segment duration
 --resample-rate F             resample rate
 --f-min F                     minimal frequency
 --f-max F                     maximal frequency
 --hann-duration SEC           hann duration
 --ifo-one IFO                 ifo for first stream
 --ifo-two IFO                 ifo for second stream
 --data-cache-one FILE         cache file for first stream
 --data-cache-two FILE         cache file for second stream
 --calibration-cache-one FILE  first stream calibration cache
 --calibration-cache-two FILE  second stream calibration cache
 --apply-mask                  apply frequency masking
 --inject                      inject a signal into the data
 --scale-factor N              scale factor for injection
 --seed N                      seed
 --overlap-hann                overlaping hann windows for data windowing
 --high-pass-filter            perform high pass filtering
\end{verbatim}

\item[Description] \verb$lalapps_stochastic$ runs the standalone
stochastic analysis code.

\item[Author] 
Adam Mercer, Tania Regimbau
\end{entry}

%\chapter{Package \texttt{framedata}}

Package for reading frame-format data files.

\newpage\input{FrameDataH}
\newpage\input{SpecBufferH}
\newpage\input{DataBufferH}

%\chapter{Package \texttt{comm}}

This package contains routines for using MPI to exchange LAL data structures.

\newpage\input{CommH}

%\newcommand{\ospsd}{\ensuremath{S\left(\left|f_{k}\right|\right)}}

\chapter{Package \texttt{findchirp}}

This package contains LAL routines to search for binary inspiral chirps
using templatated matched filtering and the $\chi^2$ veto. The
\texttt{findchirp} package is designed to allow the user to construct code to
filter interferometer data and produce a list of candidate inspiral events. It
also contains functionality to perform simulation and testing of the inspiral
search. Conceptually the package is divided into the following parts:

\begin{itemize}
\item Processing the raw interferometer input data into a form that can be
used for the filtering process.

\item Processing an inspiral chirp template into a form that can be used by
the filter, generating the chirp template internally, if necessary.

\item Using the processed input data, construction of a statistic on which to
search for chirps and searching for candidate events.

\item Constructing a veto statistic to apply to candidate events to reduce the
possibility of false alarms.

\item Code to tie all this functionality together and provide an engine to
execute these functions and perform a full flat or hierarchical inspiral
search.

\item Functionality to search for an inspiral waveform in the outputs of a
pair of interferometers.
\end{itemize}

We introduce the conventions used in the package and then describe the theory
and implementation of the code. An overview of the package is as follows:
\begin{enumerate}
\item The header \texttt{FindChirp.h} and modules grouped therein provide the
core functionality of the package. This includes the code to perform matched
filtering and search for chirps with a signal to noise ratio above a given
threshold.

\item The header \texttt{FindChirpChisq.h} and associated module provide
functionality to perform a $\chi^2$ veto on candidate events generated by the
\texttt{FindChirpFilter()} function.

\item The header \texttt{FindChirpSP.h} provides functionality to condition
the input interferometer data so that it can be used by the 
\texttt{FindChirpFilter()}
or \texttt{FindChirpBCVFilter()} function. 
It also provides code to generate 2.5
post-Newtonian inspiral chirps using the stationary phase approximation to the
inspiral waveform.

Simulation code is also provided in the engine, so that various Monte Carlo
simulations may be run using the same code that is used to actually search for
the chirps.

If LAL is configured with the \texttt{--enable-mpi} option it can
be used in an MPI environment, such the LDAS \texttt{wrapperAPI}. An
implementation of this is provided in the \texttt{inspiral} package of
LALWrapper. The MPI code required the modules grouped under the header
\texttt{FindChirpExch.h}, which are built when the appropriate configure
options are specified.

The \texttt{findchirp} engine called from standalone code, such as code
written to run under the Condor enviornment, although the functionality for
this ths not yet complete. An implementation of this is under development in
the \texttt{finchirp} program in LALApps. To run under Condor, LAL must be
configured with the \texttt{--disable-mpi} option.

\end{enumerate}

At the present time, the conditioning of the input data is performed by the
function \texttt{FindChirpSPData()} or the function \texttt{FindChirpBCVData()}
, so the order of the documentation is
silightly reversed. Documantation of the algorithm used to condition the input
data appears after the documentation of the filtering functions that use this
data.

The next stage of development will be to modify the package so that the code
that performs conditioning of the input data that is indepenent of the
implementation used for chirp generation is moved from the
\texttt{FindChirpSPData()}/texttt{FindChirpBCVData()} 
 to a function \texttt{FindChirpData()} under the
\texttt{FindChirp.h} header. The function \texttt{FindChirpSPData()} 
/ \texttt{FindChirpBCVData()} will then
be reduced to contain only the code that performs data conditioning specific
to the stationary phase approximation / BCV templates. A new header will be 
created called
\texttt{FindChirpTD.h} that contains the necessary functions
\texttt{FindChirpTDData()} and \texttt{FindChirpTDFilter()} to allow the use
of different time domain generate inspiral signal, such as those provided by
the \texttt{inspiral} package. At this stage, the ordering of the
documentation will be corrected.
\newpage

\section{Conventions}

As mandated, we follow the standards for LSC code in the standards
document T010095. Before we discuss the code itself, we present a detailed
overview of the standards. The reason for this is two fold; frist, this code is
designed to interoperate with other code used by the LSC, such as recieving
power spectral densities computed by the LDAS \texttt{datacondAPI}. Second, it
is very easy to loose track of normalisation of the filter output, causing
events to be reported with incorrect parameters. Without strict adherence to
the standards interoperability and accuracy would not be possible.

All the \texttt{findchirp} functions measure mass in units of $M_\odot$ 
and time in units of seconds.

\subsection{The Fourier Transform}

We define the forward Fourier transform $\tilde{h}(f)$ of a time domain
quantity $h(t)$ to be
\begin{equation}
\tilde{h}(f)=\int_{-\infty}^\infty dt\,h(t)\, e^{- 2 \pi i f t}
\end{equation}
and the inverse Fourier transform to be 
\begin{equation}
h(t)=\int_{-\infty}^\infty df\,\tilde{h}(f)\, e^{2 \pi i f t}.
\end{equation}
If the function $h(t)$ is sampled at $N$ consecitive points with sampling
interval $\Delta t$, that is
\begin{equation}
h_j \equiv h(t_j)\ ,\quad t_j = j\Delta t,
\end{equation}
we only have $N$ input values, so we can only produce $N$ independent values
for the Fourier transform. Further, we can only produce values in the interval
$(-f_c,f_c)$ where $f_c$ is the Nyquist critical frequency
\begin{equation}
f_c = \frac{1}{2\Delta t}.
\end{equation}
We compute estimates of the Fourier transform at the $N + 1$ discrete values
\begin{equation}
f_k \equiv \frac{k}{N\Delta t}\quad k = -\frac{N}{2},\ldots,\frac{N}{2}.
\end{equation}
There are only $N$ independent values here as the extremevales of $k$
correspond to the upper and lower limites of the Nyquist critical frequency
range and are equal. We now proceed to define the discrete Fourier transform.
Consider
\begin{eqnarray}
\tilde{h}(f_k) &=& \int_{-\infty}^\infty dt\,h(t)\, e^{-2 \pi i f t} \\
&\approx& \sum_{j=0}^{N-1} \Delta t\, h(t_j) e^{-2 \pi i f_k t_j} \\
&\approx& \Delta t \sum_{j=0}^{N-1} h_j e^{-2 \pi i j k / N} .
\end{eqnarray}
According to T010095, we define the discrete Fourier transform (DFT) to be
\begin{equation}
\tilde{h}_k = \sum_{j=0}^{N-1} h_j e^{-i 2 \pi j k / N}
\end{equation}
and then we can recover $\tilde{h}(f_k)$ by
\begin{equation}
\tilde{h}(f_k) = \Delta t\, \tilde{h}_k.
\end{equation}
The inverse Fourier transform is
\begin{equation}
h(t)=\int_{-\infty}^\infty dt\,\tilde{h}(f)\, e^{2 \pi i f t} .
\end{equation}
Using
\begin{equation}
\Delta f = f_{k+1} - f_k = \frac{k+1}{N\Delta t} - \frac{k}{N\Delta t} =
\frac{1}{N\Delta t}
\end{equation}
we may write
\begin{eqnarray}
h(t_j) &\approx& \sum_{k=0}^{N-1} \tilde{h}(f_k) e^{2 \pi i f_k t_j / N}
\Delta f \\
&=& \sum_{k=0}^{N-1} \Delta \tilde{h}_k e^{2 \pi i j k / N}\frac{1}{N\Delta t}
\\
&=& \frac{1}{N} \sum_{k=0}^{N-1} \tilde{h}_k e^{2 \pi i j k / N}
\end{eqnarray}
which is the inverse DFT according to T010095. Note that the LAL ``reverse''
DFT functions do not include the factor $1/N$ in their output.

\subsection{Power Spectral Densities}

Consider a signal, $n(t)$, containing Gaussian noise and dimensions $U$, which
may be voltage, strain, etc. We define the (one sided) power spectral density,
$S(|f|)$, of this signal by the equation
\begin{equation}
\left\langle\tilde{n}(f) \tilde{n}^\ast(f')\right\rangle = 
\frac{1}{2}S\left(\left|f\right|\right)\delta(f-f).
\end{equation}
The total power in a signal is independent of whether it is computed in the
time or the frequency domain (Parseval's theorem). The power in a signal in
the interval $(0,T)$  is given by
\begin{equation}
P = \frac{1}{T} \int_{0}^{T} dt\, \left|h(t)\right|^2 = 
\int_{0}^{f_c} df\, S\left(\left|f\right|\right).
\end{equation}
For discretely sampled quantities we have
\begin{equation}
\left\langle\tilde{n}(f_k) \tilde{n}^\ast(f_{k'})\right\rangle = 
\frac{1}{2}\ospsd\delta(f_k-f_{k'})
\end{equation}
which gives
\begin{equation}
\label{findchirp:eq:ospsddef}
\left\langle\tilde{n}_k \tilde{n}_{k'}^\ast\right\rangle = 
\frac{N}{2\Delta t}\ospsd\delta_{kk'}
\end{equation}
which defines \ospsd in terms of the discrete frequency domain quantities.
Parsevals theorem becomes
\begin{equation}
\Delta t \sum_{j=0}^{N-1} |h_j|^2 
= \sum_{k=0}^{[N/2]} S\left(\left|f_k\right|\right),
\end{equation}
the power spectral density having units of $\mathrm{time}\times U^2$. The
definition in equation [\ref{findchirp:eq:psddef}] is equivalent to that in
the standards document T010095%
\footnote{Note that we write \ospsd insted of $S_k$.}:
\begin{equation}
\ospsd = \left\{
\begin{array}{ll}
\frac{\Delta t}{N} | \tilde{h}_0 |^2 & k = 0, \\
\\
\frac{\Delta t}{N} \left[ | \tilde{h}_k |^2 + | \tilde{h}_{N-k} |^2 \right] &
k\neq 0.
\end{array}
\right.
\end{equation}

\newpage\input{FindChirpH}
\newpage\input{FindChirpChisqH}
\newpage\input{FindChirpSPH}
\newpage\input{FindChirpBCVH}
\newpage\input{FindChirpBCVSpinH}

%\chapter{Package \texttt{pulsar}: common routines}
Teviet Creighton
\bigskip

This package provides routines for timing, metric calculation and
mesh-generation relevant for pulsar searches.

\newpage\input{PulsarTimesH}
\newpage\input{LALBarycenterH}
\newpage\input{FlatMeshH}
\newpage\input{TwoDMeshH}
\newpage\input{TwoDMeshPlotH}
\newpage\input{ResampleH}

\newpage\begin{thebibliography}{0}
\bibitem{Brady_P:2000}
  P.~R. Brady and T. Creighton, Phys. Rev. D\textbf{61}, 082001
  (2000).
\end{thebibliography}

\chapter{Package \texttt{pulsar}: amplitude folding routines}
Greg Mendell
\bigskip

Contains function LALFoldAmplitudes: folds amplitudes into phase bins.

\begin{verbatim}
Files:

FoldAmplitudes.h        header file
FoldAmplitudes.c        source code
FoldAmplitudesTest.c    test code
foldamplitudes.tex      overview
\end{verbatim}

Periodic sources of gravitational radiation will produce measured strains of the following form:
$$
c[i] = A(t_i,\vec{\lambda}) \sin[\Phi(t_i,\vec{\lambda})] + n(t_i)
$$
In this equation $c[i]$ is the discrete time series output of the detector (perhaps after some data conditioning, such as
being resampled, narrow banded, or with instrument line noise removed).
The amplitude, $A(t_i,\vec{\lambda})$, is assumed roughly constant at the gravity wave source,
but is modulated by variation in the detector's response due to the Earth's motion.  The phase, $\Phi(t_i,\vec{\lambda})$,
is modulated by both the intrinsic spin down of the source, and the changes in relative motion between the source
and the detector.  This can be calculated for known pulsars.  The vector $\vec{\lambda}$ is a vector of parameters
that describe the sky position, etc., of the source and location, etc., of the detector.
Finally, $n(t_i)$ is the noise, which also includes any other signals that are not coherent with
the phase $\Phi(t_i,\vec{\lambda})$.

The folded amplitude is given by
$$
c_{\rm F} [j] = \sum_{i'}
\left \{ A(t_i,\vec{\lambda})\sin[\Phi(t_i,\vec{\lambda})] + n(t_i) \right \} ,
$$
where the sum over $i'$ means sum over all $i$'s with $\Phi$ in phase bin $j$.
If the bin sizes are sufficiently small, then $c_{\rm F} [j]$ can be approximated as
$$
c_{\rm F} [j] = \sin\Phi_j\sum_{i'} A(t_i,\vec{\lambda}) + \sum_{i'} n(t_i) ,
$$
where $\Phi_j$ is representive of the phase for bin $j$ (e.g., the phase corresponding to the midpoint of the bin).
However, because of amplitude modulation, the amplitudes that are added to a phase bin are not guaranteed to enter
with the same sign.  Thus, some sort of amplitude demodulation should be done.

If we demodulate $A(t_i,\vec{\lambda})$ (for example, in a minimum way such as multiplying by the sign
of the response function) we multiply each element of the vector $c[i]$ by an amplitude demodulation factor $D(t_i)$
$$
c_{\rm D\, , F} [j] = \sin\Phi_j \sum_{i'} D(t_i) A(t_i,\vec{\lambda}) + \sum_{i'} D(t_i) n(t_i) ,
$$
If the average value of $D(t_i)$ is zero, and is not correlated with the noise, then
$$
\sum_{i'} D(t_i) n(t_i) \approx 0
$$
However, the average value of $D(t_i)$ is probably not zero.
The following is a very preliminary suggestion of how to further reduce the noise.
Consider folding the measured strains, $c[i]$, again,
but this time shifting the phase bins by $\pi$.  Define this phase shifted folded amplitude as:
$$
c_{\pi, \, \rm D\, , F} [j] = \sin(\Phi_j + \pi) \sum_{i''} D(t_i) A(t_i,\vec{\lambda}) + \sum_{i''} D(t_i) n(t_i) ,
$$
where the sum over $i''$ means sum over all $i$'s with $\Phi + \pi$ in phase bin $j$.
This will reverse the sign of the sum of the amplitudes that enter into each phase bin, but
the sum of the noise contributions into each bin should be roughly the same.
If the signal we are searching for is present, then amplitudes, $A(t_i,\vec{\lambda})$ are correlated with $D(t_i)$ such that
$$
\sum_{i'} D(t_i) A(t_i,\vec{\lambda}) \approx \bar{A} = {\rm constant}
$$
Thus,
$$
c_{\rm D\, , F} [j] - c_{\pi, \, \rm D\, , F} [j] \approx 2 \bar{A}\sin\Phi_j ,
$$
plus residual noise.  In practice, one needs to fold the amplitudes only once, and then make the replacement
$$
c_{\rm D\, , F} [j] \rightarrow c_{\rm D\, , F} [j] - c_{\rm D\, , F} [(j + N/2) \, \% \, j] ,
$$
where $N$ is the number of phase bins.  We can then statistically analyze the hypothesis that
the demodulated folded amplitudes correspond to a sinusoid.

\newpage\input{FoldAmplitudesH}

\chapter{Package \texttt{pulsar}: Coherent search routines}
Steven Berukoff, M. Alessandra Papa
\bigskip

This package provides a routine to perform a demodulation on a set of data.
In particular, this routine works with frequency domain data by combining
short timescale Fourier Transforms (SFTs) into longer time baseline
demodulated Fourier Transforms (DeFTs). If the assumptions under which the
method was developed are met (\cite{Williams:1999}), then the demodulation
procedure concentrates the total power (within $5\%-10\%$) in a single
frequency bin. In practice, due to the discretization of frequency space, this
power may be shared between two neighbouring bins.

The procedure follows that outlined in \cite{Williams:1999} and is part of the
continuous-wave search algorithm outlined in \cite{Schutz:1999}.  Briefly, the
routine takes input SFTs, corrects for modulation effects due to intrinsic
frequency spindown and Earth's motion, and outputs a DeFT of long time
baseline. In general the routine can be easily adapted to correct for an
arbitrary modulation effect simply by the use of a suitable timing routine,
here \verb+tdb()+ .

The package is organized under the headers \verb+LALDemod.h+, \verb+LALComputeAM.h+, and
\verb+ComputeSky.h+ and the modules \verb+LALDemod.c+ and \verb+ComputeSky.c+.


\newpage\input{LALDemodH}
\newpage\input{ComputeSkyH}
\newpage\input{ComputeSkyBinaryH}
\newpage\input{LALComputeAMH}

\newpage\begin{thebibliography}{0}
\bibitem{Williams:1999}
        Peter R. Williams, Bernard F. Schutz.  gr-qc/9912029.
\bibitem{Schutz:1999}
        Bernard F. Schutz, M. A. Papa.  gr-qc/9905018. 
\end{thebibliography}

\chapter{Package \texttt{pulsar}: known pulsar time-domain search routines}

This package provides routines for a time domain search of gravitational wave
signals from known pulsars.  The documentation and functionality of this
package is \textbf{incomplete}.
\newpage\input{HeterodynePulsarH}
\newpage\input{BinaryPulsarTimingH}
\newpage\begin{thebibliography}{0}
\bibitem{TaylorWeisberg:1989}
        J. Taylor and J. Weisberg, \it{Ap. J.}, \bf{345}, 1989.
\bibitem{BlandfordTeukolsky:1976}
        R. Blandford and S. Teukolsky, \it{Ap. J.}, \bf{205}, 1976.
\bibitem{ChLangeetal:2001}
        Ch. Lange {\it et. al.}, \it{Mon. Not. R. Astron. Soc.}, \bf{326}, 2001.
\bibitem{DamourDeruelle:1985}
        T. Damour and N. Deruelle, \it{Ann. Inst. H. Poincar\'e (Phys. Th\'eorique)}, \bf{43}, 1985.
\bibitem{Wex:1998}
        N. Wex, \it{Mon. Not. R. Astron. Soc.}, \bf{298}, 1998.
\end{thebibliography}
\newpage\input{FitToPulsarH}
\newpage\input{PulsarCatH}



\part{Some Coding Details:}
\chapter{The RCSID goes in every code file}
In every code file (i.e., \texttt{file.h, file.c}) there is macro that looks
like
\input{LALHelloNRCSID}
This macro assign a character variable (in this case \texttt{LALHELLOC} in the
file \texttt{LALHello.c}) the string with the cvs-supplied revision numbers.
The reason we use this macro is that it gives no "unused variable" warnings
form the \texttt{.h}-files when the code is compiled with the \texttt{-Wall}
option.

\chapter{The status macro}

\printindex

\end{document}
