% This file is meant to be included in another
\documentclass{article}
\begin{document}
\section{TFR package}

\subsection{Purpose}
Compute a time-frequency representation (TFR) of a given signal. Four
types of TFRs are currently available. The algorithms are inspired by
the Time-Frequency toolbox for Matlab \cite{tftb}.

\subsection{Synopsis}

% Syntax: argument definitions, calling signature

\begin{verbatim}
#include "TimeFreq.h"

void CreateTimeFreqRep (Status*, TimeFreqRep**, CreateTimeFreqIn*);
void CreateTimeFreqParam (Status*, TimeFreqParam**, CreateTimeFreqIn*);
void DestroyTimeFreqRep (Status*, TimeFreqRep**);
void DestroyTimeFreqParam (Status*, TimeFreqParam**);
void TfrSp (Status*, REAL4Vector*, TimeFreqRep*, TimeFreqParam*);
void TfrWv (Status*, REAL4Vector*, TimeFreqRep*, TimeFreqParam*);
void TfrPswv (Status*, REAL4Vector*, TimeFreqRep*, TimeFreqParam*);
void TfrRsp (Status*, REAL4Vector*, TimeFreqRep*, TimeFreqParam*);
void Dwindow (Status*, REAL4Vector*, REAL4Vector*);
\end{verbatim}

\subsection{Description}
This package is part of the so-called ``Signal Track Algorithm'' which
is a time-frequency based detector of unmodeled gravitational wave
sources. It provides a LAL-style C version of some functions of the
Time-Frequency toolbox for Matlab \cite{tftb}.

Four different TFRs are currently available (see
\cite{flandrin99:tf_ts} for a detailed description of what the
following designations stand for)~:
\begin{enumerate}
\item the spectrogram (function:
  \texttt{TfrSp()}, type: \texttt{Spectrogram})
  $S_x^h(t,f)=|F_x^h(t,f)|^2$ is defined as the
  square modulus of the short-time Fourier transform~:
\begin{equation}
\label{sp}
F_x^h(t,f)=\int{x(s)h(t-s)e^{-2i\pi sf}\,ds}.
\end{equation}
\item the Wigner-Ville distribution (function: \texttt{TfrWv()}, type:
\texttt{WignerVille}) is defined as~:
\begin{equation}
\label{wv}
W_x(t,f)=\int{x(t+s/2)x(t-s/2)e^{-2i\pi sf}\,ds}.
\end{equation}
\item the pseudo smoothed Wigner-Ville distribution (function:
\texttt{TfrPswv()}, type: \texttt{PSWignerVille}) is a smoothed
version of the previous TFR. Its definition reads~:
\begin{equation}
\label{pswv}
PSW_x^{h,g}(t,f)=\int{h(\tau)\int{g(s-t)x(t+s/2)x(t-s/2)\,ds}\:e^{-2i\pi \tau
f}\,d\tau}.
\end{equation}
\item the reassigned spectrogram (function: \texttt{TfrRsp()}, type:
\texttt{RSpectrogram}) is modified version of the spectrogram (see
above) by the reassignment method through the following transformation
operation~:
\begin{equation}
\label{rsp}
\check{S}_x^h(t,f)=\int{\!\!\!\int{S_x^{h}(s,\xi)\delta(t-\hat{t}(s,\xi),f-\hat{f}(s,\xi))\,dsd\xi}},
\end{equation}
where the reassignment operators can be written for the time operator as~:
\begin{equation}
\hat{t}_x^{h}(t,f)=t+\Re\left\{\frac{F_x^{{\cal T}h}(t,f)}{F_x^h(t,f)}\right\},
\end{equation}
and for the frequency operator as~:
\begin{equation}
\hat{f}_x^{h}(t,f)=f-\Im\left\{\frac{F_x^{{\cal D}h}(t,f)}{F_x^h(t,f)}\right\}.
\end{equation}

These two operators require the computation of the short-time Fourier
transforms with the two following windows~:${\cal T}h(t)=t
h(t)$ and ${\cal D}h(t)=dh/dt$.
\end{enumerate}

The first three algorithms are inspired by a C translation of the
Time-Frequency toolbox for Matlab by M. Davy (IRCyN, Ecole Centrale de
Nantes) and E. Leroy which is currently under development.

\subsubsection{Memory allocation}
A TFR and its associated parameters are thought as a distinct data
entities. Specific structures have been designed in that purpose. In
order to create and allocate memory space to these structures, it is
mandatory to first define and fill the \texttt{CreateTimeFreqIn} input
structure:
\begin{verbatim}
typedef struct tagCreateTimeFreqIn {
  TimeFreqRepType type;             /* type of the TFR */
  INT4 fRow;                        /* number of freq bins in the TFR matrix */
  INT4 tCol;                        /* number of time bins in the TFR matrix */
  INT4 wlengthT;                    /* (Sp, Pswv and Rsp) Window length */
  INT4 wlengthF;                    /* (Pswv) Window */
} CreateTimeFreqIn;
\end{verbatim}
and then run the following routines \texttt{CreateTimeFreqRep()} and
\texttt{CreateTimeFreqParam()} to create and allocate space for the
time-frequency representation structure~:
\begin{verbatim}
typedef struct tagTimeFreqRep {
  TimeFreqRepType type;             /* type of the TFR */
  INT4 fRow;                        /* number of freq bins in the TFR matrix */
  INT4 tCol;                        /* number of time_bins in the TFR matrix */
  INT4 *timeInstant;                /* time instant for each column of the TFR */
  REAL4 *freqBin;                   /* freqs for each row of the matrix */
  REAL4 **map;                      /* TFR */
} TimeFreqRep;
\end{verbatim}
and its associated parameters~:
\begin{verbatim}
typedef struct tagTimeFreqParam {
  TimeFreqRepType type;                   /* type of the TFR */
  REAL4Vector *windowT;                   /* (Sp, Rsp and Pswv) Window */
  REAL4Vector *windowF;                   /* (Pswv) Window */
} TimeFreqParam;
\end{verbatim}

\vspace{3mm}
Notes:
\begin{enumerate}
\item The number of frequency bins \texttt{fRow} must be a power of 2.
\item The (time-smoothing) window length \texttt{wlengthT}
(frequency-smoothing) window length \texttt{wlengthF} must be odd
integers (so that the window time center is on the time grid).
\end{enumerate}

\subsubsection{TFR computation}
Next, the user needs to specify in \texttt{timeInstant} the time
instants at which the TFR will be computed and the window(s)
(\texttt{windowT} and \texttt{windowF}) if required.

The routines \texttt{TfrSp()}, \texttt{TfrWv()}, \texttt{TfrPswv()}
and \texttt{TfrRsp()} perform the computation of the chosen TFR at the
time-frequency points $(time,freq)$, where $time$ describes the
\texttt{tCol} given time locations and $freq=k/\mathtt{fRow}$, $k = 0
\ldots\mathtt{fRow}/2$ scans the \texttt{fRow/2} frequency bins
spanning from $0$ to $1/2$ (sampling frequency assumed to be $1$).

The result is stored in the structure \texttt{TimeFreqRep} under the
label \texttt{map}: \texttt{map[column][row]} is a matrix with
\texttt{tCol} columns and \texttt{fRow} rows.

We summarize the memory allocation and the intput and output of the
computation stage in Table \ref{tfr_tablea}.

\vspace{3mm}
Notes:
\begin{enumerate}
\item An additional routine \texttt{Dwindow()} is used internally in
\texttt{TfrRsp()} to compute numerically the derivative of the
spectrogram window.
\end{enumerate}

\subsubsection{Memory desallocation}
The memory can be finally freed using the following routines~:
\texttt{DestroyTimeFreqRep()} and \texttt{DestroyTimeFreqParam()}.

\begin{table}
\begin{tabular}{|l||l|l|l|}
\hline
Function & Memory allocation & Input & Output\\
\hline
\texttt{TfrSp()} & & &\\
& $\bullet$ In \texttt{TimeFreqIn}: & $\bullet$ In \texttt{TimeFreqRep}:  &$\bullet$ In \texttt{TimeFreqRep}:\\
& \texttt{type = Spectrogram} & \texttt{timeInstant*} & \texttt{map**} \\
&\texttt{tCol} &  & \texttt{freqBin*} \\
& \texttt{fRow} &$\bullet$ In \texttt{TimeFreqParam}:&\\
& \texttt{wlengthT}& \texttt{windowT*} $\equiv h(t)$ in eq. (\ref{sp})&\\
\hline
\texttt{TfrWv()} & & &\\
& $\bullet$ In \texttt{TimeFreqIn}:& $\bullet$ In \texttt{TimeFreqRep}:  &$\bullet$ In \texttt{TimeFreqRep}:\\
& \texttt{type = WignerVille} & \texttt{timeInstant*} & \texttt{map**} \\
& \texttt{tCol} & & \texttt{freqBin*} \\
& \texttt{fRow} & $\bullet$ In \texttt{TimeFreqParam}:& \\
\hline
\texttt{TfrPswv()} & & &\\
& $\bullet$ In \texttt{TimeFreqIn}: & $\bullet$ In \texttt{TimeFreqRep}:  & $\bullet$ In \texttt{TimeFreqRep}:\\
& \texttt{type = PSWignerVille} & \texttt{timeInstant*} & \texttt{map**} \\
&\texttt{tCol} & & \texttt{freqBin*} \\
& \texttt{fRow} &$\bullet$ In \texttt{TimeFreqParam}:&\\
& \texttt{wlengthT} &\texttt{windowT*} $\equiv g(t)$ in eq. (\ref{pswv}))&\\
& \texttt{wlengthF} &\texttt{windowF*} $\equiv h(t)$ in eq. (\ref{pswv})&\\
\hline
\texttt{TfrRsp()} & & &\\
& $\bullet$ In \texttt{TimeFreqIn}: & $\bullet$ In \texttt{TimeFreqRep}:  &$\bullet$ In \texttt{TimeFreqRep}:\\
& \texttt{type = RSpectrogram} & \texttt{timeInstant*} & \texttt{map**} \\
&\texttt{tCol} & & \texttt{freqBin*} \\
& \texttt{fRow} & $\bullet$ In \texttt{TimeFreqParam}:&\\
& \texttt{wlengthT}& \texttt{windowT*}$ \equiv h(t)$ in eq. (\ref{rsp})&\\
\hline
\end{tabular}
\caption{\label{tfr_tablea} This table lists what need to be specified in the
\texttt{TimeFreqIn}, \texttt{TimeFreqRep} and \texttt{TimeFreqParam}
structures for the memory allocation, before starting the computation
(input)  and what results after the computation (output).}
\end{table}

\subsection{Operating Instructions}

% Detailed usage

\begin{verbatim}
const INT4 Nsignal=16;
const INT4 NwindowT=3;
const INT4 NwindowF=5;
const INT4 Nfft=8;

static Status status;

REAL4Vector*  signal = NULL;
CreateTimeFreqRepIn tfrIn;
TimeFreqRep*tfr = NULL;
TimeFreqParam *param = NULL;

INT4 column;
INT4 row;

tfrIn.type=Spectrogram; /* for ex. the spectrogram */
tfrIn.fRow=Nfft;
tfrIn.tCol=Nsignal;
tfrIn.wlengthT=NwindowT;
tfrIn.wlengthF=NwindowF;

SCreateVector(&status, &signal, Nsignal);
CreateTimeFreqRep(&status, &tfr, &tfrIn);
CreateTimeFreqParam(&status, &param, &tfrIn);

/* assign data ... */

/* perform TFR */

TfrSp(&status,signal,tfr,param);

/* TfrWv(&status,signal,tfr,param); */
/* TfrPswv(&status,signal,tfr,param); */
/* TfrRsp(&status,signal,tfr,param); */

%/* destroy input vectors, TimeFreqRep and TimeFreqParam structures */
SDestroyVector(&status,&signal);
DestroyTimeFreqRep(&status,&tfr);
DestroyTimeFreqParam(&status,&param);
\end{verbatim}

\subsubsection{Arguments}

% Describe meaning of each argument

\begin{itemize}
\item \texttt{status} is a universal status structure. Its contents are
assigned by the functions.
\item  \texttt{signal} is the input signal
\item \texttt{tfrIn} is the time-frequency input structure (for memory
allocation).
\begin{itemize}
\item \texttt{type} is the type of the chosen TFR (see Table \ref{tfr_tablea}).
\item \texttt{fRow} is the number of frequency bins.
\item \texttt{tCol} is the number of time bins.
\item \texttt{wlengthT} is the (time-smoothing) window length (for Sp,
Rsp and Pswv).
\item \texttt{wlengthF} is the (frequency-smoothing) window length (for Pswv).
\end{itemize}
\item \texttt{tfr} is the time-frequency representation structure.
\begin{itemize}
\item \texttt{type} is the type of the chosen TFR (see Table \ref{tfr_tablea}).
\item \texttt{fRow} is the number of frequency bins.
\item \texttt{tCol} is the number of time bins.
\item \texttt{freqBin} defines the centers of each frequency bins.
\item \texttt{timeInstant} defines the centers of each time bins.
\item \texttt{map} is the TFR matrix.
\end{itemize}
\item \texttt{param} is the time-frequency parameter structure.
\begin{itemize}
\item \texttt{type} is the type of the chosen TFR (see Table \ref{tfr_tablea}).
\item \texttt{windowT} is the (time-smoothing) window (for Sp,
Rsp and Pswv).
\item \texttt{windowF} is the (frequency-smoothing) window length (for
Pswv).
\end{itemize}
\end{itemize}

\subsubsection{Options}

None.

\subsubsection{Error conditions}

% What constitutes an error condition? What do the error codes mean?

These functions all set the universal status structure on return.
Error conditions are described in the tables \ref{tfr_tableb} and
\ref{tfr_tablec}.

\begin{table}
\begin{tabular}{|r|l|p{2in}|}
\hline
status  & status          & Description\\
code    & description     & \\
\hline
CREATETFR\_ENULL 1  & Null pointer
& an argument is NULL or contains a NULL pointer\\
CREATETFR\_ENNUL 2  & Non-null pointer
& trying to create a TFR structure that already exists\\
CREATETFR\_EFROW 4  & Illegal number of freq bins
& the number of frequency bins is not a power of 2\\
CREATETFR\_ETCOL 8  & Illegal number of time instants
& the number of time bins is not a positive integer\\
CREATETFR\_EMALL 16 & Malloc failure
& Malloc refuses to allocate memory space\\
\hline
\hline
DESTROYTFR\_ENULL 1  & Null pointer
& an argument is NULL or contains a NULL pointer\\
\hline
\hline
CREATETFP\_ENULL 1  & Null pointer
& an argument is NULL or contains a NULL pointer\\
CREATETFP\_ENNUL 2  & Non-null pointer
& trying to create a TFR parameter structure that already exists\\
CREATETFP\_ETYPE 4  & Unknown TFR type
& the chosen TFR type is available\\
CREATETFP\_EWSIZ 8 & Invalid window size
& the window length is not a positive odd integer\\
CREATETFP\_EMALL 16  & Malloc failure
& Malloc refuses to allocate memory space\\
\hline
\hline
DESTROYTFP\_ENULL 1  & Null pointer
& an argument is NULL or contains a NULL pointer\\
DESTROYTFP\_ETYPE 2  & Unknown TFR type
& the chosen TFR type is not available\\
\hline
\end{tabular}
\caption{\label{tfr_tableb}Error conditions for memory allocation
functions of the TFR package}
\end{table}

\begin{table}
\begin{tabular}{|r|l|p{2in}|}
\hline
status  & status          & Description\\
code    & description     & \\
\hline
TFR\_ENULL 1 & Null pointer
& an argument is NULL or contains a NULL pointer\\
TFR\_ENAME 2 & TFR type mismatched
& the function does not compute this TFR type\\
TFR\_EFROW 4 & Invalid number of freq bins
& the number of frequency bins is not a power of 2\\
TFR\_EWSIZ 8 & Invalid window length
& the window length is not a positive odd integer\\
TFR\_ESAME 16 & Input/Output data vectors are the same
& input and output vectors need to be distinct\\
TFR\_EBADT 32 & Invalid time instant
& request of a computation at a time instant
when the signal does not exist\\
\hline
\end{tabular}
\caption{\label{tfr_tablec} Error conditions for computation
functions of the TFR package}
\end{table}

\subsection{Algorithms}
% Describe algorithm by which work is done
The algorithm are taken from the Time-Frequency toolbox for Matlab
\cite{tftb}.

\subsection{Accuracy}
% For numerical routines address issues related to accuracy:
% approximations, argument ranges, etc.

\subsection{Tests}
% Describe the tests that are part of the test suite
The test files provide the TFR of a random signal (16 time samples).
The resulting TFRs are identical to the ones given by the Time-Frequency Toolbox.

\subsection{Uses}
% What LAL, other routines does this one call?
\begin{itemize}
\item\texttt{SCreateVector()}
\item\texttt{SDestroyVector()}
\item\texttt{CCreateVector()}
\item\texttt{CDestroyVector()}
\item\texttt{EstimateFwdRealFFTPlan()}
\item\texttt{DestroyRealFFTPlan()}
\item\texttt{MeasureFwdRealFFTPlan()}
\item\texttt{RealPowerSpectrum ()}
\item\texttt{FwdRealFFT()}
\end{itemize}

\subsection{Notes}

\subsection{References}

% Any references for algorithms, tests, etc.
\begin{thebibliography}{0}

\bibitem{tftb}
F.~Auger, P.~Flandrin, P.~Gon\c{c}alv\`es, O.~Lemoine.
\newblock Time-frequency toolbox for M\textsc{atlab},
user's guide and reference guide.
\newblock Available on the Internet at the URL
\texttt{http://iut-saint-nazaire.univ-nantes.fr/ $\tilde{ }$\,auger/tftb.html}.

\bibitem{flandrin99:tf_ts}
P.~Flandrin.
\newblock {\em Time-Frequency/Time-Scale Analysis}.
\newblock Academic Press, San Diego (CA), 1999.
\end{thebibliography}

\end{document}

%%% Local Variables:
%%% mode: latex
%%% TeX-master: t
%%% End:

